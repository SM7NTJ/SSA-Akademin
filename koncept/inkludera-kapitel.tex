% Kapitel 1 Ellära
\chapter{Ellära}
\label{ellära}

% Avsnitt 1.1 Elektriska grundbegrepp
\chapter{Ellära}
\label{ellära}

\nobalance

\section{Elektriska grundbegrepp}

\harecsection{\harec{a}{1.1}{1.1}}

Elektrisk laddning, spänning och ström hänger samman med hur materian är
uppbyggd.
Den förmåga ett material har att leda laddningar, det vill säga ström, kallas
konduktivitet.

\subsection{Grundämnen}
\index{grundämnen}

Det finns många former av materia.
Ofta är en form av materia sammansatt av andra former med enklare uppbyggnad.

Sammansatt materia kan sönderdelas på kemisk väg, men däremot inte de enklaste
formerna.
All materia är uppbyggd av atomer.
De enklaste materieformerna, som kallas \emph{grundämnen}, innehåller endast
ett slags atomer.
Över 100~grundämnen är kända.

Vart och ett av grundämnena har sin speciella atomuppbyggnad och därmed en
materialstruktur, som skiljer sig från varje annat grundämne.

Tre fjärdedelar av alla grundämnen är metaller (elektriska ledare) medan de
flesta övriga är icke-metaller (isolatorer).
Det finns även en liten mellangrupp som kallas för halvledare.

\subsection{Atomernas uppbyggnad}
\index{elektron}
\index{valenselektron}

Länge ansågs atomerna vara de minsta beståndsdelarna i materian.
Men omkring förra sekelskiftet upptäcktes att atomerna i sin tur består av ännu
mindre beståndsdelar, så kallade elementarpartiklar såsom protoner, neutroner,
elektroner med flera.
Det gemensamma namnet för alla dessa partiklar är \emph{nukleoner}.

En atom består dels av en kärna som är sammansatt av protoner och neutroner,
dels av elektroner, som kretsar omkring kärnan.

\begin{itemize}
	\item Protonerna är positivt (+) laddade.
	\item Neutronerna är neutrala, ej laddade.
	\item Elektronerna är negativt (--) laddade.
\end{itemize}

\smallfig{images/cropped_pdfs/bild_2_1-01.pdf}{Atomernas uppbyggnad}{fig:BildII1-1}

Elektronerna kretsar i banor omkring atomkärnorna, liksom
planeterna kretsar i banor omkring sina solar, vilket illustreras i bild
\ssaref{fig:BildII1-1}.

Banor med samma avstånd till atomkärnan är på samma energinivå och sägs bilda
ett elektronskal.

Det kan finnas flera elektronskal.
Ju fler elektroner som finns i ett elektronskal, desto starkare är elektronerna
i skalet bundna till atomen.
Det yttersta skalet kan emellertid aldrig innehålla fler än 8~elektroner.

Elektronerna i det yttersta skalet kallas för \emph{valenselektroner}, vilka
används även av angränsande atomer vid den kemiska bindningen till
atomstrukturer, molekyler och ämnen.
För bindningen behövs ett visst antal valenselektroner.

De valenselektroner som ej behövs för bindningen kan röra sig fritt genom
materia/strukturen.
De kallas fria elektroner och är vad vi kallar elektrisk ström.

Valenselektronerna är alltså inte bara av betydelse för materialets kemiska
struktur utan också för dess elektriska egenskaper.

Atomernas massa och volym är ytterst liten.
Tag som exempel en kub av koppar med volymen \qty{1}{\cubic\centi\metre} och
vikten \(8,9\ gram\).
Den består av ca \(8,5 \cdot 10^{22}\) kopparatomer, dvs.
\(85\, 000\, 000\, 000\, 000\, 000\, 000\, 000\) stycken.
Fenomenet metallbindning gör att antalet fria elektroner i kuben är ungefär lika
med antalet atomer i den.

Varje elementarpartikel har en massa och en atoms totala massa är summan av
elementarpartiklarnas massor.
Den enklaste atomen är väteatomen med en proton och en elektron.
Väteatomens totala massa har kunnat beräknas till \(1,66 \cdot 10^{-24}\) gram.

Nästan hela massan i atomen är samlad till kärnans protoner och neutroner.
Var och en av dem har en massa som är ungefär 2000 gånger större än massan i en
elektron.

\subsection{Elektrisk laddning och kraftverkan}
\index{elektrisk laddning}

Enligt sägnen upptäckte Thales från Milteus redan för 2500~år sedan, att en bit
bärnsten drog till sig små grässtrån, sedan stenen gnidits mot en bit ylle.
Det grekiska ordet för bärnsten är ELEKTRON och de krafter som uppstod kom att
kallas ''elektriska''.
Av den elektriska spänningen mellan kroppar med olika laddning, verkar krafter
mellan dem och deras omgivning.
Krafterna kallas för elektriska fält och är det som gör att elektriskt laddade
kroppar kan komma i rörelse.


Ett exempel får man varje gång man kammar sig med en kam av isolerande material.
Då kommer håret att dras mot kammen därför att håret och kammen har
fått olika slags elektriska laddningar.
Samtidigt har hårstråna sinsemellan samma slags laddning och stöter bort
varandra -- håret ''reser sig''.
\emph{Lika laddningar stöter bort varandra -- olika laddningar drar varandra till sig.}

\subsection{Konduktivitet -- ledare, halvledare och isolator}
\harecsection{\harec{a}{1.1.1}{1.1.1}}
\index{konduktivitet}
\label{konduktivitet}

En elektrisk ström sägs flyta, när de fria laddningsbärarna i ett material -- en
strömledare -- fås att röra sig samtidigt i samma riktning.
Hur många som rör sig beror på strömledarens egenskaper och spänningen mellan
ledarens ändar.

Alla material har någon grad av elektrisk ledningsförmåga som beror på
materialets atomstruktur, dimensioner och temperatur.
Vissa material (t.ex. metaller, kol, halvledare) leder elektrisk ström bättre
än andra (t.ex. glas, gummi, plast).
Mängden av fria laddningsbärare i materialet begränsar hur stor strömmen kan
bli.

\subsubsection{Ledare}
\index{ledare}
\index{konduktivitet!ledare}
\label{ledare}

Metaller har god elektrisk ledningsförmåga och kallas ledare.
Bäst ledande är de metaller, vars atomer har det minsta antalet
valenselektroner i det yttersta elektronskalet.
Koppar-, silver- och guldatomerna har en enda valenselektron och därmed mycket
god ledningsförmåga.
Järn, zink och magnesium har två valenselektroner och därmed något sämre
ledningsförmåga.
Tabell~\ssaref{table:metaller} ger exempel på metallernas resistivitet.
Ännu sämre ledare är de så kallade halvledarna med 3 till 5 valenselektroner.

\begin{table}
  \begin{center}
\begin{tabular}{l|l}
  Ämne & Resistivitet vid \qty{20}{\degreeCelsius} \\
   & \([\unit{\ohm} \cdot mm^2 / m]\) \\
  \hline
  Aluminium   & 0,028 \\
  Bly         & 0,22  \\
  Guld        & 0,024 \\
  Järn        & 0,105 \\
  Koppar      & 0,018 \\
  Kvicksilver & 0,958 \\
  Nickel      & 0,078 \\
  Platina     & 0,108 \\
  Silver      & 0,016 \\
  Tenn        & 0,115 \\
  Volfram     & 0,056 \\
  Zink        & 0,058 \\
\end{tabular}
\end{center}
  \caption{Metallernas resistivitet}
          \label{table:metaller}
\end{table}

\subsubsection{Isolatorer}
\index{isolator}
\index{konduktivitet!isolator}
\label{isolator}

Glas, plast, porslin och vissa mineraler har mycket dålig ledningsförmåga och
kallas isolatorer.
Isolatorerna är dåliga ledare på grund av att de har många valenselektroner i
sitt yttersta skal.
Maximalt ryms 8 valenselektroner.

I icke ledande material är elektronerna mycket hårt bundna till sitt valensskal
och därför svåra att flytta.
I fasta material är också positiva laddningar svåra att flytta, eftersom de är
bundna i atomkärnorna.
Atomerna är i sin tur bundna i en struktur som kännetecknar vart och ett
material.

\subsubsection{Halvledare}
\index{halvledare}
\index{konduktivitet!halvledare}
\index{halvledare!intrinsisk}
\index{halvledare!dopning}
\index{halvledare!ledningsförmåga}
\label{halvledare}

Några grundämnen har en elektrisk ledningsförmåga som ligger mellan gränsvärdena
för att kallas elektriska ledare eller isolatorer.
Dessa ämnen tillhör gruppen halvledare och har en elektrisk ledningsförmåga som
varierar med ämnets struktur, renhet och temperatur.

En ren kristall av mineralen germanium [Ge] eller av kisel [Si] bildar ett
kristallgitter där atomerna binds till varandra med kovalenta bindningar.
Ämnena delar sina fyra valenselektroner med fyra andra atomer så att det
bildas en full oktett med åtta elektroner i valensskalet.

Då valensskalet innehåller åtta elektroner är det fullt, det finns inga fria
elektroner och ämnet leder inte elektrisk ström.
Båda dessa mineraler kan därför i denna form ses som isolatorer.
(\emph{intrinsisk halvledare})

Om några atomer av ett främmande material som till exempel arsenik, antimon,
indium eller gallium blandas in, (\emph{dopas in}), i kristallstrukturen så
förändras egenskaperna och den elektriska ledningsförmågan ökar tusenfalt.

\subsubsection{N-ledning}
\index{halvledare!N-ledning}

Man talar om N-ledande material respektive N-led\-ni\-ng; ''elektronled\-ni\-ng''.

Germanium, kisel m.fl. halvledare har fyra elektroner med ''fasta platser'' i
valensskalet -- förutsatt att materialet är helt rent.
Då finns det inga fria elektroner för laddningstransport.

För att skapa fria elektroner kan det rena materialet förorenas -- dopas -- med
atomer av till exempel arsenik [As] eller antimon [Sb].
Båda dessa material är 5-värdiga.
De har 5 elektroner i valensskalet 4 elektroner är fast bundna medan
den 5:e är löst bunden till atomen.
Den 5:e elektronen kan lossgöras från atomen med yttre kraft, till exempel värme eller
elektrisk spänning och då skapas en fri elektron.
När en spänning läggs på materialet kommer den fria elektronen att vandra mot
den positiva polen.
Materialet är N-ledande.

\subsubsection{P-ledning -- ''hålledning''}
\index{halvledare!P-ledning}

När germanium eller kisel dopas med indium [In] eller gallium [Ga] blir de
P-ledande.
Indium och gallium är 3-värdiga -- deras valensskal innehåller 3 elektroner.
Men för en fast bindning med germanium eller kisel saknas det en elektron och
det uppstår då ett ''hål'' -- en ''bristelektron''.
Hålet kan fyllas ut av en elektron från en annan atom.
I den atom som elektronen lämnar bildas det i sin tur ett hål osv.
När en spänning läggs på, kommer ''hålet'' att vandra mot den negativa polen.
Materialet är då P-ledande.

\subsection{Elektrisk spänning -- enheten volt}
\label{spänning}
\harecsection{\harec{a}{1.1.2}{1.1.2b}, \harec{a}{1.1.3}{1.1.3b}}
\index{elektrisk spänning}
\index{spänning}
\index{volt (V)}
\index{enheter!volt (V)}
\index{symbol!\(U\) spänning}
\index{symbol!\(V\) spänning}
\index{Likspänning}
\index{Växelspänning}
\index{AC alternating current}
\index{AC}
\index{DC direct current}
\index{DC}

\mediumfig{images/cropped_pdfs/bild_2_1-02.pdf}{Tankeförsök med kulor i ett rör}{fig:BildII1-2}

Bild~\ssaref{fig:BildII1-2} illustrerar ett tankeförsök med ett rör med kulor i.
Materialet i röret tänks motsvara atomstrukturen i en strömledare och kulorna
de fria elektronerna.
Tänker man sig ett slag mot en ände av röret så flyttar det sig av den energi
som tillförs.
På grund av obundenheten till röret följer av masströgheten kulorna inte med
röret, utan hamnar i dess ena ände.

Att kulorna samlas i ena änden av röret tänks motsvara ett elektronöverskott i
ena änden av en ledare och ett motsvarande underskott i den andra änden.

Man kallar änden med elektronöverskott för minuspol och änden med
elektronunderskott för pluspol.
Olika stora elektriska laddningar vid polerna innebär att de sinsemellan har
olika potential.
Potentialskillnaden kallas spänning.

Likspänning DC (eng. \emph{Direct Current}) innebär ett överskott av elektroner
och alltid vid samma anslutningspol.

Växelspänning AC (eng. \emph{Alternating Current}) innebär ett överskott av
elektroner, omväxlande vid den ena anslutningspolen och den andra.

Måttenheten för spänning är \(\mathrm{volt\ [V]}\).
I formler betecknas spänning med
\begin{itemize}
  \item \(U\) för effektivvärdet
  \item \(u\) för momentanvärdet (ögonblicks-)
  \item \(\hat{u}\) för toppvärdet (amplitud-).
\end{itemize}
Bild~\ssaref{fig:BildII1-16} i avsnitt 1.6 illustrerar relationen mellan värdena
för en sinuskurva.

Spänningen över ändpunkterna på en strömledare är \(1\ \mathrm{volt\ [V]}\), då
ledaren genomflyts av en likström av \(1\ \mathrm{ampere\ [A]}\) under
effektutvecklingen \(1\ \mathrm{watt\ [W]}\).

\subsection{Symboler}
\label{spänning.symboler}

När man ritar scheman för elektriska kretsar används symboler.
Symbolen i bild~\ssaref{fig:bildII2-batteri} visar ett elektriskt batteri med en
enda cell.

\smalltikz{
  \begin{circuitikz}
    \draw
    (3,1) to[battery1] (1,1)
    ;
  \end{circuitikz}
}{Schemasymbol för batteri}{fig:bildII2-batteri}


Förtydligande kommentarer och skrivtecknen invid symbolen förekommer.
Ofta refererar dessa till en komponentlista.
Se även kapitel~\ssaref{komponenter}.

\subsection{Elektrisk ström -- Enheten ampere}
\label{elektrisk_ström}
\harecsection{\harec{a}{1.1.2}{1.1.2a}, \harec{a}{1.1.3}{1.1.3a}}
\index{elektrisk ström}
\index{ström}
\index{ampere (A)}
\index{enheter!ampere (A)}
\index{symbol!\(I\) ström}

När en sluten strömkrets innehåller en spänningskälla, kan en
laddningsutjämning ske genom kretsen.
Det innebär att fria elektroner förflyttar sig genom kretsen i riktning från
spänningskällans minuspol till dess pluspol.
Vid pluspolen är det nämligen brist på negativa laddningar och naturen söker
alltid en utjämning.
Under utjämningsförloppet är spänningskällan även en strömkälla.

I gaser och elektrolyter (elektriskt ledande vätskor och geler) samt i
halvledare består strömmen av joner (positiva eller negativa laddningar),
i metaller däremot av elektroner (negativa laddningar).

Av tradition anses strömriktningen vara positiv i jonströmmens riktning -- den
så kallade tekniska strömriktningen -- medan elektronströmmens riktning är den
motsatta -- den så kallade fysikaliska strömriktningen.

Måttenheten för ström är \emph{ampere} \(\mathrm{A}\) \cite{SIbrochure8}.
I formler betecknas ström med:

\begin{description}
\item[\(I\)] för effektivvärdet
\item[\(i\)] för momentanvärdet (ögonblicks-)
\item[\(\hat{i}\)] för toppvärdet (amplitud-)
\end{description}

Strömmen är \qty{1}{\ampere} när \(6,25 \cdot 10^{18}\) elektroner per sekund
flyter genom ett givet ledartvärsnitt, vilket motsvarar laddningen
\(1\ \mathrm{coulomb}\).

\subsection{Strömkrets}
\index{strömkrets}
\label{strömkrets}

\mediumfig{images/cropped_pdfs/bild_2_1-03.pdf}{Potential och spänning i en strömkrets}{fig:BildII1-3}

Bild~\ssaref{fig:BildII1-3} visar potential och spänning i en strömkrets.

En elektrisk strömkrets består av en eller flera energikällor och
energiförbrukare.
Källor kan vara batterier, nätaggregat etc.
Förbrukare kan vara lampor, ledningar etc.
Varje energiförbrukare har en resistans och de elektriska laddningarna ''köar''
före förbrukaren, strax efter förbrukaren finns ingen kö.
Det uppstår en skillnad i laddningsmängd (en potentialskillnad) mellan varje
punkt i en strömkrets, när det flyter ström.
Man talar om spänningsfall.

\subsection{Strömförlopp}
\index{strömförlopp}

Likströms- och växelströmsförloppen kan vara sammansatta av ett huvudförlopp och
underordnade förlopp.

Likström kan ha konstant styrka eller den kan variera enligt något förlopp, men
växlar aldrig riktning.

Växelström kan variera enligt något visst förlopp, till exempel sinusvåg,
fyrkantsvåg, och växlar ständigt riktning.

\subsection{Resistans -- Enheten ohm}
\harecsection{\harec{a}{1.1.2}{1.1.2c}, \harec{a}{1.1.3}{1.1.3c}}
\index{resistans}
\index{ohm (\unit{\ohm})}
\index{enheter!ohm (\unit{\ohm})}
\index{symbol!\(R\) resistans}
\label{resistans}

När fria elektroner tvingas fram genom atomstrukturen i en ledare, till exempel
glödtråden i en lampa, så avgår energi i form av värme.
Detta fenomen kallas för resistans (av latinets resistere som betyder att
motstå).
Resistansen och därmed förlusterna i en strömkrets fördelas i
förhållande till de ingående materialen och deras dimensionering.

Resistans uttrycks i enheten \emph{ohm} \cite{SIbrochure8} och betecknas med
den grekiska bokstaven omega (\unit{\ohm}).

I formler betecknas resistansen i en elektrisk krets eller en del av den med
\(R\).

Resistansen i en resistor är \qty{1}{\ohm}, när en spänning av \qty{1}{\volt}
driver en ström av \qty{1}{\ampere} genom den resistorn.

\subsection{Ohms lag}
\harecsection{\harec{a}{1.1.4}{1.1.4}}
\index{Ohms lag}
\index{resistor!Ohms lag}
\label{ohms_lag}

Ohms lag beskriver sambandet mellan grundbegreppen ström
\(I\ \mathrm{[ampere]}\), spänning \(U\ \mathrm{[volt]}\) och resistans
\(R\ \mathrm{[ohm]}\).
Sambandet gäller både för likspänning och för effektivvärdet av växelspänning och
växelström.

I en ledare med resistansen \(R\) är strömstyrkan \(I\) genom resistansen
proportionell mot den pålagda spänningen \(U\).
%%
\[
\begin{array}{lllll}U=I \cdot R & & I=\dfrac{U}{R} & & R=\dfrac{U}{I}\end{array}
\]
%%
\subsection{Kirchhoffs lagar}
\harecsection{\harec{a}{1.1.5}{1.1.5}}
\index{Kirchhoffs lagar}
\index{Kirchhoffs strömlag}
\index{Kirchhoffs spänningslag}

Den tyske fysikern G~R~Kirchhoff (1824--1887) formulerade sina välkända lagar
först 1845 och sedan 1847.

Kirchhoffs strömlag: \emph{Den algebraiska summan av alla strömmar, som flyter till eller från varje punkt i en elektrisk krets, är lika med noll.}
%%
\[I_1 + I_2 + I_3 + \cdots + I_n = 0\]
%%
Kirchhoffs spänningslag: \emph{I varje sluten strömkrets är den algebraiska summan av alla spänningskällor lika med det totala spänningsfallet i alla resistorer.}

Uttryckt på ett annat sätt är algebraiska summan av spänningarna i en
strömkrets lika med noll.

\subsection{Elektrisk effekt -- enheten watt}
\harecsection{\harec{a}{1.1.6}{1.1.6}, \harec{a}{1.1.7}{1.1.7}}
\index{elektrisk effekt}
\index{effekt}
\index{voltampere (VA)}
\index{enheter!voltampere (VA)}
\index{watt (W)}
\index{enheter!watt (W)}
\index{symbol!\(P\) effekt}
\label{elektrisk_effekt}

När en ström flyter genom en resistans utvecklas värme.
Värme är en form av effekt, som är högre ju starkare strömmen och högre
spänningen är.

Måttenheten \emph{voltampere} \(\mathrm{[VA]}\) för elektrisk effekt härleds ur
produkten av volt \(\mathrm{[V]}\) och ampere \(\mathrm{[A]}\).

För effekt som alstras av likström används enheten \emph{watt} \(\mathrm{[W]}\)
\cite{SIbrochure8} i stället för \emph{voltampere} \(\mathrm{[VA]}\).
Vid sidan om grundenheten \qty{1}{\watt} används delar och multipler av denna.
%%
\[1\ \mathrm{volt\ [U]}\ \cdot\ 1\ \mathrm{ampere\ [I]}\ =\ 1\ \mathrm{watt\ [P]}\]
%%
Effektformeln \(P = U \cdot I\) gäller i första hand för likström men även för
växelström om belastningen är resistiv och ström och spänning inte är
fasförskjutna.
Formeln kan för att underlätta beräkningar skrivas om på flera sätt.

Vi börjar med att lösa ut $I$ ur Ohms lag $U = R \cdot I$:
%%
\[
I = \dfrac{U}{R}
\]
%%
Vi sätter sedan in uttrycket för $I$ i effektformeln
%%
\[
\begin{array}{lllll}
P=U \cdot I & \Rightarrow & P= \dfrac{U \cdot U}{R} & \Rightarrow & P= \dfrac{U^2}{R}
\end{array}
\]
%%
På motsvarande sätt kan vi ersätta $U$ med $R \cdot I$:
%%
\[
\begin{array}{lllll}
P=U \cdot I & \Rightarrow & P = R \cdot I \cdot I  & \Rightarrow & P = R \cdot I^2
\end{array}
\]
%%
Med hjälp av dessa formler kan effekten beräknas ur resistans- och strömvärdena
respektive ur resistans- och spänningsvärdena.
För övriga formler se formelsnurran bild~\ssaref{fig:BildII1-4}.

\subsection{Elektrisk arbete -- enheten joule}
\index{elektriskt arbete}
\index{joule (J)}
\index{enheter!joule (J)}
\index{symbol!\(W\) energi, arbete}

Energi finns i olika former, alltid och överallt.
Energi kan varken skapas eller förstöras, bara omvandlas från en form till en
annan.
Formen kan vara mekanisk, kemisk, elektrisk etc.

Arbete är omvandlingsprocessen från en energiform till en annan.

Arbetsmängden i alla energiformer kan mätas med samma enhet \emph{joule}
\(\mathrm{[J]}\) \cite{SIbrochure8} och anges med symbolen \(W\) för Work.

\(1\ \mathrm{joule}\) motsvarar det arbete som utvecklas när ett föremål
förflyttas \(1\ \mathrm{meter}\) med kraften \(1\ \mathrm{newton\ [N]}\),
det vill säga \(1\ \mathrm{newtonmeter\ [Nm]}\).

\[W = l \cdot F \ \ \ \mathrm{[J] = [Nm]}\]

Arbetet \(W\ \mathrm{[J]}\) är mer ju längre tid \(t\ [s]\) en viss effekt
\(P\ \mathrm{[W]}\) utvecklas.

\[W = t \cdot P \ \ \ \mathrm{[J] = [sW]}\]

\subsection{Joules lag}
\harecsection{\harec{a}{1.1.8}{1.1.8}}
\index{Joules lag}
\label{joules_lag}


Eftersom effekten uttrycks som \(P = U \cdot I\) kan det elektriska arbetet
uttryckas som \(W = U \cdot I \cdot t\), vilket också är Joules lag.

\[\textit{Arbete} = \textit{Effekt} \cdot \textit{tid}\qquad \mathrm{[W]} = \mathrm{[P]} \cdot \mathrm{[s]}\]

Om grundenheterna för volt \(\mathrm{[U]}\), ampere \(\mathrm{[I]}\) och
sekund \(\mathrm{[s]}\) sätts in i formeln fås en måttenhet, uttryckt som
voltamperesekunder \(\mathrm{[VAs]}\) eller wattsekunder \(\mathrm{[Ws]}\)
eller joule\ \(\mathrm{[J]}\).

Måttenheten för elektriskt arbete är 1~joule, som vanligen
kallas \(1\,\mathrm{wattsekund}\ 1\,\mathrm{[Ws]}\).
Vid sidan av grundenheten används multipler av denna.

\begin{center}
\small
\noindent
\begin{tabular}{@{}l@{\,}l@{}l@{}l@{}}
   1 kilowattsekund & = $1\ \text{kWs}$ & = $1000\ \text{Ws}$ & = $1,0\cdot 10^3\ \text{Ws}$ \\
   1 wattimme       & = $1\ \text{Wh}$  & = $3600\ \text{Ws}$ & = $3,6\cdot 10^3\ \text{Ws}$ \\
   1 kilowattimme   & = $1\ \text{kWh}$ & = $1000\ \text{Wh}$ & = $3,6\cdot 10^6\ \text{Ws}$
\end{tabular}
\end{center}

\pagebreak[3]

\subsection{Formelsnurran}
\index{formelsnurran}

\smallfig[0.4]{images/cropped_pdfs/bild_2_1-04.pdf}{''Snurra'' för Ohms och Joules lagar}{fig:BildII1-4}

Så här finner man rätt formel i formelsnurran (bild~\ssaref{fig:BildII1-4}):
Välj ett segment med önskad storhet \(I\), \(U\), \(R\) eller \(P\) som det
första ledet i formeln.
Inom valt segment finns tre alternativ för det andra ledet i formeln.
Välj det alternativ som innehåller två kända storheter.

%%Exempel:
%% TODO: Var är exemplet?

\subsubsection{Ohms lag}

\(R\) söks, \(U\) och \(I\) är kända;
Om \(U = \qty{230}{\volt}\) och \(I = \qty{2}{\ampere}\), så blir
%%
\[R=\dfrac{U}{I}=\dfrac{230}{2}=\qty{115}{\ohm}\]
%%
\subsubsection{Joules lag}

\(P\) söks, \(U\) och \(I\) är kända.
Om \(U = \qty{230}{\volt}\) och \(I = \qty{2}{\ampere}\), så blir
%%
\[P = U \cdot I = 230 \cdot 2 = \qty{460}{\watt}\]
%%
\subsection{Amperetimmar (Ah) och batterikapacitet}
\harecsection{\harec{a}{1.1.9}{1.1.9}}
\index{amperetimmar (Ah)}
\index{batterikapacitet}
\index{batteri}
\index{batteri!primärärbatteri}
\index{batteri!primärärcell}
\index{ackumulator}
\index{ackumulator!sekundärcell}
\index{ackumulator!sekundärbatteri}
\index{polspänning}
\index{elektrisk cell}
\label{batterikapacitet}

Det finns flera sätt att lagra energi.
Ett sätt är att göra det i kemisk form i speciella celler, där man kan ta ut
energin i elektrisk form.

Det finns celler som kan laddas upp och laddas ur upprepade gånger.
Sådana celler kallas vanligtvis ackumulator, sekundärbatteri eller
sekundärcell.

\newpage
Det finns också sådana celler som endast kan användas en gång och som normalt
inte kan laddas upp igen.
Sådana celler kallas vanligtvis primärcell eller primärbatteri.

Energi i form av en elektrisk laddning kan även lagras i en kondensator.
Energin kan då lagras och tas ut utan omvandling.

Kapaciteten i en elektrisk cell uttrycks som produkten av den ström \unit{\ampere}
som cellen avger och under den tid \(\mathrm{[s, h]}\) detta kan ske.
Uttryckt med tidsenheten timmar blir då kapaciteten \unit{Ah}.

Den kapacitet som anges i en cells produktdata är den nominella.
Denna kapacitet gäller endast under vissa normerade förhållanden såsom
celltemperatur, strömstyrka och urladdningstid.

Den praktiska kapaciteten i en cell begränsas av användningen.
En elektrisk cell avger sålunda regelmässigt mindre energimängd, ju högre
urladdningsströmmen är.
Kapaciteten i en elektrisk cell skiljer sig i det avseendet från den i till
exempel en oljetank, där man kan ta ut lika mycket energimängd som man häller i
och oberoende av hur fort man gör det.

Elektriska celler kan samlas till batterier, varvid cellerna oftast
seriekopplas.
Batteriets polspänning är då summan av cellernas pols\-pänningar.

Hur stort arbete ett batteri avger, beror såväl på hela batteriets
polspänning som på de enskilda cellernas kapacitet.
Exempel:
Ett batteri med polspänningen \qty{12}{\volt} och cellkapaciteten
\qty{100}{Ah} kan nominellt avge
\(P = U \cdot I = 12 \cdot 100 = \qty{1200}{VAh} = \qty{1,2}{kWh}\).

Hur länge batteriet ''räcker'' per laddning beror som sagt bland annat på
vilken strömstyrka man tar ut.
Tar man ut \qty{1}{\ampere} ur \qty{100}{Ah}-cellen här ovan, så blir
urladdningstiden nominellt \(t = \qty{100}{Ah}/\qty{1}{\ampere} =
\qty{100}{\hour}\).

% Avsnitt 1.2 Elektriska kraftkällor
\section{Elektriska kraftkällor}
\harecsection{\harec{a}{1.2}{1.2}}

\subsection{Elektromotorisk kraft -- EMK}
\harecsection{\harec{a}{1.2.1}{1.2.1}}
\index{elektromotorisk kraft (EMK)}
\index{EMK}
\index{volt (V)}

Det som driver ström genom en elektrisk strömkrets är kretsens elektromotoriska
kraft (EMK).
Måttenheten för EMK är \(volt\ [\unit{\volt}]\).
EMK är summan av de potentialökningar som uppstår i kretsen.
De vanligaste slagen av EMK är:

\begin{itemize}
\item elektromagnetisk EMK som uppkommer i strömledare i magnetfält som
varierar (t.ex. lindningarna i en roterande generator)
\item elektrokemisk EMK som uppkommer i beröringsytan mellan en metallisk
  ledare och en elektrolyt (t.ex. battericell)
  \newpage
\item elektrostatisk EMK, till exempel i kondensatorer
\item kontakt-EMK i beröringsytan mellan metaller med olika termoelektrisk
potential eller mellan metall och luftens syre (t.ex. korrosion mellan metaller)
\item termo-EMK som uppkommer i en strömkrets där två sammanlödda metaller med
olika temperatur ingår (t.ex. termokors för strömmätning).
\end{itemize}

\subsection{Polspänning}

Den spänning, som kan mätas mellan kretsens anslutningspoler då kretsen är öppen.

\subsection{Inre resistans}
\label{inre_resistans}
\index{resistans!inre}

I likhet med att komponenterna i en strömkrets har en viss resistans, har också
en strömkälla en inre resistans.
Den inre resistansen i en strömkälla ingår i kretsens totala resistans.

% \newpage % layout

\subsection{Kortslutningsström}
\index{kortslutningsström}
\index{kortslutning}
\label{subsec:kortslutningsstroem}

Om man på kortaste väg förbinder strömkällans anslutningspoler blir kretsen
totala resistans lika med källans inre resistans.

Den kortslutningsström som då uppstår begränsas enbart av strömkällans
polspänning och inre resistans.

Eftersom den inre resistansen oftast är mycket liten blir kortslutningsströmmen
motsvarande hög.

\subsection{Serie- och parallellkopplade kraftkällor}
\harecsection{\harec{a}{1.2.2}{1.2.2}}
\index{kraftkällor}
\label{kraftkällor_serie_parallell}

\subsubsection{Seriekopplade kraftkällor}
\index{kraftkällor!seriekopplade}

För att uppnå en högre total spänning (EMK) kan flera kraftkällor
(delspänningar) kopplas i en slinga efter varandra. Detta kallas seriekoppling.

Seriekopplade delspänningar verkar med eller mot varandra, beroende på
deras inbördes polariteter.

Den totala spänningen över kopplingen är summan av de ingående
delspänningarna, med hänsyn taget till deras polariteter.

\subsubsection{Parallellkopplade kraftkällor}
\index{kraftkällor!parallellkopplade}

För att erhålla högre ström, kan flera svagare kraftkällor parallellkopplas.
Vid parallellkoppling erhålls däremot inte högre spänning.

Vid parallellkoppling av kraftkällor \textbf{måste} deras polaritet vara lika.

För minsta utjämningsström mellan parallellkopplade kraftkällor bör även deras
polspänning och inre resistans vara så lika som möjligt.

\begin{center}
\begin{minipage}{0.19\columnwidth}
\Huge{\fontencoding{U}\fontfamily{futs}\selectfont\char 66\relax}
\end{minipage}
\begin{minipage}{0.7\columnwidth}
Parallellkoppling av kraftkällor är ofta direkt olämpligt eftersom det
i praktiken är svårt att få en balans, varvid enbart den ena källan levererar.
Det finns kraftaggregat utformade för att parallellkopplas.
\end{minipage}
\end{center}

% Avsnitt 1.3 Elektriskt fält
\section{Elektriskt fält}
\harecsection{\harec{a}{1.3}{1.3}}
\label{elektriskafält}
\index{elektriska fält}

\subsection{Potential}
\index{elektrisk potential}

Potentialskillnaden -- spänningen -- mellan olika laddade kroppar skapar
krafter mellan dem samt mellan dem och deras omgivning.
Detta fenomen kallas elektriskt kraftfält och är orsaken till att elektriskt
laddade kroppar kan komma i rörelse.

\subsection{Elektrisk laddning}
\index{elektrisk laddning}
\index{symbol!\(e\) elementarladdning}
\index{symbol!\(Q\) laddning}
\index{coulomb (C)}
\index{enheter!coulomb (C)}

Elektriska laddningar är grunden för elektricitetsläran.
Varje proton i atomkärnan är bärare av en positiv laddning.
Neutronerna i atomkärnan är elektriskt neutrala.
Antalet protoner i kärnan bestämmer därför kärnans totala positiva
laddning, kallat för kärnladdningstalet.
Elektronerna som kretsar omkring atomkärnan är bärare av var sin negativ
laddning.

Elementarladdningen [\emph{e}] är den laddning som finns i en elektron och har
länge ansetts vara den minsta möjliga laddningen.
Nutida elektronfysik konstaterar ännu mindre enheter, men det går vi inte in på
här.

Antalet protoner och elektroner i en atom är lika och elektronernas
negativa laddning blir då lika stor som protonernas positiva laddning.
När laddningar med olika polaritet är lika stora väger de ut varandra och blir
elektriskt neutrala till sin omgivning.

Måttenheten för elektrisk laddning är \(coulomb\ [C]\).
Laddningsmängden \(1\ coulomb\) motsvarar 6,25 triljoner (\(6,25\cdot10^{18}\))
elementarladdningar. Sambandet mellan laddning och ström är:
%%
\[Q = I \cdot t\]
%%
Laddning $[Q]$ är ström $[I]$ under tiden $[t]$:
%%
\[1\ C ~=~ 1\ A \cdot 1\ s ~=~ 1\ \textit{amperesekund}\ [1\ As]\]
\[1\ \textit{coulomb} ~=~ 1\ \textit{ampere} \cdot 1\ \textit{sekund}\]
%%

\newpage
\smallfig{images/cropped_pdfs/bild_2_1-05.pdf}{Elektriska kraftfält}{fig:BildII1-5}

\subsection{Kraftfält omkring elektriska laddningar}

%%Bild \ssaref{fig:BildII1-5} visar elektriska kraftfält.

\noindent
Mellan elektriska laddningar bildas krafter (bild~\ssaref{fig:BildII1-5}).

\begin{itemize}
  \item Varje laddning är omgiven av ett elektriskt kraftfält.
  \item Mellan positiva (+) elektriska laddningar och (--) negativa laddningar
  bildas krafter.
  \item Fältkrafternas styrka och riktning symboliseras som linjer mellan
  positiva och negativa laddningar, där styrkan är densamma utmed respektive
  linje.
\end{itemize}

%%(även 1.1) --- TODO: VA??

\begin{quote}
\emph{Kroppar med olika slags laddningar dras till varandra}

\emph{Kroppar med lika slags laddningar stöter bort varandra}

\emph{Oladdade kroppar påverkas inte och ger ingen kraftverkan.}
\end{quote}

\subsection{Elektrisk fältstyrka}
\harecsection{\harec{a}{1.3.1}{1.3.1}, \harec{a}{1.3.2}{1.3.2}}
\index{elektrisk fältstyrka}
\index{symbol!\(E\) elektrisk fältstyrka}
\label{elektrisk_fälststyrka}

\smallfig{images/cropped_pdfs/bild_2_1-06.pdf}{Elektrisk fältstyrka}{fig:BildII1-6}

I en trådformad ledare, som det flyter likström igenom, fördelas strömmen lika
över tvärsnittet.
Om ledaren i stället är ett tunt plan, så blir strömfördelningen annorlunda.

Bild~\ssaref{fig:BildII1-6} visar ett plan med två elektroder, som anslutits
till en spänningskälla.

Utmed sträckan mellan elektroderna fördelas strömmen över planet så som
strömlinjerna på bilden.
Fördelningen beror på elektrodernas utformning och polaritet.
Strömtätheten är inte lika över hela planet, eftersom planet kan ses som många
parallellkopplade resistorer vars resistanser ökar med tilltagande
strömlinjelängd.

Strömtätheten i planet är större där resistansen mellan elektroderna är liten.
Närmast elektroderna där alla strömlinjer samlas är strömtätheten extremt hög.
Där strömtätheten är som störst finns den största potentialskillnaden
(spänningen) per längdenhet strömlinje.
Man kan mäta potentialerna i planet.
Spänningen mellan två punkter utmed en tänkt strömlinje är därvid proportionell
mot linjens längd mellan punkterna.
Halva spänningen finner man mitt emellan punkterna.

Elektriska fält är upplagrad energi.
Fältstyrkan kan bli så hög, att det blir en urladdning mellan polerna.
Koronaurladdning från ändarna av en antenn är ett annat tecken på hög
fältstyrka.
För att försvåra urladdning kan man öka elektrodytan, till exempel göra den
klotformad.
Omvänt kan man medverka till urladdning genom att minska elektrodytan.
Ett exempel är åskledarens spets.

I bild~\ssaref{fig:BildII1-6} \(U = f(l)\) visas spänningarna utmed
''mittströmslinjen'' igenom plus- och minuspolerna.
Kurvutseendet är typiskt även för omkringliggande linjer, oavsett längd.

Bilden framställer en ledare som ett idealt plan, medan den i praktiken är en
volym.
För att efterlikna en volym föreställer vi oss att bilden roterar omkring
mittströmslinjen, med fältlinjerna oförändrade.
Även om resistansen i den rotationskropp som uppstår är så hög att ingen ström
flyter, så är spänningsbilden fortfarande densamma.

Spänningsbilden gäller även för isolerande fasta material, gaser och vakuum.
Det finns alltså spänning mellan olika punkter även i ''friska luften''.
Denna spänningfältstyrka- kan mätas med särskilda instrument, så kallade
fältstyrkemätare.

\newpage
Av brantheten på spänningskurvan i bilden framgår vilken delspänningen är per
dellängd av en spänningslinje.
Kvoten av delspänning och avståndet mellan mätpunkterna kallar man för
elektrisk fältstyrka.

I formler betecknas elektrisk fältstyrka med bokstaven \(E\).
Elektrisk fältstyrka mäts i volt per meter.
%%
\[
\begin{array}{ccc}
E=\dfrac{\Delta U}{\Delta l} &\quad& \dfrac{[volt]}{[meter]}
\end{array}
\]
%%
\subsection{Skärmning av elektriska fält}
\harecsection{\harec{a}{1.3.3}{1.3.3}}
\index{elektriska fält!skärmning}
\label{elektrostatik skärmning}

I grunden finns det två slags fält, det elektriska och det magnetiska.
Dessutom finns det även elektromagnetiska fält, som är sammansatt av båda dessa.
Fält kan vara statiska eller dynamiska, varav här avses dynamiska.
Ett dynamiskt elektriskt fält genererar ett magnetiskt fält.
Omvänt generar ett dynamiskt magnetiskt fält ett dynamiskt elektriskt fält.
Denna växelverkan gör att fälten kan hållas igång av varandra med tillskott av
yttre energi.

Fält i rörelse alstrar elektromagnetisk strålning, som påverkar omgivningen.
När påverkan inte är önskvärd måste fältet skärmas av.
Ett sätt att skärma av ett elektriskt fält är en metallisk kapsling som
anslutits till apparatens jordreferens.
Skärmen behöver inte vara tät, men utförd så att all magnetiskt inducerad ström
i den bryts. (Jfr \ssaref{elektromagnetisk skärmning})

% Avsnitt 1.4 Magnetiskt fält
\section{Magnetiskt fält}
\harecsection{\harec{a}{1.4}{1.4}}
\label{elektromagnetiskafält}
\index{elektromagnetiska fält}

\subsection{Magnetism}
\index{magnetism}
\index{Plinius}
\index{Magnes}
\index{Lithos herakleia}
\index{Herakleia}
\index{Magnesia}
\index{Magnetes}

\begin{quote}
\emph{Enligt den romerske författaren \emph{Plinius} lär, vid tiden ungefär
160~år f.Kr. herden \emph{Magnes} en dag ha känt hur järnstiften i
sandalerna häftade vid en viss sorts sten.
Det kunde ha varit svart järnmalm, som grekerna i äldsta tider benämnde
\emph{Lithos herakleia} efter staden \emph{Herakleia} i Lydien,
där sådan malm förekommer.
Staden fick sedermera namnet \emph{Magnesia} och man kan tänka sig att stenen
kom att kallas \emph{Magnetes}.
En hel mineralgrupp med liknande egenskaper, såsom järn, nickel m.fl. kallas
magnetiska.}
\end{quote}

\emph{Magnetism} uppstår av elektriska laddningar i rörelse.
Elektronernas rörelser i en atom skapar nämligen magnetfält.
Det gör att atomerna var för sig fungerar som en magnetisk dipol -- en magnet.
I de flesta material är atomerna orienterade så att deras magnetiska krafter
tar ut varandra.
Materialet som helhet är då omagnetiskt och utövar inga yttre krafter.
Men vid påverkan från ett yttre magnetfält kan dipolerna (atomerna) i ett
material orienteras i samma riktning och deras magnetfält kommer då att
samverka. Hela materialet blir då magnetiskt.
När det yttre magnetfältet avlägsnas, kvarstår orienteringen endast delvis --
\emph{magnetisk remanens}.
I ferromagnetiska legeringar kvarstår en större del av orienteringen, även om
påverkan från det yttre magnetfältet har upphört.
Materialet är då permanentmagnetiskt.

\smallfig{images/cropped_pdfs/bild_2_1-07.pdf}{Kraftfält omkring magneter}{fig:BildII1-7}

\subsection{Kraftfält i och omkring magneter}

Bild~\ssaref{fig:BildII1-7} visar kraftfält omkring magneter.
Varje magnet omges av ett magnetiskt kraftfält.
Magnetfältets fördelning, styrka och riktningar beskrivs som kraftlinjer med
slutna kretslopp.

Utanför magneten går kraftlinjerna från nord- till sydpol och inne i magneten
motsatt riktning.
Kraftriktningen i varje punkt av fältet är den som nordändan på en kompassnål
skulle peka åt.
Om man hänger upp en magnet i en tråd, så kommer den att inta samma riktning
som jordens magnetfält.

\begin{itemize}
	\item Poler med samma polaritet stöter bort varandra (repellerar).
	\item Poler med olika polaritet dras till varandra (attraherar).
\end{itemize}

\newpage
\tallfig{images/cropped_pdfs/bild_2_1-08.pdf}{Magnetiska fält omkring strömledare}{fig:BildII1-8}

\subsection{Magnetiska fält omkring strömbanor}
\harecsection{\harec{a}{1.4.1}{1.4.1}}
\label{magfält_ström}

Bild~\ssaref{fig:BildII1-8} visar magnetiska fält omkring strömledare.
Omkring varje ledare, som det flyter en elektrisk ström igenom, alstras det ett
magnetiskt kraftfält.
Magnetiska kraftlinjerna fördelar sig koncentriskt omkring en rak ledare och
vinkelrätt mot denna.
Mellan ändarna av en ledare med bågformad utsträckning bildas kraftlinjer som
verkar med varandra.
En strömgenomfluten cylindrisk spole -- induktor -- uppvisar samma magnetiska
fältbild som en stavformad permanentmagnet.

\subsection{Bestämma magnetiska fältriktningen}

Magnetfältets riktning omkring en ledare kan bestämmas med
\emph{högerhandsregeln}.
När en \emph{ledare} fattas med höger hand och med tummen i strömmens
riktning, kommer fingrarna att peka i fältriktningen (B).

I bild~\ssaref{fig:BildII1-8} (övre) så går strömmen från pluspolen (+) till
minuspolen (--) varvid strömmen kommer gå nedåt i bilden på ovansidan,
det vill säga precis så tummen pekar om man greppar ledaren med tummen nedåt,
och magnetfältet kommer att snurra som pilarna precis som de övriga fingrarna
på högerhanden.

När en ledare formas som en spole och en elektrisk ström flyter genom den,
kommer magnetfältet att ha ett utseende som liknar det omkring en
permanentmagnet.
En sådan spole kallas \emph{elektromagnet}.

Magnetfältets riktning i en spole kan också bestämmas med högerhandsregeln.
När \emph{en spole} fattas med höger hand och med fingrarna i strömmens
riktning, kommer den utsträckta tummen att peka mot spolens nordpol.

I bild~\ssaref{fig:BildII1-8} (undre) så går strömmen från pluspolen (+) till
minuspolen (--) varvid strömmen kommer gå inåt i bilden på ovansidan, dvs.
precis så fingrarna pekar när man lägger handen på spolen, och magnetfältet
kommer att peka mot nord (N) precis som tummen på högerhanden.

Fälten omkring alla slags magneter, såväl permanentmagnetiska som
elektromagnetiska, återverkar på varandra.
Även enkla elektriska ledare är elektromagneter.

\tallfig{images/cropped_pdfs/bild_2_1-09.pdf}{Exempel på elektromagneter}{fig:BildII1-9}

\subsection{Exempel på elektromagneter}

Bild~\ssaref{fig:BildII1-9} visar exempel på elektromagneter.

\subsubsection{Elektromagnet}
Det bildas ett magnetfält genom en spole så länge som det flyter ström genom
den.
En järnkärna i spolen koncentrerar fältet på grund av den större magnetiska
ledningsförmågan.

Elektromagneter används för att sätta magnetiska material i rörelse eller hålla
fast dem.

\subsubsection{Elektrisk ringklocka}
Anordningen består av en elektromagnet och en järnplatta på en fjäder.
På plattan sitter en självbrytande kontakt samt en kläpp som kan slå på en
klocka.

Kontakten åstadkommer en växelvis brytning och slutning av strömmen genom
elektromagneten.
Armaturen med kläppen kommer då i svängning och slår på klockan.

\subsubsection{Telefon}
I en enkel telefon finns bland annat en mikrofon, ett batteri och en
hörtelefon.

Särskilt i äldre telefoner består mikrofonen av en kolkornskammare med ett
membran.
Tryckvariationer (ljud) får membranet att vibrera, varvid resistansen genom
kolkornen varierar i motsvarande grad.
Därmed varierar talströmmen genom mikrofonen.

Hörtelefonen består av en elektromagnet och ett membran av mjukjärn.
Variationer i talströmmen genom mikrofonen passerar även hörtelefonen och får dess
magnetfält att variera.
Hörtelefonens membran alstrar då trycksvariationer, det vill säga ljud.

\subsubsection{Elektromagnetiskt relä}
Reläet består av en elektromagnet, en järnplatta (ankare) på en fjäder och en
elektrisk kontakt.
Med en svag ström / låg spänning genom spolen i manöverkretsen kan man med
reläets arbetskontakt styra starkare ström / högre spänning i huvudkretsen.

\subsection{Magnetisk fältstyrka}
\label{magnetisk_fältstyrka}
\index{magnetisk fältstyrka}
\index{symbol!\(H\) magnetisk fältstyrka}

Som magnetisk fältstyrka förstår man flödet per meter fältlinje, det vill säga:

\begin{equation*}
  H = \frac{\Theta}{l} = \frac{I \cdot N}{l} \\
\end{equation*}
%
\begin{equation*}
  H = \frac{\Theta}{l} = \frac{I \cdot N}{l} \\
\end{equation*}

\begin{table}[H]
	\centering
	\begin{tabular}{rl}
	$H$ & [A/m]\\
	$I$ & [A]\\
	$N$ & varvtal\\
	$l$ & fältlinjelängd\\
\end{tabular}
\end{table}

\emph{Magnetisk fältstyrka uttrycks således som ampere per meter flödesväg.}

\subsection{Magnetisk flödestäthet}
\index{magnetisk flödestäthet}
\index{tesla (T)}
\index{enheter!tesla (T)}
\index{symbol!\(B\) magnetisk flödestäthet}
\index{permeabilitet}
\index{symbol!\(\mu\) permeabilitet}

Den magnetiska flödestätheten \(B\) mäts i enheten tesla \(\text{[T]}\) (förut gauss):
%%
\[B = \mu \cdot H \quad B\, \text{[Vs/m$ ^2 $]}\quad H\, \text{[A/m]}\]
%%
%%Flödestäthet \(B\ [Vs/m^2]\) Fältstyrka \(H\ [A/m]\)

\(\mu\) är permeabilitetstalet för materialet.
\(\mu_0\) är permeabilitetstalet (fältkonstanten) för den magnetiska
ledningsförmåga för vakuum.

För järn eller annat magnetiskt ledande material tillkommer permeabilitetstalet
\(\mu_r\).
Det anger hur många gånger bättre än luft etc., som materialet leder ett
magnetisk flöde.
Permeabilitetstalet kan då skrivas
\(\mu = \mu_r\mu_0\):
%%
\[B = \mu_0 \cdot \mu_r \cdot H\]
%%

% \newpage %layout 

\subsection{Magnetiskt flöde}
\index{magnetiskt flöde}
\index{symbol!\(\Phi\) magnetiskt flöde}

Det magnetiska flödet är produkten av flödestätheten \(B\) och tvärsnittsytan
\(A\) av flödesvägen:
%%
\[\Phi = B \cdot A\]
\[\Phi\, \text{[weber eller Vs]}\quad B\, \text{[T eller tesla]} \quad A\, [\text{m}^2]\]

%%
\subsection{Skärmning av magnetiska fält}
\harecsection{\harec{a}{1.4.2}{1.4.2}}
\index{magnetiska fält!skärmning}
\label{elektromagnetisk skärmning}

I grunden finns det två slags fält, det elektriska och det magnetiska.
Det finns även elektromagnetiska fält som är sammansatta av båda dessa.
Fält kan vara permanenta eller rörliga, varav här avses de rörliga.
Ett rörligt magnetiskt fält genererar ett elektriskt fält.
Omvänt generar ett rörligt elektriskt fält ett rörligt magnetiskt fält.
Denna växelverkan gör att fälten kan hållas igång med tillförsel av yttre
energi.

Fält i rörelse alstrar elektromagnetisk strålning, som påverkar funktioner i
omgivningen.
När påverkan inte är önskvärd, måste fältet skärmas av.
Ett sätt att skärma magnetiska fält är en metallisk kapsling.
Kapslingen ska vara tät och bilda en sluten magnetisk krets.
Kapslingen ska vara utförd i ett material som är en god ledare av magnetiskt
flöde.
(Jämför \ssaref{elektrostatik skärmning})

% Avsnitt 1.5 Elektromagnetiska vågor
\section{Elektromagnetiska vågor}

\smallfig{images/cropped_pdfs/bild_2_1-10.pdf}{Vågor längs en linje}{fig:BildII1-10}

\harecsection{\harec{a}{1.5}{1.5}}
\index{elektromagnetiska fält}

\subsection{Vågutbredning}
\harecsection{\harec{a}{1.5.1}{1.5.1}}
\index{vågutbredning}

En tillståndsändring i ett medium innebär att energi tillförs eller tas bort.
Om detta sker växelvis uppstår förlopp såsom pendling, svängning, vågbildning etc.
Eftersom naturen söker jämvikt, så breder förloppet ut sig genom mediet efter
någon modell.

Energi kan inta olika tillstånd.
I en pendel växlar energin mellan lägesenergi och rörelseenergi.
Vågor på en vätskeyta liksom fjädring i fasta material är exempel på detta.
Det kan även innebära trycksvängningar i gaser och så vidare.

I detta avsnitt behandlas elektromagnetiska fält.
Sådana uppstår av svängningar i elektriska och magnetiska fält.
För att förklara pendling och utbredning används här modeller.

\subsection{Utbredningsmodeller}
\label{utbredningsmodeller}

\subsubsection{Vågutbredning längs en linje}

Bild~\ssaref{fig:BildII1-10} visar vågor längs en linje.
När änden av en tråd sätts i pendling med en frekvens \(f\), så kommer till sist
hela tråden i svängning med den frekvensen.
Den pendling, som först skapades, vandrar längs tråden med
utbredningshastigheten \(v\).
Våglängden är \(\lambda\) (lambda), som är avståndet mellan två närliggande
punkter med samma svängningsläge och svängningsriktning.

\subsubsection{Vågutbredning på en yta}
\index{vågutbredningshastighet}
\index{symbol!\(v\) vågutbredningshastighet}

\smallfig{images/cropped_pdfs/bild_2_1-11.pdf}{Vågutbredning på en yta}{fig:BildII1-11}

Bild~\ssaref{fig:BildII1-11} visar vågutbredning på en yta.
När ett föremål släpps genom en vätskeyta, så bildas vågor som breder ut sig
som cirklar i varandra (koncentriska).

De punkter på vågen, som för ögonblicket har samma svängningsläge, och är lika
långt från energikällan, kallas för vågfront.

Sambandet mellan utbredningshastighet \(v\), våglängd \(\lambda\) och frekvens
\(f\) är:
%%
\[
\begin{array}{llll}
v = \lambda \cdot f & v \ \text{[m/s]} & \lambda \ \text{[m]} & f \ \text{[Hz]}
\end{array}
\]

\begin{tcolorbox}[title=Exempel]
När våglängden \(\lambda = 2\ \text{[m]}\) och antalet svängningar per sekund
\(f = 10\ \text{[Hz]}\), så breder vågen ut sig med hastigheten \(v = 20\ \text{[m/s]}\).
\end{tcolorbox}

\smallfigpad{images/cropped_pdfs/bild_2_1-12.pdf}{Vågutbredning i rummet}{fig:BildII1-12}

\subsubsection{Vågutbredning i rummet}

Bild~\ssaref{fig:BildII1-12} visar vågutbredning i rummet.

Ljud är energi i form av tryckvågor i luften.
När en mekanisk kropp sätts i svängning (stämgaffel, dricksglas etc), överförs
svängningarna till den omgivande luftmassan som börjar att svänga med.
I luftmassan bildas det omväxlande över- och undertryckszoner, som breder ut
sig åt alla håll.
De mekaniska svängningarna i ljudkällan omvandlas alltså till tryckvågor.

Det mänskliga örat uppfattar tryckvågor inom frekvensområdet ca
\SIrange{15}{18000}{\hertz} som ljud.
Dessa vågor kallas ljudvågor.
Utbredningshastigheten för ljudvågor är \(v = \text{ca } 340\ \text{m/s}\) vid
\qty{15}{\degreeCelsius} och normalt lufttryck.

\subsection{Elektromagnetiska fält}
\harecsection{\harec{a}{1.5.2}{1.5.2}}
\index{elektromagnetiska fält}
\label{elektromagnetiska_fält}

Tabell~\ssaref{tab:elektromagnetiskt_spektrum} visar elektromagnetiskt spektrum.
I detta avsnitt görs i huvudsak endast jämförelse mellan ljusvågor och
radiovågor, vilka båda är elektromagnetisk strålning.
Hur ett elektromagnetiskt fält frigörs från en ledare framgår av kapitel
\ssaref{vågutbredning}.

Elektromagnetiska fält är energi, som är sammansatt av mycket snabbt svängande
elektriska och magnetiska fält.
När elektrisk ström genom en ledare ändras i styrka bildas ett magnetfält
omkring ledaren.
Detta magnetfält alstrar en elektromotorisk kraft (EMK), som är motriktad den
som driver fram strömmen.
Magnetfältet motverkar således strömändringen.
På liknande sätt alstrar en ändring av magnetfältet omkring ledaren en EMK i
form av ett elektriskt fält.
Detta driver en motriktad ström och därmed ett motverkande magnetiskt fält.

Både det elektriska och det magnetiska fältet har således alstrats av ändringar
i det andra och existerar därför bara tillsammans.

De båda fälten kombineras till ett elektromagnetiskt fält, som har egenskapen
att kunna stråla (breda ut sig) i alla tre dimensioner.
Beroende på frekvensen har elektromagnetiska fält olika egenskaper och
användning, vilket framgår av tabell~\ssaref{tab:elektromagnetiskt_spektrum}.
%% k7per: Det stod "bilden" ovan, jag antog att det var tabelle (tab:elektromagnetisk_spektrum)?

\begin{table}[t]
\begin{center}
\begin{tabular}{|rl|rl|l|}
\hline
\multicolumn{2}{|c|}{\multirow{2}{*}{Frekvens}} & \multicolumn{2}{c|}{\multirow{2}{*}{Våglängd}} & \multicolumn{1}{c|}{Egenskaper/} \\
 & & & & \multicolumn{1}{c|}{användning} \\ \hline
300 & Hz  & 100 & mil & \\
  1 & kHz & 300 & km & ULF \\ \cline{5-5}
  3 & kHz & 100 & km & \\
 10 & kHz &  30 & km & VLF \\ \cline{5-5}
 30 & kHz &  10 & km & \\
100 & kHz &   3 & km & LF \\ \cline{5-5}
300 & kHz &   1 & km & \\
  1 & MHz & 300 & m & MF \\ \cline{5-5}
  3 & MHz & 100 & m & \\
 10 & MHz &  30 & m & HF \\ \cline{5-5}
 30 & MHz &  10 & m & \\
100 & MHz &   3 & m & VHF \\ \cline{5-5}
300 & MHz &   1 & m & \\
  1 & GHz & 300 & mm & UHF \\ \cline{5-5}
  3 & GHz & 100 & mm & \\
 10 & GHz &  30 & mm & SHF \\ \cline{5-5}
 30 & GHz &  10 & mm & \\
100 & GHz &   3 & mm & EHF\\ \cline{5-5}
300 & GHz &   1 & mm & \\\
  1 & THz & 300 & \unit{\um} & Infrarött \\
  3 & THz & 100 & \unit{\um} & ljus \\
 10 & THz &  30 & \unit{\um} & (värme- \\
 30 & THz &  10 & \unit{\um} & strålning) \\
100 & THz &   3 & \unit{\um} & \\ \cline{5-5}
300 & THz &   1 & \unit{\um} & Synligt ljus \\ \cline{5-5}
  1 & PHz & 300 & nm & \\
  3 & PHz & 100 & nm & Ultraviolett \\
 10 & PHz &  30 & nm & ljus \\ \cline{5-5}
 30 & PHz &  10 & nm & \\
100 & PHz &   3 & nm & Rönt-\\
300 & PHz &   1 & nm & gen-\\
  1 & EHz & 300 & pm & strålning\\ \cline{5-5}
  3 & EHz & 100 & pm & \\
 10 & EHz &  30 & pm & Gamma-\\
 30 & EHz &  10 & pm & strål-\\
100 & EHz &   3 & pm & ning\\
300 & EHz &   1 & pm & \\
\hline
\end{tabular}
\end{center}
\caption{Elektromagnetiskt spektrum}
\label{tab:elektromagnetiskt_spektrum}
\end{table}

\subsubsection{Ljusvågor}
\index{ljusvågor}
\index{ljushastighet}
\index{symbol!\(c\) ljushastighet i vakuum}
\index{symbol!\(\lambda\) våglängd}

Ögat uppfattar elektromagnetisk strålning bara inom ett visst frekvensområde
som ljus.
Ljusets utbredningshastighet beror av vilket material, som det passerar igenom.
I vakuum är hastigheten störst, \(c = 299\, 792\, 458\ \text{[m/s]}\)
(= ca \(3 \cdot 10^8\ \text{[m/s]}\)) \cite{SIbrochure8}.

I tätare ämnen är hastigheten lägre, till exempel i glas ca
\qty{200000000}{\metre\per\second}.
Det för människan synliga ljuset har våglängder mellan \(7,7 \cdot 10^{-7}\)
och \(3,9 \cdot 10^{-7}\ \text{[m]}\), motsvarande 7,7 till 3,9 tiotusendels mm.

Sambandet mellan ljusets utbredningshastighet \(c\) i vakuum, frekvensen \(f\)
och våglängden \(\lambda\) är
%%
\[
\begin{array}{llll}
c = \lambda \cdot f & c \ \text{[m/s]} & \lambda \ \text{[m]} & f \ \text{[Hz]}
\end{array}
\]
%%
\subsubsection{Radiovågor}
\index{radiovågor}

Även radiovågor är elektromagnetisk strålning, men inom ett lägre
frekvensområde än det för ljus.
Men utbredningshastigheten för radiovågor genom olika material följer ändå
samma lagar som de för till exempel ljusets utbredning.

Radiovågor anses omfatta ett frekvensområde från ca \qty{10}{\kilo\hertz}
(\(\lambda = 30\ \text{[km]}\)) till \qty{300}{\giga\hertz} (\(\lambda = 1\ \text{[mm]}\)).

Rundradio tilldelas frekvenser i intervallet \qty{100}{\kilo\hertz} till
\qty{1000}{\mega\hertz}.
Amatörradio tilldelas ett antal frekvensområden i intervallet
\qty{136}{\kilo\hertz} till \qty{250}{\giga\hertz}.

Att märka är att elektromagnetiska fält, som sagts ovan, förekommer så långt
ner i frekvens som ett fåtal \unit{\kilo\hertz}.
Detta ska självklart inte förväxlas med ljudtryck med samma frekvens.

\subsubsection{Egenskaper hos elektromagnetiska vågor}

Elektromagnetiska vågor med högre frekvens än radiovågor uppfattas som
värmestrålning, vågor med ännu högre frekvens som ljus etc., men fortfarande är
huvudegenskaperna samma.
Som exempel kan nämnas polariserade vågor.
Dessutom kan man finna motsvarigheten till sådana egenskaper såsom interferens
och överlagring även i andra vågtyper, till exempel i ljud.

\newpage
\mediumtopfig{images/cropped_pdfs/bild_2_1-14.pdf}{Polarisation av elektromagnetiska vågor}{fig:BildII1-14}

\subsection{Vågpolarisation}
\harecsection{\harec{a}{1.5.3}{1.5.3}}
\label{vågpolarisation}
\index{vågpolarisation}

Bild~\ssaref{fig:BildII1-14} visar polarisation av elektromagnetiska vågor.

\subsubsection{Vågor längs en linje (tråd e.d.)}
En vågrörelse i ett plan kallas linjärt polariserad.
Om änden på en horisontell tråd sätts i rörelse uppåt-nedåt, uppstår på tråden
en linjärt polariserad vågrörelse i vertikalplanet -- vertikal polarisering.
Om tråden sätts i rörelse höger-vänster kommer dess svängning att vara
horisontellt polariserad.
Om tråden sätts i svängning i ett plan och detta plan ständigt vrider sig,
kommer även vågrörelsen utmed tråden att vrida sig.
En vågrörelse, vars polarisering vrider sig roterar -- kallas för cirkulärt
polariserad.
Vridning mot- respektive medurs kallas för vänster- respektive högervriden
polarisering.

\subsubsection{Elektromagnetiska vågor}

De magnetiska och elektriska fälten omkring en ledare är vinkelrätt orienterade
mot varandra.
Det elektromagnetiska fält som de bildar tillsammans bildar en vågfront som
är vinkelrätt orienterad mot dem.

Polariseringsriktningen för en elektromagnetisk våg definieras som den riktning
dess elektriska fält har:
\begin{itemize}
  \item vertikalt elektriskt fält -- vertikal polarisering
  \item horisontellt elektriskt fält -- horisontell polarisering.
\end{itemize}

\subsubsection{Ljusvågor}

Ljus är elektromagnetiska vågor.
När dagsljus, som för övrigt är opolariserat, belyser ett polariseringsfilter
passerar endast de vågkomposanter genom filtret, som har samma polarisering
som filtret.

När det polariserade ljuset därefter sänds mot ett efterföljande filter,
passerar ljuset genom filtret endast när det har samma polarisering som ljuset.
När de båda filtren är vridna \ang{90} i förhållande till varandra, passerar
inget ljus alls.

\mediumplustopfig{images/cropped_pdfs/bild_2_1-15.pdf}{Våginterferens}{fig:BildII1-15}

\subsubsection{Radiovågor}

Radiovågor är elektromagnetiska vågor inom det frekvensområde som lämpar sig
för radiokommunikation.

Beroende på sändarantennens utformning avger den vågor med en polarisation.
På samma sätt är en mottagarantenn mest mottaglig för vågor med en viss
polarisation.
Överföringsförlusterna blir lägst mellan antenner med samma polarisation.

I det högre frekvensområdet för radio (VHF, UHF, SHF) är polariseringsvridning
under överföringen mindre vanlig.
Genom att utforma antennerna med horisontell, vertikal eller cirkulär (höger-
alternativt vänstervriden) polarisation fås överföringsegenskaper för olika
syften.

Cirkulärt polariserade antenner ger lägst överföringsförluster när
polariseringsriktningen är lika i sän\-d\-ar- och mottagarantennen.

I det lägre frekvensområdet för radio (HF och lägre) utnyttjas oftast
rymdvågsutbredning.
Eftersom de utsända vågorna då reflekteras mot jonosfärskikt, uppstår
polariseringsvridningar som inte kan förutses.
Då är det en fördel att kunna växla mellan antenner med olika polarisation.

\subsection{Våginterferens}

Bild~\ssaref{fig:BildII1-15} visar våginterferens.
När vågor från olika energikällor blandas med varandra (överlagras), kommer
de att antingen samverka eller motverka.
Beroende av det tidsmässiga läget mellan vågorna och deras amplituder blir
resultatet en förstärkning eller en försvagning.
Om har samma frekvens och lika stora, motriktade amplituder, så uppstår en
utsläckning, vilket kallas fädning (eng. \emph{fading}).

Denna vågmekanism är liknande i gaser (luft), vätskor, elektromagnetiska fält
etc.
Ett försök kan göras med en stämgaffel som man slår an och håller intill örat.
När man vrider stämgaffeln runt sin längdaxel, kommer avståndet mellan
vart och ett av gaffelbenen och örat att variera.
Då uppstår en växelvis med- och motverkan mellan tonerna från gaffelbenen och
därmed varierande tonstyrka.

Detta fenomen utnyttjas bland annat i antenner för riktad sändning respektive
mottagning av radiovågor.

% \mediumfig{images/cropped_pdfs/bild_2_1-15.pdf}{Våginterferens}{fig:BildII1-15}

% Avsnitt 1.6 Sinusformade signaler
\newpage
\section{Sinusformade signaler}
\harecsection{\harec{a}{1.6}{1.6}, \harec{a}{1.6.1}{1.6.1}}

\mediumfig{images/cropped_pdfs/bild_2_1-16.pdf}{Alstring av en sinusformad signal}{fig:BildII1-16}

Bild~\ssaref{fig:BildII1-16} visar alstring av en sinusformad signal.
I detta avsnitt behandlas några grundbegrepp inom växelströmsläran.
Förloppen framställs med vektor- och linjediagram.
För närmare beskrivning används sådana begrepp som momentanvärde,
toppvärde, topp- till toppvärde, effektivvärde, fasläge, fasförskjutning och
båghastighet.

\subsection{Momentanvärde}
\harecsection{\harec{a}{1.6.2a}{1.6.2a}}
\index{momentanvärde}
\index{symbol!\(u\) momentan spänning}
\index{symbol!\(i\) momentan ström}
\index{symbol!\(t\) tidpunkt}

Momentanvärdet är storheten på en spänning \(u\), en ström \(i\) etc. vid en
viss tidpunkt \(t\).
(Storheter som ändrar sig som en funktion av tiden kännetecknas ofta med gemena
bokstäver.)

Bild~\ssaref{fig:BildII1-16} visar en sinusformad växelspänning med frekvensen
\qty{50}{\hertz}.
Spänningen \(u\) är \(+230\ \text{V}\) vid tidpunkten 2,5~millisekunder efter en
positiv nollgenomgång.
Efter totalt \qty{5}{\milli\second} uppnås toppvärdet \(u_{max}\) dvs. \(+325\ \text{V}\).
Efter totalt \qty{10}{\milli\second} sker en negativ nollgenomgång.
Efter totalt \qty{12,5}{\milli\second} är spänningen \(-u\), dvs. \qty{-230}{\volt} osv.

\subsection{Toppvärde eller amplitud}
\label{toppvärde}
\harecsection{\harec{a}{1.6.2b}{1.6.2b}}
\index{toppvärde}
\index{amplitud}

Toppvärdet \(u_{max}\) är det högsta värdet över eller under noll.
På bild~\ssaref{fig:BildII1-16} är de högsta värdena \(+325\ \text{V}\) och
\qty{-325}{\volt}.

\subsection{Topp-till-toppvärde}
\label{peak-to-peak-värde}
\index{topp-till-toppvärde}

Topp-till-toppvärde är summan av toppvärdena över och under noll.
På bild~\ssaref{fig:BildII1-16} är detta värde \qty{650}{\volt}.

\subsection{Effektivvärde}
\harecsection{\harec{a}{1.6.2c}{1.6.2c}, \harec{a}{1.6.2d}{1.6.2d}}
\index{effektivvärde}
\label{effektivvärde}

Effektivvärdet av en växelspänning \(u\) är det värde, som medför samma
effektutveckling som en likspänning \(U\).

För ett sinusformat förlopp gäller följande samband mellan toppvärdet och
effektivvärdet (det s.k. kvadratiska medelvärdet), vilket motsvarar amplituden
vid vinklarna 45, 135, 225 och \ang{270}.
%%
\[
\begin{array}{lllll}
U=\dfrac{\hat{u}}{\sqrt{2}} & & I=\dfrac{\hat{i}}{\sqrt{2}} & & (\sqrt{2} = 1,414)
\end{array}
\]
%%
\subsection{Fasläge}
\index{fasläge}
\index{symbol!\(\phi\) fasläge}

Fasläget \(\varphi\) är när inom en period, som ett givet momentanvärde
uppträder.
Tidpunkten för varje momentanvärde motsvarar en andel av \ang{360} elektriska
grader.
Till exempel uppnås värdet noll volt vid \ang{0}, \ang{180} och \ang{360} (= 0\degree).

\subsection{Bågmått}
\index{bågmått}
\index{radianer}
\index{båghastighet (\(\omega\))}
\index{vinkelhastighet (\(\omega\))}
\index{symbol!\(\omega\) vinkelhastighet}

I beräkningar av växelströmskretsar används ofta inte vinkelmått för fasläget
(gradtal) utan i stället begreppet bågmått.

I en så kallad enhetskrets med radien \(r = 1\) motsvaras vinkeln \ang{360} av
en båge med längden \(2 \cdot \pi \cdot r= 2 \cdot \pi \cdot 1 = 2 \pi =\)
omkretsen.
Vid \(f\) perioder per sekund blir båglängden \(= 2\pi f\).
Denna storhet kallas båghastighet eller oftare vinkelhastighet och betecknas
med \(\omega\) (uttalas omega):
%%
\[\omega= 2\pi f\qquad [\text{rad/s}]\]
%%
\subsection{Period}
\harecsection{\harec{a}{1.6.3a}{1.6.3a}}
\index{period}
\label{period}

En period har passerat, när en storhet (spänning, ström osv.) återtagit samma
tillstånd eller värde efter att ha gjort en fullständig växling, till exempel en hel
pendelrörelse eller ett helt varv vid rotation.

\newpage
\subsection{Periodtid T}
\harecsection{\harec{a}{1.6.3b}{1.6.3b}}
\index{periodtid (T)}
\index{symbol!\(T\) periodtid}

Periodtid \(T\) är den tid som åtgår för att strömmen eller spänningen ska
genomlöpa en period. Periodtiden är det inverterade värdet av frekvensen.

Måttenheten för periodtid är sekund [s].

$$\text{Periodtid} \qquad (T) = \dfrac{1}{f}$$

\noindent
\paragraph{Exempel:}~\\[1ex]
\begin{small}
\begin{tabular}{@{}lll}
\(T_1=\dfrac{1}{10}\) s & = \qty{0,100}{\second} & = \qty{100}{\milli\second} (\qty{10}{\hertz})\\
\\
\(T_2=\dfrac{1}{50}\) s & = \qty{0,020}{\second} & = \qty{20}{\milli\second} (\qty{50}{\hertz})\\
\\
\(T_3=\dfrac{1}{1000}\) s & = \qty{0,001}{\second} & = \qty{1}{\milli\second} (\qty{1}{\kilo\hertz})\\
\\
\(T_4=\dfrac{1}{1000000}\) s & = \qty{0,000001}{\second} & = \qty{1}{\micro\second} (\qty{1}{\mega\hertz})\\
\end{tabular}
\end{small}

\subsection{Frekvens}
\harecsection{\harec{a}{1.6.4}{1.6.4}}
\index{frekvens}
\label{frekvens}

Frekvens är antalet perioder per tidsenhet.
Följande begrepp demonstreras med hjälp av pendeln:

Period = en fullständig fram- och tillbakasvängning i ett system, till exempel
pendelns väg mellan punkterna 2-3-2-1-2-3- osv.
Bild~\ssaref{fig:BildII1-33} visar pendelrörelse som illustration av frekvens.

\mediumfig[0.65]{images/cropped_pdfs/bild_2_1-33.pdf}{Pendel som illustration av fr\-e\-kv\-en\-s}{fig:BildII1-33}

Periodtid \(T\) = tidsåtgången för en fullständig svängning.
Amplitud \(A\) = den största avvikelsen från viloläget.
Frekvens \(f\) = antal svängningar/tidsenhet.
Sambandet mellan frekvensen \(f\) och periodtiden \(T\) är:
%%
\[f=\dfrac{1}{T}\]
till exempel:
\[5 [\text{Hz}] = \dfrac{1}{5}\quad [\text{sekunder}]\]
%%

\newpage
\subsection{Enheten hertz}
\harecsection{\harec{a}{1.6.5}{1.6.5}}
\index{hertz}
\index{enheter!hertz (Hz)}
\index{symbol!\(f\) frekvens}
\index{HF}

Måttenheten för frekvens är hertz [\unit{\hertz}].
I formler betecknas frekvensen med \(f\).

\begin{center}
\begin{tabular}{ll}
\qty{1}{\hertz}      & = 1 period per sekund (p/s) \\
\qty{10}{\hertz}     & = 10 perioder per sekund \\
\qty{50}{\hertz}     & = 50 perioder per sekund \\
\qty{1000}{\hertz}  & = \qty{e3}{\hertz} = \qty{1}{\kilo\hertz} (kilohertz) \\
\qty{1000}{\kilo\hertz} & = \qty{e6}{\hertz} = \qty{1}{\mega\hertz} (megahertz) \\
\qty{1000}{\mega\hertz} & = \qty{e9}{\hertz} = \qty{1}{\giga\hertz} (gigahertz) \\
\end{tabular}
\end{center}

Nätfrekvensen för elkraft är i Europa \qty{50}{\hertz}.
Andra nätfrekvenser förekommer, till exempel \qty{60}{\hertz} i USA och Japan.
Frekvensområdet vid överföring av kvalitativt tal och musik, lågfrekvens LF, är
mellan ca \qty{16}{\hertz} och \qty{16}{\kilo\hertz}.
Frekvensområdet för talöverföring, till exempel över telefonlinjer eller
kommunikationsradio, är ca 300 till \qty{3000}{\hertz}.
Frekvensområdet för radioöverföring, högfrekvens HF, är i huvudsak mellan
\qty{50}{\kilo\hertz}, så kallad långvåg, och 100-tals \unit{\giga\hertz}, så
kallad mikrovåg.


\newpage
\mediumtopfig{images/cropped_pdfs/bild_2_1-18.pdf}{Ren sinusvåg och övertonshaltig våg}{fig:BildII1-18}
\smallfig{images/cropped_pdfs/bild_2_1-17.pdf}{Vektorer och fasförskjutning}{fig:BildII1-17}
\subsection{Fasförskjutning}
\harecsection{\harec{a}{1.6.6}{1.6.6}}
\index{fasförskjutning}

% \pagefig[0.6]{images/cropped_pdfs/bild_2_1-19.pdf}{Uppdelning av en signal i grundton och övertoner}{fig:BildII1-19}

Bild~\ssaref{fig:BildII1-17} visar vektorer och fasförskjutning.
Med fasförskjutning menas tidsskillnaden mellan förlopp, till exempel spänningar
och/eller strömmar.
Fasförskjutningen mellan vektorerna kallas även fasvinkel och uttrycks som ett
gradtal mellan 0 och \ang{360}.

\subsection{Vektorer}
\index{vektorer}

En spänning, ström, kraft osv. kan beskrivas som en vektor med en storhet och
riktning.
På bilden~\ssaref{fig:BildII1-17} har vektorerna \(X_L\), \(R\) och \(X_C\) en
inbördes fasförskjutning av \ang{90}.
De motsvarar spänningsfallen i en krets med en induktor, en resistor och en
kondensator kopplade i serie, där den gemensamma strömmen är en sinus.

Antag att vektorerna roterar i ett oförändrat inbördes läge och med en
vinkelhastighet av \(\omega= 2\pi f\).
Systemet roterar då 360\(\degree\) = 2\(\pi\) radianer = 1 varv/per\-iod.

Vid varje tidpunkt har vektorsystemet uppnått en viss vridningsvinkel.
Momentanvärdet på vektorernas spänningar avsätts till höger i bilden.
Avståndet mellan en vektorspets och noll-linjen är vektorns momentana värde,
som kan vara positivt eller negativt.

% Avsnitt 1.7 Icke sinusformade signaler
\newpage
\section{Icke sinusformade signaler}
\harecsection{\harec{a}{1.7}{1.7}}

\subsection{Grundton, övertoner och kantvågor}
\harecsection{\harec{a}{1.7.2}{1.7.2}, \harec{a}{1.7.3}{1.7.3}, \harec{a}{1.7.4b}{1.7.4b}}
\index{grundton}
\index{överton}
\index{kantvåg}
\label{subsec:oevertoner}

% \mediumfig[0.7]{images/cropped_pdfs/bild_2_1-18.pdf}{Ren sinusvåg och övertonshaltig våg}{fig:BildII1-18}

Bild~\ssaref{fig:BildII1-18} visar en ren sinusvåg och övertonshaltig våg.
Ett sinusformat förlopp med en enda frekvens -- en enda ton -- sägs vara
spektralt ren.
En sådan svängning kallas för grundton.

Varje signal som inte är sinusformad är sammansatt av flera sinussvängningar.
Det är signalens grundton samt dess harmoniska övertoner, vilka kan ha 2, 3
osv. gånger högre frekvens än grundtonen.
Den inbördes styrkan på grundton och övertoner avgör signalens form.
Om signalen ligger inom det hörbara området, kan man märka hur den ändrar
karaktär beroende på övertonshalten.
Man kan säga att övertonerna modulerar grundtonen.

\pagefig[0.6]{images/cropped_pdfs/bild_2_1-19.pdf}{Uppdelning av en signal i grundton och övertoner}{fig:BildII1-19}
\pagefig[0.6]{images/cropped_pdfs/bild_2_1-20.pdf}{Uppdelning av en fyrkantsvåg i grundton och övertoner}{fig:BildII1-20}

Bild~\ssaref{fig:BildII1-19} visar uppdelning av en signal i grundton och
övertoner.
Oscillatorsignalen i exemplet på bilden har 1~volts amplitud på grundtonen
\(f_0\) (1:a harmoniska), 0,7~volts amplitud på de 1:a övertonen
(2:a harmoniska) och 0,2~volts amplitud på den 2:a övertonen (3:e harmoniska).
Den totala amplituden blir emellertid inte summan av 1, 0,7 och 0,2~volt
eftersom de olika delspänningarnas toppvärden inte uppträder samtidigt.
I stället måste delspänningarna adderas vid varje tidpunkt för sig.

\mediumfig{images/cropped_pdfs/bild_2_1-21.pdf}{Överlagrade spänningar}{fig:BildII1-21}

Bild~\ssaref{fig:BildII1-20} visar uppdelning av en fyrkantsvåg i grundton och
övertoner.

\index{Fourier, Joseph}
\index{Fourieranalys}
\index{Fourier!Fourieranalys}
\index{Fouriertransform (FT)}
\index{Fourier!Fouriertransform (FT)}
\index{invers Fouriertransform (IFT)}
\index{Fourier!invers Fouriertransform (IFT)}
\index{Discrete Fourier Transform (DFT)}
\index{Fourier!Discrete Fourier Transform (DFT)}
\index{DFT}
\index{Fourier!inverse Discrete Fourier Transform (IDFT)}
\index{IDFT}
\index{Fourier!Fast Fourier Transform (FFT)}
\index{FFT}
\index{Fourier!inverse Fast Fourier Transform (IFFT)}
\index{IFFT}

%%\infobox{
Denna analys av vågor uppfanns av Jean-Baptiste Joseph Fourier (1768--1830)
vid analys av värmeutbredning och vibration som presenterades 1822.
Denna metod är kraftfull och har haft stort inflytande på vetenskapen och
utvecklingen både som matematiskt verktyg och som praktiskt analys med
spektrumanalysatorer och vid modern modulation och demodulation.
Man pratar om \emph{fourieranalys} (eng. \emph{Fourier analysis}) och
\emph{fouriertransform (FT)} för omvandling från tid till frekvens och
\emph{invers fouriertransform} för omvandling från frekvens till tid.
För tidsdiskret (samplad) form är termerna
\emph{diskret fouriertransform (DFT)} och
\emph{invers diskret fouriertransform (IDFT)} respektive.
Senare optimeringar av beräkningar har resulterat i
\emph{Fast Fourier Transform (FFT)} och
\emph{Inverse Fast Fourier Transform (IFFT)}.
%%}

Det finns olika karaktärer av förlopp såsom sinusvåg, triangelvåg, sågtandsvåg,
fyrkantsvåg och så vidare.

Fyrkantsvågen är sammansatt av sinusvågor med grundfrekvensen och dess udda
övertoner, varvid amplituderna fördelar sig som \(1/1\), \(1/3\), \(1/5\),
\(1/7\), \(1/9\), \(1/11\) osv.
Teoretiskt når övertonsspektrum upp till oändligt höga frekvenser, medan de
motsvarande amplituderna minskar till oändligt små värden.

En ideal fyrkantsvåg, vilken inte kan uppnås i praktiken, skulle bestå av ett
oändligt antal udda övertoner med fallande amplitud.
Ju fler av de högre övertonerna som filtreras bort, desto mer lutar
fyrkantsvågens flanker, desto rundare blir hörnen på vågen och desto vågigare
blir kurvans topp.

\subsection{Överlagrade spänningar (likspänningskomposant)}
\harecsection{\harec{a}{1.7.4a}{1.7.4a}}

Bild~\ssaref{fig:BildII1-21} visar överlagrade spänningar.
I signalkretsar förekommer det mycket ofta, att växelspänning överlagras på
likspänning eller omvänt.
Likspänningen kallas då för likspänningskomposant.
Olika åtgärder behövs för att överlagra spänningar på varandra och att sedan
skilja dem åt.

Bilden visar ett avsnitt av en AM-mottagare.
Från vänster hämtas en AM-modulerad signal från MF-förstärkaren för att
demoduleras, det vill säga för att återvinna den modulerande LF-signalen.
MF-signalen halvvågslikriktas.
Kvar blir den positiva delen av MF-signalen och den modulerande LF-signalen,
sammanlagrade.
LF-signalen ska nu skiljas ut och förstärkas.
Alltså filtreras MF-komposanten bort.
Kvar blir LF-signalen, men överlagrad på en likspänning.
Likspänningen stoppas och kvar blir slutligen LF-signalen som förstärks.

\subsection{Brus}
\harecsection{\harec{a}{1.7.5}{1.7.5}, \harec{a}{7.19}{7.19}}
\label{termisktbrus}

\subsubsection{Termiskt brus}
\index{brus}
\index{termiskt brus}
\index{brus!termiskt}

Resistorer och resistans, i alla dess former, uppvisar en egenskap av
en varierande spänning även när ingen ström går genom motståndet.
Denna extra spänning innehåller ett brett spektrum av toner, men är också ett
tätt spektrum, sådant att ingen enskild ton kan särskiljas från någon annan.
Istället för att tänka sig en grundton och dess övertoner med ingen energi
emellan dem så är det istället ett kontinuerligt spektra med oändligt många
toner.
Detta spektra begränsas dock av bandbredden.

\smallfig{images/cropped_pdfs/bild_2_1-36.pdf}{En resistor kan ses ha brusekvivalenter som spänning eller ström}{fig:BildII1-36}

%% k7per: Add more description of what the figure is describing?  What should you be looking for?
\smallfig{images/cropped_pdfs/bild_2_1-34.pdf}{Brus innebär en ostabilitet över tid}{fig:BildII1-34}

%% k7per: Add more description of what the figure is describing?  What should you be looking for?
\smallfig{images/cropped_pdfs/bild_2_1-35.pdf}{Brus innehåller alla frekvenser, vitt brus har samma amplitud}{fig:BildII1-35}

\index{vitt brus}
\index{brus!vitt}
\index{white noise}
\index{Johnson noise}
\index{brus!Johnson}
Man kallar detta spektrum i daglig tal för \emph{termiskt brus}
(eng. \emph{thermal noise}), eftersom det beror på temperaturen hos motståndet,
eller \emph{Johnson noise}, efter J.~B. Johnson som 1928 fann att detta brus
fanns i alla ledare \cite{ott1988}.
Brus skapar en variation i spänning och ström, som illustreras i bild
\ssaref{fig:BildII1-34}.

I dagligt tal pratar man dock bara om \emph{vitt brus} (eng. \emph{white noise})
eller \emph{brus}.
Med vitt brus menas brus som inte ''färgats'', och det betyder i det här
sammanhanget att det har samma amplitud för alla frekvenser, så som illustreras
i bild~\ssaref{fig:BildII1-35}.
I praktiken är allt brus begränsat med bandbredden på kanalen, men man
betraktar det som vitt inom kanalen om det är jämnt inom bandet.

Effekten \(P_n\) av detta brus beror på Boltzmanns konstant
\(k\ =\ 1,38 \cdot 10^{-23}\) J/K, den absoluta temperaturen \(T\) i
kelvin samt bandbredden \(B\) i hertz och anges enligt formeln:
%%
\[P_n = k T B\]
%%
Varje motstånd med den absoluta temperaturen T kan modelleras som att ha en
ekvivalent spänning \(e_n\) och ström \(i_n\) för resistansen \(R\),
så som illustreras i bild~\ssaref{fig:BildII1-36} är
%%
\[e_n = \sqrt{4kTBR}\quad\text{och}\quad i_n = \sqrt{\dfrac{4kTB}{R}}.\]
%%
\subsubsection{Brusbandbredd}
\index{brus!brusbandbredd}

Medan vi initialt antagit att brusets bandbredd är för frekvenser
från DC till övre gränsfrekvensen, är det inte nödvändigt.
Formeln är även relevant för bruset på ett band och bandbredden för det
bandpassfilter vi har för att enbart lyssna på detta band.

Exempelvis behöver tal på SSB hantera \qty{300}{\hertz} till
\qty{3}{\kilo\hertz}, det vill säga \qty{2,7}{\kilo\hertz} bandbredd, och därmed
kommer även mottagarens bandbredd att behöva vara så stor.
Vi kommer då att ta emot brus för motsvarande bandbredd.
Ett filter för telegrafi kan till exempel vara \qty{350}{\hertz} och kommer
därmed också att ha ett motsvarande förhållande lägre bruseffekt.

Detta är dock en förenkling, eftersom filtret inte filtrerar med branta kanter
och är helt plant.
Filtrets egentliga brusbandbredd beror på hur filtret filtrerar över
alla frekvenser och summan av dessa.
Beroende på typ av filter behövs därför en korrigeringsfaktor
från den normala bandbredden till brusbandbredden.
För ett normalt 12\,dB/oktav lågpassfilter är korrigeringsfaktorn 1,22.

% Avsnitt 1.8 Modulation
\section{Modulation}
\harecsection{\harec{a}{1.8}{1.8}}
\label{modulation}
\index{modulation}

\subsection{Allmänt}
\index{modulation}
\index{modulerande signal}
\index{basband}
\index{modulerad signal}
\index{bärvåg}

\emph{Modulera} (lat. \emph{modulari}, rytmiskt avmäta, eng. \emph{modulate})
är att med hjälp av en oftast högfrekvent elektrisk signal (bärvågen) överföra
informationen i en lågfrekvent signal.
På så sätt kan lågfrekvens, till exempel tal och musik, först omvandlas till en
elektrisk signal, som får påverka (modulera) en högfrekvent elektrisk signal.
Denna modulerade signal strålas ut från antennen som ett elektromagnetiskt fält.

Den signal som innehåller informationen kallas \emph{modulerande signal},
\emph{basband} eller \emph{underbärvåg}.

Den signal som informationen överförts till kallas \emph{modulerad signal},
\emph{bärvåg} eller \emph{huvudbärvåg}.

\subsection{Modulationssystem}
\label{modulationssystem}

Den största gruppen av modulationssystem är definierad med avseende på hur
huvudbärvågen är modulerad.
Vanligast är då amplitud- och vinkelmodulation.
Av vinkelmodulation finns främst två slag, frekvensmodulation och fasmodulation.
Därutöver finns system för pulsmodulation.

\subsection{Sändningsslag}
\index{sändningsslag}
\label{sändningsslag}

Sätten att modulera kallas \emph{sändningsslag}.
Gemensamt för sändningsslagen är att en givare -- det kan vara en mikrofon, en
telegrafnyckel, en fjärrskriftsmaskin, en dator, en TV-kamera -- alstrar
en analog eller digital signal.
Denna styr underbärvågen så att huvudbärvågen moduleras med den avsedda
informationen och sänds ut.

Det enklaste sändningsslaget får anses vara morsetelegrafi med
''nycklad bärvåg''.
Då förekommer bara två tillstånd, nedtryckt och icke nedtryckt telegrafnyckel,
dvs. antingen bärvåg med någon varaktighet eller ingen bärvåg alls.
Kombinationer av bärvågselement med olika längd motsvarar skrivtecken.

För att återge tal, musik etc. behövs en noggrannare tillståndsstyrning av
bärvågen.
Det innebär att bärvågen måste moduleras av en underbärvåg och att denna
motsvarar lufttrycksvariationerna i ljudet.

\subsection{Kännetecken för modulerade signaler}
\label{kännetecken_modulerade_signaler}
\harecsection{\harec{a}{1.8.5}{1.8.5a}}
\index{amplitudmodulation}
\index{frekvensmodulation}
\index{fasmodulation}
\index{pulsmodulation}

\mediumfig{images/cropped_pdfs/bild_2_1-22.pdf}{Modulerade signaler}{fig:BildII1-22}

Bild~\ssaref{fig:BildII1-22} illustrerar modulerade signaler.
En modulerad signal kännetecknas av dess amplitud, frekvens och fasläge.

Vid \emph{amplitudmodulation} påverkas huvudbärvågens amplitud, så att den i
varje tidpunkt motsvarar den modulerande signalens variation.

Vid \emph{frekvensmodulation} påverkas huvudbärvågens frekvens, så att den i
varje tidpunkt motsvarar den modulerande signalens variation.

Vid \emph{fasmodulation}, som är besläktad med frekvensmodulation, påverkas i
stället för frekvensen huvudbärvågens fasläge i förhållande till en
referenssignal, så att fasläget i varje tidpunkt motsvarar den modulerande
signalens variation.

Frekvens- och fasmodulation liknar varandra och kan sammanfattas som
vinkelmodulation, eftersom fasvinkeln mellan bärvågens spänning och ström
varierar i båda fallen.

Vid \emph{pulsmodulation} används pulståg (korta upprepade bärvågspaket), till
exempel pulsamplitud-, pulslängds-, pulsläges- och pulskodmodulation.
Pulskodmodulation används till exempel vid samtidig överföring av flera
telesamtal på samma linje, bärvåg etc.

% \newpage % layout

\subsection{Bandbredd vid olika sändningsslag}
\harecsection{\harec{a}{1.8.5}{1.8.5b}}
\index{bandbredd}
\index{frekvenseffektivitet}
\label{bandbredd_modulation}

Varje radiosändning tar upp plats omkring den nominella bärvågsfrekvensen --
tillsammans \emph{bandbredden}.

Radioamatören måste veta detta ''platsbehov'', främst för att inte sända utanför
de frekvensband som är tilldelade för amatörradioanvändning, men även för att
kunna umgås med annan trafik inom banden.

I alla sändningsslag ökar den använda bandbredden med ökad modulation.
Eftersom största \emph{frekvenseffektivitet} alltid ska eftersträvas så upptar
en sändare med kraftigare modulation än vad som behövs för en överföring alltid
onödigt frekvensutrymme.

\subsection{Beskrivningskod för sändningsslagen}
\index{sändningsslag}
\label{modulation_beskrivningskod}

Vid 1979 års radioförvaltningskonferens (WARC 79) i Geneve reviderades det
internationella radioreglementet (RR), som i huvudsak trädde i kraft 1982.
Däri ingår bland annat ett nytt system för klassindelning och beteckning av
sätten att utsända information över radio med mera.
Reglementet har reviderats senare, men i detta stycke gäller det ännu.

Indelningen i \emph{sändningsslag} behövs för att känneteckna utsändningarna,
till exempel i frekvenslistor, författningar och föreskrifter.
Indelningen är också av stort värde vid teknisk beskrivning av apparater och
system för radiokommunikation.

Emellertid används av många även äldre benämningar, vilka lever kvar i
litteraturen, i märkning av manöverdonen på sändare och mottagare.

Dessa äldre benämningar är dock inte entydiga och skapar lätt missförstånd,
varför beskrivningskoden enligt WARC~79 bör användas för tydlighetens skull.

Här följer avkortade koder enligt WARC~79 för några av de sändningsslag som
amatörer använder mest, samt för jämförelse även de benämningar som fortfarande
används jämsides (se vidare i bilaga~\ssaref{sändslag}).

\mediumfig[0.67]{images/cropped_pdfs/bild_2_1-23.pdf}{Modulerande signaler}{fig:BildII1-23}

\begin{description}
\item[NON] Bärvåg utan modulerande signal. Ingen information.

\item[A1A] Bärvåg med dubbla sidband. En enda kanal med kvantiserad bärvåg.
Ingen modulerande underbärvåg. Telegrafi. Även kallat nycklad bärvåg (CW).

\item[A3E] Linjärt modulerad huvudbärvåg. Dubbla sidband. En enda kanal med
analog information. Telefoni. Även kallat amplitudmodulation (AM).

\item[J3E] Linjärt modulerad huvudbärvåg. Ett sidband med undertryckt bärvåg.
  En enda kanal med analog information. Telefoni.
  Även kallat enkelt sidband, Single Side Band (SSB).

\item[F3E] Vinkelmodulerad bärvåg. Frekvensmodulering. En enda kanal med analog
information. Telefoni. Även kallat frekvensmodulering (FM).

\item[G3E] Vinkelmodulerad bärvåg. Fasmodulering. En enda kanal med analog
information. Telefoni. Även kallat fasmodulering (PM).
\end{description}

Såväl A1A, A3E som J3E är sändningsslag där amplituden moduleras.
Därför är termen \emph{amplitudmodulation} inte tillräcklig för att beskriva
flera likartade sändningsslag.

\subsection{Modulerande signaler}
\harecsection{\harec{a}{1.7.1}{1.7.1}}
\index{modulerande signaler}

\subsubsection{Basband}
\index{basband}

Basband är ett frekvensområde för en modulerande signal.
Det finns ett basband för alla slags modulerande signaler, vare sig de är
analoga eller digitala.
Det kan finnas mer än ett basband i en komplett modulationsprocess.
Till exempel är en nycklad ton, som går till sändaren genom mikrofoningången,
dess analoga basband medan nycklingspulserna till tongeneratorn är dess
digitala basband.

Bild~\ssaref{fig:BildII1-23} illustrerar modulerade signaler.
Ett vanligt sätt att överföra information över radio är med telefoni, det vill
säga tal.

Frekvensområdet \SIrange{300}{3000}{\hertz} räcker för god förståelighet av tal.
Dels är örat känsligast inom det området och dels finns där den mesta energin
i talet.

Mikrofonen tar upp de lufttrycksvariationer som uppstår när man talar och
omvandlar dem till elektriska svängningar.
Svängningarna varierar mellan positiva och negativa spänningsvärden.

\bigskip

\textbf{Försök}

\begin{enumerate}
\item Anslut en mikrofon till ett oscilloskop och studera spänningsförloppen
  för olika slags ljud, toner, tal osv. som funktion av tiden.
  På bilden är dessa svängningar mycket förenklade, till exempel sinusformade.

\item Anslut en högtalare och ett oscilloskop till en LF-generator, vars
frekvens och amplitud kan ändras. Lyssna på ljud med låg och hög frekvens samt
på svaga och starka ljud.
En baston har låg frekvens och en diskantton har hög frekvens.
En svag ton har liten amplitud och en stark ton har stor amplitud.
\end{enumerate}

\subsection{Sändningsslaget A3E (AM)}
\harecsection{\harec{a}{1.8.2}{1.8.2}, \harec{a}{1.8.6b}{1.8.6b}, \harec{a}{1.8.7b}{1.8.7b}}
\index{amplitudmodulation}
\index{A3E}
\index{AM|see {amplitudmodulation}}
\label{modulation_am}

\mediumfig{images/cropped_pdfs/bild_2_1-24.pdf}{Sidband vid A3E-modulation}{fig:BildII1-24}

Bild~\ssaref{fig:BildII1-24} visar frekvensspektrum av en signal vid
amplitudmodulation med

\begin{enumerate}[label=\alph*.,noitemsep]
\item en sinuston,
\item en blandning av tre sinustoner,
\item ett frekvensspektrum.
\end{enumerate}

\noindent\textbf{Försök}
%
Modulera en A3E-sändare med en \qty{3}{\kilo\hertz}-signal.
Med en mottagare utrustad med ett smalt filter för telegrafi, kan man urskilja
och påvisa bärvågen och de båda sidbanden.

\subsubsection{A3E-modulation med en ton}

\mediumfig{images/cropped_pdfs/bild_2_1-25.pdf}{A3E-modulation med toner med olika styrka och frekvens}{fig:BildII1-25}

Bild~\ssaref{fig:BildII1-25} visar A3E-modulation med toner av olika styrka och
frekvens.
En omodulerad bärvåg har konstant amplitud.
En amplitudmodulerad signal är i grunden resultatet av svävning mellan
frekvenser eller av icke linjär blandning av frekvenser.
När bärvåg och basband blandas är särskilt tre blandningsprodukter av
intresse.

Dessa är:
\begin{itemize}
\item bärvågen
\item det lägre sidbandet (förkortat LSB)
\item det övre sidbandet (förkortat USB).
\end{itemize}

AM-signalen består således inte bara av bärvågsfrekvensen \(f_{HF}\) utan även
av övre och nedre sidofrekvenser, vilka är summan och skillnaden av
bärvågsfrekvensen \(f_{HF}\) och den modulerande frekvensen \(f_{LF}\).
Alltså \(f_{HF} + f_{LF}\) (övre sidofrekvens) och skillnadsfrekvensen
\(f_{HF} - f_{LF}\) (undre sidofrekvens).

Eftersom tal inte bara omfattar en enda frekvens utan ett helt frekvensspektrum
(ca \SIrange{0,3}{3}{\kilo\hertz}) uppstår inte bara två sidofrekvenser utan två
sidband, det lägre sidbandet (LSB, Lower Side Band) och det övre (USB, Upper
Side Band).

LF-signalens frekvens bestämmer sidofrekvensens avstånd från bärvågen.
Bandbredden på en amplitudmodulerad signal med full bärvåg och två sidband är
dubbelt så stor som den högsta modulerande LF-frekvensen:
\(b= 2 \cdot f_{LFmax}\)

Om de modulerande LF-frekvenserna är mellan 0,3 och \qty{3}{\kilo\hertz} blir
sändningens totala bandbredd \qty{6}{\kilo\hertz}.

LF-signalernas amplitud påverkar sidbandens och sidofrekvensernas amplitud.
Vid maximal modulation (100~\% modulationsgrad) varierar signalamplituden mellan
noll och dubbla värdet av det för en omodulerad bärvåg.

Som mest kan vardera sidbandet överföra en fjärdedel så mycket effekt som
bärvågen, dvs. en sjättedel av den totalt utsända effekten.
Då avger sändaren dubbelt så stor medeleffekt som utan modulation.
Toppeffekten (PEP, Peak Envelope Power) är till och med fyra gånger så stor.

Slutförstärkaren och kraftförsörjningen måste dimensioneras för toppeffekten vid
full modulation eller att modulationsgraden anpassas så att överbelastning inte
sker.

\subsubsection{Fördelar med A3E-modulation}

En A3E-sändare är enkel jämfört med en J3E-sändare, vilken har en mer
komplicerad signalbehandling.

\pagefig{images/cropped_pdfs/bild_2_1-26.pdf}{Amplitudmodulation med morsetecken}{fig:BildII1-26}

\subsubsection{Nackdelar med A3E-modulation}

Eftersom samma information finns i båda sidbanden och ingen finns i bärvågen,
så sänds effekten i bärvågen och ett av sidbanden ut till ingen nytta.
I talpauser sänds endast bärvågseffekten och till ingen nytta.
Även frekvensutrymme slösas bort.
Då en annan, alltför närliggande sändares bärvåg blandas med den egna,
alstras interferenstoner i mottagarna.

\mediumplustopfig{images/cropped_pdfs/bild_2_1-27.pdf}{Sidband vid DSB}{fig:BildII1-27}

\subsection{Sändningsslaget A1A (CW)}
\harecsection{\harec{a}{1.8.1}{1.8.1}, \harec{a}{1.8.6a}{1.8.6a}, \harec{a}{1.8.7a}{1.8.7a}}
\index{A1A}
\index{CW}
\label{modulation_cw}


Bild~\ssaref{fig:BildII1-26} visar amplitudmodulation med morsetecken.
Man kan överföra meddelanden med morsetelegrafi på olika sätt.
Det enklaste sättet är att koppla in och ur sändarens bärvåg i takt med
teckendelarna i morsetecknen.
Man kan kalla det för bärvågstelegrafi.
Förfarandet kallas sedan mycket länge även för CW (continous waves), vilket
egentligen anger att bärvågen svänger med konstant amplitud, om man bortser
från att den nycklas.
Detta står i motsats till de dämpade bärvågssvängningar som var fallet i sedan
mycket länge förbjudna gnistsändare.

Fastän en sändare ''moduleras utan ton'', har den en viss bandbredd.
Det beror på att den takt, som sändaren nycklas med, egentligen är en ton --
låt vara med låg frekvens.
Antag att sändaren nycklas med en serie korta morsetecken.
Vid telegraferingshastigheten 60~tecken/minut alstrar bärvågspulserna en kantvåg
med frekvensen \qty{5}{\hertz}.
Som tidigare beskrivits, består en sådan kantvåg av summan av sinussignaler med
frekvenserna \qty{5}{\hertz}, \qty{15}{\hertz}, \qty{25}{\hertz},
\qty{35}{\hertz} och så vidare.

Det innebär att det uppstår sidofrekvenser över och under bärvågens frekvens och
med ett avstånd till bärvågen av \qty{5}{\hertz}, \qty{15}{\hertz},
\qty{25}{\hertz}, \qty{35}{\hertz} osv.
Telegrafisändaren har alltså liksom vid A3E en bandbredd, som dels står i
förhållande till nycklingshastigheten och dels till ''kantigheten'' på tecknen,
vilket bestämmer övertonshalten i bärvågen.
Vid så kallad mjuk nyckling kan den 9:e övertonen antas vara den högsta som
uppfattas av en motstation.
Med en nycklingsfrekvens av \qty{5}{\hertz} blir bandbredden inte större än
\(2 \cdot 10 \cdot 5 = \qty{100}{\hertz}\).

En hård (kantig) och snabb teckengivning ökar bandbredden och kan resultera i
att så kallade nycklingsknäppar kan uppfattas långt vid sidan om
sändningsfrekvensen.
Ju hårdare nycklingen är, desto längre bort från bärvågsfrekvensen hörs
nycklingsknäpparna.
Detta stör andra stationer.

Kännetecken för sändningsslaget A1A, telegrafi genom nycklad bärvåg:

Mycket liten bandbredd, extremt gott utnyttjande av sändareffekten, stor
överföringssäkerhet, lång räckvidd, enkla sändare.

\subsection{Sändningsslaget J3E (SSB)}
\harecsection{\harec{a}{1.8.3c}{1.8.3c}, \harec{a}{1.8.6c}{1.8.6c}, \harec{a}{1.8.7c}{1.8.7c}}
\index{Single Side Band (SSB)}
\index{J3E}
\index{SSB}
\label{modulation_ssb}

\subsubsection{Princip}

Som sagts är det onödigt att sända ut två sidband, eftersom båda innehåller
samma information.

Signaler med endast ett sidband och undertryckt bärvåg kan alstras på flera
sätt.
Numera är den så kallade filtermetoden i särklass vanligast och den enda som
behandlas här.

Bild~\ssaref{fig:BildII1-27} illustrerar sidband vid DSB-modulation.
Med filtermetoden blandas HF- och LF-signalerna i en speciell blandare.
Där undertrycks båda dessa signaler medan blandningsprodukterna med deras summa-
och skillnadsfrekvenser blir kvar, dvs. det övre och nedre sidbandet.

Utsignalen från blandaren benämns DSB-signal (Double Side Band).
Till skillnad från i A3E-signalen saknas dock bärvågen i DSB-signalen.
För att även undertrycka det ena sidbandet före sändningen följs blandaren
av ett bandpassfilter med bandbredd och frekvensläge för avsett sidband.

Den signal som sänds ut innehåller därför endast ett sidband (Single Side Band).

\newpage
% \tallfig[0.45]{images/cropped_pdfs/bild_2_1-28.pdf}{Sidbandsval vid SSB}{fig:BildII1-28}
\mediumtopfig{images/cropped_pdfs/bild_2_1-28.pdf}{Sidbandsval vid SSB}{fig:BildII1-28}

\paragraph{Exempel}


Bild~\ssaref{fig:BildII1-28} illustrerar sidbandsval vid SSB-modulering.
Ett SSB-filter har ett passband av \SIrange{9000,3}{9003}{\kilo\hertz}.
Vid bärvågsfrekvensen \qty{9000}{\kilo\hertz} sträcker sig det övre sidbandet
från \SIrange{9000,3}{9003}{\kilo\hertz} och släpps igenom.
Däremot blir bärvågsfrekvensen undertryckt.

Det undre sidbandet \SIrange{8997}{8999,7}{\kilo\hertz} faller utanför filtrets
passband och blir också undertryckt.

Ska däremot det undre sidbandet kunna passera igenom samma filter, så måste
bärvågsfrekvensen höjas med \qty{3}{\kilo\hertz}, alltså till
\qty{9003}{\kilo\hertz}.
Då faller det undre sidbandet, \SIrange{9002,7}{9000,0}{\kilo\hertz} inom
filtrets passband.

Det övre sidbandet \SIrange{9003,3}{9006,0}{\kilo\hertz} faller nu utanför
passbandet och blir undertryckt.

%% k7per: Make this bigger.
\mediumtopfig{images/cropped_pdfs/bild_2_1-29.pdf}{Sidbandslägen vid SSB}{fig:BildII1-29}

Bild~\ssaref{fig:BildII1-29} illustrerar sidbandslägen vid SSB.
LF-signalens amplitud bestämmer amplituden på sidofrekvensen.

LF-signalens frekvens bestämmer sidofrekvensens avstånd från bärvågsfrekvensen
(bärvågen undertryckt).

Bandbredden på den utsända signalen är skillnaden mellan högsta och lägsta
modulerande frekvens i signalen:

till exempel \(b = \qty{3}{\kilo\hertz} - \qty{0,3}{\kilo\hertz} =
\qty{2,7}{\kilo\hertz}\)

\subsubsection{Fördelar med J3E-modulation}
Bra verkningsgrad vid J3E-modulation jämfört med vid A3E-modulation
(traditionell AM).
Effekten i det utsända sidbandet motsvarar den i ett av sidbanden vid A3E.
Hela den utsända effekten finns alltså i ett enda sidband,
som överför hela informationen.

I sändningspauserna sänds ingen effekt ut.
Bandbredden är mindre än hälften av den vid A3E.
Vid mottagning av en J3E-sändning (SSB) är det mindre besvär med
interferenstoner från J3E-sändningar på närliggande frekvenser, eftersom ingen
bärvåg och endast ett sidband sänds ut.

\subsubsection{Nackdelar med J3E-modulation}
J3E-modulation medför mera komplicerade apparater, både för mottagning och
sändning.
En J3E-signal blir förvrängd och hörs i fel tonläge om mottagaren
inte är inställd på exakt rätt frekvens.

\subsection{Vinkelmodulation}
\harecsection{\harec{a}{1.8.3a}{1.8.3a}}
\index{vinkelmodulation}
\label{modulation_vinkel}

Termen vinkelmodulation är samlingsnamnet för frekvensmodulation (FM) och
fasmodulation (PM).
Ofta sägs utrustningar vara för frekvensmodulation när de antingen är för
frekvens- eller fasmodulation.
Det finns alltså skillnader och likheter mellan dessa system, vilka emellertid
inte är oberoende av varandra, eftersom frekvensen i en signal inte kan
varieras utan att fasen också varieras, och vice versa.

Hur effektiv kommunikationen då är beror mest på mottagningsmetoderna.
I båda fallen uppfattas ändringar i den mottagna signalens frekvens och fasläge.
Amplitudändringar uppfattas däremot inte.
De flesta störningar -- särskilt pulserande sådana som från tändningssystem --
kommer därför att skiljas bort.

För att effektivt utnyttja fördelarna med vinkelmodulation, antingen det är
frekvens eller fasmodulation, behövs tillräckligt frekvensutrymme.
Det innebär att främst högre frekvensband kommer i fråga.

\newpage
\subsection{Frekvensmodulation (FM)}
\harecsection{\harec{a}{1.8.3b}{1.8.3b}, \harec{a}{1.8.6d}{1.8.6d}}
\index{frekvensmodulation}
\index{FM|see {frekvensmodulation}}
\label{modulation_fm}

\mediumfig[0.8]{images/cropped_pdfs/bild_2_1-30.pdf}{Frekvensmodulation}{fig:BildII1-30}

Bild~\ssaref{fig:BildII1-30} (överst och i mitten) visar frekvensmodulation.

Vid frekvensmodulation varierar bärvågens frekvens i takt med den modulerande
signalens amplitud och polaritet.
På bilden ökar bärvågens frekvens när den modulerande signalen är positiv
(första halvperioden) och minskar när den modulerande signalen är negativ
(andra halvperioden).
Bilden visar att perioderna i den modulerade bärvågen tar kortare tid (har
högre frekvens), när den modulerande signalen är positiv, och mer tid (har lägre
frekvens) när den modulerande signalen är negativ.
Bärvågen kommer alltså att pendla omkring ett medelvärde, dvs. vara
frekvensmodulerad.

Frekvensavvikelsen \(\Delta f\) (deviationen) från bärvågens vilofrekvens är
vid varje tillfälle proportionell mot den modulerande signalens amplitud.
Sålunda är deviationen liten när den modulerande signalens amplitud är liten
och störst när amplituden når sitt toppvärde, antingen amplituden är positiv
eller negativ.
Vid en modulationsfrekvens av \qty{300}{\hertz} varierar bärvågsfrekvensen 300
gånger per sekund, vid \qty{3}{\kilo\hertz} varierar den 3000 gånger per sekund.

Likspänningsnivåer kan överföras med FM, eftersom en motsvarande
frekvensavvikelse kan framställas.

Bilden visar också vad som oftast sägs, att bärvågsamplituden inte ändras av
modulationen.
Detta är emellertid bara delvis sant, eftersom såväl bärvågsamplitud som
sidbandsamplitud varierar med modulationsindex, vilket förklaras nedan.

\subsubsection{Sidbanden vid vinkelmodulation}

Vid AM produceras endast ett sidbandspar med samma innehåll, ett över och ett
under bärvågsfrekvensen.
Vid vinkelmodulation, både vid FM och PM, produceras däremot flera sidbandspar
över och under bärvågsfrekvensen.
Dessa sidband uppträder på multiplerna av varje modulerande frekvens.
Vid basband med samma frekvensomfång har därför en vinkelmodulerad signal
större bandbredd än en AM-signal.

Vid vinkelmodulation beror antalet sidband på sambandet mellan den modulerande
frekvensen, frekvensdeviationen och modulationsindex.

\mediumtopfig{images/cropped_pdfs/bild_2_1-31.pdf}{Sidbandsspektrum vid FM-modulering med 1 sinuston}{fig:BildII1-31}

\subsubsection{Bandbredden vid vinkelmodulation}

Bild~\ssaref{fig:BildII1-30} (nederst) visar bandbredd på vinkelmodulation.
Vi gör tankeexperimentet att en FM-sändare moduleras med en fyrkantsvåg.
Frekvensen kommer då att hoppa växelvis mellan frekvenserna \(f\) och
\(f + \Delta f\).
Sättet kallas FSK (frekvensskiftnyckling) och används till exempel vid sändning
av radiofjärrskrift (RTTY, AMTOR, Paketradio etc.).

Vi föreställer oss två sändare, som sänder varannan gång, varav den ena sänder
frekvensen \(f\) och den andra sänder \(f + \Delta f\).
Båda sändarnas HF-signaler kommer då att bilda ett frekvensspektrum, som
förutom \(f\) och \(f + \Delta f\) även innehåller sidofrekvenser.

Bredden på detta spektrum beror bland annat på nycklingsfrekvensen.
Eftersom en fyrkantsvåg innehåller summan av dess grundfrekvens och övertoner,
kommer alla dessa toner att modulera vardera sändaren.
De högsta modulerande LF-frekvenserna alstrar sidofrekvenserna längst ut från
vilofrekvensen.
LF-signalens frekvensspektrum påverkar alltså HF-signalens bandbredd.

Spektrum nederst i bilden är en förenklad framställning av
frekvensskiftnyckling.

Vid modulation med en sinussignal istället för med en fyrkantssignal, uppstår
ett frekvensspektrum som på överst i bilden.

%% k7per: Ska inte detts också vara en subsubsection?
\subsubsection{Frekvensdeviation och modulationsindex}
\harecsection{\harec{a}{1.8.4}{1.8.4}}
\index{frekvensdeviation}
\index{modulationsindex}
\index{symbol!\(m\) modulationsindex}

%% k7per: Find a solution for words that already have a hyphen. quote-dash?
Bild~\ssaref{fig:BildII1-31} visar sidbandsspektrum vid FM-moduler\-ing med 1
sinuston.
Vid vinkelmodulation uppstår talrika sidofrekvenser, som beror av den
modulerande frekvensen \(f_{LF}\).
Amplitudfördelningen mellan sidofrekvenserna står i förhållande till
deviationen, varvid deras amplitud blir mindre ju längre bort från bärvågen
de är.

I praktiken anses en sidofrekvens försumbar när dess amplitud är mindre än 1~\%
av amplituden för omodulerad bärvåg.

För beräkning av bandbredden används begreppet modulationsindex \(m\), vilket är
kvoten av maximal deviation \(\Delta f\) och högsta frekvensen \(f_{LF}\).
%%
\[m = \dfrac{\Delta f_{max}}{f_{LFmax}}\]
%%
Inom amatörradion är det vanligt att arbeta med \(\Delta f_{max} =
\qty{3}{\kilo\hertz}\) och \(f_{LFmax} = \qty{3}{\kilo\hertz}\), dvs. \(m = 1\).

Vid modulationsindex \(m = 1\), gäller följande formel för bandbredden \(b\)

% k7per: Make this a formula?
\medskip
\(b = 2 \cdot ( \Delta f_{max} + f_{LFmax}) = 2 \cdot \Delta f_{max}
 + 2 \cdot f_{LFmax}\)
 \medskip
 
Med ovan nämnda värden blir bandbredden \(b = 2 \cdot (\qty{3}{\kilo\hertz} +
\qty{3}{\kilo\hertz}) = \qty{12}{\kilo\hertz}\)

Bandbredden ökar således både med ökande deviation och ökande modulerande
frekvens.
För att inte interferera med trafik på grannkanalerna måste såväl deviation som
frekvensen på den modulerande signalen begränsas.
En deviationsbegränsare begränsar amplituden på denna signal.
Ett lågpassfilter reducerar den distorsion, som uppstår av begränsningen.
Vidare undertrycks modulerande frekvenser högre än \qty{3}{\kilo\hertz}, vilket
är tillräckligt för överföring av tal.

\paragraph{Jämförelse}

En VHF-rundradiosändare är tilldelad ett större frekvensutrymme och kan därför
använda mycket större bandbredd.

Där är \(\Delta f_{max} = \qty{75}{\kilo\hertz}\) och \(f_{LFmax} =
\qty{15}{\kilo\hertz}\), därmed är \(m = \frac{75}{15} = 5\) och \(b = 2 \cdot
(75 + 15) = \qty{180}{\kilo\hertz}\).

Som framgår av tabell~\ssaref{tab:ampmod} varierar bärvågens liksom
sidofrekvensernas inbördes amplitud med modulationsindex.
Detta ska jämföras med AM där bärvågens amplitud är konstant och endast
sidbandens amplitud varierar.

Vid vinkelmodulation utsläcks bärvågen \(A_0\) vid modulationsindex 2,404.
Den blir sedan ''negativ'' vid högre index, vilket betyder att den återkommer,
men att dess fasläge blir omvänt.
I vinkelmodulation tas energin i sidbanden från bärvågen, vilket innebär att
den totala effekten förblir densamma oavsett modulationsindex.

%\paragraph{Kännetecken för sändningsslaget F3E (FM)}
%\index{F3E}

\paragraph{Fördelar med sändningsslaget F3E (FM)}
F3E-sän\-daren är enkel till sin uppbyggnad och hög överföringskvalitet
uppnås vid stor bandbredd, störningar från amplitudmodulerade signaler såsom
tändgnistor undertrycks i mottagaren.

\paragraph{Nackdelar med sändningsslaget F3E (FM)}
En relativt stor bandbredd behövs för överföring av ett basband med stort
frekvensomfång.
Sändaren måste avge full effekt, även när modulation inte sker.

\begin{table*}[ht]
\begin{center}
  %\begin{tabular}{ll|S|S[table-format=-1.3]|S[table-format=-1.3|S[table-format=-1.3]|S[table-format=-1.3]|S[table-format=-1.3]|S[table-format=-1.3]|S[table-format=-1.3]|}
  \begin{tabular}{ll|S[table-format=-1.3]|S[table-format=-1.3]|S[table-format=-1.3]|S[table-format=-1.3]|S[table-format=-1.3]|S[table-format=-1.3]|S[table-format=-1.3]|l|}
\cline{3-9}
&\multicolumn{1}{l}{}  & \multicolumn{7}{|c|}{Modulationsindex} \\ \cline{3-9}
&\multicolumn{1}{l|}{}  &  \multicolumn{1}{c|}{1}   &   \multicolumn{1}{c|}{2}   &    \multicolumn{1}{c|}{3}   &    \multicolumn{1}{c|}{4}   &    \multicolumn{1}{c|}{5}   &    \multicolumn{1}{c|}{6}   &    \multicolumn{1}{c|}{7}   \\ \hline
\multicolumn{1}{|c|}{\multirow{11}{*}{\rotatebox[origin=c]{90}{Relativ amplitud på}}}&\(A_0\) & 0,765 & 0,224 & \num{-0,260} & \num{-0,397} & \num{-0,178} &  0,151 &  0,300 \\
\multicolumn{1}{|c|}{}&\(A_1\) & 0,440 & 0,577 &  0,334 & \num{-0,066} & \num{-0,328} & \num{-0,277} & -0,005 \\
\multicolumn{1}{|c|}{}&\(A_2\) & 0,115 & 0,353 &  0,486 &  0,364 &  0,047 & \num{-0,243} & -0,301 \\
\multicolumn{1}{|c|}{}&\(A_3\) & 0,020 & 0,129 &  0,309 &  0,430 &  0,365 &  0,115 & -0,168 \\
\multicolumn{1}{|c|}{}&\(A_4\) &       & 0,034 &  0,132 &  0,281 &  0,391 &  0,358 &  0,158 \\
\multicolumn{1}{|c|}{}&\(A_5\) &       & 0,016 &  0,043 &  0,132 &  0,261 &  0,362 &  0,348 \\
\multicolumn{1}{|c|}{}&\(A_6\) & \multicolumn{2}{c|}{} &  0,011 &  0,049 &  0,131 &  0,246 &  0,339 \\
\multicolumn{1}{|c|}{}&\(A_7\) & \multicolumn{3}{c|}{} &  0,015 &  0,053 &  0,130 &  0,234 \\
\multicolumn{1}{|c|}{}&\(A_8\) & \multicolumn{4}{c|}{}           &  0,018 &  0,057 &  0,128 \\
\multicolumn{1}{|c|}{}&\(A_9\) & \multicolumn{4}{c}{} &        &  0,021 &  0,059 \\
\multicolumn{1}{|c|}{}&\(A_{10}\) & \multicolumn{5}{c}{Tomma fält för \(A_n\) under 0,01 (1 \%)} &  &  0,024 \\ \hline
\end{tabular}
\end{center}
\caption{Relativa amplituden på bärvåg $A_0$ och sidofrekvenser $A_1$--$A_{10}$ vid
modulationsindex 1--7. (Vid omodulerad bärvåg är modulationsindex 0. Då är
bärvågens relativa amplitud 1,0.)}
\label{tab:ampmod}
\end{table*}


\subsection{Fasmodulation (PM)}
\index{fasmodulation}
\index{PM}

Vid fasmodulation varierar bärvågens fasläge i förhållande till ett
referensvärde.
Vid PM är frekvensändringen -- deviationen -- direkt proportionell mot hur
snabbt fasläget på den modulerande frekvensen ändras och till den totala
fasändringen.
Hastigheten på fasändringen är direkt proportionell mot frekvensen på den
modulerande frekvensen och till den momentana amplituden på den modulerande
signalen.

Det betyder att deviationen i PM-system ökar både med den momentana amplituden
och frekvensen på den modulerande signalen.
Detta att jämföras med FM-system där deviationen är proportionell mot den
momentana amplituden på den modulerande signalen.

I PM-system uppfattar demodulatorn i mottagaren endast momentana ändringar i
bärvågsfrekvensen.
Till skillnad från vid FM, så kan därför ändringar i likspänningsnivåer
överföras endast om en fasreferens används.

Med konstant amplitud på insignalen till modulatorn är vid PM
modulationsindex konstant oavsett modulerande frekvens, medan vid FM
modulationsindex varierar med den modulerande frekvensen.

\subsection{Frekvens- och fasmodulation jämförs}

\begin{itemize}
\item Frekvensmodulation (FM) alstras genom att sändarens oscillatorfrekvens
  varieras (devieras) i takt med den modulerande signalen (t.ex. tal).
  Det gör man genom att variera resonansfrekvensen i den resonanskrets som
  styr oscillatorfrekvensen.

\item Fasmodulation (PM) alstras vanligen genom att efter sändaroscillatorn
  variera den modulerande signalens fasläge i förhållande till en omodulerad
  bärvåg -- så kallad fasmodulering.
  Det gör man genom att variera resonansfrekvensen i en resonanskrets efter
  oscillatorn, dvs. utan att påverka oscillatorfrekvensen.

\item I båda fallen ändrar man alltså resonansfrekvensen i en resonanskrets i
  takt med frekvensen i den modulerande spänningen, men denna krets har
  olika placering i FM-sändare respektive PM-sändare.

\item I sändaren alstras det i båda fallen utfrekvenser som devierar från
  oscillatorns vilofrekvens.
  Graden av deviation skiljer emellertid vid FM och PM.
  Vid FM är deviationen proportionell mot amplituden på den modulerande
  underbärvågen medan deviationen vid PM är proportionell mot produkten av den
  modulerande underbärvågens amplitud och frekvens.

\item Den hörbara skillnaden mellan FM och PM är därför en annorlunda
  frekvensgång.
  Vid samtidig användning av PM-sändare och FM-mottagare är det alltså lämpligt
  att justera frekvensgången i PM-sändarens modulator, lämpligen
  med \qty{6}{\decibel} dämpning per oktav ökad frekvens.
\end{itemize}

\subsection{Pulsmodulation}
\index{pulsmodulation}
\index{PWM}
\index{PAM}
\index{PPM}

Pulsmodulation används mest i mikrovågsområdet.
Pulsmodulerade signaler sänds vanligen som en serie korta pulser åtskilda av
relativt långa pauser utan modulering.

En typisk sändning kan bestå av pulser med en längd av \qty{1}{\micro\second}
och en frekvens av \qty{1000}{\hertz}.
Toppeffekten på en pulssändning är därför mycket högre än dess medeleffekt

Före WARC~79 var symbolen för all pulssändning P.
Därefter används P endast för omodulerade pulståg.
Annan pulsmodulation har följande symboler

\begin{description}
\item[K] -- puls-/amplitudmodulation (PAM)
\item[L] -- pulsviddmodulation (PWM)
\item[M] -- pulsposition/fasmodulation (PPM)
\item[Q] -- vinkelmodulation under pulsen
\item[V] -- kombination av dessa eller annat sätt.
\end{description}

\begin{table*}[ht]
\begin{center}
\begin{tabular}{|L{.12\textwidth}|L{.18\textwidth}|L{.18\textwidth}|L{.18\textwidth}|L{.19\textwidth}|}
\hline
  Sändningsslag &
    Amplituden på LF-signalen &
    Tonhöjden på LF- signalen påverkar & 
    Bandbredden b förhåller sig till &
    För stor amplitud på LF-signalen medför \\
  \hline % =======================================================
  A3E (AM) & 
    amplituden i båda sidbanden &
    sidofrekvensernas avstånd från bärvågen &
    LF-signalens högsta frekvens & 
    övermodulering och för stor bandbredd \\
\hline
  J3E (SSB) & 
    amplituden på utsänt sidband & 
    sidofrekvensernas avstånd från bärvågen  & 
    skillnaden mellan LF-signalens högsta och lägsta frekvens & 
    för stor bandbredd, överstyrning av förstärkarsteg \\
\hline
  F3E (FM) &
    deviationen & 
    hastigheten på bärvågens frekvensändring &
    dubbla summan av största deviation och högsta LF-frekvens & 
    för stor deviation, för stor bandbredd \\
  \hline % =======================================================
\end{tabular}
\end{center}
\caption{Jämförelse mellan några vanliga sändningsslag inom amatörradio}
\end{table*}

% \newpage % layout

\subsection{Digital modulation}
\harecsection{\harec{a}{1.8.8}{1.8.8}}
\index{digital modulation}
\label{modulation_digital}

Utöver de klassiska analoga modulationsmetoderna finns ett antal digitala
modulationsformer.
De är anpassade för transmission av binära data.
I viss mån kan CW ses som digital modulation där 0 moduleras utan bärvåg
och 1 moduleras med bärvåg.
Det finns dock flera andra modulationsmetoder som FSK, 2-PSK/BPSK, 4-PSK och
QAM vilka presenteras i följande delavsnitt.

\subsubsection{Frekvensskiftsmodulation -- FSK}
\harecsection{\harec{a}{1.8.8a}{1.8.8a}}
\index{frekvensskiftsmodulation}
\index{Frequency Shift Keying (FSK)}
\index{FSK}
\index{frekvensmodulation}
\index{GFSK}
\index{Gaussian Frequency Shift Keying (GFSK)}
\index{Gaussiskt filter}
\index{C4FM}
\index{JT65}
\index{JT9}

\emph{Frekvensskiftsmodulation} (eng. \emph{Frequency Shift Keying (FSK)})
skiljer sig från CW-modulationen genom att den ändrar frekvensen, dvs. är en
variant av frekvensmodulation. I den enklaste formen, binär FSK växlar man
mellan två frekvenser, där en frekvens får representera 0 och den andra får
representera 1. Denna metod har används för modem på telefonförbindelser,
såsom Bell 103.

Eftersom varje växling mellan frekvenser ger avbrott i bägge signalerna, likt
nycklingen i CW, så kommer de att skapa sidband. Av det skälet filtrerar man
gärna signalen, och använder man ett Gaussiskt filter får man
\emph{Gaussian Frequency Shift Keying (GFSK)} som används av till exempel GSM-telefoni.

Man kan använda fler än två frekvenser, till exempel används fyra frekvenser i
Continuous 4 level FM (C4FM), i Phase 1 radios, i Project 25 samt Fusion.

Frekvensskift används även för att sända långsamma meddelanden där JT65
använder 65 frekvenser som den skiftar mellan, medan JT9 använder 9~frekvenser.

\subsubsection{Binär fasskiftsmodulation -- 2-PSK \& BPSK}
\harecsection{\harec{a}{1.8.8b}{1.8.8b}}
\index{binär fasskift modulation}
\index{fasskift modulation!binär}
\index{2-PSK}
\index{fasskift modulation!2-PSK}
\index{BPSK}
\index{fasskift modulation!BPSK}
\index{Costas loop}

Istället för att modulera frekvensen kan man modulera polariteten eller fasen.
En sådan modulationsform är \emph{binär fasskift modulation} (eng.
\emph{Binary Phase Shift Keying (BPSK)} eller \emph{2-state Phase Shift Keying
(2-PSK)}.
Förenklat kan man säga att bärvågen moduleras med \num{+1} eller \num{-1}, ofta
med \num{+1} representerande \num{0} och \num{-1} representerande \num{1}.

En nackdel med BPSK är att om polariteten blir förväxlad kommer meddelandet
att bli inverterat, dvs. 0 blir 1 och 1 blir 0.
BPSK behöver därför också kompletteras med annan digital modulation för att
hantera polariteten, något som i allmänhet kan åstadkommas enkelt.

BPSK används av satellitnavigationssystem som GPS, GLONASS och Galileo.
För att återvinna BPSK behöver man ofta en speciell variant av PLL-loop känd
som \emph{Costas loop}, eftersom en normal PLL-loop inte klarar av
teckenändringarna på signalen.

\subsubsection{Fyrnivå fasskiftmodulation -- 4-PSK}
\harecsection{\harec{a}{1.8.8c}{1.8.8c}}
\index{4-PSK}
\index{fasskift modulation!4-PSK}
\index{kvadratur-modulering}
\index{quadrature modulation}
\index{In phase (I)}
\index{Quadrature (Q)}
\index{I/Q modulation}

Fasskiftmodulation kan även göras med flera nivåer. När fyra olika faslägen
används kallas det för \emph{fyrnivå fasskiftmodulation} (eng.
\emph{4-state Phase Shift Keying (4-PSK)}).

Istället för 180~graders fasskift (0 och 180 grader) som används vid 2-PSK/BPSK
så använder man 360/4 det vill säga 90 graders fasskift mellan symbolerna.
Ett effektivt sätt att avkoda det är att göra \emph{kvadraturmodulering}
(eng. \emph{quadrature modulation}) där man modulerar en signal till två
komponenter, i \emph{fas} (eng. \emph{In Phase (I)}) och förskjuten 90 grader \emph{kvadratur} (eng.
\emph{Quadrature (Q)}), ofta kallat I/Q modulering.

De fyra faslägena kan nu enkelt förklaras som amplituder i de olika faslägena
som anges av tabell~\ssaref{tab:4-PSK}.

\begin{table}[t]
\begin{center}
\begin{tabular}{|r|r|r|r|}
\hline
Symbol & Vinkel & I & Q \\ \hline
0 &   0 & +1       &  0 \\
1 &  90 &  0       & +1 \\
2 & 180 & \num{-1} &  0 \\
3 & 270 &  0       & \num{-1} \\ \hline
\end{tabular}
\end{center}
\caption{4-PSK i kvadratur-modulering}
\label{tab:4-PSK}
\end{table}

Amplituden är densamma för alla fyra symbolerna, men med olika vinkel.
I likhet med 2-PSK/BPSK behöver man återvinna fasen och sedan kunna avgöra
vad som är 0 grader, men givet att det görs i den övriga modulationen så
kan informationen avkodas korrekt.

\subsubsection{Kvadratur-amplitudmodulation -- QAM}
\harecsection{\harec{a}{1.8.8d}{1.8.8d}}
\label{QAM}
\index{kvadratur-amplitudmodulation}
\index{QAM}
\index{16QAM}
\index{DAB}
\index{DVB-T}
\index{DVB-T2}
\index{Wi-Fi}

Medan fasskiftning kan göras för fler fassteg har man funnit att det inte
är lika enkelt för högre upplösningar.
Redan vid åtta steg behöver man ha I- och Q-värden som är \(\sqrt{1/2}\), vilket
i och för sig går att approximera.
En smidigare modulationsform är istället att låta även amplituden variera,
och genom att låta några bitar modulera I och några bitar modulera Q kan
man enkelt få ett symbolmönster som är effektivt att implementera.
Denna modulationsform kallar man \emph{kvadratur-amplitudmodulation}
(eng. \emph{Quadrature Amplitude Modulation (QAM)}).

Ofta benämner man olika varianter med antalet olika positioner, så att 16QAM
har 16 olika lägen i fas och amplitud tillsammans.
Ett exempel på hur 16QAM kan moduleras finns i tabell~\ssaref{tab:16QAM}.

\begin{table*}[ht]
\begin{center}
\begin{tabular}{|r|r|r|r|r|r|r|}
\hline
Symbol & Isym & Qsym & Amplitud      & Vinkel &  I &   Q \\ \hline
     0 &    0 &    0 & \(3\sqrt{2}\) &    +45 & +3 &  +3 \\
     1 &    0 &    1 & \(\sqrt{10}\) &    +72 & +3 &  +1 \\
     2 &    0 &    2 & \(\sqrt{10}\) &   +108 & +3 &  \num{-1} \\
     3 &    0 &    3 & \(3\sqrt{2}\) &   +135 & +3 &  \num{-3} \\
     4 &    1 &    0 & \(\sqrt{10}\) &    +18 & +1 &  +3 \\
     5 &    1 &    1 &  \(\sqrt{2}\) &    +45 & +1 &  +1 \\
     6 &    1 &    2 &  \(\sqrt{2}\) &   +135 & +1 &  \num{-1} \\
     7 &    1 &    3 & \(\sqrt{10}\) &   +162 & +1 &  \num{-3} \\
     8 &    2 &    0 & \(\sqrt{10}\) &   +342 & \num{-1} &  +3 \\
     9 &    2 &    1 &  \(\sqrt{2}\) &   +315 & \num{-1} &  +1 \\
    10 &    2 &    2 &  \(\sqrt{2}\) &   +225 & \num{-1} &  \num{-1} \\
    11 &    2 &    3 & \(\sqrt{10}\) &   +198 & \num{-1} &  \num{-3} \\
    12 &    3 &    0 & \(3\sqrt{2}\) &   +225 & \num{-3} &  +3 \\
    13 &    3 &    1 & \(\sqrt{10}\) &   +252 & \num{-3} &  +1 \\
    14 &    3 &    2 & \(\sqrt{10}\) &   +288 & \num{-3} &  \num{-1} \\
    15 &    3 &    3 & \(3\sqrt{2}\) &   +315 & \num{-3} &  \num{-3} \\ \hline
\end{tabular}
\end{center}
\caption{Exempel på 16QAM i kvadraturmodulering}
\label{tab:16QAM}
\end{table*}

Medan både amplituder och vinklar kan kännas udda, så är det enkelt
att mappa bitarna över till I- och Q-amplituder och faslägen via Isym- och
Qsym-delarna av symboler.

QAM-modulering används av DAB, DVB-T, DVB-T2, IEEE~802.11 (Wi-Fi),
mikrovågslänkar och många andra moderna system såsom EDGE
(efterföljaren till GSM med högre datatakt), UMTS när man kör
höghastighet (HSPA) liksom i LTE där man kör relativt långsamma
symboler men i stället väldigt många parallellt fördelat över ett
större frekvensband.
I mobiltelefonisystem använder man bland annat 64QAM och 256QAM.

Mikrovågslänkar använder upp till 2048QAM.
En fördel med QAM-moduleringen är att det är enkelt att få samma avstånd mellan
de olika symbolpositionerna, och därmed kan också modulationen anpassas
till störningen.
Detta nyttjas av många moderna modulationssystem så att QAM-modulationen
anpassas utifrån mottagarens rapportering om störning.
Denna dynamiska anpassning gör att kommunikationen kan upprätthållas även om
kapaciteten tillåts variera.

\subsection{Begrepp vid digital modulation}
\harecsection{\harec{a}{1.8.9}{1.8.9}}
\index{digital modulation}

Digital modulation innebär också att signalerna som sänds har lite
andra egenskaper än de analoga.
Istället för varierande spänningsnivåer som för till exempel tal skickar vi
diskreta fixa nivåer, ofta i form av bitar.
Det är därför lämpligt att diskutera några grundläggande begrepp kring digital
modulation.

% k7per
% \newpage % layout

\subsubsection{Bit rate}
\harecsection{\harec{a}{1.8.9a}{1.8.9a}}
\index{bit}
\index{byte}
\index{informationsmängd}
\index{informationsöverföringskapacitet}
\index{bit rate}

Informationen som vi skickar har vi kodat i bitar (eng. \emph{bit (b)}),
\emph{informationsmängden} vi har är därför ett visst antal bitar och takten på
denna informationsmängd blir därmed \emph{informationsöverföringskapaciteten}
(eng. \emph{bit rate}) i bitar per sekund.

Ofta brukar vi referera till informationsmängden som mängden \emph{byte (B)}
som till exempel att en fil är 2\,kB eller en bild är 1,25\,MB.
Då en byte innehåller åtta bitar motsvarar det 16\,kb respektive 10\,Mb.
I dagligt tal talar vi då om storleken på en fil.

Överföringskapaciteten, eller i dagligt tal hastigheten, brukar vi ofta prata
om i termer av \emph{bit rate} som 10\,Mb/s (ofta skrivet \emph{bps -- bits per
second}), dvs. man klarar av att överföra upp till 10 miljoner bitar per sekund.

Det är ofta som man talar om den råa överföringskapaciteten, medan den
verkliga överföringskapaciteten för nyttotrafik är något lägre på grund av
olika former av packningsformat och protokollbehov, så kallad \emph{overhead}.
Man ska därför vara noga med att skilja dessa åt.

\subsubsection{Symboltakt -- Baud rate}
\harecsection{\harec{a}{1.8.9b}{1.8.9b}}
\index{symbol}
\index{symboltakt}
\index{symbol rate}
\index{Baud rate}
\index{Baudot, Emile}
\index{enheter!baud (Bd)}

Som vi redan sett exempel på kan ibland bitar skickas en och en, eller
ihopklumpade. Varje sådan ihopklumpning kallas \emph{symbol}, och en symbol
kan bära en eller flera bitar, ibland inte ens ett jämnt antal.

Om man kan artikulera något i två olika \emph{nivåer} (av amplitud, fas,
frekvens eller kombination), så kan man representera en bit.
Om man kan artikulera något i fyra olika nivåer, kan man representera två bitar.
På samma sätt ger åtta nivåer support för tre bitar.
Varje representation kallar man en symbol och varje symbol bär alltså en, två
eller tre bitar information.
Strikt räknat är det logaritmen med bas två (2-logaritm eller $\log_{2}$) av
antalet nivåer som anger antalet bitar som en symbol kan bära.
Tre nivåer brukar sägas kunna bära 1,5~bitar, vilket är en slarvig approximation
men visar principen.

Den takt varmed symboler överförs \emph{symboltakten} (eng. \emph{symbol rate}),
benämns även \emph{Baud rate}, efter Emile Baudot, med enheten \emph{baud (Bd)}.
Enheten baud (förkortat Bd) anger antalet symboler per sekund.
Genom att multiplicera antalet symboler per sekund med antalet bitar per symbol
fås överföringskapaciteten bitar per sekund.

\subsubsection{Bandbredd}
\harecsection{\harec{a}{1.8.9c}{1.8.9c}}
\index{bandbredd}
\index{Nyquist-Shannons samplingsteorem}

Genom att justera antalet bitar per symbol kan man ändra antalet symboler
per sekund utan att ändra överföringskapaciteten. En anledning till att man
vill göra det är att bandbredden som används av en överföring är ungefär
proportionerlig mot symboltakten, det vill säga hur många baud man överför.
Detta påverkar hur stor del av radiospektrat man upptar, och därmed också hur
nära en annan signal man kan ligga i spektrat utan att störa varandra, dvs.
det påverkar frekvensplaneringen av bandet ifråga.

Ofta används begreppet bandbredd synonymt med överföringskapaciteten, eftersom
det finns en proportionell relation dem emellan, men bandbredden är inte den
enda parametern som krävs, så i mer strikta sammanhang ska dessa begrepp
hanteras som separata för att undvika missförstånd.

Bandbredden för en digital ström är relaterad till nyquistteoremet, som säger
att samplingstakten måste vara minst dubbelt så hög som den högsta frekvens
som överförs.

\subsection{Bitfel -- detektion och korrigering}

Hittills har vi diskuterat digital modulation utan att ta hänsyn till
störningar och hur dessa påverkar våra överförda data. Precis som
vår CW eller SSB kan vara störd av atmosfäriska störningar, andra sändare
eller helt enkelt vara svaga signaler så att det interna bruset blir en
begränsning, så kommer mottagningen av digitala signaler att bli störd.
Vi ska titta på dessa grundläggande begrepp såsom bitfel, bitfelssannolikhet,
felupptäckt samt korrigering med återsändning eller korrigeringskoder.

\subsubsection{Bitfel}
\index{bitfel}
\index{bit error}

Av olika orsaker kommer en eller flera bitar ofta att bli fel.
Vi kallar varje sådant fel för att \emph{bitfel} (eng. \emph{bit error}).
Störningar kan göra att vi tolkar en symbol fel, vilket kan resultera i en eller
flera felaktiga bitar.

Om vi i till exempel 16QAM-koden i kapitel~\ssaref{QAM} får in +0.2 i I och +1.1
i Q, ser vi i tabell~\ssaref{tab:16QAM} att närmaste symbolen är symbol 5 med +1
i I och +1 i Q.
Vi skulle kunna anta att om I är större än 0 och mindre än 2, samt Q är större
än 0 och mindre än 2 så är symbol 5 den enda vettiga symbolen, och det är precis
den tolkning vi i allmänhet gör, för det är den symbolen vars avstånd är lägst
och därmed rimligast.
Det kan dock vara så att man egentligen sände symbol 9 med \num{-1} i I och +1 i
Q, och därmed fick för stor störning på I för att man ska tolka det som rätt
symbol.
Vi kommer då lägga ut 9 istället för 5, vilket innebär att två bitar har
ändrats.

Genom att granska tabell~\ssaref{tab:16QAM} vidare ser man att värdena för
I och Q för de olika symbolerna är gjorda så att minsta avstånd är 2 mellan
alla närliggande symboler, i respektive I- och Q-riktning.
Det förenklar tolkning av symbolerna.
Är dock störningen större än 1 i någon riktning kommer man tolka den symbolen
fel, och det kan då leda till 1 eller fler bitfel.

\subsubsection{Bitfelssannolikhet}
\index{bitfelssannolikhet}
\index{bit error rate}
\index{BER|see {bit error rate}}
\index{gaussiskt brus}
\index{brus!gaussiskt}
\index{gaussian noise}
\index{effektiv-värde}
\index{Root Mean Square}
\index{RMS|see {Root Mean Square}}
\index{Error Function (erf)}
\index{erf}

Om vi antar att vi inte har störning från några andra signaler, utan enbart har
brus som störning, så kan vi estimera \emph{bitfelssannolikheten} (eng.
\emph{bit error rate (BER)}) ur hur starkt bruset är i förhållande till vårt
steg. Eftersom bruset antas vara vitt brus, så har det egenskaperna av
\emph{Gaussiskt brus} (eng. \emph{Gaussian noise}).

Gaussiskt brus har en statistisk fördelning med hög sannolikhet nära
medelvärdet och avtar sedan med avståndet. Sannolikheten att man tolkar en
signal som vara på ena eller andra sidan av en gräns beror på hur långt bort
från medelvärdet den gränsen, ofta benämnd kvantiseringsgränsen, är i
förhållande till den effektiva värdet (eng. \emph{Root Mean Square (RMS)}) i
amplitud hos bruset. Detta kan uttryckas i form av den matematiska funktionen
\emph{error function (erf)}.

När gränsen är 1~sigma, det vill säga 1 gånger RMS-värdet för brusamplituden,
från medelvärdet så är det 67~\% sannolikhet att värdet ligger inom
gränsvärdet, det vill säga en bitfelssannolikhet på 33~\%.
Ligger det inom 2 sigma har sannolikheten ökat till 97~\%, en
bitfelssannolikhet på 3~\%, och vid 3 sigma är den 99,7~\% med en
bitfelssannolikhet på ringa 0,3~\%, vilket ofta används för många
ingenjörsapplikationer. Dock, för överföring av information har vi högre krav.
För en bitfelssannolikhet på \(10^{-12}\), ofta benämnt BER på 1E-12, behövs
det 14~sigma bort till gränsen, dvs. brusmängden får max vara 1/14 av
kvantiseringsgränsen.
Den råa radiokanalen uppvisar dock sällan så bra egenskaper, men det kan uppnås
i kabel och fiber.

\subsubsection{Detektion}
\harecsection{\harec{a}{1.8.10a}{1.8.10a}}
\index{bitfelsdetektion}
\index{paritet}
\index{CRC}
\label{bitfel_detektion}

Eftersom störningar förekommer och man har behov av lägre bitfelssannolikhet
än vad den råa kanalen medger är det lämpligt att identifiera när det har
blivit bitfel. Detta kan utföras på många sätt, men ett sätt är att räkna fram
checksummor som skickas med datat. Det kräver visserligen en del av
informationsöverföringskapaciteten, men tjänsten det medger är att försäkra sig
om att informationen är rimligt korrekt.

En enkel form av checksumma är paritet, där bitarna i ett ord har summerats ihop
binärt (med XOR) för att bilda en checksumma. I mottagaränden görs samma
kombination och sedan jämförs det med paritetsbiten, och om de överensstämmer så
har inget bitfel upptäckts. Denna enkla metod har en svaghet i att ett jämnt
antal bitfel kommer att kompensera varandra, varvid det döljer bitfel från
upptäckt.
Det är med andra ord inte en särdeles stark checksumma.
Paritet används till exempel i seriekommunikation så som RS-232.

Ett flertal checksummor finns, för olika ändamål, olika mängd fel och olika
typer av fel. För lite större meddelanden är det vanligt att summera bytes
till en checksumma antingen additivt eller med XOR. För större meddelanden
används en lite mer intrikat metod som heter Cyclic Redundancy Check (CRC)
där man återmatar överskjutande del på checksumman till sig själv och får en
starkare kod den vägen. CRC används till exempel i Ethernet.

\subsubsection{Omsändning}
\harecsection{\harec{a}{1.8.10b}{1.8.10b}}
\index{omsändning}
\index{ARQ}
\index{TCP}

En enkel åtgärd att vidta när man konstaterat att ett block data man
tagit emot har fel, är att begära omsändning.
Genom att sändaren håller en buffert med meddelanden som den skickat, och
mottagaren meddelar sändare om den mottagit meddelandet eller behöver ha det
omskickat, så kan omsändning realiseras.
Automatisk omsändningsbegäran (eng. \emph{Automatic Repeat reQuest (ARQ)}) är en
typ av protokoll som gör automatisk omsändningsbegäran om ett enskilt datablock,
även kallat paket, inte kommit fram rätt eller helt försvunnit.
Ett sådant protokoll är TCP, som ingår i internetsviten av TCP/IP-protokollet.

\subsubsection{Korrigeringskod -- FEC}
\harecsection{\harec{a}{1.8.10c}{1.8.10c}}
\index{korrigeringskod}
\index{felrättandekod}
\index{FEC}
\index{AMTOR}
\index{Hamming-koder}
\index{paritet}
\index{Reed-Solomon (RS)}

En annan form av korrigering är att helt enkelt skicka för mycket data redan
från början, som mottagaren kan använda för att korrigera meddelandet utan att
skicka någon begäran till sändaren.
Detta är praktiskt antingen om det skulle ta för mycket tid eller om det helt
enkelt inte finns någon kommunikation från mottagaren till sändaren, till
exempel för satellitmottagare.

En enkel form av felrättande kod används i AMTOR FEC, där man helt enkelt
sänder samma tecken två gånger.
Liknande används i Bluetooth där meddelandet sänds tre gånger, varvid man kan
göra majoritetsröstning.

Andra system för FEC är Hamming-koder, paritets-paket och Reed-Solomon (RS).

%% k7per
%% \newpage % layout

\subsection{Digitala sändningsslag}

Här ges exempel på digitala modulationstekniker för kortvågstillämpningar inom
amatörradio.

De flesta digitala sändningsslagen för kortvåg är smalbandiga och bandbredden
kan i vissa fall endast vara några hertz.

Signalbehandlingen sker i den dator som programmet körs på och där datorns
in- och utgång för dess ljudkort är kopplade till amatöradioutrustningen.
Oftast är programmets styrning av sändning och mottagning också kopplad till
lämplig serieport, till exempel via dess USB-anslutning.

\subsubsection{RTTY}
\index{RTTY}
\index{FSK}
\index{Frequency Shift Keying (FSK)}
\index{Audio Frequency Shift Keying (AFSK)}
\index{AFSK}

\paragraph{Historia}

Ett av de första digitala trafiksätten som användes av radioamatörer var \emph{RTTY},
uttytt ''RadioTeleTYpe'', där man använde sig av så kallade teleprintrar,
automatiska skrivmaskiner som skrev ut text.

Emile Baudot konstruerade år 1874 ett system baserat på fem bitar,
som fortfarande används idag.
I augusti 1922 testade The US Department of Navy ''skriven telegrafi'' mellan
ett flygplan och en markstation.
Amerikanska kommersiella RTTY-system fanns aktiva så tidigt som 1932.
Under 50-talet började surplusutrustning komma ut på den amerikanska
marknaden och radioamatörerna var inte sena att prova den nya tekniken på
kortvågsbanden.
De kommersiella systemen körde med 50~baud, 75~baud eller 100~baud.

Amatörerna i USA körde med 45,45~baud, vilket motsvarar 300 tecken per minut.
De europeiska utrustningarna, bland annat framtagna av Siemens, arbetade med
50~baud men gick att justera ner till 45,45~baud.
45~baud är idag den vedertagna standarden över världen.

\paragraph{Teknik}

RTTY använder FSK-modulering.
För att åstadkomma detta behöver man styra frekvensen så att den hoppar mellan
två frekvenser med en skillnad, ett så kallat ''skift'', på \qty{170}{\hertz}.

Äldre sändare behövde modifieras för att åstadkomma detta frekvensskift, men
med en SSB-sändare kunde man istället mata sändaren med två toner, som gav
samma resultat -- så kallad Audio Frequency Shift Keying (AFSK).

Med nyare amatörradiotransceivrar blev det senare den vanligaste förekommande
metoden att modulera sändaren.
Det innebar att man slapp modifiera utrustningen.

Idag kör de flesta radioamatörer RTTY med en dator och använder sig ofta av
AFSK-tekniken med hjälp av programvaror, med samma uppkoppling som man använder
för andra digitala trafiksätt.

\subsubsection{SSTV}
\index{SSTV}

\emph{Slow Scan Television (SSTV)} är en blandning av analog och digital teknik.
En SSTV sändning sker långsamt jämfört med traditionell TV, men är i grunden
rätt lik.
Varje linje sänds en efter en, men modulerad så att den kan sändas över en
SSB-radiolänk.
Intensiteten för varje pixel anger tonhöjden som moduleras, vilket därmed
innebär en FM-modulation.
Denna FM-modulerade ton skickas sedan över SSB.
I början på varje linje skickas ett 7-bitars tal med jämn paritet som anger
vilken modulationsform som används.
De olika modulationsformerna kan sedan hantera olika upplösningar samt
variera med avseende på svart-vitt eller färg.

\subsubsection{APRS}
\index{APRS}
\index{AX.25}
\index{TNC}
\label{modulation_aprs}

\emph{Automatic Packet Reporting System (APRS)} är en teknik för att
huvudsakligen över VHF och UHF förmedla GPS-position, väderdata, enkla
meddelanden och annat.
Den bygger på en teknik som heter \emph{AX.25}, som är en amatörradiospecifik
version av telekomstandarden X.25.
AX.25 moduleras över 1200 baud Bell 202 AFSK teknik på vanlig talkanal.
Ofta används en Terminal Node Controller (TNC) som gränssnitt mellan dator och
radio.

\subsubsection{PSK31}
\index{PSK31}
\label{modulation_psk31}

\paragraph{Historia}

Namnet beskriver modulationstypen och överföringshastigheten i baud.
Det första programmet utvecklades specifikt för windowsbaserade datorer med
ljudkort av den engelska radioamatören Peter Martinez, G3PLX, och
introducerades i amatörradiovärlden 1998.

\paragraph{Teknik}

Modulationen som används i PSK31 utvecklades från en idé av den polske
radioamatören Pawel Jalocha, SP9VRC, som hade tagit fram en mjukvara
''SLOWBPSK'' för Motorolas EVM-radio, vilket var ett radiosystem för att
utvärdera olika modulationsformer.
Istället för att använda den gängse metoden med frekvensskift baserades
''SLOWBPSK'' på polaritetsskiftning av fasläget.
Ett bra utformat PSK-baserat system kan ge bättre resultat än FSK, och kan
arbeta med smalare bandbredd än FSK.
Överföringshastigheten 31~baud valdes för att passa en genomsnittlig
skrivhastighet hos den gemene amatören.

\subsubsection{WSPR}
\index{WSPR}

\paragraph{Historia}

WSPR släpptes i sin första version 2008.
Programmet skrevs initialt av Joe Taylor, K1JT, men är nu ett öppen
källkodsprogram och utvecklas av ett litet team.
Joe Taylor fick sin utbildning i astronomi vid Harvard University.
Han var sedan verksam inom området astrofysik vid Princeton University,
varifrån han pensionerades 2006.
Joe Taylor tilldelades Nobelpriset i fysik år 1993.

Programmet är i huvudsak tänkt för vågutbredningstester inom kortvågsområdet.

\paragraph{Teknik}

WSPR står för Weak Signaling Propagation Reporter och uttalas ''Whisper''.
WSPR är ett sändningsslag som använder amatörradiostationen som en radiofyr, en
så kallad beacon.
Sändning och mottagning sker i tvåminuterspass och efter varje sändningspass
rapporterar de stationer som mottagit signalen in sitt resultat till en databas
över internet.
Den sändande stationen kan därefter studera resultatet.
WSPR använder låga effekter, det går att nå europeiska stationer med effekter
under \qty{100}{\milli\watt}, och andra kontinenter med effekter under några
watt, även med modesta antenner.

\subsubsection{WSJT}
\index{WSJT}
\index{FSK441}
\index{JT6M}
\index{JT65}
\index{JT9}
\index{FT8}
\index{FSK}
\index{Frequency Shift Keying (FSK)}
\index{meteorer}
\index{troposcatter}
\index{EME}
\index{månstuds}
\index{8FSK}

WSJT är liksom WSPR ett program som används inom amatörradiohobbyn för så
kommunikation med svaga signaler.
Även detta program är utvecklat av Joe Taylor, K1JT.
De flesta av dessa sändningsslag (se nedan) är så smalbandiga, att de inte
upptar större bandbredd än några hertz.

\paragraph{Historia}

WSJT presenterades för amatörradiovärlden år 2001 och har undergått ett flertal
revisioner.
Olika sändningsslag har under åren lagts till och tagits bort.
Sedan 2005 har programmet öppen källkod och utvecklas av ett litet team.

\paragraph{Teknik}

WSJT erbjuder en plattform för ett flertal olika tillämpningar där olika
varianter av i huvudsak FSK-modulering används.

FSK441 används för att utvärdera överföringar via radiovågsreflekterande skikt
av laddade joner, som uppkommer från de spår som meteorer lämnar efter sig.

JT6M introducerades år 2002 och är avsett för kommunikation via bland annat
meteorreflektioner på \qty{6}{\metre}-bandet.

JT65, utvecklat och släppt år 2003, används för kommunikation via troposfären,
så kallat troposcatter, men också för kommunikation via reflektion mot månen
så kallad EME-trafik.

JT9 används för kortvågstrafik och är snarlikt JT65, men använder sig av en
FSK-signal med nio toner.
JT9 använder sig av mindre än \qty{16}{\hertz} bandbredd.

FT8 utvecklades och släpptes år 2017 och använder sig av en 8FSK-signal.

FT8 är att föredra vid så kallat multi-hop via E-skikt, där signalerna utsätts
för fädning och där öppningarna mot andra stationer är korta så att man behöver
slutföra kommunikationen inom en kort tid.

\subsubsection{FreeDV}
\index{FreeDV}

FreeDV skiljer sig mot de sändningsslag som nämnts ovan genom att detta är tänkt
för digitalt tal på kortvåg.

\paragraph{Historia}

FreeDV skapades av en grupp radioamatörer från olika länder som arbetade
med kodning, utformning, användargränssnitt och testning.
FreeDV släpptes år 2015.

\paragraph{Teknik}

FreeDV är tänkt att användas på kortvåg med SSB-modulerade radiostationer,
men kan också användas med AM- eller FM -modulering.
Fördelen ska vara att överföringen blir mer robust samt att signaleringen är
utformad för att motverka påverkan av fädning.

FreeDV använder en något mer komplex modulering.
Man använder sig av ett flertal bärvågor inom dess bandbredd på
\qty{1,25}{\kilo\hertz}.
Bärvägorna ligger med \qty{75}{\hertz} mellanrum och varje bärvåg moduleras med
varianter av PSK-modulering.
Bandbredden är hälften (\qty{1,25}{\kilo\hertz}) av en normal SSB-bandbredd
(\qty{2,4}{\kilo\hertz}).

% Avsnitt 1.9 Effekt och energi
\mediumtopfig{images/nomogram_db_effekt.png}{Nomogram för omvandling mellan effekt och decibel}{ellära-nomogram-db-effekt}

\section{Effekt och energi}
\harecsection{\harec{a}{1.9}{1.9}}
\label{effect och energi}
\index{effekt}
\index{energi}

\subsection{Effekt i en sinusformad signal}
\harecsection{\harec{a}{1.9.1}{1.9.1}}

För beräkning av effekten av en sinusformad signal använder man effektivvärdet
av spänning och ström.
%%
\[\begin{array}{ccc}
    U_{\textit{eff}} = \dfrac{U_{max}}{\sqrt{2}} &
    \quad I_{\textit{eff}} = \dfrac{I_{max}}{\sqrt{2}} &
    \quad P = U_{\textit{eff}} \cdot I_{\textit{eff}} \\
    \end{array}\]


\subsection{Effektändring uttryckt i dB}
\harecsection{\harec{a}{1.9.2}{1.9.2}}
\label{decibel}
\index{dB}
\index{dB!effekt}
\index{effekt!dB}
\label{effekt_db}

Måtten i det metriska systemet är alldagliga och ingen finner det märkligt att
det till exempel går tio decimeter på en meter.
Däremot är begreppet decibel ovant för många.

I detta avsnitt förklaras det mycket användbara begreppet decibel.
Decibel (dB) är en tiondedel av grundenheten Bel (B).

Räkning med decibel grundas på logaritmer, som är ett bekvämt sätt att uttrycka
och behandla talvärden.

\begin{quote}\emph{
Decibel är ett dimensionslöst uttryck för graden av dämpning alternativt
förstärkning.
}\end{quote}

\emph{Effektdämpning} är följden av att vissa komponenter bromsar elektrisk
ström.
Den bromsande faktorn kan vara en resistans R, induktans L, kapacitans C eller
sammansatta nätverk av R, L och C.

\emph{Effektförstärkning} innebär att en transistor, ett elektronrör eller
annan så kallad aktiv komponent kan styra en större elektrisk ström och därmed
större effekt än den själv styrs med.
Vad som förorsakar effektförändringarna går vi inte in på i detta sammanhang,
utan byggdelarna betraktas som ''svarta lådor'' med anslutningsklämmor.

En byggdel med två ingångs- och två utgångsklämmor kallas för ''fyrpol'', se bild~\ssaref{fig:BildII1-32}.

\smallfig[0.4]{images/cropped_pdfs/bild_2_1-32.pdf}{Effektförhållande}{fig:BildII1-32}

Antag att den inmatade effekten P är \qty{1}{\watt}.
Om effekten inte ändras vid passagen genom fyrpolen, så är även den uttagna
effekten \qty{1}{\watt}.

\emph{Effektförhållandet} mellan in- och utgångarna är då:
%%
\[\dfrac{P_{in}}{P_{ut}} = \dfrac{1\ \textit{watt}}{1\ \textit{watt}} = 1 (\text{kvoten} = 1)\]
%%
Oförändrad effekt varken dämpas eller förstärks, varför både dämpningen och
förstärkningen har talvärdet 0.
Enheten på talvärdet är Bel, dämpningen eller förstärkningen är således 0~Bel.
En tiondel därav är 0~decibel (\qty{0}{\decibel}).

Omräkning av kvoten av en effektändring till \unit{\decibel} görs så, att
10-logaritmen för kvoten söks och resultatet blir effektändringen uttryckt i Bel
(B).
Om resultatet uttrycks i \unit{\decibel}, ska Bel-värdet multipliceras med 10.

Logaritmer förklaras i bilaga~\ssaref{logaritmer}.

För att förenkla beräkningen av dB-talet divideras det högre effekttalet med det
lägre.
Bokstaven a i följande formler betyder antingen förstärkning (+a) eller
dämpning (-a) beroendet på vilket förtecken som sätts.
%%
\[a[B] = \log \dfrac{P_{\text{hög}}}{P_{\text{låg}}}\]
\[a[dB] = 10\log \dfrac{P_{\text{hög}}}{P_{\text{låg}}}\]
%%
Att addera och subtrahera värden på en logaritmisk skala motsvarar att
multiplicera respektive dividera värden på en linjär skala.
Huvudskalorna på en räknesticka är logaritmiska.
(Räknestickan är ett enkelt och förut mycket använt hjälpmedel).

Med hjälp av nomogrammet i bild~\ssaref{ellära-nomogram-db-effekt} kan en
\emph{effektändring}, uttryckt som kvot (effekterna dividerade med varandra),
omvandlas till decibel och omvänt.


Följande avrundade värden kan utläsas:

\begin{center}
\begin{tabular}{rlrlrl}
0 dB & = 1 &  1 dB & =  1,25 & 2 dB = 1,6 \\
3 dB & = 2 &  4 dB & =  2,5  & 5 dB = 3,2 \\
6 dB & = 4 &  7 dB & =  5    & 8 dB = 6,3 \\
9 dB & = 8 & 10 dB & = 10    & 11 dB = 12,5
\end{tabular}
\end{center}

Det vill säga vid ökning fördubblas effekten för var 3:e dB och vid minskning
halveras effekten för var 3:e \unit{\decibel}.

Om kvoten är en eller flera 10-potenser högre än 10, så kan nomogrammet utökas
enligt följande tabell.

\begin{center}
\begin{tabular}{rllr}
Kvot av & Analys             & Skriv            & dB \\
\(P_{\text{hög}}/P_{\text{låg}}\) &          &                  &    \\
     1 & 1 har 0 nollor      & \(0 \cdot 10\) = &  0 \\
    10 & 10 har 1 nolla      & \(1 \cdot 10\) = & 10 \\
   100 & 100 har 2 nollor    & \(2 \cdot 10\) = & 20 \\
 1 000 &  1 000 har 3 nollor & \(3 \cdot 10\) = & 30 \\
10 000 & 10 000 har 4 nollor & \(4 \cdot 10\) = & 40
\end{tabular}
\end{center}

\subsection{Strömändring uttryckt i dB}
\index{ström!dB}
\index{dB!ström}

%% k7per: These paragraphs are hard to use, convert the math to real formulas?
Förhållandet mellan strömmar liksom mellan spänningar kan även uttryckas i
\unit{\decibel}, men annorlunda än mellan effekter.
En fyrpol med inbördes lika ingångs- och utgångsimpedans är förutsättningen för
jämförelse.

Enligt Joules lag är \(P = I^2 \cdot R\) (\(P = U \cdot I\)), således

\[\dfrac{P_{\text{hög}}}{P_{\text{låg}}} = \dfrac{I_{\text{hög}}^2 \cdot R}{I_{\text{låg}}^2 \cdot R}\].

R kan avkortas \emph{om in- och utgångsimpedanserna (resistanserna) är lika}.

En jämförelse uttryckt i dB kan endast göras under samma förutsättningar;
här att impedanserna (resistanserna) är lika, således

\[\dfrac{P_{\text{hög}}}{P_{\text{låg}}} = \dfrac{I_{\text{hög}}^2}{I_{\text{låg}}^2}\].

Effektförhållandet eller kvadratvärdet på strömförhållandet kan uttryckas
logaritmiskt i B eller dB

\[a[dB] = 10\log \dfrac{I_{\text{hög}}^2}{I_{\text{låg}}^2}\]

Eftersom \(\log x^2 = 2 \cdot \log x\), fås slutligen

\[a[dB] = 20\log \dfrac{I_{\text{hög}}}{I_{\text{låg}}}\].

\subsection{Spänningsändring uttryckt i dB}
\index{spänning!dB}
\index{dB!spänning}

Förhållandet mellan spänningar kan uttryckas i dB på ett liknande sätt som med
strömmar.

Enligt Joules lag är \(P = \frac{U^2}{R}\) (\(P = U \cdot I\))

Två effekter kan ställas i förhållande till varandra på följande sätt:
%%
\[\dfrac{P_{\text{hög}}}{P_{\text{låg}}}=\dfrac{U_{\text{hög}}^2 \cdot R}{U_{\text{låg}}^2 \cdot R}\]
%%
\(R\) avkortas och efter omskrivning fås en formel som liknar den för strömmar
%%
\[\dfrac{P_{\text{hög}}}{P_{\text{låg}}} = \dfrac{U_{\text{hög}}^2}{U_{\text{låg}}^2}\]
\[a[dB] = 20\log \dfrac{U_{\text{hög}}}{U_{\text{låg}}}\]
%%
Med nomogrammet i bild~\ssaref{ellära-nomogram-db-spänning} kan kvoten av en
ström- eller spänningsändring omvandlas till decibel och tvärt om.

\mediumfig{images/nomogram_db_spanning.png}{Nomogram för omvandling mellan spänning och decibel}{ellära-nomogram-db-spänning}

Följande avrundade värden kan utläsas:

\begin{center}
\begin{tabular}{rlrlrl}
0 dB & = 1   &  1 dB & = 1,12 &  2 dB = 1,25 \\
3 dB & = 1,4 &  4 dB & = 1,6  &  5 dB = 1,8 \\
6 dB & = 2   &  7 dB & = 2,24 &  8 dB = 2,5 \\
9 dB & = 2,8 & 10 dB & = 3,2  & 11 dB = 3,6
\end{tabular}
\end{center}

Det vill säga vid ökning fördubblas strömmen resp. spänningen för var 6:e
\unit{\decibel} och att vid minskning halveras strömmen resp. spänningen för var
6:e \unit{\decibel}.

Om kvoten är en eller flera 10-potenser högre än 10, så kan nomogrammet utökas
enligt följande tabell.

\begin{center}
\begin{tabular}{rllr}
  Kvot av & & & \\
\(U_{\text{hög}}/U_{\text{låg}}\) &          &                  &    \\
\(I_{\text{hög}}/I_{\text{låg}}\) & Analys             & Skriv            & dB \\
  \hline
     1 & 1 har 0 nollor      & \(0 \cdot 20\) = &  0 \\
    10 & 10 har 1 nolla      & \(1 \cdot 20\) = & 20 \\
   100 & 100 har 2 nollor    & \(2 \cdot 20\) = & 40 \\
 1 000 &  1 000 har 3 nollor & \(3 \cdot 20\) = & 60 \\
10 000 & 10 000 har 4 nollor & \(4 \cdot 20\) = & 80
\end{tabular}
\end{center}

\section{dB med miniräknare}

%% k7per: Hard to read, turn into math...
När beräkningen av förstärkning eller dämpning uttryckt i \unit{\decibel} görs
med miniräknare låter man räknaren sköta hela beräkningen.

Med miniräknare skrivs uttrycken för \unit{\decibel} så här:

För effekt \(a[dB] = 10\log \dfrac{P_{\text{ut}}}{P_{\text{in}}}\)

För spänning \(a[dB] = 20\log \dfrac{U_{\text{ut}}}{U_{\text{in}}}\)

\emph{Lägg märke till hur värdena ut och in används i ekvationerna.}

Detta gör att man vid beräkningen automatiskt får positiva svar för
förstärkning och negativa svar för dämpning.

\section{Decibel över 1 mW vid 50 ohm [dB(m)]}
\label{dBm}

Det är mycket vanligt att in- och utgångarna i HF-utrustningar utförs
med en impedans av \qty{50}{\ohm}.
För god anpassning väljs då koaxialkablarna mellan apparaterna med en
karaktäristisk impedans av \qty{50}{\ohm}.

Det har utvecklats en praxis, att referensvärdet vid jämförelse av signalnivåer
i radiosystem ska vara en milliwatt (\qty{1}{\milli\watt}) utvecklad i en
belastning med impedansen \qty{50}{\ohm}.
Signalnivåer över belastningen \qty{50}{\ohm} kan uttryckas i dB(m) eller ofta
dBm, där (m) står för milliwatt, varvid referenseffekten \qty{1}{\milli\watt} är
0\,dB(m) vid \qty{50}{\ohm}.

Det spänningsfall som bildas över belastningen \qty{50}{\ohm} vid effektnivån
0\,dB(m) är
%%
\[U = \sqrt{P\cdot R} = \sqrt{1\cdot 10^{-3} \cdot 50} \approx \qty{0,224}{\volt}\]
%%
Den ström som flyter genom belastningen \qty{50}{\ohm} vid effektnivån 0\,dB(m)
är
%%
\[
I = \sqrt{\frac{P}{R}} = \sqrt{\frac{1\cdot 10^{-3}}{50}} \approx \qty{0,0045}{\ampere} = \qty{4,5}{\milli\ampere}
\]
%%
Strömmen \qty{4,5}{\milli\ampere} genom belastningen \qty{50}{\ohm} motsvarar
således 0\,dB(m).

Varje annan effekt, spänningsfall och ström som uppstår vid en belastning av
\qty{50}{\ohm} kan jämföras med respektive referensvärden \qty{1}{\milli\watt},
\qty{0,22}{\volt} och \qty{4,5}{\milli\ampere}.
\emph{dB(m) är ett absolut och logaritmiskt mått.}

\vspace*{1ex}
\noindent
Effekt
%%
\begin{gather*}
	a [dB(m)] = 10 \log\frac{P_{[\qty{50}{\ohm}]}}{1[mW_{\qty{50}{\ohm}}]} \\
	P_{50} = 1 [mW] \cdot 10^{\dfrac{a}{10}}
\end{gather*}
%%
Ström
%%
\begin{gather*}
	0 dB(m) \approx 4,47 mA_{50} \\
	a [dB(m)] \approx 20 \log\frac{I_{50}}{4,47}
\end{gather*}
%%
Spänning
%%
\begin{gather*}
	0 dB(m) \approx 0,224 V_{50} \\
	a [dB(m)] \approx 20 \log\frac{U_{50}}{0,224} \\
	U_{50} \approx 0,224 \cdot 10^{\dfrac{a}{20}}
\end{gather*}

\section[Sambandet spänning och dBm]{Sambandet mellan spänning och dBm}

\begin{center}
\begin{tabular}{S[table-format=-2]|lp{1cm}S[table-format=2]|l}
	\multicolumn{1}{c|}{dBm} & V & & \multicolumn{1}{c|}{dBm} & V \\
	\cline{1-2} \cline{4-5}
	\num{-40} & 0,00224 & & & \\
	\num{-30} & 0,00707 & & & \\
	\num{-20} & 0,0224  & & & \\
	\num{-10} & 0,0707  & & & \\
	0   & 0,224   & & & \\
	1   & 0,251   & & 11 & 0,793 \\
	2   & 0,282   & & 12 & 0,890 \\
	3   & 0,316   & & 13 & 0,999 \\
	4   & 0,354   & & 14 & 1,121 \\
	5   & 0,398   & & 15 & 1,257 \\
	6   & 0,446   & & 16 & 1,411 \\
	7   & 0,501   & & 17 & 1,583 \\
	8   & 0,562   & & 18 & 1,776 \\
	9   & 0,630   & & 19 & 1,993 \\
	10  & 0,707   & & 20 & 2,236
\end{tabular}
\end{center}

%%\emph{dB(W) är ett annat absolut mått.}

\noindent
Effektnivåer över en belastning kan också uttryckas i dB(W), där (W)
står för watt.
Referenseffekten är då \qty{1}{\watt}, dvs. 0\,dB(W).
Liksom med dB(m) anges impedansen i den belastning, som effekten utvecklas över.

\subsection{Ändring uttryckt i dB vid förstärkande eller dämpande anordningar kopplade i serie}
\harecsection{\harec{a}{1.9.3}{1.9.3}}

Ett räkneexempel på effektändringar:
Fråga:
Vi har en enkel sändaranläggning med ett drivsteg som matas med \qty{10}{\watt}
HF.
Drivsteget förstärker med \qty{6}{\decibel}.
Vidare har vi ett effektslutsteg som förstärker med \qty{10}{\decibel}.
Antennkabeln dämpar med \qty{1}{\decibel}.

Fråga: Med vilken effekt matas själva antennen?

Svar: (två sätt att lösa uppgiften)
\begin{enumerate}
\item Drivsteget förstärker fyra gånger, slutsteget förstärker tio gånger och
kabeln dämpar \(1/1,25 = 0,8\) gånger. Antennen matas då med
\(10 \cdot 4 \cdot 10 \cdot 0,8 = \qty{320}{\watt}\).
\item Drivstegets \qty{6}{\decibel} plus slutstegets \qty{10}{\decibel} minus antennkabelns \qty{1}{\decibel} = \qty{15}{\decibel}.
\qty{15}{\decibel} är \(10 + \qty{5}{\decibel}\) dvs. \(10 \cdot 3,2 = 32\ ggr\). Antennen matas med
\(\qty{10}{\watt} \cdot 32 = \qty{320}{\watt}\).
\end{enumerate}

\subsection{Impedansanpassning}
\harecsection{\harec{a}{1.9.4}{1.9.4}}
\index{impedansanpassning}
\index{impedans!anpassning}

Impedansanpassning är av stor betydelse inom kommunikationstekniken.
Normalt vill man nämligen överföra mesta möjliga effekt från energikällan
(t.ex. sändaren) till förbrukaren (t.ex. antennen).

Varje spänningskälla har en inre resistans \(R_i\). Det innebär som först att
källan inte kan avge oändligt stor ström.
För att förenkla det hela antar vi nu att en sändare med den inre resistansen
\(R_i\) ansluts direkt till en antenn med resistansen \(R_a\).

Målet med anpassningen är att finna det optimala förhållandet mellan
sändarresistansen och antennresistansen för att kunna överföra maximal effekt.
Vi har de två ytterlighetsfallen obelastad sändare respektive kortsluten
sändare.
Sändarens elektromotoriska kraft (EMK) betecknas som \(E\ [V]\) och sändarens
utspänning som \(U\ [V]\).

Fall 1.
En obelastad sändare avger ingen ström när ingen antenn eller en med oändligt
stor resistans har anslutits.
Alltså vid obelastad sändare:
%%
\[
\begin{array}{lllll}
R_a = \infty & & I = 0 & & U = E
\end{array}
\]
%%
Fall 2.
När sändarutgången är kortsluten, det vill säga belastningen
(antennresistansen) är noll ohm, avger sändaren en ström som beror av EMK och
inre resistans.
Eftersom sändarutgången är kortsluten är utspänningen \(U\) noll.
Alltså vid kortsluten sändare:
%%
\[
\begin{array}{lllll}
R_a = 0 & & I = \dfrac{E}{R_i} & & U = O
\end{array}
\]
%%
I båda ytterlighetsfallen är den effekt som omsätts i \(R_a\) lika med noll.
För att få ut någon effekt måste man alltså söka ett värde på \(R_a\) som
ligger mellan ytterlighetsvärdena.

Enligt formeln för spänningsdelare är utspänningen
%%
\[U = E \cdot \dfrac{R_a}{R_a+R_i}\]
%%
Formeln för uteffektens effektivvärde är
%%
\[P_{ut} = \dfrac{U^2}{R_a}\]
%%
Efter insättning får man
%%
\[P_{ut} = \dfrac{E^2 \cdot R_a}{(R_a + R_i)^2}\]
%%
För att finna det optimala förhållandet mellan \(R_i\) och \(R_a\), det vill
säga när \(R_a\) tar upp maximal effekt, måste man differentiera formeln med
\(d\ P_a/d\ R_a\), men vi hoppar över denna utflykt i matematiken.

I stället konstaterar vi helt enkelt att \emph{maximal effektöverföring sker när
\(R_i = R_a\)}.

\subsection{Förhållandet mellan in- och uteffekt uttryckt som procent verkningsgrad}
\harecsection{\harec{a}{1.9.5}{1.9.5}}
\index{verkningsgrad}
\label{verkningsgrad}

Antag att en antennkabel har en effektförlust av \qty{1}{\decibel}.
Det innebär en effektdämpning av 1,25 gånger, det vill säga 0,8.
Nu matar vi in \qty{10}{\watt} i kabeln och får alltså ut \qty{8}{\watt}.
Hur stor verkningsgrad har kabeln uttryckt i procent?
Lösning:
%%
\[\eta = \dfrac{8}{10} \cdot 100 = 80\ \%\]
%%

%
%
% Kapitel 2 Komponenter
\chapter{Komponenter}
\label{ch:komponenter}
\index{komponenter}

Komponenter är samlingsnamnet för de delar som tillsammans eller var för sig
bestämmer hur spänning och ström påverkar funktionen i en elektrisk eller
elektronisk apparat.
För att kunna förstå de enskilda komponenterna funktion är det oftast nödvändigt
att studera deras uppbyggnad.
% Avsnitt 2.1 Resistorn
\chapter{Komponenter}
\label{komponenter}

\section{Resistorn}
\harecsection{\harec{a}{2.1}{2.1}}
\index{resistor}
\index{resistans}
\label{resistorn}

\subsection{Allmänt}

Strömkretsar består av komponenter med olika egenskaper.
Den vanligaste egenskapen, åtminstone i likströmskretsar, är resistansen.
För att få avsedd funktion, så anpassar man resistansen i komponenterna.

\textbf{Exempel:} En krets med strömkälla, lampa, kopplingsledningar och smältsäkring.
Kopplingsledningarna mellan komponenterna bör ha låg resistans och därför lågt
spänningsfall (små förluster).
Lampan ska däremot ha hög resistans och därmed höga förluster för att kunna bli
het och lysa.
Smältsäkringen ska skydda ledningarna från för hög ström.
Säkringen ges därför en resistans som gör att den smälter när strömmen
överstiger ett tillåtet värde.

Som hjälpmedel för att fördela spänningar och strömmar i en krets används
en komponenttyp kallad \emph{resistor}.
Dess utmärkande egenskap är \emph{resistans} (eng. \emph{resistance}) --
även kallad ohmskt motstånd.

\subsection{Enheten ohm}
\harecsection{\harec{a}{2.1.1}{2.1.1}}
\index{ohm (\unit{\ohm})}
\index{enheter!ohm (\unit{\ohm})}
\index{symbol!\(R\) resistans}
\label{enheten_ohm}

%%(Se även kapitel \ssaref{ellära}.)

Resistansen mellan två punkter i en strömkrets är 1~ohm som även skrives
\qty{1}{\ohm} (uttalas ''en åm''), när spänningen \qty{1}{\volt} mellan
punkterna gör att en ström av \qty{1}{\ampere} (en ampere) flyter i kretsen.

Inom elektroniken används höga resistansvärden och därför även följande
multipler av enheten

\begin{center}
\begin{tabular}{lll}
  1 kiloohm & (\qty{1}{\kilo\ohm}) & = \(10^3\) ohm \\
  1 megaohm & (\qty{1}{\mega\ohm}) & = \(10^6\) ohm \\
\end{tabular}
\end{center}

\subsection{Resistans i strömledare}
\harecsection{\harec{a}{2.1.2}{2.1.2}}
\index{resistivitet}
\index{specifik resistans}
\index{symbol!\(\rho\) resitivitet}

För att bestämma resistansen i exempelvis en tråd, behöver man veta dess 
resistivitet, tvärsnittsyta, längd och temperatur.

\emph{Resistivitet} (eng. \emph{resitivity}) är ett materials
strömledningsegenskaper.
Ett annat namn för resistivitet är \emph{specifik resistans}.
Symbolen för resistivitet är \(\rho\) (uttalas ''rå'').
Formeln for resistivitet är:
%%
\[\rho = \dfrac{R A}{l}\qquad \left[\dfrac{ohm \cdot mm^2}{m}\right]\]
%%
där resistansen \(R\) på en längd \(l\) av en strömledare med en
genomsnittsarea \(A\) (som oftast anges i kvadratmillimeter).

Resitiviteten för material finns ofta att finna i tabeller i formelsamlingar.
I tabellen~\ssaref{table:metaller} finns ett antal vanliga metallers resitivitet
angivna.

Följande formel gäller för beräkning av resistansen i en strömledare med linjär
ström/spänningskaraktär.
%%
\[\begin{array}{c}
    R = \rho \dfrac{l}{A} \qquad \left[\rho = \frac{\unit{\ohm} \cdot A}{m} \right] \qquad l=\text{meter}; A=\unit{\milli\metre\squared}
\end{array}\]
%%
\noindent
\textbf{Exempel}

\(l = \qty{4}{\metre}\) koppartråd

\(A = \qty{2}{\milli\metre\squared}\)

\(\rho \text{ (koppar)} = 0,017\)

\[\begin{array}{c}
R = \dfrac{\rho \cdot l}{A} \qquad R = \dfrac{0,017 \cdot 4}{2} = \qty{0,034}{\ohm}
\end{array}\]

\noindent
\textbf{Not.} \emph{Förväxla inte A [tvärsnittsytan] i denna formel med enheten ampere.}

\subsection{Resistiva material}

Resistorer kan utföras med olika typer av resistiva material, vilket bestämmer
användningsområdet.
En resistor, vars resistans är oberoende av ström, spänning och annan yttre
påverkan, till exempel temperatur och ljus, sägs ha linjär karaktär.
Om resistansen däremot beror av yttre påverkan sägs resistorn ha olinjär
karaktär.
Man skiljer mellan tre huvudgrupper av resistiva material.
Det kan vara en kropp av pressat kol eller ett ledande ytskikt på ett isolerande
underlag eller en metalltråd på en isolerande stomme.
På senare tid har det tillkommit resistornät med integrerade resistorer, det
vill säga flera resistorer av resistiva skikt på ett gemensamt isolerande
underlag.
Här beskrivs i korthet olika typer av resistorer.

\subsection{Utförandeformer}

Resistorer kan utföras med fast eller ställbart resistansvärde.
Här följer först en översikt över resistorer med olika resistiva material och
fast resistansvärde.

\subsection{Fasta resistorer med linjär karaktär}
\label{fasta_resistorer_linjära}

\subsubsection{Massaresistor}
\index{massaresistor}
\index{resistor!massa-}

Det resistiva materialet består av kolmassa med bindemedel (kolkomposit).
Massan är bakad till en stav eller ett rör.
Anslutningsledningarna är inbakade i materialet.
\emph{Massaresistorer} är lämpliga för lik- och växelströmskretsar med
låga krav på temperaturberoende och egenbrus.
Den homogena kroppen gör att egeninduktansen är låg.
Å andra sidan uppstår vid höga frekvenser en skineffekt, det vill säga en
strömkoncentration vid ytan, som medför viss resistansökning.

\subsubsection{Kolfilmsresistor}
\index{kolfilmresistor}
\index{resistor!kolfilm-}

Det resistiva materialet består av ett kolskikt, som genom förångning överförts
till ett keramiskt rör.
Resistansen bestäms av tjockleken på skiktet samt av spiralformade spår i
detta.
Genom spiraliseringen tillförs en induktans, som dock i någon mån uppvägs av
egenkapacitansen.

\subsubsection{Metallfilmresistor}
\index{metallfilmresistor}
\index{resistor!metallfilm-}

I denna typ är kolfilmen ersatt av ett metallskikt.
Eftersom egenkapacitansen är liten är typen lämpad för höga frekvenser.

\subsubsection{Tjockfilmsresistor}
\index{tjockfilmsresistor}
\index{resistor!tjockfilm-}

Det resistiva materialet består av en film av bland annat metalloxid, som
screentrycks på ett keramiskt underlag.
Typen har god tålighet mot pulser och höga temperaturer, men har relativt högt
egenbrus.
Ytmonterade resistorer är oftast tillverkade av tjockfilm.

\subsubsection{Tunnfilmsresistor}
\index{tunnfilmresistor}
\index{resistor!tunnfilm-}

Det resistiva materialet består av en tunn metallfilm, som genom förångning
överförts till ett underlag av glas eller keramik. Denna resistortyp har över
lag god stabilitet och används ofta i apparater med hög precision.
Egenskaperna vid höga frekvenser är dock inte så goda.

\subsubsection{Metalloxidresistor}
\index{metalloxidresistor}
\index{resistor!metalloxid-}

Denna resistortyp har ett spiralformat skikt av metalloxid.
Temperatur- och spänningsberoendet är måttligt.
Tåligheten mot pulser och höga temperaturer är stor.
Typen kan i någon mån ersätta trådlindade resistorer.

\subsubsection{Resistornät}
\index{resistornät}
\index{resistor!-nät}

Resistornät (integrerade resistorer) består av flera resistiva skikt på ett
gemensamt isolerande underlag, det vill säga en liknande teknik som för tjock- 
och tunnfilmsresistorer.

\subsubsection{Trådlindad resistor}
\index{trådlindad resistor}
\index{resistor!trådlindad}

Det resistiva materialet är en metalltråd lindad på en stomme som tål hög
temperatur.
Stommen kan vara av keramik, glas eller liknande.
Tåligheten mot pulser och höga temperaturer är stor.

\newpage % layout

\subsection{Fasta resistorer med olinjär karaktär}
\harecsection{\harec{a}{2.1.3}{2.1.3}}
\index{olinjära resistorer}
\index{resistor!olinjär}
\label{fasta_resistorer_olinjära}

Vanligast är att materialet i resistorer har linjär ström- och
spänningskaraktär, men det finns även sådana med olinjär karaktär.
I resistorer med olinjär karaktär är det ingående materialet av halvledartyp.

\subsubsection{Spänningsberoende resistor -- Voltage Dependent Resistor (VDR)}
\index{spänningsberoende resistor}
\index{resistor!spänningsberoende}
\index{VDR}
\index{resistor!VDR}

Linjära resistorer påverkas knappast av den pålagda spänningen.
Resistorer av kiselkarbid har däremot en hög resistans vid låg spänning och
omvänt en låg resistans vid hög spänning.
Sådana spänningsberoende resistorer används till exempel för begränsning av
spänningstoppar.

\subsubsection{Ljusberoende resistor, fotoresistor -- Light Dependent Resistor (LDR)}
\index{ljusberoende resistor}
\index{resistor!ljusberoende}
\index{fotoresistor}
\index{resistor!foto-}
\index{LDR}
\index{resistor!LDR}

Ledningsförmågan i halvledare påverkas inte bara av värme utan även av ljus.
Halvledare av germanium och särskilt sammansatta halvledare av kadmiumoxid,
blysulfid och indiumantimonid har särskilt stor ljuskänslighet. Kadmiumsulfid
är känsligast för synligt ljus medan andra material är känsligast i det
infraröda området.

\subsubsection{Magnetfältberoende resistor (fältplatta)}
\index{magnetfältberoende resistor}
\index{resistor!magnetfältberoende}
\index{hallresistor}
\index{resistor!Hall-}

Resistansen ökar med längden på strömledaren. Denna egenskap används i
\emph{magnetfältsberoende fältplattor} som utnyttjar \emph{halleffekten}, även
kända som \emph{hallresistor}. En sådan består av en keramisk bärarplatta med
en yta av indiumantimonid.
I ytan är ytterst smala parallella metallbanor inlagda på ett avstånd av någon
\unit{\micro\metre}.
Normalt går strömmen kortaste vägen tvärs över banorna, men när ett magnetfält
träffar vinkelrätt mot plattans yta avlänkas elektronerna.
De får då längre väg över till nästa metallbana och den totala resistansen
ökar.

\subsubsection{Temperaturberoende resistor}
\index{temperaturberoende resistor}
\index{resistor!temperaturberoende}
\index{NTC}
\index{resistor!NTC}
\index{PTC}
\index{resistor!PTC}

Se nedan om NTC och PTC i resistorer.

\subsection{Temperaturkoefficienten för resistorer}
\index{temperaturkoefficient i resistor}
\index{resistor!temperaturkoefficient}
\index{NTC}
\index{resistor!NTC}
\index{PTC}
\index{resistor!PTC}
\label{resistor_temperaturkoefficient}

Resistansen i ingående material påverkas av temperaturen, varvid det skiljer
mellan materialen.

Amorft kol och de flesta halvledande material leder bättre när de är varma -- de
har en negativ temperaturkoefficient (NTC). Sådana material finns till exempel i
dioder och transistorer.

Däremot leder metaller och speciella halvledarmaterial bättre när de är kalla
-- de har en positiv temperaturkoefficient (PTC). Glödtråden i glödlampor och
elektronrör är resistorer med positiv temperaturkoefficient (PTC).

I vissa metallegeringar kan resistansen däremot vara nästan konstant vid
varierande temperatur.
Ett exempel är konstantan, som är en legering mellan koppar, nickel och mangan.

Alla material har en temperaturkoefficient, som anger hur mycket resistansen
ändras per grad. Resistansen vid någon annan temperatur kan därför beräknas med
följande formel, där man sätter in begynnelsetemperaturen [\(\vartheta\)]
(\unit{\degreeCelsius}), temperaturändringen [\(\Delta \vartheta\)] och
temperaturkoefficienten [\(\alpha\)].
%%
\[R_{varm} = R_{kall} \pm \alpha \cdot \Delta \vartheta \cdot R_{kall}\]
%%
Resistansändringen är ledet
%%
\[ \Delta R = \pm \alpha \cdot \Delta \vartheta \cdot R_{kall}\]
%%
Temperaturkoefficienten kan vara positiv (PTC) eller negativ (NTC).
I principscheman har PTC- respektive NTC-resistorer symboler som i bild~\ssaref{fig:BildII2-1}.

\mediumfig{images/cropped_pdfs/bild_2_2-01.pdf}{Schemasymboler för resistorer}{fig:BildII2-1}

\subsection{Variabla resistorer}
\index{variabla resistorer}
\index{resistor!variabel}

En resistor kan även utföras med variabelt resistansvärde. Då används endast
den andel av det resistiva materialet som finns mellan en resistors ena ände
och ett uttag någonstans mellan ändarna. En sådan anordning kallas för reostat.
Om en variabel resistor används som spänningsdelare kallas den för
potentiometer.

I en potentiometer används dels hela resistansen mellan ändpunkterna och dels
andelen mellan uttaget och någon av ändpunkterna.
Uttagets mekaniska utförande beror oftast av hur bekvämt inställningen ska
kunna ske.
En potentiometer, där det resistiva materialet är lagt på en cirkulär bana och
uttaget är fäst vid en axel i banans centrum, medger enkel inställning med
mejsel, ratt eller liknande.
Ett enklare slags uttag är en släpkontakt eller ett spännband som kan flyttas
utmed en stavformad resistor.

\subsubsection{Resistiva material i variabla resistorer}

Banan i en variabel resistor består i princip av liknande resistiva material som
i en fast resistor.
Billigast och enklast är en bana av kol, som är tryckt på ett enkelt underlag.
Nackdelar är låg effekttålighet, dålig upplösning och linjäritet, högt brus och
kort livslängd. Fördelen är lågt pris.
Bättre än en kolbana är en bana av kolkomposit, det vill säga kolpulver med
bindemedel, som är tryckt på ett underlag.
Nackdel är högre pris och låg effekttålighet, medan fördelarna är god
upplösning, lågt brus och lång livslängd.
Vill man ha god effekttålighet och temperaturstabilitet, utöver kolkompositens
egenskaper, så erbjuder en bana av cermet sådana fördelar.
En cermetbana består av en blandning av metaller och keramik, som trycks på ett
underlag.
Trådlindad bana har främst god tålighet mot hög effekt.
Tålighet vid hög ström genom uttaget är en annan fördel.

\subsubsection{Linjära och olinjära potentiometrar}

En potentiometers resistansändring som funktion av uttagets rörelseväg utmed
resistansbanan kan beskrivas med en kurva.
Kurvformen kan utföras linjär, logaritmisk, eller på något annat sätt.
Olinjära kurvor består oftast av en följd av linjära segment, som tillsammans
någorlunda motsvarar den önskade olinjära formen.

\subsection{Effektutveckling i resistorer}
\harecsection{\harec{a}{2.1.4}{2.1.4}}
\index{effektutveckling i resistorer}
\index{resistor!effektutveckling}

I resistorer utvecklas värme av den ström som flyter igenom dem.
Värmeutvecklingen sker enligt Joules lag, som återges i kapitel~\ssaref{ellära}.
Hur mycket effekt i form av värme som strålas ut från resistorn beror på
storleken på dess yta och egentemperatur samt på omgivningens temperatur.
Det finns en övre gräns för hur mycket värme det ingående materialet tål innan
det förstörs och eventuellt fattar eld.
En resistors effekttålighet framgår i vissa fall av påstämplade värden.
I övriga fall är man hänvisad till kataloguppgifter eller en bedömning, som
eventuellt kan grundas på höljets utseende och dimensioner.

\subsection{Standardiserade komponentvärden}
\index{resistor!standardiserade värden}

Resistorer tillverkas vanligen med standardiserade värden från en talserie.

\subsection{Märkning av resistorer}
\index{färgmärkning}
\index{färgmärkning!resistor}
\index{färgmärkning!kapacitans}
\label{färgmärkning}

Resistorer märks med hjälp av siffror och bokstäver eller med en färgkod så att
resistorns huvuddata kan avläsas.
Ofta finns märkningen förklarad i komponentleverantörernas kataloger.

\subsubsection{Färgmärkning av resistorer}

Ett vanligt sätt att märka resistorer är genom att ha färger på ringar
runt kroppen. Detta var vanligare på den tiden man hade hålmonterade
resistorer och i dag med ytmonterade brukar man i stället använda
siffror tryckta på motståndskroppen.

Färgkoden är dock bra att känna till för att kunna identifiera resistorer
och ibland även andra komponenter.

Färgkoden består av tre olika scheman, de kan finnas 4, 5 eller 6 band runt
komponenten.
Första banden ger värdet hos komponenten och det två sista banden har särskild
betydelse, det näst sista är en multiplikator och det sista bandet är
toleransen.
Ofta är det sista bandet också tryckt med en viss distans från de andra banden.

Om färgkoden har n stycken band kan man beskriva den med tabell~\ssaref{tab:rcolors} nedan.

\begin{table}[H]
\begin{tabular}{lrrr}
	\textbf{Färg}    & \textbf{Sifferkod} &     \textbf{Multiplikator} 
	&     \textbf{Tolerans} \\ \hline \hline
	Svart   &         0 &    $10^0$ &              \\ \hline
	Brun    &         1 &    $10^1$ &    $\pm 1\%$ \\ \hline
	Röd     &         2 &    $10^2$ &    $\pm 2\%$ \\ \hline
	Orange  &         3 &    $10^3$ &              \\ \hline
	Gul     &         4 &    $10^4$ &              \\ \hline
	Grön    &         5 &    $10^5$ &  $\pm 0,5\%$ \\ \hline
	Blå     &         6 &    $10^6$ & $\pm 0,25\%$ \\ \hline
	Violett &         7 &    $10^7$ &  $\pm 0,1\%$ \\ \hline
	Grå     &         8 &    $10^8$ & $\pm 0,05\%$ \\ \hline
	Vit     &         9 &    $10^9$ &              \\ \hline
	Guld    &           & $10^{-1}$ &    $\pm 5\%$ \\ \hline
	Silver  &           & $10^{-2}$ &   $\pm 10\%$ \\ \hline
	Saknas  &           &           &   $\pm 20\%$ \\ \hline
\end{tabular}
\caption{Färgmärkning av resistorer och deras betydelse}
\label{tab:rcolors}
\end{table}

\textbf{Exempel:} Gul, violett, orange, silver. Första är siffran 4, nästa är siffran 7 och den tredje orange är multiplikatorn $10^3$. Resultatet blir då 47\,kOhm. Till sist har vi toleransen som är silver och innebär $\pm 10\%$.

% Avsnitt 2.2 Kondensatorn
\section{Kondensatorn}
\harecsection{\harec{a}{2.2}{2.2}}
\index{kondensator}

\subsection{Allmänt}
\label{kondensator_allmänt}

Så snart det finns en elektrisk potentialskillnad -- en spänning -- mellan två
kroppar uppstår ett elektriskt kraftfält mellan dem.
Ett sådant fält lagrar elektrisk energi.
Kropparna måste då isoleras från varandra.

Elektrisk energi lagras mellan olika delar av en strömkrets, även om de inte är
direkt avsedda för det.
Särskilt vid mycket höga frekvenser har detta stor betydelse för utformningen av
en strömkrets.
Vid låga frekvenser och likström har kretsens utformning däremot mindre
inverkan.
Då behövs i stället särskilda anordningar för ta upp eller avge energi på
önskade ställen i strömkretsen.

En sådan anordning kallas kondensator. Den består i princip av två band eller
plattor med anslutningsledningar samt ett isolerande skikt -- dielektrikum --
däremellan.

Kapacitansen är näst efter resistansen den vanligaste egenskapen i en
strömkrets.

\subsection{Kapacitans}
\harecsection{\harec{a}{2.2.1}{2.2.1}}
\index{kapacitans}
\index{dielektricitet}
\index{symbol!\(C\) kapacitans}
\index{symbol!\(\epsilon_0\) dielektricitetskonstanten}
\index{symbol!\(\epsilon_r\) relativa dielektricitetskonstanten}

Förmågan att lagra elektrisk energi (elektrisk laddning) kallas
\emph{kapacitans} (eng. \emph{capacitance}).
Ordet kommer från latinets capax, som betyder rymlig eller duglig.
Kapacitansen betecknas i formler med bokstaven C.

En kondensators kapacitans bestäms av ytan på kondensatorns plattor,
avståndet mellan dessa ytor, den absoluta dielektricitetskonstanten
\(\epsilon_0\) och den relativa dielektricitetskonstanten \(\epsilon_r\), som är
den faktor kapacitansen ökar med när dielektrikum utgörs av annat än vakuum.

Det isolerande materialet mellan plattorna kallas för \emph{dielektrikum}
(eng. \emph{dielectric}) och egenskaperna hos detta material påverkar
kondensatorns kapacitans.
Materialets egenskap kallas för \emph{dielektricitet}
(eng. \emph{dielectric property}) och kännetecknas av dess
\emph{dielektricitetskonstant} (eng. \emph{dielectric constant}).
Dielektricitetskonstanten för vakuum är definierad som:
%%
\[\epsilon_0 = \dfrac{1}{c_0^2\mu_0} \approx 8,854187 \cdot 10^{-12}\]
%%
Den \emph{relativa dielektricitetskonstanten} \(\epsilon_r\) varierar för olika
material och dess värde går att finna i tabeller.
Genom att multiplicera den relativa dielektricitetskonstanten för ett material
med dielektricitetskonstanten för vakuum \(\epsilon_0\) får man den
\emph{absoluta dielektricitetskonstanten} \(\epsilon\).
%%
\[\epsilon = \epsilon_0\epsilon_r\]
%%

% \newpage % layout

\subsection{Kapacitans, dimension och dielektrikum}
\harecsection{\harec{a}{2.2.3}{2.2.3}}
\index{dielektrum}
\index{kapacitans!dimension}
\index{kapacitans!dielektrum}

Kapacitansen är proportionell mot kondensatorplattornas yta och omvänt 
proportionell mot plattavståndet.

Följande formler gäller för kapacitansen i en enkel kondensator med två
plattor. När en kondensator är uppbyggd av n stycken plattor, ökar kapacitansen
med faktorn (n-1).
Med vakuum som dielektrikum gäller
%%
\[C = \varepsilon _0 \dfrac{A}{d}\]
%%
Med ett godtyckligt dielektrikum gäller
%%
\[C = \varepsilon _0 \cdot \varepsilon _r \frac{A}{d}\]
\[C\ [\textit{farad}]\quad A\ [\unit{\metre\squared}]\qquad d\ [\unit{\metre}]\quad \varepsilon\ [\dfrac{F}{m}]\]

\subsection{Enheten farad}
\harecsection{\harec{a}{2.2.2}{2.2.2}}
\index{farad (F)}
\index{enheter!farad (F)}

Kapacitans är elektricitetsmängden per volt där måttenheten är \emph{farad} [F].
Eftersom denna enhet är mycket stor används inom elektroniken oftast bråkdelar
av den.
Se bilaga~\ssaref{app:mattenheter}.

\begin{center}
\begin{tabular}{ll}
1~mikrofarad & \((\qty{1}{\micro\farad}) = \qty{e-6}{\farad}\) \\
1~nanofarad & \((\qty{1}{\nano\farad}) = \qty{e-9}{\farad}\) \\
1~pikofarad & \((\qty{1}{\pico\farad}) = \qty{e-12}{\farad}\) \\
\end{tabular}
\end{center}

\subsection{Kondensatorn i likströmskretsen}
\index{laddningsmängd}
\index{symbol!\(Q\) laddning}

En kondensator i en likströmskrets har alltid samma polaritet.
Därvid förhåller sig kondensatorns polspänning \(U\) till dess
\emph{laddningsmängd} \(Q\) och kapacitans \(C\) enligt sambandet
%%
\[U = \frac{Q}{C} \]
\[U\ [\unit{\volt}]\quad Q\ [As]\quad C\ [\unit{\farad}]\]
%%
Laddningsmängd har enheten ampere gånger sekund och är alltså ett mått på hur
många laddningar som har lagrats.

När en ansluten spänningskälla har högre spänning än kondensatorn flyter en 
ström till kondensatorn och laddar upp den. Ju högre spänningen är desto större 
blir laddningen. Ju kortare uppladdningstiden är desto högre effekt
utvecklas under den tiden.

Efter att kondensatorn har uppladdats så flyter inte längre någon ström genom
den. Denna egenskap gör att man med kondensatorns hjälp kan spärra likspänning
och enbart släppa igenom växelspänning. Det används när man till exempel har
överlagrad likspänning som man vill ta bort från en signal eller av annat skäl
inte önskar likspänning gå vidare i en krets.

När en uppladdad kondensator ansluts till en krets med lägre spänning urladdas 
kondensatorn till kretsen. Ju kortare urladdningstiden är desto högre
effekt utvecklas under den tiden.

Laddningen i en kondensator kan resultera i en hög polspänning.
Om kondensatorns kapacitans är stor kan laddningsmängden bli betydande.
Varning för elektriska stötar och brännskador!

\subsection{Kondensatorn i växelströmskretsen}

I en likströmskrets förhåller sig kondensatorns polspänning till
laddningsmängden. Men i en växelströmskrets växlar spänningen och polariteten 
ständigt och därmed växlar även kondensatorns laddning och polaritet.

\textbf{Not:} \emph{Vissa kondensatortyper kan inte användas i rena
  växelströmskretsar.}

\textbf{Försök:} En glödlampa och en kondensator kopplas i serie
med varandra och ansluts till en växelströmskrets. Med lämpligt
valda värden på komponenterna kommer lampan att lysa.

Det visar att en kondensator inte hindrar elektronflödet i en växelströmskrets.
Man brukar säga att kondensatorn ''släpper igenom växelström'', men i själva 
verket är det så att laddningar pendlar mellan kondensatorns plattor genom den
strömkrets som kondensatorn är ansluten till.

Använd för säkerhets skull låg spänning exempelvis från en ringledningstransformator!

\subsection{Kapacitiv reaktans}
\harecsection{\harec{a}{2.2.4}{2.2.4}}
\index{kapacitiv reaktans}
\index{kapacitans!reaktans}
\index{kondensator!reaktans}
\index{reaktans!kapacitiv}
\index{symbol!\(X_c\) kapacitiv reaktans}
\label{kapacitiv_reaktans}

Strömstyrkan i en växelströmskrets beror bland annat på hur stor kondensatorns
kapacitans är, det vill säga på dess \emph{kapacitiva reaktans} (eng.
\emph{capacitive reactance}) \(X_c\).

Ordet reaktans kommer från latinets \emph{re} (åter) \emph{agere} (verka).

Större kapacitans innebär större förmåga att ta upp elektrisk laddning och ger
därmed en lägre reaktans. Resultatet blir ett kraftigare elektronflöde.
En mindre kapacitans innebär ett svagare elektronflöde.
%%
\[X_C = \dfrac{1}{2\pi fC}\quad \text{eller}\quad X_C = \dfrac{1}{\omega C}\]
\[X_C\ [\unit{\ohm}]\quad f\ [\unit{\hertz}]\quad C\ [\unit{\farad}]\]
%%
\begin{exempelbox}
\[C = \qty{10}{\micro\farad}\quad f = \qty{50}{\hertz}\quad X_c = ?\]
\tcblower
\[X_c = \dfrac{1}{2\pi fC} = \dfrac{1}{2\pi 50 \cdot 10 \cdot 10^{-6}} = \qty{318,3}{\ohm}\]
\end{exempelbox}
\begin{exempelbox}
\[C = \qty{10}{\micro\farad}\quad f = \qty{5}{\kilo\hertz}\quad X_c = ?\]
\tcblower
\[X_c = \dfrac{1}{2\pi fC} = \dfrac{1}{2\pi 5 \cdot 10^3 \cdot 10 \cdot 10^{-6}}
= \qty{3,183}{\ohm}\]
\end{exempelbox}
En kondensators reaktans är således omvänt proportionell mot dess kapacitans
och frekvensen i kretsen.
Jämför detta med en induktor (spole) där reaktansen är proportionell mot frekvensen.

När en ström flyter genom en resistor uppstår värmeförluster.
När ström flyter genom en ideal reaktans -- en induktor eller en kondensator --
uppstår däremot inga värmeförluster.

\subsection{Fasförskjutning i en kondensator}
\harecsection{\harec{a}{2.2.5}{2.2.5}}
\index{fasförskjutning!kondensator}
\index{kondensator!fasförskjutning}

Med fasförskjutning menas här den tidsmässiga förskjutningen mellan ström- och
spänningsförloppen.
I en kondensator når nämligen strömmen inte sitt toppvärde samtidigt som
spänningen.
I en ideal kondensator är spänningen fasförskjuten \ang{90} efter strömmen.

\subsection{Förlustvinkel}
\index{förlustvinkel}

I praktiken är fasförskjutningen i en kondensator något mindre än \ang{90} på
grund av att laddning läcker igenom dielektrikum.
Man talar om en förlustvinkel.
Läckningen kan ses som en resistor kopplad parallellt över kondensatorn.

\subsection{Läckström}
\index{läckström}
\index{kondensator!läckström}

Med sitt extremt tunna dielektrikum har elektolytkondensatorn en mycket högre
kapaciatans än andra sorters kondensatorer, men den har också några nackdelar.
Bland annat kan den normalt endast användas med likspänning, den har hög 
förlustfaktor på grund av läckström och den värme som utvecklas av läckströmmen 
skapar övertryck till följd av gasbildning.

\subsection{Utförandeformer}
\index{glimmer}

Kondensatorer kan utföras med fast kapacitansvärde.
Dielektrikum består då av ett skikt av glimmer, impregnerat papper eller liknande.
Glimmer (eng. \emph{mica}) är ett mineral som lätt kan spaltas i tunna skivor.
Glimmer har mycket goda isolations- och värmeledningsegenskaper.

Kondensatorer kan även utföras med variabelt kapacitansvärde.
Dielektrikum består då oftast av luft, men kan även bestå av ett fast material.
Bild~\ssaref{fig:BildII2-2}.

\subsubsection{Fasta kondensatorer}

\mediumfig{images/cropped_pdfs/bild_2_2-02.pdf}{Schemasymboler för kondensatorer}{fig:BildII2-2}

Kondensatorer har oftast namn efter utförande och det dielektriska materialet.
\begin{description}
\item[Pappers- och plastkondensatorer]
Plattorna i dessa typer består av aluminiumremsor med anslutningstrådar.
Däremellan finns en pap\-pe\-rs- re\-spe\-k\-ti\-ve plastremsa som dielektrikum.
För att spara plats rullas det hela ihop och skyddas med en plastingjutning.

\item[Keramiska kondensatorer]
I keramiska kondensatorer består dielektrikum av något keramiskt material.
På ömse sidor om detta sätts en metallbeläggning med anslutningstrådar.

\item[Glimmerkondensatorer]
I denna kondensatortyp består dielektrikum av tunna glimmerskivor.

% \newpage % layout
\item[Elektrolytkondensatorer]
Elektrolytkondensatorer har elektroder av aluminium eller tantal där pluspolen
(anoden) ges ett mycket tunt oxidskikt.
Detta är inte ledande och fungerar som dielektrikum.
Mellan oxidskiktet och minuspolen (katoden) läggs en elektrolyt med låg
resistivitet.
\end{description}

\noindent
Elekrolytkondensatorer har särskilt högt kapacitansvärde. Till skillnad från
andra kondensatortyper är elektolytkondensatorer polariserade. Utom i ett
specialfall innebär det att polariteten på den pålagda spänningen inte får
kastas om.
Flera olika slags elektrolytkondensatorer finns, bland andra våta och torra
aluminiumelektrolytkondensatorer samt tantalelektrolytkondensatorer.

\subsubsection{Variabla kondensatorer}
Variabla kondensatorer har oftast sitt namn efter utförandeformen, som
vridkondensator och trimbar kondensator (trimmer).

\subsection{Temperaturkoefficient}

På liknande sätt som med resistorer påverkas kapacitansen i kondensatorer av
temperaturen.
Att sambandet mellan kapacitans och temperatur är viktigt förstås av att
temperaturkoefficienten i den frekvensbestämmande kapacitansen i en
oscillatorkrets är en av faktorerna för stabil frekvens.

Temperaturkoefficienten \(\alpha _c\) anger kapacitansändringen per grad temperaturändring.
Kapacitansändringen blir då
%%
\[\Delta C = \pm \alpha _c \cdot C_k \cdot \Delta\vartheta\]
%%
varvid \(C_k\) är kapacitansvärdet vid den lägre temperaturen (oftast
\qty{20}{\degreeCelsius}) och \(\Delta\vartheta\) är temperaturändringen i
kelvin.
Kelvin [K] är den normerade måttenheten för absolut temperatur.
En ändring med \qty{1}{\kelvin} motsvarar en ändring med \qty{1}{\degreeCelsius}.
Är \(\alpha _c\) positivt betyder det att kapacitansen ökar med ökande
temperatur.
Är \(\alpha _c\) negativt betyder det att kapacitansen minskar med ökande
temperatur.
En kondensator som är märkt med N 100 betyder
\(\alpha _c = -100 \cdot 10^{-6}\ 1/K\)

\subsection{Standardiserade komponentvärden}
\index{kondensator!standardiserade värden}

Kondensatorer tillverkas vanligen med standardiserade värden från en talserie.

\subsection{Märkning av kondensatorer}

Kondensatorer märks med hjälp av siffror och bokstäver eller med en färgkod så
att kondensatorns huvuddata kan avläsas.
Ofta finns märkningen förklarad i komponentleverantörernas kataloger.

Se exempelvis tabellen för resistorer i avsnitt~\ssaref{färgmärkning}.

% Avsnitt 2.3 Induktorn
\mediumfig{images/cropped_pdfs/bild_2_2-03a.pdf}{Försök 1 med induktion}{fig:BildII2-3a}
\smallfig{images/cropped_pdfs/bild_2_2-03b.pdf}{Försök 2 med induktion}{fig:BildII2-3b}

\section{Induktorn}
\harecsection{\harec{a}{2.3}{2.3}}
\index{induktor}

\subsection{Allmänt}
\label{induktor_allmänt}

När elektrisk ström flyter genom en ledare alstras ett magnetfält omkring den.
Så snart strömmens styrka eller riktning ändras uppstår en motsvarande så kallad
elektromotorisk kraft (EMK) som motverkar ändringen.
Kraften finns i magnetfältet i form av lagrad magnetisk energi.


\subsection{Självinduktion -- induktans}
\harecsection{\harec{a}{2.3.1}{2.3.1}}
\index{induktans}
\index{självinduktion}
\index{elektromotorisk kraft (EMK)}
\index{EMK}
\index{induktor}

Magnetfältets förmåga att alstra en motverkande EMK kallas
\emph{självinduktion} (eng. \emph{self inductance}) eller
\emph{induktans} (eng. \emph{inductance}).
%% k7per: Macro for latin?
Ordet induktans kommer från latinets \emph{inducere}, som betyder införa.

När en ledare som ingår i en sluten krets rör sig i ett magnetfält, kommer
en ström att flyta genom ledaren på grund av den EMK (spänning) som alstras.
Varje ändring av strömmen motverkas av det magnetfält som strömmen själv
alstrar.

När det uppstår självinduktion i en ledare kallas ledaren \emph{induktor}
(eng. \emph{inductor}).
Självinduktionen är jämnt utbredd över ledarens hela längd. När ett större
induktansvärde behövs på något särskilt ställe i strömkretsen, kan ledarens
längd ökas just där och lindas upp till en spole med lämplig form.
Hela spolen kallas då för induktor.

När ett motverkande magnetiskt fält alstras omkring en ledare genom att strömmen
i den ändras, påverkas kretsens egenskaper och därmed dess utformning på olika
sätt.
Vid snabba strömändringar, exempelvis vid hög frekvens, är motverkan större än
vid långsamma ändringar.
Vid konstant likström uppstår däremot ingen motverkan -- självinduktion.

Induktansen är efter resistansen och kapacitansen den vanligaste egenskapen i
en strömkrets.

\subsection{Försök med induktion}

%\mediumfig{images/cropped_pdfs/bild_2_2-03.pdf}{Försök med induktion}{fig:BildII2-3}

\paragraph{Försök 1:}
I bild~\ssaref{fig:BildII2-3a} är ett känsligt vridspoleinstrument kopplat
till en induktor.
Instrumentet bör ha noll på skalans mitt, så att strömriktningen syns.
En permanentmagnet används för att visa att självinduktion uppstår när
magneten förs fram och tillbaka genom induktorn.

Instrumentet ger utslag när magneten är i rörelse.
Utslaget blir större vid snabbare hastighetsändring.
Utslagsriktningen växlar när magneten förs in i respektive dras ut ur induktorn
-- det uppstår en växelström.

En växelspänning uppstår över induktorn även när den ingår i en strömkrets som
sluts och bryts -- alltså utan en magnet som rör sig.

\paragraph{Försök 2:}
I bild~\ssaref{fig:BildII2-3b} har permanentmagneten bytts mot ännu en
induktor.
Utöver den första induktorn, som vi nu kallar sekundärlindning, kallar vi den
nya induktorn för primärlindning.

När vi släpper ström genom primärlindningen alstrar den ett magnetfält.
Först är strömmen noll för att sedan ändras till ett högt värde och därefter
återgå till noll. Det blir en strömstöt.

Varje ändring alstrar en mot-EMK, som bygger upp ett magnetfält, först i en
riktning och sedan i den andra. I båda fallen passerar fältet genom båda
lindningarna. Fältet från primärlindningen inducerar en spänningsstöt i
sekundärlindningen. Stöten har en riktning när primärlindningens strömkrets
sluts och motsatt riktning när den bryts -- en växelspänning alstras.
När sekundärlindningen ingår i en sluten krets uppstår en växelström genom
sekundärlindningen.

\smallfig{images/cropped_pdfs/bild_2_2-03c.pdf}{Försök 3 med induktion}{fig:BildII2-3c}
\paragraph{Försök 3:}
I bild~\ssaref{fig:BildII2-3c} ställer vi oss frågan vad som händer när
primärlindningen i försök 2 ansluts till en växelspänning, till exempel
med nätfrekvensen \qty{50}{\hertz}.
Använd för säkerhets skull en skyddstransformator mellan nätet och lindningen!

I sekundärlindningen uppstår då spänningsstötar vars polaritet i detta fall
växlar 100 gånger per sekund.
Det uppstår alltså en växelspänning över sekundärlindningen och om denna ingår i
en sluten strömkrets uppstår det en motsvarande växelström.

\mediumfig{images/cropped_pdfs/bild_2_2-04.pdf}{Schemasymboler för induktorer}{fig:BildII2-4}

\subsection{Olika utföranden}
\index{drossel}

Bild~\ssaref{fig:BildII2-4} visar schemasymboler för ett antal vanliga induktorer.
Utöver dessa finns elektromagneter, drosslar, induktorer för resonanskretsar,
ramantenner och så vidare.

En drossel (eng. \emph{choke}) är en induktans, ofta lindad kring en 
magnetisk 
kärna, med uppgift att begränsa strömmen i en ledare.

% \newpage % layout

\subsection{Enheten henry (H)}
\harecsection{\harec{a}{2.3.2}{2.3.2}}
\index{henry (H)}
\index{enheter!henry (H)}
\index{symbol!\(L\) induktans}
\label{enheten_henry}

Måttenheten för självinduktion är \emph{henry} (\unit{\henry}).
1~henry (\qty{1}{\henry}) är självinduktionen i en induktor som alstrar en
motspänning av 1~volt vid en strömändring av 1~ampere under 1~sekund.
I formler betecknas induktans med symbolen L.
Sambandet är:
%%
\[\textit{volt} = \textit{henry} \cdot \textit{ampere}/\textit{sekund}\]
%%
\qty{1}{\henry} är en stor måttenhet.
För elektroniktillämpningar används därför ett mer hanterligt format.
Se bilaga~\ssaref{app:mattenheter}.

\noindent\textbf{Exempel:}

\begin{center}
\begin{tabular}{ll}
\qty{1}{\henry} & = \qty{1000}{\milli\henry} \\
\qty{1}{\milli\henry} & = \(1 \cdot 10^{-3}\)\,H \\
\qty{1}{\milli\henry} & = \qty{1000}{\micro\henry} \\
\qty{1}{\micro\henry} & = \(1 \cdot 10^{-3}\)\,mH = \(1 \cdot 10^{-6}\)\,H
\end{tabular}
\end{center}

\subsection{Hur induktansen påverkas}
\harecsection{\harec{a}{2.3.3}{2.3.3}}
\index{permeabilitet}
\index{relativa permeabiliteten}
\index{symbol!\(\mu_0\) permeabilitetskonstanten}
\index{symbol!\(\mu_r\) relativa permeabiliten}

Induktansen beror på induktorns mekaniska dimensioner, antalet lindningsvarv och
materialet i kärnan.

Induktansen i en cylindrisk induktor är proportionell mot tvärsnittsytan, omvänt
proportionell mot längden och proportionell mot kvadraten på lindningsvarvtalet.

Induktansen ökar om induktorn förses med en kärna av järn och minskar med en
kärna av omagnetisk, ledande metall, till exempel koppar, mässing eller
aluminium.

Precis som för kondensatorn har materialet i en induktors kärna betydelse,
då dess \emph{permeabilitet} kan anta olika värden. Den absoluta permeabiliteten
\(\mu\) brukar delas upp i permeabiliteten för vakuum \(\mu_0\) och den
\emph{relativa permeabiliteten} \(\mu_r\) som gives av
%%
\[\mu = \mu_0\mu_r\]
%%
Den relativa permeabiliteten går att hitta i tabeller och varierar med material.
Permeabiliteten för vakuum är definierad som
%%
\[\mu_0 = 4\pi 10^{-7} \approx 1,256637 \cdot 10^{-6}\]

\subsection{Induktiv reaktans}
\harecsection{\harec{a}{2.3.4}{2.3.4}}
\index{induktiv reaktans}
\index{reaktans!induktiv}
\index{symbol!\(X_L\) induktiv reaktans}
\label{induktiv_reaktans} 

Till skillnad från när en resistor ansluts till en spänning, så blir
strömökningen i en induktor fördröjd. Orsaken är att en induktor inte bara har
en resistans, vilken ju inte påverkas av strömvariationer, utan har även en
\emph{induktiv reaktans} (eng. \emph{inductive reactance}) \(X_L\).
Ordet reaktans kommer från latinets re (åter) agere (verka).

\emph{Reaktans} -- växelströmsmotstånd eller skenbart motstånd -- uppträder så
länge som strömmen genom induktorn ändras.
En induktor gör således också motstånd mot varje strömändring och detta motstånd
ökar med ökande ändringshastighet.

En fullbordad pendling i en växelström kan ses som ett varv i en cirkel --
\ang{360} -- och en fullbordad sådan pendling kallas en period.

En period motsvarar omkretsen i en cirkel med radien r, där omkretsen är
\(2 \cdot \pi  \cdot r\). När strömmen växlar 1 gång per sekund har
pendlingen en frekvens [f] av 1~hertz [Hz].
Vid 50 växlingar per sekund har pendlingen en frekvens av \qty{50}{\hertz}.

Den \emph{Induktiva reaktansen \(X_L\)} -- växelströmsmotståndet i en induktor -- 
är en funktion av strömmens så kallade vinkelhastighet \(\omega = 2 \cdot \pi  \cdot f\)
och av induktansen L.

Den induktiva reaktansen är proportionell mot strömmens frekvens och mot
induktorns induktansvärde.
Inga förluster uppstår i en ideal induktor, det vill säga en induktor som
teoretiskt saknar resistans.
Sambandet är:
%%
\[X_L = 2\pi fL = \omega L\]
\[X_L [\unit{\ohm}] \quad f [\unit{\hertz}] \quad L [\unit{\henry}]\]
%%
\textbf{Exempel:}
%%
\[L = \qty{1}{\henry} \quad f = \qty{50}{\hertz} \quad X_L = ?\]
\[X_L = 2\pi fL = 2\pi \cdot 50 \cdot 1 = \qty{314}{\ohm}\]
%%
\textbf{Exempel:}
%%
\[L = \qty{1}{\henry} \quad f = \qty{5}{\kilo\hertz} \quad X_L = ?\]
\[X_L = 2\pi fL = 2\pi  \cdot 5 \cdot 10^3 \cdot 1 = \qty{31400}{\ohm}\]

\subsection{Fasförskjutning mellan spänning och ström i en induktor}
\harecsection{\harec{a}{2.3.5}{2.3.5}}
\index{induktor!fasförskjutning}

Med fasförskjutning menas den tidsmässiga förskjutningen mellan ström- och
spänningsförlopp.
Strömmen genom en induktor når inte sitt toppvärde samtidigt
som spänningen över den.
Orsaken är växlingarna mellan elektrisk och magnetisk energi i induktorn.
Detta illustreras i bild~\ssaref{fig:BildII3-11}.

I en ideal induktor är spänningen fasförskjuten \ang{90} före strömmen.
I praktiken är dock förskjutningen något mindre än \ang{90} på grund av
resistansen i induktorn.

\subsection{Q-faktor -- godhetstal}
\harecsection{\harec{a}{2.3.6}{2.3.6}}
\index{Q-faktor!induktor}

\emph{Q-faktorn} kan avse två olika saker, som inte ska förväxlas.
Dessa är Q-faktorn för en komponent respektive Q-faktorn för en hel strömkrets.

Q-faktorn för en induktor är kvoten av dess reaktans och dess serieresistans.

\[Q_{\text{komponent}} = \dfrac{X_{\text{komponent}}}{R_{\text{komponent}}}\]

Q-faktorn för en hel resonanskrets beror däremot på bredden på det
frekvensband som en viss komponentkombination ger.
Q-faktorn för en resonanskrets är därför ett mått på dess
selektivitet (se kapitel~\ssaref{Q-faktor}).

Q-faktorn för en ingående komponent påverkar Q-faktorn för en hel krets.
Däremot gäller inte det omvända.

\subsection{Yteffekt -- skin-effect}
\index{yteffekt}
\index{skin-effect}

I en ledare av homogent material fördelar sig en likström lika över hela
tvärsnittet.
Men för en växelström minskar strömtätheten i ledarens mitt och ökar i stället
vid ytan.
Ju högre frekvensen är desto större är strömtätheten vid ytan.
Fenomenet kallas \emph{yteffekt} (eng. \emph{skin effect}) och uppträder i alla
ledare.

Det djup i ledarmaterialet där laddningstätheten sjunkit till \qty{37}{\percent}
av värdet vid ytan kallas \emph{skin depth}.
För koppar är detta djup ca \qty{70}{\milli\metre} vid \qty{100}{\hertz}.
Vid \qty{1}{\mega\hertz} har djupet minskat till \qty{0,07}{\milli\metre} och
vid \qty{100}{\mega\hertz} till \qty{0,0067}{\milli\metre}.
På grund av yteffekten är alltså materialet i mitten av homogena ledare
elektriskt mindre verksamt vid höga frekvenser.
Resistansen för en viss ledare blir alltså större för växelström än för likström.

Utöver frekvensen påverkas yteffekten av ledarmaterialets elektriska och
magnetiska ledningsförmåga.
För att få låg resistans i ledare för högfrekvent ström är det viktigt att
omkretsen är stor och att materialskiktet vid ytan har hög ledningsförmåga.
Därför är induktorerna i sändarslutsteg ofta försilvrade och består av rör med
stor diameter eller av breda band.

\subsection{Temperaturkoefficient}

Liksom med resistorer påverkas även induktansen av temperaturen.
Att sambandet mellan induktans och temperatur är viktigt förstås av att
temperaturkoefficienten i den frekvensbestämmande induktorn i en oscillatorkrets
påverkar frekvensstabiliteten.

Eftersom metallen koppar utvidgar sig vid temperaturökning och induktorns
tvärsnittsyta då blir större, är temperaturkoefficienten vanligen positiv.
Temperaturkoefficienten \(\alpha_L\) anger induktansändringen per grad
temperaturändring.

Induktansändringen blir då $\Delta L = \pm \alpha _L \cdot L_k \cdot \Delta\vartheta$
där \(L_k\) är induktansvärdet vid den lägre temperaturen (oftast
\qty{20}{\degreeCelsius}) och \(\Delta\vartheta\) är temperaturändringen i
kelvin.

Kelvin [K] är den normerade måttenheten för absolut temperatur.
En ändring med \qty{1}{\kelvin} motsvarar en ändring med \qty{1}{\degreeCelsius}.

Induktorer kan innehålla kärnor av någon metallegering vars egenskaper också är
temperaturberoende.

I praktiken kan man knappast påverka temperaturkoefficienten i en induktor.
Eftersom en resonanskrets för det mesta även innehåller kondensatorer kan
man kompensera en positiv temperaturkoefficient i induktorn med en negativ
temperaturkoefficient i en kondensator.

\subsection{Förluster i kärnmaterial}

När ett magnetiskt växelfält passerar ett kärnmaterial kommer atomerna (som
är permanentmagneter) att ständigt inta nya lägen i materialet i takt med
fältets frekvens.
Då uppstår virvelströmmar, så kallade järnförluster, som dels påverkar
materialets ledningsförmåga och som dels höjer temperaturen i kärnan och därmed
i hela induktorn.

% Avsnitt 2.4 Transformatorn
\mediumtikz{
      \begin{circuitikz}
        \draw
        (1,1) node[transformer](T1) {}
        (T1.base) node{1}
        ;
        \draw[european]
        (4,1) node[transformer](T2) {}
        (T2.base) node{2}
        ;
        \draw
        (7,1) node[transformer core](T3) {}
        (T3.base) node{3}
        ;
      \end{circuitikz}
%%      \\
%%      \begin{tabular}{rl}
%%        1, 2 & Allmänna symboler \\
%%        3 & Transformator med kärna
%%      \end{tabular}
}{Schemasymboler för transformatorer: 1 och 2 är allmänna symboler och 3 transformator med kärna.}{fig:BildII2-5}

\mediumfigpad{images/cropped_pdfs/bild_2_2-06.pdf}{Obelastad transformator}{fig:BildII2-6}

\section{Transformatorn}
\harecsection{\harec{a}{2.4}{2.4}}
\label{sec:transformator}
\index{primärlindning}
\index{transformator!primärlindning}
\index{sekundärlindning}
\index{transformator!sekundärlindning}

\subsection{Allmänt}

En \emph{transformator} (eng. \emph{transformer}) består av en eller flera
lindningar eller spolar av elektriska ledare.
Lindningarna är magnetiskt kopplade till varandra.
Det innebär att de är anordnade så att ett magnetfält som alstrats i någon
av lindningarna även passerar genom övriga lindningar.

När en växelspänning läggs över en lindning kallas den \emph{primärlindning}
(eng. \emph{primary coil}).
I och omkring primärlindningen alstras då ett magnetiskt fält som växlar i takt
med spänningen. Primärfältet passerar även genom övriga lindningar --
\emph{sekundärlindningarna} (eng. \emph{secondary coil}) -- och alstrar där
spänningar och strömmar.

Den så kallade kopplingsfaktorn mellan lindningarna varierar för olika frekvenser.
Den är lägre vid låga frekvenser (hundratals \unit{\hertz}) och högre vid höga
frekvenser (tusentals \unit{\hertz}).
Speciellt vid låga frekvenser behövs en större kopplingsfaktor för att avsedd
effekt ska kunna överföras mellan lindningarna. Då kan ledningsförmågan i den
magnetiska flödesvägen ökas med hjälp av en järnkärna.

Bild~\ssaref{fig:BildII2-5} illustrerar flera vanligt förekommande schemasymboler
för transformatorer med två lindningar.


% \newpage % layout
\subsection{Utföranden}
\index{spänningstransformator}
\index{transformator!spännings-}
\index{strömtransformator}
\index{transformator!ström-}
\index{impedanstransformator}
\index{transformator!impedans-}

Transformatorn kan utföras för olika ändamål, till exempel som
\emph{spänningstransformator} (eng. \emph{voltage transformer}),
\emph{strömtransformator} (eng. \emph{current transformer}) eller
\emph{impedanstransformator} (eng. \emph{impedance transformer}).

Utförandet påverkas även av frekvens och av vilken effekt som ska överföras.

\subsection{Terminologi}
\index{varvtalsomsättning}
\index{transformator!varvtalsomsättning}
\index{impedansomsättning}
\index{transformator!impedansomsättning}

\begin{center}
\begin{tabular}{ll}
primärkrets & sekundärkrets \\
primärlindning & sekundärlindning \\
primärspänning \(u_1\) &  sekundärspänning \(u_2\) \\
primärström \(i_1\) & sekundärström \(i_2\) \\
lindningsvarvtal n & primärt \(n_1\) sekundärt \(n_2\)
\end{tabular}
\end{center}

\begin{tabular}{rcl}
varvtalsomsättning &=& \(\dfrac{n_1}{n_2}\) eller \(\dfrac{n_2}{n_1}\) \\
&&\\
impedansomsättning &=& \(\dfrac{Z_1}{Z_2}\) eller \(\dfrac{Z_2}{Z_1}\) \\
\end{tabular}

\newpage
\subsection{Den ideala (förlustfria) transformatorn}
\harecsection{\harec{a}{2.4.1}{2.4.2.1}, \harec{a}{2.4.2.2}{2.4.1}, \harec{a}{2.4.2.1}{2.4.2.2}}
\index{varvtalsomsättning}
\index{transformator!varvtalsomsättning}
\index{impedansomsättning}
\index{transformator!impedansomsättning}
\label{ideal_transformator}

% \mediumfig{images/cropped_pdfs/bild_2_2-06.pdf}{Obelastad transformator}{fig:BildII2-6}

I bild~\ssaref{fig:BildII2-6} är transformatorn är obelastad när sekundärkretsen
är bruten.

När primärlindningen ansluts till en växelspänning induceras växelspänningar
både över primär- och sekundärlindningarna.
Det uppstår även en ström i primärlindningen, men däremot inte i
sekundärlindningen när sekundärkretsen är bruten.
För den obelastade transformatorn gäller sambandet
%%
\[\dfrac{u_1}{u_2} = \dfrac{n_1}{n_2}\]
%%
det vill säga att spänningen över lindningarna är proportionell mot lindningsvarvtalen.

\mediumplustopfig{images/cropped_pdfs/bild_2_2-07.pdf}{Belastad transformator}{fig:BildII2-7}
% \mediumminustopfig{images/cropped_pdfs/bild_2_2-08.pdf}{Sparkopplad transformator}{fig:BildII2-8}
I bild~\ssaref{fig:BildII2-7} är transformatorn belastad när sekundärkretsen
är sluten.

När någon av transformatorns sekundärlindningar ingår i en sluten strömkrets 
uppstår en sekundärström där.
%
Sekundärströmmen alstrar ett magnetfält som motverkar primärströmmens fält,
hindrar dess växlingar och tar ut energi från primärkretsen.
%
Strömförbrukningen på primärsidan ökar således i proportion mot
strömförbrukningen på sekundärsidan. Transformatorn reglerar själv hur mycket
energi som den tar från strömkällan och lagrar i fältet för att föra över
till sekundärkretsen.

\newpage
För den belastade transformatorn gäller att strömmen genom lindningarna är
omvänt proportionell mot lindningsvarvtalet, det vill säga omvänt proportionell 
mot varvtalsomsättningen.
%%
\[\dfrac{i_1}{i_2} = \dfrac{n_2}{n_1}\]
%%
Av föregående formler följer att:
%%
\[\dfrac{u_1}{u_2} = \dfrac{i_2}{i_1}\]
%%
Av \(P_1 = u_1 \cdot i_1\) och \(P_2 = u_2 \cdot i_2\) följer att \(P_1 = P_2\).

Om man bortser från förlusterna i transformatorn, är den effekt som den tar
från kraftkällan lika med den effekt som transformatorn avger.

Eftersom transformatorn transformerar både spänningar och strömmar, kommer
även impedansen att transformeras genom transformatorn.
Denna impedanstransformation följer impedansomsättningen, det vill säga
%%
\[\dfrac{Z_1}{Z_2} = \dfrac{n_1^2}{n_2^2}~.\]

\newpage
\subsection{Transformatortillämpningar}
\harecsection{\harec{a}{2.4.2.4}{2.4.2.4}}

\subsubsection{Sparkopplade transformatorer}
\index{galvanisk förbindelse}
\index{spartransformator}
\index{transformator!spar-}

\mediumfig{images/cropped_pdfs/bild_2_2-08.pdf}{Sparkopplad transformator}{fig:BildII2-8}
\mediumfigpad[0.9]{images/cropped_pdfs/bild_2_2-09.pdf}{Strömtransformator}{fig:BildII2-9}

I bild~\ssaref{fig:BildII2-7} har transformatorn beskrivits så att primär- och
sekundärlindningarnas enda förbindelse med varandra är över ett magnetfält,
alltså utan galvanisk förbindelse.

Varje lindning kan förses med godtyckliga uttag. Mellan uttagen finns 
då en spänning som är proportionell mot antalet lindningsvarv.

Detta är en metod för att spara in på antalet lindningar.
För att till exempel omsätta nätspänningen \qty{230}{\volt} till
\qty{115}{\volt} används ibland en \emph{spartransformator}.

Med en spartransformator kommer olika strömkretsar i galvanisk förbindelse med
varandra, vilket visas i bild~\ssaref{fig:BildII2-8}.
Särskild försiktighet ska därför iakttas vid användning av sparkopplade
transformatorer, på grund av risken för elolycksfall.
Spartransformatorer bör därför inte användas i amatörradiosammanhang.
Säkrast är skyddstransformatorer med galvaniskt skilda ledningar och dessutom
med speciellt bra isolering och kapsling.

\subsubsection{Strömtransformatorer}
\index{strömtransformator}
\index{transformator!ström-}

Hög sekundärström under låg sekundärspänning kännetecknar en
\emph{strömtransformator} (eng. \emph{current transformer}),
som illustreras i bild~\ssaref{fig:BildII2-9}.
Strömtransformatorer används i elektriska svetsningsutrustningar,
induktionsugnar och liknande.
Strömtransformatorer används även för mätning av höga växelströmmar.

%% \mediumplustopfig{images/cropped_pdfs/bild_2_2-09.pdf}{Strömtransformator}{fig:BildII2-9}

\subsubsection{Högspänningstransformatorer}
\index{spänningstransformator}
\index{transformator!spännings-}
\index{högspänningstransformator}
\index{transformator!högspännings-}

Hög sekundärspänning under förhållandevis låg sekundärström kännetecknar en
\emph{spänningstransformator} (eng. \emph{voltage transformer}).
Bild~\ssaref{fig:BildII2-10} visar en transformator med ett gnistgap i
sekundärkretsen för tändning av gas.

\emph{Högspänningstransformatorer} (eng. \emph{high voltage transformer})
används i distributionsnät, neonskyltar, tändsystem för förbränningsmotorer,
anodspänningsaggregat för sändare och så vidare.

\mediumplustopfig[0.9]{images/cropped_pdfs/bild_2_2-10.pdf}{Högspänningstransformator}{fig:BildII2-10}

\newpage
\subsubsection{Låg- och klenspänningstransformatorer}
\index{spänningstransformator}
\index{transformator!spännings-}
\index{lågspänningstransformator}
\index{transformator!lågspännings-}
\index{skyddstransformator}
\index{transformator!skydds-}

\mediumfigpad[0.9]{images/cropped_pdfs/bild_2_2-11.pdf}{Klenspänningstransformator}{fig:BildII2-11}

En \emph{lågspänningstransformator} (eng. \emph{low voltage transformer}) med
spänningen 400/\qty{230}{\volt} används i lokala distributionsnät, det vill säga
de elektriska ledningar som går från en transformator till vanliga bostäder och
kontor.

För ökad säkerhet mot elektrisk chock krävs dock att vissa apparater drivs med
klenspänning via en \emph{skyddstransformator}
(eng. \emph{safety isolating transformer}).
Det är en transformator med skyddsseparation mellan primär- och
sekundärlindningarna.
Sekundärspänningen i en klenspänningstransformator,
bild~\ssaref{fig:BildII2-11}, får inte överstiga \qty{50}{\volt}.

\newpage
\subsection{Sambandet mellan varvtal och impedans}
\harecsection{\harec{a}{2.4.2.3}{2.4.2.3}}
\index{impedans!transformator varvtal}
\index{impedansomsättning}
\index{transformator!impedansomsättning}
\index{impedanstransformator}
\index{transformator!impedans-}


Transformatorn kan även användas för anpassning av impedanser.
Impedansen Z i en lindning är proportionell mot kvadraten av dess
lindningsvarvtal n.

Om effekten i sekundärlindningen är lika stor som i primärlindningen, gäller
formeln:
%
\[\dfrac{Z_p}{Z_s} = \dfrac{n_p^2}{n_s^2}~.\]

% Avsnitt 2.5 Halvledardioden
\newpage
\section{Halvledardioden}
\harecsection{\harec{a}{2.5}{2.5}}
\index{halvledardiod}
\index{diod}
\index{diod!halvledardiod}

\subsection{Allmänt}
\label{dioden_allmänt}

I en strömkrets kan av olika anledningar ström tillåtas att flyta i en riktning
men kanske inte i den motsatta.
En anordning med en sådan funktion kallas för en diod.

Först bestod en diod av två elektroder i vakuum (se avsnitt
\ssaref{vakuumdioden}). Därav namnet vakuumdiod.
Numera består en diod oftast av någon halvledare. Därav namnet halvledardiod.

\mediumfig{images/cropped_pdfs/bild_2_2-12.pdf}{Spärrskiktet i en halvledardiod}{fig:BildII2-12}

Bild~\ssaref{fig:BildII2-12} överst illustrerar en halvledardiod bestående av ett
P-ledande och ett N-ledande materialskikt som fogats samman.

Mellan de båda skikten utbildas ett tunt gränsskikt som inte innehåller
laddningsbärare. Detta skikt kan vara ledande eller icke ledande -- ett
spärrskikt -- beroende på polariseringen.

\subsection{Halvledardiodens karaktär}
\harecsection{\harec{a}{2.5.1.2}{2.5.1.2}}

\subsubsection{Diod i framriktningen}
\index{diod!framriktning}
\index{diod!framström}
\index{diod!framspänningsfallet}

Förbinder man den positiva polen på en spänningskälla med P-skiktet i en diod
och den negativa polen med N-skiktet så är dioden polariserad i
\emph{framriktningen}, detta illustreras i bild~\ssaref{fig:BildII2-12} mitten.
Spärrskiktet upplöses då och en \emph{framström}
(eng. \emph{forward current}) flyter genom dioden.
Elektronerna flyter till den positiva polen och hålen till den negativa polen.
Över anslutningarna ligger en spänning, \emph{framspänningsfallet}
(eng. \emph{forward voltage}), som varierar med strömmen och temperaturen.
Spänningsfallet och strömmen ger på normalt sett diodens \emph{förlusteffekt}.

\subsubsection{Diod i backriktningen}
\index{diod!spärriktningen}
\index{diod!backriktningen}
\index{diod!zenereffekt}

\emph{Backspänning, backström, läckström, spärriktning}

Förbinder man i stället den negativa polen på en spänningskälla med P-skiktet i
en diod och den positiva polen med N-skiktet så är dioden polariserad i
\emph{spärriktningen} eller \emph{backriktningen}, så som illustreras i
bild~\ssaref{fig:BildII2-12} underst.
Spärrskiktet blir då ännu kraftigare.

Endast en obetydlig ström \(I_{SP}\) flyter genom dioden i spärriktningen, även
vid ökande spänning \(U_{SP}\).
Men över en viss spänning ökar strömmen snabbt -- den så kallade zenereffekten
uppstår.
Dioden kan då lätt förstöras av en alltför hög ström.

\subsubsection{Diod i strömkrets}
\index{diod!inkoppling}
\index{diod!anod}
\index{diod!katod}
\index{diod!PANK}

När en diod kopplas in i en strömkrets är det nödvändigt att dioden vänds så att
ström kan flyta igenom den i önskad riktning.

Anslutningen till en diods P-skikt kallas för \emph{anod} och kopplas normalt
mot strömkretsens positiva pol.

Motsvarande anslutning från N-skiktet på en diod kallas \emph{katod} och kopplas
normalt mot strömkretsens negativa pol.

\newpage
För att komma ihåg hur en diod ska vändas så används anod och katod i en
minnesregel som lyder Positiv~Anod~Negativ~Katod vilket förkortas \textbf{PANK}.

En diods katod märks ut med ett streck på eller en markering i höljet som ska
motsvara strecket framför pilen i diodens schemasymbol.

\subsubsection{Diodens ström-spänningsförhållande}

\mediumfig{images/cropped_pdfs/bild_2_2-13.pdf}{Halvledardiodens karaktäristik}{fig:BildII2-13}

Bild~\ssaref{fig:BildII2-13} visar en diods ström-spänningsförhållande.

Strömmen \(I_D\) börjar att flyta när spänningen \(U_D\) har nått ett
tröskelvärde (vid kiseldioder \qty{0,6}{\volt}).
När spänningen ökar ytterligare däröver, ökar även strömmen.

Produkten av spänningsfallet över dioden och strömmen genom den kallas
förlusteffekt. Denna värmer upp dioden. Vid för hög temperatur förstörs
kristallstrukturen.
En kiselkristall kan klara upp till \qty{200}{\degreeCelsius} medan en
germaniumkristall bara klarar \qty{75}{\degreeCelsius}.\\

\smalltikz{
		\begin{circuitikz}[american voltages]
	% diodetyper bild 2.16
	% diod
	\draw (0,0) to [D-, a_=$1$, l2^=K and A] (0,1);
	% zenerdiod
	\draw (1.5,0) to [zD-, l_=$2$] (1.5,1);
	% kapacitansdiod
	\draw (3,0) to [VC-, l_=$3$] (3,1);
	% lysdiod
	\draw (4.7,0) to [leD-, l_=$4$] (4.7,1);
\end{circuitikz}
}{Schemasymboler för dioder}{fig:BildII2-14}

\newpage
\subsection{Diodtillämpningar}
\harecsection{\harec{a}{2.5.1.1}{2.5.1.1}}

Bild~\ssaref{fig:BildII2-14} illustrerar flera olika schemasymboler för dioder.\\
1 Diod allmän symbol med Katod och Anod\\
2 Zenerdiod\\
3 Kapacitansdiod\\
4 Lysdiod (LED)\\

% \noindent
Likriktning är den vanligaste tillämpningen för dioder (se
kapitel~\ssaref{kraftaggregat}).
Halvledardioder utförs även för en rad andra ändamål och finns i en mängd
varianter.

\subsubsection{Dioder för spänningsstabilisering (zenerdiod)}
\index{zenerdiod}
\index{diod!zener}
\label{diod_zener}

  Inom ett visst område är spänningsfallet över en zenerdiod i en strömkrets
  i det närmaste konstant medan strömmen varierar. Denna egenskap kallas
  zenereffekt och används för konstanthållning av spänning.

  Det finns zenerdioder för många olika spänningar och effekter.

\subsubsection{Dioder som variabla kondensatorer (kapacitansdiod, VariCap)}
\label{varicap}
\index{varicap}
\index{diod!varicap}

  När en diod är polariserad i spärriktningen bildas ett spärrskikt.
  Olika polariseringsspänning alstrar olika tjocka spärrskikt och en spärrad diod
  har på så sätt egenskaper som liknar dem i en variabel kondensator.
  Det finns dioder där reglerbarheten av kapacitansen är speciellt utvecklad.

\newpage
\subsubsection{Lysdioder (LED)}
\index{lysdiod}
\index{diod!lysdiod}
\index{LED}
\index{diod!LED}
\index{laserdiod}
\index{diod!laserdiod}
\label{diod_led}

\emph{Lysdiod} (eng. \emph{Light Emitting Diode, LED}) är en diod anpassad för
att leverera ljus, ofta synligt sådant.
Lysdioder finns tillgängliga med infrarött, rött, orange, gult, grönt,
blått och vitt ljus.
En variant av lysdiod är laserdioden, som bland annat används för överföring
över optisk fiber.

När en diod är polariserad i passriktningen frigörs energi i spärrzonen.
Det sker genom rekombination av par av laddningsbärare, varvid det normalt avgår
energi i form av värme.
Vid en viss inblandning av främmande atomer avgår istället ljus.

Spänningfallet över en lysdiod är ungefär dubbelt så stort som över en
kiseldiod, det vill säga ungefär 1,5~volt.
Det normala spänningsfallet bör alltid kontrolleras för korrekt dimensionering
av kretsen.
Ljusstyrkan är proportionell mot strömmen, som normalt har värden mellan 10 och
\qty{50}{\milli\ampere}.
En lysdiod bör ha ett strömbegränsande motstånd i serie för att strömmen
inte ska bli för stor och lysdioden åldras i förtid eller rent av gå sönder.

Moderna högeffektslysdioder kräver en konstant\-strömsmatning och kan ha betydligt
högre spänning.
Dessa har blivit tillgängliga till lågt pris och populära för experiment.

\subsection{Vakuumdioden i jämförelse med halvledardioden}

\smalltikz{
\begin{circuitikz}[american voltages]
	% dioder polarisering bild 2.17
	% passriktning
	%vakuumdiode
	\draw (0,0) node[diodetube]{};
	\draw (0,1) to (2,1);
	\draw (2,1) to [R, a^=$R$] (2,-1);
	\draw (2,-1) to [american voltage source] (-0.3,-1);
	%diod
	\draw (6,1) to [D-] (4,1);
	\draw (4,1) to (4,-1);
	\draw (6,1) to [R, a^=$R$] (6,-1);
	\draw (6,-1) to [american voltage source] (4,-1);
\end{circuitikz}\\
Passriktning\\

\begin{circuitikz}[american voltages]
	% dioder polarisering bild 2.17
	% Spärriktning
	%vakuumdiod
	\draw (0,0) node[diodetube]{};
	\draw (0,1) to (2,1);
	\draw (2,1) to [R, a^=$R$] (2,-1);
	\draw (-0.3,-1) to [american voltage source] (2,-1);
	%diod
	\draw (6,1) to [D-] (4,1);
	\draw (4,1) to (4,-1);
	\draw (6,1) to [R, a^=$R$] (6,-1);
	\draw (4,-1) to [american voltage source] (6,-1);
\end{circuitikz}\\
Spärriktning\\

}{Dioders polarisering i kretsen}{fig:BildII2-15}

Bild~\ssaref{fig:BildII2-15} visar principen för hur de båda diodtyperna ingår i
en strömkrets.
Den stora skillnaden är att arbetsspänningen för en vakuumdiod är mångfalt
högre än den för en halvledardiod samt att vakuumdiodens ena elektrod (katoden)
behöver hettas upp för att avge elektroner.

% Avsnitt 2.6 Transistorn
\newpage
\section{Transistorn}
\harecsection{\harec{a}{2.6}{2.6}}
\label{transistorn}
\index{transistor}

\smallfig{images/cropped_pdfs/bild_2_2-16.pdf}{Schemasymboler}{fig:BildII2-16}

\subsection{Allmänt}
\label{transistor_allmänt}

En transistor består av skikt av dopade halvledarelement som sammanfogats.
Vanligt är två N-skikt och ett mellanliggande P-skikt (NPN-transistor) eller två
P-skikt och ett mellanliggande N-skikt (PNP-transistor).
Skikten är försedda med anslutningar.

Bild \ssaref{fig:BildII2-16} visar schemasymboler för de vanliga tran\-sistor\-typerna 
NPN-transistorer (bipolära), PNP-\-tran\-sistorer (bipolära) och
FET-transistorer (fälteffekt-).

\newpage
\subsection{NPN-transistorer}
\harecsection{\harec{a}{2.6.1b}{2.6.1b}}
\index{NPN-transistor}
\index{transistor!NPN}

Halvledarskikten kallas emitter (E), bas (B) och kollektor (C).
Bild \ssaref{fig:BildII2-17a} visar en klassisk hålmonterad småsignaltransistor.

\smallfigpad[0.06]{images/cropped_pdfs/bild_2_6-37.pdf}{Transistor}{fig:BildII2-17a}


\subsubsection{Spärrzonerna}

%% k7per: Another candidate of a picture to split apart into 3 separate ones, so the
%% picture can come close to the text that explains it.
\tallfig[0.30]{images/cropped_pdfs/bild_2_2-17.pdf}{Skikten i en bipolär transistor}{fig:BildII2-17}

Bild \ssaref{fig:BildII2-17} överst visar hur mellan skikten B och E respektive
mellan B och C bildas zoner vars ledningsförmåga kan styras elektriskt över
anslutningarna.

\begin{figure*}[p]
  \begin{center}
    \includegraphics[width=0.66\textwidth]{images/cropped_pdfs/bild_2_2-18.pdf}
    \caption{Emitterkopplad transistor}
    \label{fig:BildII2-18}
  \end{center}
\end{figure*}

\begin{figure*}[p]
  \begin{center}
    \includegraphics[width=0.66\textwidth]{images/cropped_pdfs/bild_2_2-19.pdf}
    \caption{Karaktäristika för transistor BC 107}
    \label{fig:BildII2-19}
  \end{center}
\end{figure*}

%%\clearpage

\subsubsection{Spänningskällan \(U_{BE}\)}

Bild \ssaref{fig:BildII2-17} mitten visar att mellan bas och emitter finns en
diodsträcka.
När en positiv spänning läggs på basen och en negativ spänning på emittern
polariseras diodsträckans spärrzon i passriktningen.
Spärrzonen upplöses då och det flyter en så kallad basström \(I_B\).

\subsubsection{Spänningskällan \(U_{CE}\)}

Bild \ssaref{fig:BildII2-17} nederst visar att när en positiv spänning läggs på
kollektorn och en negativ spänning läggs på emittern polariseras diodsträckan i
spärriktningen.
Spärrzonen förstärks då och det flyter ingen ström.

\newpage
\subsubsection{Inverkan av både \(U_{BE}\) och \(U_{CE}\)}

Bild \ssaref{fig:BildII2-18} visar hur två spänningskällor \(U_{BE}\) och
\(U_{CE}\) ansluts till en emitterkopplad NPN-transistor.
Ur den starkt dopade emitterzonen strömmar elektronerna in i den svagt dopade
baszonen (spänning: \(U_{BE}\)).
De flesta elektronerna blir emellertid inte kvar i basen.
De stöter igenom det tunna basskiktet och når fram till
kollektorskiktet med spänningen \(U_{CE}\). Det flyter en kollektorström.

För strömmen \(I_E\) (emitterström), \(I_B\) (basström) och \(I_C\)
(kollektorström) gäller:
%%
\[I_E = I_B + I_C\quad \text{där} I_B \ll I_C\ (\ll \text{mycket mindre än})\]
%%
Kollektorströmmen \(I_C\) kan styras med basspänningen \(U_{BE}\).
En liten ändring i basspänningen ger stor förstärkande verkan i
kollektorströmmen.

\subsection{Förstärkningsfaktor}
\harecsection{\harec{a}{2.6.2}{2.6.2}}
\index{förstärkningsfaktor!transistor}
\index{transistor!förstärkningsfaktor}
\label{transistor_förstärkningsfaktor}

Om strömmen i ingångskretsen för en transistor ändras kan strömmen i
utgångskretsen ändras mer.
Vi får en förstärkning.

Av sambandet \(I_C = f(I_B)\) framgår strömförstärkningsfaktorn \(\beta\) eller
\(h_{FE}\) som är kvoten mellan ändringen i utgångsströmmen och ändringen i
ingångsströmmen i transistorns aktiva (linjära) område.

Bild \ssaref{fig:BildII2-19} visar ström-spänning-diagram för transistorn BC~107
för olika basströmmar.
För emitterkoppling gäller:
%%
\[h_{FE} = \dfrac{\Delta I_C}{\Delta I_B}~.\]
%%
\begin{tabular}{ll}
  \(h_{FE}\) & strömförstärkningsfaktorn \\
  \(\Delta I_C\)   & ändringen i kollektorströmmen \\
  \(\Delta I_B\)   & ändringen i basströmmen \\
\end{tabular}

\subsection{PNP-transistorer}
\harecsection{\harec{a}{2.6.1a}{2.6.1a}}
\index{PNP-transistor}
\index{transistor!PNP}
\label{transistor_pnp}

Ersätter man de två N-skikten i en NPN-transistor med P-skikt och P-skiktet med
ett N-skikt så erhåller man en PNP-transistor.

Uppbyggnad, koppling och användning av en PNP-transistor motsvarar i övrigt den
för en NPN-transistor. Spänningskällorna måste emellertid ha motsatt polaritet.

\newpage % layout
\subsection{Fälteffekttransistorer}
\harecsection{\harec{a}{2.6.3}{2.6.3}}
\index{fälteffekttransistor}
\index{transistor!fälteffekt}
\index{FET}
\index{transistor!FET}

\subsubsection{Allmänt}

\emph{Fälteffekttransistorer (FET)} har en mycket hög ingångsimpedans och
styrströmmen blir därför mycket svag.
Man säger därför att en FET är spänningsstyrd.

Även NPN- och PNP-transistorer -- bipolära transistorer -- styrs med spänning,
men dessa typer har en relativt låg ingångsimpedans och därför högre styrström.
Man säger därför att de är strömstyrda.

\smallfig[0.15]{images/cropped_pdfs/bild_2_2-20.pdf}{Schemasymbol för en FET}{fig:BildII2-20}

Bild \ssaref{fig:BildII2-20} anger en schemasymbol för en FET.

Fälteffekttransistorn har tre anslutningar, source (S), drain (D) och gate (G).

\smallfigpad[0.25]{images/cropped_pdfs/bild_2_2-21.pdf}{Skikten i en N-kanal FET}{fig:BildII2-21}

\subsubsection{Fälteffekttransistorns uppbyggnad}

Bild \ssaref{fig:BildII2-21} visar ett N-ledande skikt (även kallat N-kanal) med
anslutningarna S och D anslutna till respektive ändar av skiktet.
N-kanalen passerar mellan två P-ledande skikt förbundna med styrelektroden G.

När en spärrspänning läggs mellan G och S breder spärrskikten ut sig och
N-kanalen blir trängre.
Läggs en negativ spänning på S och en positiv spänning på D, kommer en ström att
flyta i N-kanalen.
Strömmens styrka kan påverkas med spänningen på G.

En liten spänningsändring \(\Delta U_{GS}\) medför stor ändring av strömmen
\(\Delta I_{GS}\) i N-kanalen. Detta innebär förstärkning.

\smallfig[0.25]{images/cropped_pdfs/bild_2_2-22.pdf}{Skikten i en N-kanal MOSFET}{fig:BildII2-22}

\index{MOSFET}
\index{transistor!MOSFET}

\newpage
Bild \ssaref{fig:BildII2-22} visar skikten i en N-kanal MOSFET.

I en MOSFET (eng. \emph{Metal Oxide Semicoductor Field Effect Transistor}) är
G-elektroden (metallen) isolerad från halvledarkanalen med ett kiseloxidskikt.
Funktionssättet är samma som för en FET.
Drainströmmen kan ökas eller minskas med hjälp av en positiv respektive negativ
spänning på G.

\subsubsection{Resistansen mellan gate och source}

För att erhålla en förstärkning med en FET sätter man in en resistor \(R_0\) i
drainströmkretsen.
Över resistorn uppstår då spänningsändringar i proportion med strömändringarna.

För att fastställa viloströmmen och därmed arbetspunkten för samma transistor
sätter man in en resistor \(R_S\) i sourceströmkretsen.
Storleken på sourceresistorn ger sig av önskad gateförspänning \(-U_{GS}\).
%%
\[R_S = \dfrac{-U_{GS}}{I_D}~.\]

\smallfigpad{images/cropped_pdfs/bild_2_2-23.pdf}{Karaktäristik för N-kanal FET}{fig:BildII2-23}

\newpage
\subsection{Sambandet drain-ström och spänning}

%%Bild \ssaref{fig:BildII2-23} visar karaktäristiken för en N-kanals-FET.
För att beskriva en FET använder man sig av karaktäristiska kurvor (bild \ssaref{fig:BildII2-23}).
Vi har redan presenterat bipolära transistorers in- och utgångsegenskaper i kurvform.
Eftersom ingångsströmmen (gateströmmen) i en FET är praktiskt taget noll, är en
sådan kurva utan praktisk mening.
I stället framställer man grafiskt sammanhanget mellan styrspänningen \(U_{GS}\)
och utgångsströmmen (drainströmmen \(I_D\)).
Eftersom det finns N-kanals-FET och P-kanals-FET så skiljer sig polariteten på
\(U_{GS}\) mellan dessa båda typer.

% Avsnitt 2.7 Elektronrör
\section{Elektronrör}
\label{elektronrör}
\index{elektronrör}

\subsection{Allmänt}

Ett elektronrör består av två eller flera elektroder i en lufttom behållare,
vanligen av glas eller ett keramiskt material.

\smallfig{images/cropped_pdfs/bild_2_2-24.pdf}{Schemasymboler för dioder}{fig:BildII2-24}

\subsection{Vakuumdioden (tvåelektrodröret)}
\harecsection{\harec{a}{2.8.1}{2.8.1}}
\label{vakuumdioden}
\index{vakuumdioden}
\index{elektronrör!diod}
\index{anod}
\index{katod}
\index{diod!anod}
\index{diod!katod}
\index{diod!vakuumdiod}
\index{diod!elektronrör}

Dioden på bild~\ssaref{fig:BildII2-24} innehåller två elektroder, anod (a) och
katod (k), samt i förekommande fall en glödtråd (f) (eng. \emph{filament}).

\emph{Anoden} ska dra elektronerna från katoden.
\emph{Katoden} ska avge elektronerna och måste därför hettas upp.

Upphettningen av katoden kan göras direkt, det vill säga att katoden i sig
själv utgör glödtråd, vanligen med en 4- till 6-volts strömkälla.
Alternativt kan katoden uphettas indirekt med en separat glödtråd som omsluter
och hettar upp ett speciellt katodmaterial.
I det senare fallet är en 1,5- till 12,6-volts glödströmkälla vanlig.

\mediumfig{images/cropped_pdfs/bild_2_2-27.pdf}{Halvvågslikriktning}{fig:BildII2-27}

\smallfigpad{images/cropped_pdfs/bild_2_2-25.pdf}{Edisoneffekten}{fig:BildII2-25}

\subsubsection{Edisoneffekten}
\index{Edisoneffekten}
\index{elektronrör!Edisoneffekten}

Bild~\ssaref{fig:BildII2-25} illustrerar \emph{Edisoneffekten}.
När katoden upphettas lossnar fria elektroner från den och bildar ett moln.
Med en spänning mellan anod och katod, där anoden är positiv, kommer
elektronerna att dras mot anoden.
En anodström börjar att flyta.

\subsubsection{\(I_a/U_a\)-karaktäristikan för en vakuumdiod}


\smallfig[0.35]{images/cropped_pdfs/bild_2_2-26a.pdf}{Diodens karaktäristik}{fig:BildII2-26}
Bild~\ssaref{fig:BildII2-26} illustrerar vakuumdiodens karaktäristik.
När anoden ges positiv potential (anodspänning) flyter en elektronström från
katod till anod (anodström).
Om anodspänningen \(U_a\) ökar så ökar anodströmmen \(I_a\).
Varje par av talvärden representerar en punkt i ett diagram, som det på bilden.
När anodspänningen ökat till ett visst värde, ökar inte anodströmmen ytterligare.
I ett mellanområde, det linjära området, är kurvan i det närmaste rak.

% \mediumfig{images/cropped_pdfs/bild_2_2-27.pdf}{Halvvågslikriktning}{fig:BildII2-27}
\subsubsection{Likriktarverkan}
\index{elektronrör!likriktarverkan}

När anoden i en vakuumdiod ges positiv potential i förhållande till katoden
flyter en så kallad anodström, förutsatt att katoden upphettas så att den avger
fria elektroner.

När anoden ges en negativ potential i förhållande till katoden flyter däremot
ingen anodström.

Vakuumdioden kan därför användas för likriktning av växelströmmar.
Den har en likriktande funktion.

\newpage
\subsubsection{Halvvågslikriktning}
% \mediumbotfig{images/cropped_pdfs/bild_2_2-27.pdf}{Halvvågslikriktning}{fig:BildII2-27}

Bild~\ssaref{fig:BildII2-27} illustrerar halvvågslikriktning.
När anoden ges en omväxlande positiv och negativ potential, en växelspänning,
flyter anodström under varje positiv halvperiod av växelspänningen.
En likströmspuls uppstår under varannan halvperiod.

\newpage
\mediumfig{images/cropped_pdfs/bild_2_2-28.pdf}{Helvågslikriktning}{fig:BildII2-28}
\smallfigpad[0.45]{images/cropped_pdfs/bild_2_2-29.pdf}{Likriktande funktion}{fig:BildII2-29}

\subsubsection{Helvågslikriktning}

Bild~\ssaref{fig:BildII2-28} illustrerar helvågslikriktning.
Med ett elektronrör med dubbla anoder och en transformator med mittuttag på
sekundärlindningen, kan växelspänningens båda halvperioder utnyttjas, så att
anodström flyter i samma riktning under alla halvperioder.


Bild~\ssaref{fig:BildII2-29} illustrerar hur växelspännng via två två
halvvågslikriktningar formar en helvågslikriktning.

\subsection{Vakuumtrioden (treelektrodröret)}
\index{vakuumtrioden}
\index{trioden}
\index{elektronrör!triod}

Bild~\ssaref{fig:BildII2-30} illustrerar symboler för triod och pentod.
Bild~\ssaref{fig:BildII2-32} visar deras karaktäristik.
Trioden innehåller de tre elektroder anod (a), styrgaller (\(g_1\)) och katod
(k) samt en glödtråd (f = filament).

\newpage
\smallfigpad{images/cropped_pdfs/bild_2_2-30.pdf}{Symboler för triod och pentod}{fig:BildII2-30}

\mediumtopfig[0.75]{images/cropped_pdfs/bild_2_2-32.pdf}{Karaktäristika för elektronrör}{fig:BildII2-32}
\mediumtopfig[0.75]{images/cropped_pdfs/bild_2_2-31.pdf}{Elektronstömmen i en triod}{fig:BildII2-31}


\subsubsection{Triodens funktion}

Bild~\ssaref{fig:BildII2-31} illustrerar en triod och dess elektronström.
Styrgallret kan ges positiv, neutral eller negativ potential (förspänning) i
förhållande till katoden.
Valet av förspänning avgör triodens arbetssätt.
När styrgallret ges samma potential som katoden fungerar trioden som en diod.
Med styrgallret positivt ökar anodströmmen.
Med gallret negativt minskar den.

Trioden har en \emph{förstärkande} funktion eftersom anodströmmen kan styras med
styrgallret.
En liten ändring av gallerspänningen medför stor ändring av anodströmmen.
Vid positiv förspänning flyter en gallerström, som inte får bli för hög.
Vanligen väljs en negativ förspänning.

\subsubsection{Triodens strömkretsar och strömkällor}

\begin{center}
\begin{tabular}{lll}
Glödströmskrets      & Anodkrets        &  Gallerkrets \\
Glödbatteri          & Anodbatteri      &  Gallerbatteri \\
Glödspänning \(U_f\) & Anodsp. \(U_a\)  &  Gallersp. \(U_{g1}\) \\
Glödström \(I_f\)    & Anodstr. \(I_a\) &  Gallerstr. \(I_{g1}\) \\
\end{tabular}
\end{center}

\noindent
Vanligen används nätdrivna strömkällor i stället för batterier.
Valet av gallerförspänning är avgörande för triodens arbetssätt.

\subsection{Pentoden (femelektrodröret)}
\index{pentod}
\index{elektronrör!pentod}

Pentoden innehåller fem elektroder, se bild~\ssaref{fig:BildII2-30}.

\begin{center}
\begin{tabular}{ll}
  a       & anod \\
  \(g_3\) & bromsgaller \\
  \(g_2\) & skärmgaller \\
  \(g_1\) & styrgaller \\
  k      & katod med glödtråd (f = filament) \\
\end{tabular}
\end{center}

Bromsgallret förbinds med katoden. Skärmgallret ges en potential som är något
lägre än anodspänningen.
Broms- och skärmgallren förhindrar elektronerna att studsa tillbaka till
styrgallret efter anslaget mot anoden.

\subsection{Tetroden (fyraelektrodröret)}
\index{tetrod}
\index{elektronrör!tetrod}

Denna rörtyp innehåller fyra elektroder.
Uppbyggnaden är densamma som pentodens, men bromsgallret saknas.

%\mediumfig{images/cropped_pdfs/bild_2_2-32.pdf}{Karaktäristika för elektronrör}{fig:BildII2-32}

\subsection{Karaktäristika för elektronrör}

Bild~\ssaref{fig:BildII2-32} illustrerar ett \(I_a/U_{gt}\)-diagram för en triod
eller pentod, vid konstant \(U_a\).

\(I_a/U_a\)-diagram för en triod, vid konstant \(U_{g1}\)

\(I_a/U_a\)-diagram för en pentod, vid konstant \(U_{g1}\)

Tre kurvor visas i \(I_a/U_a\)-diagrammen, med olika värden på
\(U_{g1}\). (\(U_{g1}\) är en så kallad parameter).

\newpage
\smallfig{images/cropped_pdfs/bild_2_2-33.pdf}{Branthet}{fig:BildII2-33}

\subsection{Branthet $S$ och inre resistans $R_i$}

Bild~\ssaref{fig:BildII2-33} visar brantheten.
Om man vid konstant anodspänning ändrar gallerförspänningen med värdet
\(\Delta U_{g1}\), ändrar sig anodströmmen med värdet \(\Delta I_a\).
%%
%%\[\text{Branthet}\quad S = \dfrac{\Delta I_a}{\Delta U_{g1}} \qquad S\ [mA/V]; \Delta I_a\ [mA]; U_{g1}\ [V]\]
%% k7per Prototypical solutions for unit explanations.
\noindent%
\begin{tabularx}{\linewidth}{@{} *1{>{\hsize=.7\linewidth}X} X@{}}
\begin{align*}
\text{Branthet}\quad S = \dfrac{\Delta I_a}{\Delta U_{g1}} 
\end{align*}
&
\begin{equation*}
\begin{aligned}
S\ [mA/V]\\
\Delta I_a\ [mA]\\
U_{g1}\ [V]
\end{aligned}
\end{equation*}
\end{tabularx}
%%
Bild~\ssaref{fig:BildII2-34} visar den inre resistansen.
Om man vid konstant gallerförspänning ändrar anodspänningen med
\(\Delta U_a\), ändras anodströmmen med värdet \(\Delta I_a\).
%%
%\[\text{Inre resistans}\ R_i = \dfrac{\Delta U_a}{\Delta I_a}\qquad R_i\ [k \ohm]; \Delta U_a\ [V]; \Delta I_a\ [mA]\]
%% k7per Prototypical solutions for unit explanations.
\noindent%
\begin{tabularx}{\linewidth}{@{} *1{>{\hsize=.7\linewidth}X} X@{}}
\begin{align*}
\text{Inre resistans}\ R_i = \dfrac{\Delta U_a}{\Delta I_a}
\end{align*}
&
\begin{equation*}
\begin{aligned}
R_i\ [k \ohm]\\ \Delta U_a\ [V]\\ \Delta I_a\ [mA]
\end{aligned}
\end{equation*}
\end{tabularx}
%%
\mediumfig{images/cropped_pdfs/bild_2_2-34.pdf}{Inre resistans}{fig:BildII2-34}

\noindent
Om man vill ändra anodströmmen med \(\Delta I_a\) ges två möjligheter.
Antingen ändrar man gallerförspänningen med värdet \(\Delta U_{g1}\), eller så
ändrar man anodspänningen med värdet \(\Delta U_a\).
Genom att ändra gallerförspänningen med värdet \(U_{g1}\) kan man åstadkomma
samma anodströmsändring \(\Delta I_a\) som med en ändring av anodspänningen
med värdet \(\Delta U_a\).

\subsection{Barkhausens elektronrörsformler}
\index{Barkhausen elektronrörsformler}
\index{elektronrör!Barkhausen formler}

Förstärkningsfaktorn \(\mu \) illustreras av följande samband som gäller mellan
de så kallade rörkonstanterna
%%
\[\mu = S \cdot R_i\]
%%
\paragraph{Exempel}
Beräkna \(\mu\)  om \(S = \qty{2}{\milli\ampere\per\volt}\) \(R = \qty{10}{\kilo\ohm}\) \(\mu = ?\)

\paragraph{Svar} \(\mu = 20\) (\(\mu\)  är dimensionslös)

\subsection{Transistor jämförd med elektronrör}

Transistorer har fördelar som lågt pris, små dimensioner, lång livslängd, enkel
strömförsörjning (glödström behövs inte) och låg driftspänning (\qty{6}{\volt},
\qty{12}{\volt} \ldots ).
Vanliga nackdelar är känslighet för överbelastning och höga temperaturer.

Elektronrör har fördelen av tålighet mot överbelastning, men bland nackdelarna
kan nämnas att de kräver hög anodspänning, att de behöver glödström och att de
är utrymmeskrävande.

Transistorer ersätter numera nästan helt elektronrören, men man bör ändå känna
till elektronrörens egenskaper och arbetssätt.

Ett användningsområde där elektronrör ännu är vanliga är i större
sändarslutsteg.

% Avsnitt 2.8 Digitala kretsar
\section{Digitala kretsar}
\index{digitala kretsar}
\label{digitala kretsar}

Digital elektronik förekommer i all modern utrustning för radio- och
telekommunikation.
Ämnet är mycket omfattande och här redogörs endast för några grundläggande
digitala funktioner.

I \emph{analogtekniken} kan under ett förlopp förekomma oändligt många nivåer,
till exempel spänningar mellan noll och ett högsta värde.

I digitaltekniken förekommer bara ett bestämt antal tillstånd.
I det enklaste digitala systemet finns två tillstånd, till exempel 0 och 1 eller
Till och Från eller Hög och Låg eller Fel och Rätt.
Ett system med två tillstånd kallas binärt.
En lampa som tänds eller släcks med en enkel strömställare är ett binärt system.
Strömställaren kan ha olika utföranden.
Den kan vara en mekanisk kontakt som är styrd för hand eller av en reläspole.
Den kan också vara en transistor eller annan anordning.

\subsection{Transistorn som strömställare}
\label{transistor_strömställare}

\mediumfigpad{images/cropped_pdfs/bild_2_2-35.pdf}{Transistorn som analog förstärkare respektive digital strömställare}{fig:BildII2-35}

Bild \ssaref{fig:BildII2-35} visar två transistorkopplingar.
Den till vänster är en analog förstärkare för växelspänning.
Om det på grund av en viss basspänning flyter en kollektorström av
\qty{1}{\milli\ampere} och kollektorresistorn har värdet \qty{5}{\kilo\ohm},
blir spänningsfallet över denna resistor \qty{5}{\volt}.
Eftersom matningsspänningen är \qty{12}{\volt}, blir spänningen \qty{7}{\volt}
mellan kollektorn och minuspolen.

Kopplingen till höger fungerar som en binär strömställare.
Antag att insignalen intar ett av två spänningstillstånd, antingen
\qty{0}{\volt} (låg) eller \qty{5}{\volt} (hög).
När inspänningen är till exempel \qty{5}{\volt}, flyter så mycket basström genom
basresistorns \qty{10}{\kilo\ohm} att transistorn blir fullt utstyrd.

Därmed är spänningen mellan kollektor och emitter, det vill säga utspänningen,
nära \qty{0}{\volt} (0,1 till \qty{0,2}{\volt} beroende på transistortyp).
Man säger då att utgången är låg (L) eller 0 (noll).

Om däremot inspänningen är \qty{0}{\volt}, spärras kollektorströmmen och
utspänningen blir nära \qty{5}{\volt}.
Man säger då att utgången är hög (H) eller 1.

För NPN-transistorn i bilden gäller att hög inspänning ger låg utspänning och
vice versa.

Denna logiska funktion kallas inverterande.

\subsubsection{NOT-gate eller inverterande grind}
\index{inverterande grind}
\index{NOT-gate}

\smallfig{images/cropped_pdfs/bild_2_2-36.pdf}{NOT-gate}{fig:BildII2-36}

Logiska funktioner beskrivs med internationella symboler.
En ring vid utgången betyder att utspänningens nivå är motsatt
inspänningens vilket illustreras i bild \ssaref{fig:BildII2-36}.
Sambandet mellan in- och utnivåerna beskrivs med en \emph{sanningstabell}.


\subsection{Villkorskretsar -- s.k. grindar}

Det finns olika sätt att bygga grindar.
Idag är de flesta grindarna elektroniska lösningar.
Därutöver finns elektromekaniska grindar i form av strömbrytare och
reläkontakter.

Föregångarna till de elektroniska televäxlarna (AXE med flera) var stora system
av mestadels elektromekaniska reläer.

Att överskådligt förklara arbetssättet i de vanligaste grindarna görs enklast
med reläsymboler.
En reläkontakt kan då motsvara en transistor eller en diod.
Reläspolar kan motsvara logiska nivåer i insignaler.

Elektriska kontakter kan vara normalt öppna och sluter vid påverkan (slutande
kontakt).
Alternativt kan de vara normalt slutna och öppnar vid påverkan (brytande
kontakt).
I kretsscheman visas kontaktlägena vid systemet i vila.

\newpage
\mediumtopfig[0.65]{images/cropped_pdfs/bild_2_2-37.pdf}{OCH-grind (AND-gate)}{fig:BildII2-37}
\mediumtopfig[0.65]{images/cropped_pdfs/bild_2_2-38.pdf}{ELLER-grind (OR-gate)}{fig:BildII2-38}

Av bild \ssaref{fig:BildII2-37} framgår att samma villkor kan skapas med slutande
alternativt brytande kontakter.
Observera placeringen av resistorn på kretsens utgångssida i respektive fall.
När resistorn ligger närmast pluspolen kallas den pull-up.
När den ligger närmast minuspolen kallas den pull-down.
I båda fallen definierar resistorn den logiska nivån.

\subsubsection{OCH-grind eller AND-gate}
\index{AND-gate}

Sanningstabellen i bild \ssaref{fig:BildII2-37} säger att när alla insignaler
är 1 så är utsignalen också 1.

\subsubsection{ELLER-grind eller OR-gate}
\index{OR-gate}


Sanningstabellen i bild \ssaref{fig:BildII2-38} säger att när en eller flera av
insignalerna är 1 så är utsignalen också 1.
När alla insignaler är 0 är utsignalen 0.

\subsubsection{OCH INTE-grind eller NAND-gate}
\index{NAND-gate}

Sanningstabellen i bild \ssaref{fig:BildII2-39} säger att när ingen eller någon
insignal är 1, men inte alla, så är utsignalen 1.
När alla insignaler är 1 är utsignalen 0.

\subsubsection{INTE ELLER-grind eller NOR-gate}
\index{NOR-gate}

Sanningstabellen i bild \ssaref{fig:BildII2-40} säger att när någon eller alla
insignaler är 1 är utsignalen 0.
När alla insignaler är 0 är utsignalen 1.

\subsubsection{Inverterad ingång}

En ingång kan behöva ha en inverterad funktion i förhållande till de övriga
(\emph{low active}).
Man kan då göra som i exemplet med en OCH-grind i bild \ssaref{fig:BildII2-41}.

\mediumtopfig[0.66]{images/cropped_pdfs/bild_2_2-39.pdf}{OCH INTE-grind (NAND-gate)}{fig:BildII2-39}

\mediumtopfig[0.66]{images/cropped_pdfs/bild_2_2-40.pdf}{INTE ELLER-grind (NOR-gate)}{fig:BildII2-40}

\mediumherefig[0.65]{images/cropped_pdfs/bild_2_2-41sbs.pdf}{Inverterad ingång}{fig:BildII2-41}

%% k7per: De följande två små styckena är delade i två för att få
%% siddelninging and figurplacering OK, lägg inte ihop dem om du inte
%% håller på att fixa layout.
\newpage
Möjligheten att ha en ingång inverterad gör det ibland svårt att läsa
kretsscheman, för läslighet är det oftast bäst att ha med en explicit NOT-gate.

\newpage
Meningarna går naturligtvis isär om vad som är mest läsligt, så kolla
noga vilka grindar som används i ett visst kretsschema.

\newpage

\smallfig[0.25]{images/cropped_pdfs/bild_2_2-42.pdf}{Exklusiv ELLER-grind (EXOR-gate)}{fig:BildII2-42}
\smallfig[0.27]{images/cropped_pdfs/bild_2_2-43.pdf}{Exklusiv INTE ELLER-grind (EXNOR-gate)}{fig:BildII2-43}


\subsubsection{Exklusiv ELLER-grind (XOR-gate)}
\index{XOR-gate}

Sanningstabellen i bild \ssaref{fig:BildII2-42} säger att när alla insignaler
antingen är 1 eller 0, så är utsignalen 0.
När någon insignal är 1, men inte alla, så är utsignalen 1.

\subsubsection{Exklusiv INTE ELLER-grind (XNOR-gate)}
\index{XNOR-gate}

Sanningstabellen i bild \ssaref{fig:BildII2-43} säger att när alla insignaler
antingen är 1 eller 0, så är utsignalen 1.
När en insignal är 1, men inte alla, så är utsignalen 0.

\subsection{Grindar med dioder och transistorer}

I stället för reläer eller diskreta halvledare i grindar använder man nu ytterst
sällan något annat än integrerade digitala kretsar (se avsnitt
\ssaref{integrerade kretsar}).

\smalltikz{
  \bigskip
  \begin{circuitikz}
    \draw (-2,1) to[short,-o] (1,1);
    \draw
    (0,-2) node[npn](T) {}
    ;
    \draw (T.collector) to[R] (0,1) to (0,1);
    \draw (-2,-2) to[D] (-1,-2) to (T.base);
    \draw (-2,-2) to[D] (-4,-2) to node[label=A] {} (-4,-2);
    \draw (-2,1) to[R] (-2,-1) to (-2,-2);
    \draw (-2,-2) to (-2,-3);
    \draw (-2,-3) to[D] (-4,-3) to node[label=B] {} (-4,-3);
    \draw (T.emitter) to (0,-3) to node[ground] {} (0,-3);
    \draw (T.collector) to node[label=C] {} (1,-1.235);
  \end{circuitikz}
}{DTL-logik}{fig:BildII2-44}

%%\smallfig[0.4]{images/cropped_pdfs/bild_2_2-44.pdf}{DTL-logik}{fig:BildII2-44}

Bild \ssaref{fig:BildII2-44} visar en NAND-grind.
Den egentliga grinden består av tre dioder och en resistor.
Två av dioderna är ingångar och den tredje är utgång.
Grinden styr en digitalt arbetande transistor liksom den i bild
\ssaref{fig:BildII2-35}.
Resultatet är en så kallad DTL-logik (eng. \emph{Diode-Transistor Logic}).

\smallfig[0.4]{images/cropped_pdfs/bild_2_2-45.pdf}{TTL-logik}{fig:BildII2-45}

Bild \ssaref{fig:BildII2-45} visar en NAND-grind.
Här består den egentliga grinden av en ingångstransistor med två emittrar,
vilka motsvarar dioderna vid A och B i föregående bild.
Kollektorn i denna transistor motsvarar ingångsdioden till transistorn i bild
\ssaref{fig:BildII2-44}.
De övriga tre transistorerna i bild \ssaref{fig:BildII2-45} bildar en switch
(digital strömställare, jämför bild~\ssaref{fig:BildII2-35}), som ger snabb
övergång mellan väl definierade logiska nivåer.
Resultatet är en så kallad TTL-logik (eng. \emph{Transistor-Transistor Logic}).

% Avsnitt 2.9 IC
% Avsnitt 2.10 Operationsförstärkare
\newpage
\section{Integrerade kretsar (IC)}
\label{integrerade kretsar}

\subsection{Allmänt om IC}
\index{integrerad krets}
\index{IC}

Att integrera betyder att samla till en enhet, det kan vara komponenter,
funktioner eller verksamheter.
Integration kan ske på olika nivåer och i många olika sammanhang.

Med integration avses här integration av komponenter för elektroniska
strömkretsar.
Särskilt halvledarelement av olika slag samt resistorer och kondensatorer med
små värden kan framställas med små dimensioner.
Många komponenter kan då samlas i samma hölje.

Komponenter inom ett hölje, avsedda för en viss funktion kallas
\emph{integrerad krets} (eng. \emph{Integrated Circuit -- IC}).

Komponenterna i en IC kan i sin tur vara del av komponenterna en hel strömkrets.
Redan inom höljet kan komponenter kopplas samman för en viss funktion eller som
en del av strömkretsen.
Skrymmande eller effektkrävande komponenter, såsom induktorer, transformatorer
och så vidare får emellertid inte plats, varför även yttre kopplingar behövs.
Det kan också behövas flera IC i en strömkrets -- kanske med innehåll för en
annan funktion.

En integrerad krets är uppbyggd på en basplatta av halvledarmaterial -- ett
chipp.
På plattan framställs, med fototeknik eller etsning, kompletta eller nästan
kompletta dioder, transistorer, resistorer och kondensatorer.
Metoden, som kallas planarteknik, medger att många komponenter kan få plats på
samma platta.

\subsection{Olika slags integrerade kretsar}

Det finns stora sortiment av både standardiserade och speciella IC, varav det
finns två huvudtyper:
\begin{itemize}
  \item digitala integrerade kretsar
  \item analoga integrerade kretsar.
\end{itemize}

\subsection{Digitala IC}

Digitala IC arbetar som framgår av namnet med digitala signalnivåer.
De enklaste typerna innehåller en eller flera digitala grindar (se avsnitt
\ssaref{digitala kretsar}).
Genom att koppla samman grindar kan man skapa kretsar för ett visst ändamål.
I början av 70-talet byggdes komplicerade system av grindar i SSI- och
MSI-teknik.
Ett sådant system är emellertid inte flexibelt eftersom eventuella ändringar
måste göras ''hårdvarumässigt''.
Det innebär att kopplingsledningar måste ändras, kanske hela kretsar bytas ut
och så vidare.

I dagens digitala system används IC i form av en mikroprocessor eller till och
med flera.
En mikroprocessor är en avancerad krets som kan programmeras (konfigureras)
mjukvarumässigt inte bara för ett ändamål utan för många olika.
I system med mikroprocessorer behövs också minnesfunktioner.
Mikroprocessorn är hjärtat i en dator.
Styrd av ett program (mjukvaran) kontrollerar den kringutrustningar med uppgift
att inhämta och avge information -- att kommunicera.

\subsection{Analoga IC}

Analoga IC arbetar med analoga signalnivåer, det vill säga spänningar och
strömmar med kontinuerligt varierande nivåer och frekvenser.
En analog IC kan även arbeta med digitala signaler.

Analoga IC innehåller en eller flera balanserade förstärkare samt olika slags
hjälpkretsar.
Med yttre komponenter kan en analog IC ges olika förstärkning och frekvensgång.
Med ett gemensamt namn kallas dessa förstärkare för operationsförstärkare
(OP-amp).
Operationsförstärkare utförs vanligen i SSI- eller möjligen MSI-teknik.

\subsection{Kombinerade och speciella IC}

Utöver standardiserade IC finns kombinerade och speciella IC.
Exempel på speciella digitala IC är sådana för telekommunikationsändamål.
Ett annat exempel på digitala IC är sådana för signalbehandling, såväl på HF
som LF-nivå.
Exempel på speciella analoga IC är sådana för radiokommunikationsändamål.

Bortsett från vissa skrymmande komponenter och manöverdon kan numera till
exempel en IC innehålla en komplett radiomottagare.
Ett annat exempel på speciella analoga IC är sådana för hörapparater.
Genom programmering anpassas de för det personliga behovet.

\subsection{Utvecklingen}

Det kan sägas hur ofta som helst.
Genom den fantastiska utvecklingen av mikroelektronik öppnas även för
radioamatören möjligheter som tidigare inte var tänkbara.

Denna utveckling har vidgat utrymmet för den experimentella verksamhet som
amatörradio i grunden innebär.
Hobbyn får sålunda med tiden en allt större teknisk spännvidd.

\subsection{Aktuell litteratur}

Ökat teknikomfång inom amatörradio ställer motsvarande krav på litteratur.
På senare tid inbegripes även digitalteknik.
Mest av utrymmesskäl behandlas i denna faktabok digitaltekniken mycket
kortfattat, men ändå så mycket som nämns i CEPT-rekommendationen T/R 61-02.
För djupare studium hänvisas till andra läromedel samt till
leverantörskataloger.

\section{Operationsförstärkare}
\harecsection{\harec{a}{2.8.3}{2.8.3}}
\index{operationsförstärkare}
\index{op-amp}
\label{op-amp}

\emph{Operationsförstärkare} (eng. \emph{operational amplifier}), ofta kallad
för \emph{op-amp} är en integrerad kretstyp som har hög förstärkning.
Istället för att ha enbart en ingång så har den två, en positiv
och en negativ, och operationsförstärkaren förstärker skillnaden mellan den
positiva och negativa signalen. Förstärkningen i en modern
operationsförstärkare kan vara i storleksordningen en miljon gånger.
De två mest grundläggande kopplingarna är komparator respektive negativt
återkopplad förstärkare.

\subsection{Komparator}

I en \emph{komparator} används den höga förstärkningen för att få även små
spänningsskillnader att ge ett stort utslag.
Med referensspänningen på den negativa ingången och insignalen på den positiva
ingången, kommer utgången att vara så hög den kan vara när ingången har högre
spänning än referensspänningen.
Omvänt kommer den vara så låg den kan vara när spänningsnivån på ingången är
lägre än referensspänningen.

Det är enkelt att ändra utgångens egenskaper genom att växla signaler mellan
positiv och negativ ingång på operationsförstärkaren.

\subsection{Negativ återkoppling och förstärkare}

En operationsförstärkare som har en negativ återkoppling, det vill säga där
signal från utgången matas tillbaka till den negativa ingången, kommer att
försöka driva utgången så att spänningsskillnaden mellan den positiva och
negativa ingången jämnas ut.
Det finns en rik uppsättning kopplingar som bygger på denna jämvikt, där
operationsförstärkaren arbetar i ett linjärt driftsområde.

Denna jämvikt gör också att en snabb diagnosticering kan göras genom att mäta
spänningen mellan ingångarna.
Om spänningen ligger nära noll fungerar kopplingen förmodligen.
Men om kopplingen är felaktig på något sätt, exempelvis om själva
operationsförstäkaren eller någon komponent i återkopplingen är trasig, kommer
spänningen vara synbart annorlunda och jämvikten finns inte.

\subsubsection{Buffertförstärkare}

Den enklaste linjära kopplingen med en operationsförstärkare är en
buffertförstärkare.
I denna koppling är den negativa ingången direkt kopplad till utgången och
insignalen är kopplad till den positiva ingången.
Med denna koppling kommer operationsförstärkaren försöka få den negativa
ingången, och därmed även utgången, att följa insignalen.
En sådan koppling, där vi får samma spänning på utgången som vi har på
ingången, kallas för en spänningsföljare.
Fördelen med en spänningsföljare är att lasten på utgången kan vara oerhört
mycket högre än vad insignalen skulle kunna driva.
Om utgångsnivån skulle sjunka beroende på lasten, försöker återkopplingen driva
den tillbaka till rätt spänning.

Buffertförstärkaren gör att operationsförstärkaren kan leverera samma spänning
ut, men mot en last på bara något fåtal ohm, det vill säga med mycket större
strömstyrka.
Den relativt höga ingångsimpedansen hos operationsförstärkaren, uppemot en
teraohm, gör att drivsignalen inte påverkas av en lågohmig last.

\subsubsection{Positiv (icke-inverterande) förstärkning med op-amp}
\label{icke-inverterande förstärkning}

En enkel variant av buffertförstärkaren fås när man kopplar in en
spänningsdelare mellan utgången och den negativa ingången, så som illustreras i
bild~\ssaref{fig:BildII2-46}.

\smalltikz{
 \begin{circuitikz}
   \draw
   % op-amp
   (0, 0) node[op amp,yscale=-1] (opamp) {}
   
   % Input through R1 to op-amp -
   (opamp.+) to (-2, 0.5) to[short,o-] (-2, 0.5)
   
   % Feedback resistor and output
   (opamp.-) to[short,-] ++(0,-1) coordinate (leftC)
   to[R=$R_2$] (leftC -| opamp.out)
   to[short,-*] (opamp.out) to [short,-o] (1.5,0)

   % Opamp - ground reference
   (opamp.-) -- (-1.192,-1.5) to [R,l=$R_1$] (-1.192,-3) to (-1.192,-3) node[ground]{}
   ;
 \end{circuitikz}
}{Icke inverterande förstärkare}{fig:BildII2-46}

Om förhållandet i spänningsdelaren är 1:10 kommer spänningen på den negativa
ingången att vara en tiondel av spänningen på utgången.
För att behålla jämvikten mellan den positiva och den negativa ingången, kommer
operationsförstärkaren att driva utgången till tio gånger nivån på den positiva
ingången.
Genom att variera förhållandet i spänningsdelaren kan man kontrollera
förstärkningen hos kretsen.
Förstärkningen blir:
%%
\[G = 1+ \dfrac{R_2}{R_1}\]
%%
Genom att koppla en kondensator parallellt över återkopplingsmotståndet
(\(R_2\) i bild~\ssaref{fig:BildII2-46}) kan man skapa en bandbreddsbegränsning
för förstärkaren.
För de högre frekvenserna kommer merparten av strömmen att gå genom
kondensatorn och återkopplingen blir därför frekvensberoende.
Förstärkningen för höga frekvenser sänks mot samma nivå som för en
bufferförstärkare.

Detta är också ett sätt att undvika att kretsen självsvänger vid höga
frekvenser.

\subsubsection{Negativ (inverterande) förstärkning med op-amp}
\label{inverterande förstärkning}
\label{virtuell jord}
\label{jordning!virtuell}

Kopplingen i bild~\ssaref{fig:BildII2-47} ger en negativ förstärkning.

\smalltikz{
  \begin{circuitikz}
    \draw
    % op-amp
    (0, 0) node[op amp] (opamp) {}
    
    % Input through R1 to op-amp -
    (opamp.-) to[R,l_=$R_1$] (-3, 0.5) to[short,o-] (-3, 0.5)
    
    % Feedback resistor and output
    (opamp.-) to[short,*-] ++(0,1) coordinate (leftC)
    to[R=$R_2$] (leftC -| opamp.out)
    to[short,-*] (opamp.out) to [short,-o] (1.5,0)

    % Opamp + ground reference
    (opamp.+) -- (-1.2,-0.5) to (-1.2,-1.0) node[ground]{}
    ;
  \end{circuitikz}
}{Inverterande förstärkare}{fig:BildII2-47}

Operationsförstärkaren kommer att balansera den negativa ingången så att den
är på samma potential som jord.
Detta kallas för \emph{virtuell jord}.
Strömmen kommer att gå från ingången till utgången, men ingången kommer se
lasten från ingångsmotståndet \(R_1\) och utgången kommer att mata \(R_2\) mot
jord.
Förstärkningen kommer att vara negativ och proportionell mot kvoten mellan
motståndsvärdena:
%%
\[G = -\dfrac{R_2}{R_1}\]

% Avsnitt 2.11 Värmeutveckling
\section{Värmeutveckling}

\harecsection{\harec{a}{2.7}{2.7}}

\index{värmeutveckling}
\index{heat dissipation}

\subsection{Värmeledning}

\harecsection{\harec{a}{2.7.1}{2.7.1}}

\index{värmeledning}
\index{heat transfer}
\index{termisk resistans}
\index{symbol!\(R_\Theta\) termisk resistans}
\index{omgivande temperatur}
\index{ambient temperature}
\index{symbol!\(T_A\) ambient temperature}
\index{elsäkerhet}
\index{kallödning}

Vi har tidigare betraktat Joules lag för effektutveckling i motstånd.
Det är dags att börja utveckla en lite mer komplett syn på värmeutveckling.
Ett motstånd som utvecklar 1~watt kommer stiga i temperatur till dess att
jämvikt uppstår mellan motståndets förmåga att avleda värme och
omgivningstemperaturen.

\emph{Termisk resistans} (eng. \emph{thermal resistance}) är ett mått på
hur bra ett material är på att leda värme. Den betecknas med symbolen \(R_\Theta\),
och anges i enheten kelvin per watt.
Temperaturen \(T_k\) för en komponent beror på medeleffekten \(P\) som den
producerar i värme, den termiska resistansen samt den \emph{omgivande temperaturen}
(eng. \emph{ambient temperature}) \(T_A\) enligt:
%%
\[T_k = T_A + R_\Theta \cdot P\]
%%
De termiska resistanserna för komponent, kylpasta, isolerskiva och kylfläns
kan summeras precis som resistanser för vanliga motstånd och det
sammanlagda värdet används sedan för att beräkna temperaturen på en
komponent eller för att dimensionera en kylfläns.

\subsection{Konvektion}
\harecsection{\harec{a}{2.7.2}{2.7.2}}
\index{konvektion}
\index{kylfläns}
\index{heatpipe}

\emph{Konvektion} (eng. \emph{convection}) är när värme skapar ett
naturligt flöde i vätska eller gas, oftast luft. När luft värms upp 
vill den expandera, varvid densiteten sjunker och luften vill stiga uppåt.
Kallare luft strömmar då till och kan därmed kyla värmekällan.
En stor temperaturskillnad medför att konvektionen ökar och innebär därmed en
bättre kylning.

För exempelvis transistorer kan värmealstringen ske på en sådan liten yta att
konvektion från komponenten inte räcker för att kyla bort den producerade
värmen.
Därför monterar man dem på en \emph{kylfläns} (eng. \emph{heat sink}) som
fördelar värmen över en större yta så att verkan av konvektion ökar.

En effektiv metod för att transportera värme är via en så kallad \emph{heat pipe}.
Det är ett rör innehållande en vätska som förångas vid en temperatur strax över
rumstemperatur och som då effektivt leder överskottsvärme till ett plats där den
kan kylas bort.
Heat pipe används numera ofta i datorer och solfångare.

Om värme produceras på en liten yta kan man behöva hjälpa konvektionen, vilket
ibland kallas för \emph{forcerad konvektion} (eng. \emph{forced convection}).
Med hjälp av en fläkt blåses luft mot eller sugs förbi kylflänsen vilket ökar
värmeutbytet.
Eftersom fläktar skapar oljud brukar man försöka anpassa fläktens varvtal i
förhållande till temperaturen, men även en variation av varvtal kan uppfattas
som störande.
Andra åtgärder för att minska ljudnivån är att skapa släta ytor för luften så
det inte bildas luftvirvlar eller att styra in- och utgående luftflöde med
bafflar.

Ett problem som kan uppstå är att utrustning som är gjord för självkonvektion
blir placerad eller monterad så att luft inte kan flöda fritt runt
utrustningen. Detta kan leda till överhettning på motsvarande sätt som när
en fläkt för forcerad kylning går sönder. Dålig termisk kontakt mellan
transistor och kylfläns är ett annat exempel på hur dålig värmeledning skapar
problem med överhettning.

\subsection{Värmealstring}
\harecsection{\harec{a}{2.7.4}{2.7.4}}
\index{värmealstring}

\emph{Värmealstring} kan ske på fler ställen än i motstånd. Lite förenklat
kan man säga att alla komponenter har förluster som producerar värme. Genom
lämpligt val av komponenter och korrekt dimensionering kan vi undvika att
producera onödiga värmeförluster. Kraftaggregat och effektsteg är exempel
på apparater där det går större strömmar vilka ofrånkomligen också alstrar
mera värme. Lägre förluster skapar man genom att helt enkelt ha bättre
ledningsförmåga, lägre resistans.

Även halvledare skapar värme, och även här gäller Joules lag med spänning
gånger ström. I exempelvis ett effektsteg kommer transistorn utveckla en
effekt motsvarande spänningen över transistorn gånger strömmen genom den.
Onödigt hög spänning och ström skapar högre värmeutveckling, vilket är en
anledning till att man gärna undviker slutsteg som arbetar i klass A till fördel
för slutsteg som arbetar i klass AB, B eller C.

Bristande värmeavledning leder ofta till katastrofala fel, som till exempel
sönderbrända motstånd och transistorer. Även ledare kan brinna av när man
har för liten ledararea, och därmed för hög resistans för den ström som
ska gå genom den. Av det skälet finns dimensioneringsregler, till exempel
krav på minsta arean av koppar i ledare, helt enkelt för att det inte ska
uppstå brand.

En annan effekt av värmeledning är att det kan ibland vara svårt att löda
på kretskort, framför allt vid ledare som går mot stora kopparytor som
har en relativt god värmeledningsförmåga. Ibland konstruerar man små mönster
''thermals'' runt sådana lödpunkter för att minska värmeavledningen.
Ett effektivt sätt att kunna löda och framförallt löda av från sådana
kort är att man förvärmer hela kretskortet eller området runt om
lödpunkten. Då kommer temperaturskillnaden mellan lödpennans spets och
omgivningen att minska och det krävs inte lika stor effekt för att få
upp lödpunkten i rätt temperatur för att kunna genomföra lödningen med god
\emph{vätning} och därmed undvika att det bildas en kallödning.

\subsection{Värme i transistor}
\harecsection{\harec{a}{2.7.3}{2.7.3}}

\mediumtopfig{macros/bild_tx_heat.pdf}{Värmealstring i en transistor. Transistorspänning $U_{ce}$ och transistoreffekt $P_t$ varierar med vinkeln hos sinussignalen för resistiv last.}{fig:power1}

För att förstå värmealstring i en transistor börjar vi med att titta
på effektförbrukningen, \(P_t\), i en NPN-transistor som kan beskrivas
som

\[P_t \approx U_{be}\cdot I_b + U_{ce}\cdot I_c,\]
%
där \(U_{be}\) är spänningen från bas till emittor, \(I_b\)
strömmen genom bas,  \(U_{ce}\) spänningen från kollektor till
emittor, och \(I_c\) strömmen genom kollektor.
Oftast är strömmen genom basen försumbar.

För att gå lite djupare tänker vi oss ett exempel med en transistor i
enkel Klass A förstärkarkrets, se kapitel~\ssaref{klassabc}.
Transistorn har en \qty{12}{\volt} matningsspänning och har en
vilospänning på \qty{6}{\volt} för att få marginal mot \qty{0}{\volt}
och \qty{+12}{\volt}.
Den genererar en sinussignal med topp-till-topp-värdet \(10\ V_{pp}\)
in i en resistiv last.

Spänningen kollektor till emittor, \(U_{ce}\), och strömmen, \(I_c\),
är 180 grader ur fas så den resulterande effekten, \(P_t\), har sina
maxvärden mellan spänningens max- och minvärden, se bild
\ssaref{fig:power1}.

%
%
% Kapitel 3 Kretsar
\chapter{Kretsar}
\label{ch:kretsar}
\index{kretsar}

När flera likadana eller olika komponenter kopplas ihop bildas en elektrisk
krets.
Den sammansatta funktion i en krets är beroende på hur de enskilda komponenterna
påverkar varandra.
 
% Avsnitt 3.1 Serie och parallellt
\chapter{Kretsar}

\section[Serie och parallellt]{Komponenter i serie och parallellt}

\subsection{Seriekopplade resistorer}
\harecsection{\harec{a}{3.1.1a}{3.1.1a}, \harec{a}{3.1.2}{3.1.2a}, \harec{a}{3.1.3}{3.1.3a}}
\index{resistor!seriekopplade}
\index{seriekoppling!resistorer}
\label{seriekopplade_resistorer}

\smallfig{images/cropped_pdfs/bild_2_3-01.pdf}{Seriekopplade resistorer}{fig:BildII3-01}

Bild \ssaref{fig:BildII3-01} visar seriekopplade resistorer.
Den totala resistansen av seriekopplade resistorer är summan av resistanserna.

\[R = R_1 + R_2 + R_3 \cdots \]

Strömmen är lika stor genom alla seriekopplade resistorer i strömvägen (ingen
avgrening).

\[I = I_1 = I_2 = I_3 \cdots \]

Den totala spänningen över seriekopplade resistorer är summan av spänningen över
var och en av dem.

\[U = U_1 + U_2 + U_3 \cdots \]

Spänningen över var och en av seriekopplade resistorer förhåller sig som deras
resistanser. För två resistorer gäller

\[\dfrac{U_1}{U_2} = \dfrac{R_1}{R_2}\]


\subsection{Parallellkopplade resistorer}
\harecsection{\harec{a}{3.1.1b}{3.1.1b}}
\index{resistor!parallellkopplade}
\index{parallellkoppling!resistorer}
\label{parallellkopplade_resistorer}

\smallfigpad{images/cropped_pdfs/bild_2_3-02.pdf}{Parallellkopplade resistorer}{fig:BildII3-02}

Bild \ssaref{fig:BildII3-02} visar parallellkopplade resistorer.
Den totala resistansen av parallellkopplade resistorer är lägre än den lägsta
enstaka resistansen.

\[
\frac{1}{R} = \frac{1}{R_1} + \frac{1}{R_2} +
\frac{1}{R_3} + \frac{1}{R_4} + \cdots \frac{1}{R_n}
\]

För två parallellkopplade resistorer gäller

\begin{align*}
\frac{1}{R} &= \frac{1}{R_1} + \frac{1}{R_2} && eller \\
R &= \frac{R_1 \cdot R_2}{R_1 + R_2}
\end{align*}

Strömmen förgrenar sig mellan parallellkopplade resistorer.
Den totala strömmen är summan av grenströmmarna

\( I = I_1 + I_2 + \cdots I_n \)

Spänningen är lika stor över resistorerna

\(U = U_1 = U_2 = U_3 = \cdots U_n \)

Grenströmmarna genom parallellkopplade resistorer fördelar sig omvänt
proportionellt mot deras respektive resistanser.
För två resistorer gäller

\[\frac{I_1}{I_2} = \frac{R_2}{R_1}\]

\subsection{Spänningsdelare}
\index{spänningsdelning}
\label{spänningsdelare}

\smallfig{images/cropped_pdfs/bild_2_3-03.pdf}{Resistiv spänningsdelare}{fig:BildII3-03}

Spänningsdelare förekommer i flera former.
Bild \ssaref{fig:BildII3-03} visar en spänningsdelare med resistorer där
spänningen \(U\) delas upp i spänningen \(U_1\) över resistorn \(R_1\)
respektive \(U_2\) över \(R_2\).

Ett alternativ till spänningsdelning med fasta resistorer är \emph{potentiometern}. 
Den är en variabel spänningsdelare i form av en resistor med ett flyttbart uttag.

Om man ansluter en apparat parallellt över \(R_2\), till exempel ett instrument
vars inre resistans motsvaras av \(R_y\), kommer spänningarna över \(R_1\)
och \(R_2\) att påverkas.

Om \(R_y\) är mycket större än \(R_2\), kan man bortse från påverkan.
För att beräkna \(U_2\) kan man använda följande formel för en obelastad
resistiv spänningsdelare.

\begin{align*}
\frac{U_2}{R_2} &= \frac{U}{R_1 + R_2} \quad \text{eller} \\
U_2 &= U \cdot \frac{R_2}{R_1 + R_2}
\end{align*}

Om \(R_y\) däremot är av samma storleksordning som \(R_2\) eller lägre, är det lämpligt att först räkna ut resistansen \(R_p\) i parallellkretsen 

\[
R_p = \frac{R_2 \cdot R_y}{R_2 + R_y}
\]

och därefter räkna ut spänningen \(U_2\) 

\[
U_2 = U \cdot \dfrac{ R_p }{ R_1 + R_p } = U \cdot \dfrac{ \dfrac{R_2 \cdot R_y}{R_2 + R_y} }{ R_1 + \dfrac{R_2 \cdot R_y}{R_2 + R_y} }
\]

Härav förstås att till exempel en spänningsmätning ger olika resultat beroende
på den inre resistansen i voltmetern.

\subsection{Wheatstones brygga}
\index{Wheatstones brygga}

\smallfig{images/cropped_pdfs/bild_2_3-04.pdf}{Wheatstones brygga}{fig:BildII3-04}

En speciell tillämpning av spänningsdelare är en \emph{Wheatstones brygga}, se
bild \ssaref{fig:BildII3-04}, som används för att jämföra spänningar.

Bryggan kan ses som två parallellkopplade spänningsdelare varav den ena är en
potentiometer med en skala graderad till exempel i \unit{\ohm}.
Den andra spänningsdelaren består av en resistor med känd resistans och en
resistor med okänd resistans, det vill säga mätobjektet.

I ledningen som förbinder de respektive mittuttagen X och Y, finns en
amperemeter som nollströmsindikator.

Det flyter ström mellan X och Y när det finns en potentialskillnad -- spänning
-- däremellan.
Bryggan är då i obalans.
Det flyter däremot ingen ström där när det inte finns en potentialskillnad,
det vill säga när bryggan är i balans.
Balans (mätvärdet) får man genom justering av den graderade potentiometern
till noll ström.
Då gäller sambandet

\[\frac{R_1}{R_2} = \frac{R_3}{R_4}\]

Exemplen med spänningsdelare och bryggor visar att apparater påverkar varandra
när de kopplas samman, vilket är fallet vid mätningar.

Spänningsdelning kan även utföras med kondensatorer och induktorer förutsatt att
det är fråga om en växelströmskrets.

\subsection{Parallellkopplade kondensatorer}
\harecsection{\harec{a}{3.1.1f}{3.1.1f}}
\index{kondensator!parallellkopplade}
\index{parallellkoppling!kondensatorer}
\label{parallellkopplade kondensatorer}

\smallfig[0.3]{images/cropped_pdfs/bild_2_3-05.pdf}{Parallellkopplade kondensatorer}{fig:BildII3-05}

Bild \ssaref{fig:BildII3-05} visar parallellkopplade kondensatorer.
I stället för att använda en enda kondensator kan man parallellkoppla flera
kondensatorer för att uppnå önskad total kapacitans.

Den totala kapacitansen för parallellkopplade kondensatorer är summan av de
enskilda kapacitanserna.
%
\[
  C = C_1 + C_2 + C_3 + \cdots C_n
\]
%
%\begin{minipage}{\columnwidth}
\textbf{Räkneexempel:}
\begin{enumerate}
\item \(C_1 = \qty{5}{\micro\farad} \quad C_2 = \qty{10}{\micro\farad} \quad C =\ ?\)
  \begin{align*}
    C &= C_1 + C_2 \\[1ex]
    &= 5 + 10 \\[1ex]
    &= \qty{15}{\micro\farad}
  \end{align*}
\item \(C_1 = \qty{1}{\nano\farad} \quad C_2 = \qty{5}{\pico\farad} \quad C =\ ?\)
  \begin{align*}
    C &= C_1 + C_2 \\[1ex]
    &= 1 + 0,005 \\[1ex]
    &= \qty{1,005}{\nano\farad}
  \end{align*}
\end{enumerate}
%\end{minipage}

\subsection{Seriekopplade kondensatorer}
\harecsection{\harec{a}{3.1.1e}{3.1.1e}}
\index{kondensator!seriekopplade}
\index{seriekoppling!kondensatorer}
\label{seriekopplade_kondensatorer}

\smallfig[0.3]{images/cropped_pdfs/bild_2_3-06.pdf}{Seriekopplade kondensatorer}{fig:BildII3-06}

Bild \ssaref{fig:BildII3-06} visar seriekopplade kondensatorer.
Den totala kapacitansen för seriekopplade kondensatorer är lägre än kapacitansen
för kondensatorn med det minsta värdet.
%
\[
\frac{1}{C} = \frac{1}{C_1} + \frac{1}{C_2} +
\frac{1}{C_3} + \cdots \frac{1}{C_n}
\]
%
För två kondensatorer gäller:
%%
\[
  \frac{1}{C} = \frac{1}{C_1} + \frac{1}{C_2} \quad \text{eller} \quad
  C = \frac{C_1 \cdot C_2}{C_1 + C_2}
\]
%%
\noindent
\textbf{Räkneexempel:}

 \(C_1 = \qty{5}{\micro\farad} \quad C_2 = \qty{10}{\micro\farad} \quad C =\ ?\)
    \begin{align*}
      \frac{1}{C} &= \frac{1}{C_1} + \frac{1}{C_2} \\[1ex]
      C &= \frac{C_1 \cdot C_2}{C_1 + C_2}
      = \frac{5 \cdot 10}{5 + 10}
      = 3\frac{1}{3}\
      \approx \qty{3,33}{\micro\farad}
    \end{align*}

\subsection{Galvaniskt kopplade induktorer}
\label{galvaniskt_kopplade_induktorer}

Induktansvärdet för galvaniskt sammankopplade induktorer kan i princip
beräknas på samma sätt som för motsvarande sammankoppling av resistorer.

\subsubsection{Galvaniskt seriekopplade induktorer}
\harecsection{\harec{a}{3.1.1c}{3.1.1c}}
\index{induktor!seriekopplade}
\index{seriekoppling!induktorer}

Förutsatt att magnetfälten från de respektive induktorerna inte återverkar på
varandra -- det vill säga inte ''kopplar magnetiskt till varandra'' -- så
gäller:
%
\[L = L_1 + L_2 + L_3 + \cdots L_n\]
%
\newpage
\noindent
\textbf{Räkneexempel:}

  \(L_1 = \qty{20}{\milli\henry} \quad L_2 = \qty{50}{\milli\henry} \quad L =\ ?\)
\begin{align*}
  L = L_1 + L_2 = 20 + 50 = \qty{70}{\milli\henry}
\end{align*}

\subsubsection{Galvaniskt parallellkopplade induktorer}
\harecsection{\harec{a}{3.1.1d}{3.1.1d}}
\index{induktor!parallellkopplade}
\index{parallellkoppling!induktorer}

Förutsatt att magnetfälten från de respektive induktorerna inte återverkar på
varandra -- det vill säga inte ''kopplar magnetiskt till varandra'' -- så
gäller:

\[
\frac{1}{L} = \frac{1}{L_1} + \frac{1}{L_2} + \frac{1}{L_3} +
\cdots \frac{1}{L_n}
\]

För två induktorer gäller:

\begin{align*}
  \frac{1}{L} &= \frac{1}{L_1} + \frac{1}{L_2} \quad \text{eller} \\
  L &= \frac{L_1 \cdot L_2}{L_1 + L_2}
\end{align*}

\textbf{Räkneexempel:}

\[L_1 = \qty{50}{\milli\henry} \quad L_2 = \qty{60}{\milli\henry} \quad L =\ ?\]
\[
  L = \frac{L_1 \cdot L_2}{L_1 + L_2}
  = \frac{50 \cdot 60}{50 + 60}
  = \frac{3000}{110}
  \approx \qty{27}{\milli\henry}
\]

\subsection{Magnetiskt kopplade induktorer}
\index{induktor!magnetiskt kopplade}
\index{magnetisk koppling!induktorer}
\index{ömsesidig induktans}

\smallfig[0.35]{images/cropped_pdfs/bild_2_3-07.pdf}{Magnetiskt kopplade induktorer}{fig:BildII3-07}

I praktiken anordnas ofta induktorer så, att deras respektive magnetfält kan
återverka på varandra -- så kallad magnetisk koppling.

En \emph{ömsesidig induktans} \(M\) uppstår i induktorerna på grund av denna
koppling.
Den ömsesidiga induktansen ökar eller minskar det resulterande induktansvärdet
beroende på om induktorernas magnetfält verkar med eller mot varandra.

Beräkningen av värdet på \(M\) är emellertid relativt komplicerad och behandlas
ej här.
I stället görs en förenklad framställning.

Bild \ssaref{fig:BildII3-07} visar seriekopplade induktorer, vars magnetfält
kopplar till varandra på olika sätt.
''Pricken'' vid änden av induktorerna på bilden markerar magnetfältens inbördes
polarisering.

\subsubsection{Magnetiskt kopplade induktorer i serie}

\textbf{Formel:}
%
\[L = L_1 +L_2 \pm 2M\]
%
\textbf{Räkneexempel:}
Två induktorer har en induktans av 20 respektive \qty{10}{\micro\henry} och en
ömsesidig induktans av \qty{2}{\micro\henry}.
Induktorerna är kopplade och placerade så att deras magnetfält samverkar.

Vardera induktansen ökas därför med \(M = \qty{2}{\micro\henry}\).
\[
\begin{array}{rcl}
  L &=& L_1 + M + L_2 + M \\
    &=& 20 + 2 + 10 + \qty{2}{\micro\henry} \\
    &=& \qty{34}{\micro\henry}
\end{array}
\]
\textbf{Räkneexempel:}
Två induktorer har en induktans av 20 respektive \qty{10}{\micro\henry} och en
ömsesidig induktans av \qty{2}{\micro\henry}.
Induktorerna är kopplade och placerade så att deras magnetfält motverkar varandra.
Vardera induktansen minskas därför med \(M = \qty{2}{\micro\henry}\).
\[
\begin{array}{rcl}
  L &=& L_1 - M + L_2 - M \\
    &=& 20 - 2 + 10 - \qty{2}{\micro\henry} \\
    &=& \qty{26}{\micro\henry}
\end{array}
\]
\subsubsection{Magnetiskt kopplade induktorer i parallell}
När flera induktorer är parallellkopplade och placerade så att deras magnetiska
fält interagerar behöver man ta hänsyn till om de samverkar eller motverkar varandra.
Mer läsning om induktorer och hur de påverkar varandra finns att läsa i
\cite{letrafo}.

\textbf{Formler:}

Samverkande parallella induktorer
%
\[L = \frac{L_1 \cdot L_2 - M^2}{L_1 + L_2 - 2M}\]
%
Motverkande parallella induktorer
%
\[L = \frac{L_1 \cdot L_2 - M^2}{L_1 + L_2 + 2M}\]
%
\subsection{Upp- och urladdning av en kondensator}

\subsubsection{Uppladdning}

\tallfig{images/cropped_pdfs/bild_2_3-08.pdf}{Uppladdning av en kondensator}{fig:BildII3-08}

Bild \ssaref{fig:BildII3-08} visar uppladdning av en kondensator.
En kondensator \(C\) seriekopplas med en resistans \(R\)
och kopplas till spänningen \(U\).

% PHU: Bilden förvirrar. U i kretsschemat är samma som U_max i diagrammet.

Spänningen över kondensatorn stiger från 0~volt till \(U_{\it max}\) samtidigt
som laddningsströmmen sjunker från \(I_{\it max}\) till 0~ampere.
Spänningen över kondensatorn ökar exponentiellt under uppladdningen
%
% \[u_c = U_{\it max} \cdot ( 1 - e^{-\dfrac{t}{\tau}} )\]
\[u_c = U_{\it max} \cdot ( 1 - e^{-t/\tau} )\]
%
%\begin{tabular}{lp{0.35\textwidth}}
där \(u_c\) är spänningen över kondensatorn efter en given inkopplingstid, \(U_{\it max}\)
är slutspänningen efter minst \(t = 5\tau\), \(t\) är inkopplingstiden, och \(e\) är basen
för den naturliga logaritmem.
%\end{tabular}

I förloppet ingår storleken av resistans och kapacitans enligt följande samband,
som kallas tidskonstant:

\begin{gather*}
  \tau = R \cdot C \\
  \tau\ [\text{tidskonstant i sek}] \\
  C\ [\text{F}] \quad R\ [\unit{\ohm}]
\end{gather*}

Efter tiden \(t = 1\tau\) från inkopplingsögonblicket har spänningen över
kondensatorn ökat från noll till 63~\% av maxvärdet.
Efter tiden \(t = 5\tau\) är kondensatorn uppladdad till 99~\%.

Strömmen från kondensatorn minskar exponentiellt under uppladdningen

% \[i_c = I_{\it max} \cdot e^{-\dfrac{t}{\tau}}\]
\[i_c = I_{\it max} \cdot e^{-t/\tau}\]

där
% \begin{tabular}{lp{0.35\textwidth}}
\(i_c\) är strömmen från kondensatorn efter en given inkopplingstid och 
\(I_{\it max}\) är begynnelseströmmen.
% \end{tabular}

Efter tiden \(t = 1\tau\) från inkopplingsögonblicket har strömmen till
kondensatorn minskat till 37~\% av maxvärdet.
Efter tiden \(t = 5\tau\) återstår 1~\% av strömmens maxvärde.

\subsubsection{Urladdning}

\tallfig{images/cropped_pdfs/bild_2_3-09.pdf}{Urladdning av en kondensator}{fig:BildII3-09}

Bild \ssaref{fig:BildII3-09} visar hur en kondensator C urladdas genom en resistor \(R_2\).

Spänningen över kondensatorn minskar exponentiellt under urladdningen.

% \[u_c = U_{\it max} \cdot e^{-\dfrac{t}{\tau}}\]
\[u_c = U_{\it max} \cdot e^{-t/\tau}\]

Strömmen från kondensatorn minskar exponentiellt under urladdningen.
Strömriktningen är motsatt den vid uppladdningen.

\[i_c = - I_{\it max} \cdot e^{-t/\tau}\]

Efter tiden \(t = 1\tau\) är kondensatorn urladdad så, att 37~\% av
\(I_{\it max}\) respektive \(U_{\it max}\) återstår.
Efter tiden \(t = 5\tau\) är kondensatorn urladdad så, att mindre än 1~\% av
\(I_{\it max}\) respektive \(U_{\it max}\) återstår.

Exempel på beräkning av tidskonstanten:
\begin{enumerate}
\item \(C = \qty{10}{\micro\farad} \quad R = \qty{1}{\kilo\ohm} \quad \tau =\ ?\)
  \begin{align*}
    \tau &= R \cdot C \\
    &= 1 \cdot 10^3 \cdot 10 \cdot 10^{-6} \\
    &= 10 \cdot 10^{-3} \quad \text{dvs var 1/100 sekund.}
  \end{align*}
\item \(C = \qty{1000}{\micro\farad} \quad R = \qty{1}{\kilo\ohm} \quad \tau =\ ?\)
  \begin{align*}
    \tau &= R \cdot C \\
    &= 1 \cdot 10^3 \cdot 10^3 \cdot 10^{-6} \\
    &= 1\ \text{sekund}
  \end{align*}
\end{enumerate}

\newpage
\subsection{In- och urkoppling av en induktor}

\subsubsection{Inkoppling}

\tallfig{images/cropped_pdfs/bild_2_3-10.pdf}{Inkoppling av en induktor}{fig:BildII3-10}

Bild \ssaref{fig:BildII3-10} visar inkopplingen av en induktor.
En induktor L i serie med en resistans R kopplas in över en likspänning U.
Spänningen över induktorn minskar från \(U_{\it max}\) till 0.

Strömmen genom induktorn ökar efter inkopplingen exponentiellt från 0 till \(I_{\it max}\)
%
\[i_L = I_{\it max} \cdot (1-e^{-t/\tau} )\]
%
där
%\begin{tabular}{lp{0.35\textwidth}}
\(i_L\) är strömmen efter en given inkopplingstid, 
\(I_{\it max}\) är slutströmmen efter minst \(t = 5\tau\),
\(t\) är inkopplingstiden, och
\(e\) är basen för den naturliga logaritmen).
% \end{tabular}

I förloppet ingår storleken av resistans och induktans enligt följande samband,
som kallas tidskonstant
%
\[\tau = \frac{L}{R}\]
%
\[
L\ [\text{H}] \quad
R\ [\unit{\ohm}] \quad
s\ [\text{sek}] \quad
\tau\ [\text{tidskonstant}]
\]

Efter en tid av \(t = 1\tau\) från inkopplingsögonblicket har strömmen genom
induktorn ökat från noll till 63~\% av \(I_{\it max}\) och spänningen över
induktorn minskat till 37~\% av maxvärdet.

\newpage
\subsubsection{Urkoppling}
\label{induktor_urkoppling}

Spänningskällan kopplas bort från samma induktor som ovan.
En resistor är inkopplad över induktorn.
Energin i induktorn avleds genom resistorn som en ström med motsatt riktning än vid inkopplingen.
Strömmen är vid urkopplingstillfället \(I_{\it max} = i_L\) och minskar därefter exponentiellt

\[i_L = I_{\it max} \cdot e^{-t/\tau}\]
\noindent där
%\begin{tabular}{lp{0.42\textwidth}}
%  \raggedright
  \(i_L\) är strömmen genom induktorn efter en given ur\-kop\-p\-li\-ngs\-tid, 
  \(I_{\it max}\) är strömmen i urkopplingsögonblicket,
  \(e\) är basen för den naturliga logarithmen, och\
  \(t\) är tiden efter urkopplingsögonblicket.
% \end{tabular}

\vspace*{1ex}

\noindent
Efter en tid av \(t = 1\tau\) från urkopplingsögonblicket har strömmen genom
induktorn minskat till 37~\% av maxvärdet.

Teoretiskt kan spänningarna och strömmarna aldrig nå ett noll- eller maxvärde,
men för praktiskt bruk anses detta inträffa efter en tid av minst \(5\tau\).

All den energi som lagras i en induktor finns i dess magnetfält.
När strömmen bryts eller minskas så återgår energin omedelbart till kretsen.
I en induktor kan det således inte finnas någon kvarstående energi, vilket
det däremot kan göra i en kondensator.

Under den tid som magnetfältet i en induktor avvecklas eller byggs upp, så
induceras en motspänning i den.
Denna spänning är högre än den som finns över induktorn innan strömmen bryts
eller ändras och är proportionell mot den hastighet som ändringen har.
När en en strömkrets med induktor bryts är det vanligt att det i brytögonblicket
bildas en gnista eller ljusbåge över brytarens kontakter.

Om induktansen är stor och kretsströmmen hög ska en stor mängd energi frigöras
på mycket kort tid.
Det är därför inte ovanligt att brytarkontakter bränns eller smälter.
I likströmskretsar kan gnistan eller ljusbågen minskas eller undertryckas genom
att en kondensator i serie med en resistor kopplas över kontaktstället.
Kondensatorn fångar upp en del av energin i induktorn och resistorn minskar
hastighetsändringen.

\subsection{Växelströmskretsar}
\harecsection{\harec{a}{3.1.1}{3.1.1h}, \harec{a}{3.1.2}{3.1.2b}, \harec{a}{3.1.3}{3.1.3b}}

\subsubsection{Komponentegenskaper vid växelström}

Inom radiotekniken används mycket ofta resonanskretsar (benämns även
svängningskretsar) bestående av kondensatorer och induktorer, som är kopplade i
serie eller parallellt med varandra.
När resonanskretsens egenfrekvens sätts lika med frekvensen på den signal som
tillförs kretsen, så får kretsen särskilda egenskaper som används på olika sätt.

För att förstå hur ''LC-kretsar'' fungerar beskrivs först hur de ingående
komponenternas resistans, induktans och kapacitans förhåller sig till varandra,
när de kombineras och kopplas till en växelströmkälla.

\mediumfig[0.8]{images/cropped_pdfs/bild_2_3-11.pdf}{Faslägen och effekter i L C-kretsar}{fig:BildII3-11}

Bild \ssaref{fig:BildII3-11} visar amplituden av spänning och ström vid ett
sinusformat förlopp samt den effekt som då utvecklas.
Tidsaxeln är graderad 0--360\degree~per period.

\textbf{Fall a:} Förloppen med en resistor R.

I en resistor följer ström- och spänningskurvorna varandra tidsmässigt, även
vid riktningsändring.
När kurvorna följs åt på det sättet sägs de vara i fas med varandra.

Effekt överförs från strömkällan till resistorn.
Den effekt som utvecklas i resistorn är, vid varje tidpunkt av perioden,
produkten av strömmen och spänningen just då.
Eftersom storheterna spänning och ström är antingen positiva eller negativa
samtidigt, blir produkten alltid positiv.
Det betyder att den effekt som utvecklas pulserar två gånger per period mellan
ett noll- och maxvärde.

\textbf{Fall b:} Förloppen med en induktor L.

I en induktor är utvecklingen av ström och spänning inte samtidig.
Vid inkopplingen stiger spänningen genast till maxvärdet medan strömmen stiger
långsammare och bygger under tiden upp ett magnetfält i induktorn och omkring
övriga ledare i kretsen.

Strömmen fördröjs alltså i förhållande till spänningen.
Eftersom kurvornas max- och nollvärden inträffar vid olika tidpunkter heter
det att de är \emph{ur fas} eller \emph{fasförskjutna}.

En växelström genom en ideal induktor är förskjuten \ang{90} \emph{efter}
spänningen.
Strömmen når toppvärdet vid tidpunkten \ang{90} av perioden, när spänningen nått
ner till noll.
När spänningen minskar sjunker strömmen och tar med sig energin i magnetfältet.
Först vid \ang{180}, när spänningen har nått maxvärdet åt andra hållet, ändrar
också strömmen riktning och bygger upp ett nytt magnetfält med motsatt
polaritet.

Effekt överförs från strömkällan till induktorn när ström och spänning har
samma riktning.
När ström och spänning har olika riktning försöker induktorn i stället
''ladda'' strömkällan med energi från sitt kraftfält.
Effekt pendlar mellan strömkällan och induktorn, varvid effekten i
ena riktningen är lika stor som i andra riktningen.

Sett över en hel period upphäver därför dessa effekter varandra.
Följden blir att en ideal induktor, i motsats till en resistor, inte förbrukar
någon aktiv effekt.
Man säger att en reaktans, här en induktor, arbetar med reaktiv effekt.

I praktiken har kretsen även en viss resistans.
Därför sätts reaktansens \ang{90} fasförskjutna ström samman med resistansens
\ang{0} fasförskjutna ström.
Resultatet blir en ström som är mindre än \ang{90} ur fas och det förbrukas då
en viss aktiv effekt i resistansen.

\textbf{Fall c:} Förloppen med en kondensator C.

Inte heller i en kondensator utvecklas ström och spänning samtidigt.
Efter inkopplingen laddar strömmen upp kondensatorn, det vill säga bygger upp
ett elektriskt fält med en viss potential (spänning).
Spänningen utvecklas långsammare än strömmen -- den blir \emph{fasförskjuten}.

Strömmen till (och från) en ideal kondensator är fasförskjuten \ang{90} före
spänningen.
När kondensatorn är kopplad till en växelströmskälla når strömmen toppvärdet vid
tidpunkten \ang{90} eller \ang{270} av perioden.
Spänningen passerar då i båda fallen värdet noll.
När spänningen minskar sjunker strömmen och tar energi ur det elektriska fältet.

Sedan strömmen passerat noll vid \ang{180} eller \ang{0}/\ang{360} bygger den
upp ett nytt elektriskt fält med motsatt polaritet.

Liksom med en induktor överförs effekt från strömkällan till kondensatorn när
ström och spänning har samma riktning.
När ström och spänning har olika riktning försöker kondensatorn i stället
''ladda'' strömkällan med energi.
Effekt pendlar mellan strömkällan och kondensatorn, varvid effekten i
ena riktningen är lika stor som i andra riktningen.

Sett över en hel period upphäver därför dessa effekter varandra.
Följden blir att en ideal kondensator, i motsats till en resistor, inte
förbrukar någon aktiv effekt.
Man säger då att en reaktans, här en kondensator, arbetar med \emph{reaktiv}
effekt.

I praktiken har kretsen även en viss resistans.
Därför sätts reaktansens \ang{90} fasförskjutna spänning samman med
resistansens \ang{0} fasförskjutna ström.
Resultatet blir en spänning som är mindre än \ang{90} ur fas och det
förbrukas då en viss aktiv effekt i resistansen.
Som framgår av bilden blir variationerna i tiden de omvända med kondensator
jämfört med induktor.

\subsection{Impedans}
\harecsection{\harec{a}{3.1.3}{3.1.3c}, \harec{a}{3.2.2}{3.2.2}}
\index{impedans}
\index{resistans}
\index{reaktans}
\label{impedans}

Liten ordlista:
\begin{description}
\item[Impedans] -- hindra (lat. impedire).
\item[Resistans] -- motstå (lat. resistere).
  Del av impedansen, kallas ibland ohmskt motstånd.
\item[Reaktans] -- återverka (lat. reagere).
  Del av impedansen, samlingsord för växelströmsmotstånd.
\item[Kapacitans] -- inrymma (lat. capax). Del av reaktansen.
\item[Induktans] -- införa (lat. inducere). Del av reaktansen.
\end{description}

Hittills har storheterna resistans, induktans och kapacitans behandlats var för
sig, men i praktiken förekommer de alltid tillsammans och kallas impedans.

Resistansen är i princip oförändrad vid ström- eller spänningsändringar.
Men när strömmen genom en ledare eller induktor, liksom spänningen över en
kondensator ändras, tillkommer en reaktans som motverkar förändringarna.

Reaktansen kan från fall till fall vara kapacitiv eller induktiv och ingår i
impedansen.
Om ingen reaktans finns, är impedansen lika med resistansen.

\smallfig{images/cropped_pdfs/bild_2_3-12.pdf}{Seriekrets av L+C+R}{fig:BildII3-12}

Bild \ssaref{fig:BildII3-12} visar en induktor, en kondensator och en resistor
som är kopplade i serie.
När man vill beräkna den resulterande impedansen i kretsen
(''totala växelströmsmotståndet''), måste man ta hänsyn till att komponenternas
spänningar eller strömmar inte är i fas med varandra.
De arbetar ju inte ''i takt''.

Att då addera maxvärdena ger fel resultat.
I stället söker man den så kallade resultanten av de olika vektorer som
motsvarar ström- och spänningsvärden.
Detta kan göras grafiskt eller beräknas.

\mediumfig{images/cropped_pdfs/bild_2_3-13.pdf}{Spänningar i seriekrets L+C+R}{fig:BildII3-13}

I bild \ssaref{fig:BildII3-13} tänker vi oss att vektorerna i systemet vrider sig moturs med vinkelhastigheten
\(\omega = 2\pi f\) där \(f\) är frekvensen.
Eftersom vektorerna har samma frekvens, så är vektorernas lägen inbördes samma.
Ögonblicksvärdet av respektive vektorer följer en sinuskurva.

Spänningsvektorn i den ''induktiva reaktansen'' ligger \ang{90} före strömmen
och spänningen i resistansen.
Spänningsvektorn i den ''kapacitiva reaktansen'' ligger \ang{90} efter
strömmen och spänningen i resistansen.
Vektorerna i dessa två reaktanser är således \(2 \cdot 90 = 180\degree\)
åtskilda, det vill säga motriktade.
Man säger att de är i motfas.

\mediumfig{images/cropped_pdfs/bild_2_3-14.pdf}{Impedansen och fasvinkeln i seriekrets L+C+R}{fig:BildII3-14}

I bild \ssaref{fig:BildII3-14} visas vektorerna för komponenterna i bild
\ssaref{fig:BildII3-12} samt hur man grafiskt bestämmer impedansen för dessa
vektorer.
Vidare får man fasvinkeln mellan impedansens och resistansens vektor, varav den
senare är den så kallade riktfasen för hela seriekretsen.

Resistansen ritas som en vektor \(R\), som riktas vågrätt mot höger.
Vektorns längd motsvarar resistansens storlek i ohm.

Den induktiva reaktansen ritas på liknande sätt med vektorn \(X_L\) lodrätt
uppåt.
Slutligen ritas den kapacitiva reaktansen \(X_C\) lodrätt neråt.

Man subtraherar de motverkande reaktiva vektorerna \(X_L\) och \(X_c\) från
varandra och avsätter resultatet \(X\) på den vertikala axeln, uppåt om \(X_L\)
är större och neråt om \(X_c\) är större.

Man låter nu vektorerna \(X\) och \(R\) bilda sidor i en rätvinklig rektangel.
Längden på rektangelns diagonal är den resulterande impedansen \(Z\).
Fasvinkeln mellan impedans och resistans kan också avläsas.

Eftersom vektordiagrammet bildar en rätvinklig triangel kan den resulterande
spänningen \(U\) i kretsen även beräknas med Pythagoras sats:
%
\[C^2 = A^2 + B^2 \quad eller \quad C = \sqrt{A^2 + B^2}\]
%
Tillämpad på ovanstående vektordiagram kan satsen skrivas som
%
\[U_{LCR}^2 = U_R^2 + ( U_L - U_C)^2\]
%
Termerna ersätts med följande ekvationer:
\[
\begin{array}{rcl}
  U_{LRC} &=& I \cdot Z \\[1ex]
  U_R     &=& I \cdot R \\[1ex]
  U_L     &=& I X_L = I \omega L \\[1ex]
  U_C     &=& I X_C = I \frac{1}{\omega C} \\[1ex]
  I^2 Z^2 &=& I^2 R^2 + ( I \omega L - I\frac{1}{\omega C})^2
\end{array}
\]
%
Efter division med \(I^2\) fås
%
\[
\begin{array}{rcl}
  Z^2 &=& R^2 + ( \omega L - \frac{1}{\omega C} )^2 \quad \text{eller} \\[1ex]
  Z   &=& \sqrt{R^2 + (\omega L - \frac{1}{\omega C})^2} \quad \text{eller} \\[1ex]
  Z   &=& \sqrt{R^2 + (X_L - X_C)^2}
\end{array}
\]
I en seriekrets är den resulterande reaktansen negativ (kapacitiv) om \(X_C\) är
större än \(X_L\) och positiv (induktiv) om \(X_L\) är större än \(X_C\).

\subsection{Ohms lag vid växelström}
\harecsection{\harec{a}{1.1.4}{1.1.4b}}
\label{ohms_lag_växelström}

I formler betecknas impedansen med bokstaven \(Z\) och reaktansen med bokstaven
\(X\).
I båda fallen är enheten ohm [\unit{\ohm}].

Vid beräkning av impedans är Ohms lag inte direkt tillämplig, eftersom
reaktansen i en induktor eller kondensator uppträder annorlunda i tiden vid
ström- respektive spänningsändring än vad resistansen gör.

Om impedansen \(Z\) sätts in i Ohms lag fås följande samband, som ofta kallas
Ohms lag för växelström
\[
\begin{array}{rcl}
  U_{\it eff} &=& I_{\it eff} \cdot Z \quad \text{eller} \\
  U_{\it eff} &=& I_{\it eff} \cdot \sqrt{R^2 + X^2} \quad \text{eller} \\
  U_{\it eff} &=& I_{\it eff} \cdot \sqrt{R^2 + (X_L - X_C)^2} \quad \text{osv.}
\end{array}
\]
Av vad som framgått tidigare i detta avsnitt kan även slutsatsen dras att:
%
\[
\text{skenbar effekt} = \sqrt{(\text{aktiv effekt})^2 + (\text{reaktiv effekt})^2}
\]
%
\subsection{Parallellkopplade LC-kretsar}
\harecsection{\harec{a}{3.1.3}{3.1.3d}, \harec{a}{3.2.1}{3.2.1}}

\index{LC-krets}
\index{parallellkopplad LC-krets}
\index{LC-krets!parallellkopplad}

\smallfig{images/cropped_pdfs/bild_2_3-15.pdf}{Parallellkopplad LC-krets}{fig:BildII3-15}

En parallellkopplad LC-krets är i bild \ssaref{fig:BildII3-15} ansluten till
växelspänningen \(U\) från en signalgenerator med inställbar frekvens \(f\).
Två fall studeras.

\begin{description}

\item[Fall 1:] \(f = f_{\it res}\)

Signalgeneratorns frekvens \(f\) ställs lika med LC-kretsens resonansfrekvens
\(f_{\it res}\).
Då visar kretsen hög impedans \(Z\) mot generatorn.
En stark ström cirkulerar i LC-kretsen, men endast en svag ström flyter i
ledningen mellan generator och krets.
Jämför med modellförsöket på bild~\ssaref{fig:BildII3-17}.

\item[Fall 2:] \(f > f_{\it res}\) eller \(f < f_{\it res}\)

Frekvensen \(f\) ställs högre eller lägre än kretsens resonansfrekvens
\(f_{\it res}\).

Kretsen visar då en låg impedans \(Z\) mot generatorn.
En svag ström cirkulerar i LC-kretsen, medan en starkare ström flyter i
ledningen mellan generator och krets.

I praktiken finns även en resistans (belastning) parallellt över kretsen och en
resistans i serie med induktansen.
För enkelhetens skull bortses här från dessa resistanser.

I en parallellkopplad LC-krets är spänningen över induktans och kapacitans
densamma.
Spänningsvektorn \(U\) används därför som så kallad riktfas.

Riktfasen ritas på bilden åt höger.
Strömmen \(I_C\) genom kondensatorn är fasförskjuten \ang{90} före \(U\) och
ritas rakt uppåt (vektorerna roterar moturs).
Strömmen \(I_L\) genom induktorn är fasförskjuten \ang{90} efter \(U\) och
ritas rakt nedåt.
Den resulterande reaktiva strömmen genom kretsen är skillnaden mellan
strömmarna \(I_C\) och \(I_L\), vilka är motriktade varandra.

Formeln för parallellkopplade resistanser kan även användas för
parallellkopplade reaktanser om man tillämpar Pythagoras sats
\(A^2 + B^2 = C^2\), således
%
\begin{gather*}
	\frac{1}{R} = \frac{1}{R_1} + \frac{1}{R_2} + \cdots \\
	\left(\frac{1}{Z}\right)^2 = \left(\frac{1}{R}\right)^2 +
	\left(\frac{1}{X}\right)^2 \quad \text{eller}
\end{gather*}
\begin{align*}
	\frac{1}{Z} &= \sqrt{\left(\frac{1}{R}\right)^2 + \left(\frac{1}{X}\right)^2} \\
	\frac{1}{Z} &= \sqrt{\frac{1}{R^2} + \frac{1}{X^2}}
\end{align*}

Med \(R\) försumbart kan den totala reaktansen beräknas ur den induktiva reaktansen \(X_L\)
och den vektormässigt motriktade kapacitiva reaktansen \(X_C\) på följande sätt:
%
\begin{align*}
	\frac{1}{X} &= \frac{1}{X_L} - \frac{1}{X_C} = \frac{X_C - X_L}{X_C \cdot X_L}
	\quad \text{eller} \\
	X &= \frac{X_C \cdot X_L}{X_C - X_L}
\end{align*}

I en parallellkopplad LC-krets är den resulterande reaktansen negativ
(kapacitiv) om \(X_L\) är större än \(X_C\) och positiv (induktiv) om \(X_L\) är
mindre än \(X_C\).
\end{description}

\subsection{Seriekopplade LC-kretsar}
\index{LC-krets}
\index{seriekopplad LC-krets}
\index{LC-krets!seriekopplad}

\smallfig[0.5]{images/cropped_pdfs/bild_2_3-16.pdf}{Seriekopplad LC-krets}{fig:BildII3-16}

En seriekopplad LC-krets i bild~\ssaref{fig:BildII3-16} ansluts till
växelspänningen \(U\) från en signalgenerator med inställbar frekvens \(f\).
Två fall studeras.

\begin{description}
\item[Fall 1:] \(f = f_{\it res}\)

Signalgeneratorns frekvens \(f\) ställs lika med LC-kretsens resonansfrekvens
\(f_{\it res}\).
Impedansen \(Z\) i en seriekrets visar då ett mycket lågt värde mot generatorn.
Det flyter en stark ström i ledningen mellan generator och krets.

\item[Fall 2:] \(f < f_{\it res} \quad \text{eller} \quad f > f_{\it res}\)

Frekvensen \(f\) ställs lägre eller högre än kretsens resonansfrekvens
\(f_{\it res}\).

Eftersom LC-kretsen då visar hög impedans \(Z\) mot generatorn, flyter
endast en svag ström i ledningen mellan generator och krets.

I praktiken finns även en resistans i serie med induktansen liksom en
parallellt över kapacitansen.
För enkelhets skull bortses här från dessa resistanser.

Strömmen \(I\) är samma genom hela kretsen och strömvektorn \(I\) används
därför som riktfas.
Den ritas i bilden åt höger.
Om serieresistansen \(R\) varit med skulle ett spänningsfall \(U_R\) varit
inritat i samma riktning som \(I\) (i fas med \(I\)).
Spänningen över reaktansen \(X_C\) ligger \ang{90} efter \(I\) och ritas
rakt neråt (vektorerna roterar moturs).
Spänningen över reaktansen \(X_L\) (induktorn) ligger \ang{90} före \(I\) och
ritas rakt uppåt.
\end{description}

\subsection{Thomsons formel}
\harecsection{\harec{a}{3.2.4}{3.2.4}}
\index{Thomsons formel}
\index{svängningskrets}
\index{resonanskrets}
\index{oscillator}
\index{oscillator!Thomsons formel}

\smallfig{images/cropped_pdfs/bild_2_3-17.pdf}{Svängningskrets}{fig:BildII3-17}

Bild \ssaref{fig:BildII3-17} visar en svängningskrets som består av en
kondensator och en induktor med förskjutbar järnkärna.
En ändring av kärnans tvärsnitt ändrar den magnetiska ledningsförmågan och
därmed induktansen varför även reaktansen \(X_L\) ändras.

Med anordningen kan resonansfrekvensen alltså ställas in så att den blir högre än,
lika med eller lägre än den anslutna spänningens frekvens.
Tre fall undersöks:
\begin{enumerate}
\item \(X_L > X_c\) LA1 och LA2 lyser upp, en kraftig ström flyter genom
  kondensatorn,
\item \(X_L < X_C\) LA1 och LA3 lyser upp, en kraftig ström flyter genom
  induktorn,
\item \(X_L= X_C\) LA2 och LA3 lyser upp, LA1 lyser inte, en kraftig ström
  flyter i kretsen men inte i tilledningarna.

\end{enumerate}

När \(X_L = X_C\) är kretsen i resonans och Thomsons formel
\(\omega L 0 \frac{1}{\omega C}\) kan användas för att beräkna
resonansfrekvensen.
Formeln namngiven efter William Thomson (Lord Kelvin) beskriver resonansfallet
då de induktiva och kapacitiva reaktanserna i kretsen är lika stora och tar ut
varandra.
Kvar är kretsens resistans, vilken vi tills vidare betraktar som försumbar.

Således \(X_L = X_C\), där
%
\begin{gather*}
  X_L = 2\pi fL \quad \text{och} \\
  X_C = \frac{1}{2\pi fC} \quad \text{sätts in.} \\
  2\pi fL = \frac{1}{2\pi fC} \quad 4\pi ^2f^2LC = 1 \\
  f^2 = \frac{1}{4\pi ^2LC} \quad f = \frac{1}{2\pi \sqrt{LC}} \\
  f\text{ [Hz] }L\text{ [H] }C\text{ [F] } \\
\end{gather*}

Formeln gäller både för parallell- och seriekretsar.

\vspace*{1ex}
\noindent
\textbf{Räkneexempel:}
%
\[L = \qty{100}{\nano\henry} \quad C = \qty{10}{\pico\farad} \quad f =\ ?\]
\begin{align*}
  f &= \frac{1}{2\pi \sqrt{100 \cdot 10^{-9} \cdot 10 \cdot 10^{-12}}} \\
  &= \frac{1}{2\pi 10^{-9}} \\
  &= \frac{10^9}{2\pi } \\
  &\approx \qty{159}{\mega\hertz}
\end{align*}

\subsection{Impedansen i en resonant krets}
\harecsection{\harec{a}{3.2.1}{3.2.1b}, \harec{a}{3.2.2}{3.2.2b}, \harec{a}{3.2.3}{3.2.3}, \harec{a}{3.2.4}{3.2.4b}}
\label{impedans_resonant_krets}

\index{resonans!impedans}
\index{impedans!resonanskrets}

En enkel framställning görs av hur impedans, reaktans och resistans förhåller
sig inbördes när en resonanskrets är i resonans.
Som exempel används följande kretsdata: Induktans \qty{200}{\micro\henry},
kapacitans \qty{200}{\pico\farad}, förlustresistans \qty{10}{\ohm}.

\subsubsection{Resonansfallet i en parallellkrets}
\label{parallellresonans}
\index{parallellresonans}
\index{resonans!parallellkrets}

\tallfig{images/cropped_pdfs/bild_2_3-18.pdf}{Resonansfallet i parallellkrets}{fig:BildII3-18}

Kretsen består av parallellkopplade reaktanser, \(X_L\) och \(X_C\).
Vid resonans är dessa lika stora och motverkande.
Inom kretsen är således den resulterande reaktansen:
%
\[X_C - X_L = 0\]
%
Därför uppvisar samma krets en yttre reaktans av:
%
\[
  X = \frac{X_C \cdot X_L}{X_C - X_L}
  = \frac{X_C \cdot X_L}{0}
  = \infty
\]
%
I praktiken finns i kretsen också en resistans varför dessa extremvärden inte
uppstår.
Inne i en parallellkrets i resonans cirkulerar alltså en stark ström,
som endast begränsas av kretsens resistans.

Bild \ssaref{fig:BildII3-18} visar en parallellkrets där induktorn har resistansen
\(r_L\) och kondensatorn antas vara förlustfri.
Vidare förutsätts att kretsen är i resonans.

Vid resonans kan termen \(X_C - X_L = 0\) bytas mot \(r_L\) i formeln
\[X = \frac{X_C \cdot X_L}{X_C - X_L}\] förutsatt att \(r_L\) är försumbart
jämfört med \(X_L\).

Därtill är \(X_L = 2\pi fL\) och \(X_C = \frac{1}{2\pi fC}\) det vill säga
\(X_L \cdot X_C = \frac{L}{C}\) som sätts in.

Parallellkretsens impedans vid resonans kan då skrivas

\[
Z = \frac{X_C \cdot X_L}{r_L} = \frac{L}{r_L \cdot C}
\]

Med ovanstående kretsdata blir Z = \qty{100}{\kilo\ohm}.

Därav framgår, att impedansen i parallellkretsen är en funktion av det så
kallade L/C-förhållandet samt av kretsens resistiva förluster.

%% k7per: need to make layout guides soft, otherwise any change ahead wrecks havoc.
%% \newpage % layout

\subsubsection{Resonansfallet i en seriekrets}
\label{serieresonans}
\index{serieresonans}
\index{resonans!seriekrets}

\tallfig{images/cropped_pdfs/bild_2_3-19.pdf}{Resonansfallet i seriekrets}{fig:BildII3-19}

Bild \ssaref{fig:BildII3-19} visar en seriekrets är i resonans, så är

\begin{align*}
& X_C = X_L \quad & \text{dvs.} \quad \frac{1}{\omega C} = \omega L\\
& \text{eller} & \\
& X_C - X_L = 0 \quad & \text{dvs.} \quad \frac{1}{\omega C} - \omega L = 0
\end{align*}

Med ovanstående kretsdata blir resonansfrekvensen:
%
\[
f_0 = \frac{1}{2\pi \sqrt{LC}} \approx 796\ kHz
\]
%
Vid resonansfrekvensen blir reaktansen \qty{1000}{\ohm} både för induktansen och
kapacitansen.
Eftersom reaktansernas spänningsfall är motriktade tar de ut varandra.
Kretsens impedans i resonans blir resistansen \(r_L\) och
spänningsfallet över kretsen bestäms enbart av \(r_L\).

Antag att det alstras en spänning av \qty{5}{\milli\volt} i antennkretsen.
Strömmen genom den vid resonans blir då
\(\frac{\qty{5}{\milli\volt}}{\qty{10}{\ohm}} = \qty{0,5}{\milli\ampere}\).

Av strömmen bildas reaktiva spänningar, det vill säga \(\qty{0,5}{\milli\ampere}
\cdot \qty{1000}{\ohm} = \qty{500}{\milli\volt}\) både över induktans och
kapacitans (som tar ut varandra) och \qty{5}{\milli\volt} över resistansen.

\subsection{Q-faktorn i en parallellkrets}
\harecsection{\harec{a}{3.2.5}{3.2.5}}
\label{Q-faktor}
\index{Q-faktor}
\index{symbol!Q qualityfactor}

\smallfig{images/cropped_pdfs/bild_2_3-20.pdf}{Q-värden i parallellkrets}{fig:BildII3-20}

Bild \ssaref{fig:BildII3-20} illustrerar Q-värden för parallellkrets.
Godhetstalet Q (=Quality Factor) kan ses som den förmåga en resonanskrets har
att lagra energi, det vill säga förhållandet mellan den lagrade energin och
energiförlusten i kretsen.
Energiförlusten yttrar sig som värmeutveckling.
%
\[
Q = 2\pi \frac{\text{lagrad energi i kretsen}}{\text{energiförlusten per period}}
\]
%
Energiförluster uppstår både i kretsens kondensator och induktor, men moderna
kondensatorer har så låga förluster att induktorn ensam kan anses bestämma
Q-värdet, åtminstone i kortvågsområdet.

En växelspänning \(U_1\) ansluts till en parallellkrets.
I resonansfallet uppträder då en spänning \(U_2\) över kondensatorn och
induktorn.

\(U_2\) är mycket större än \(U_1\).
Ju högre \(Q\) är i kretsen desto större är förhållandet mellan \(U_2\) och
\(U_1\).

I kortvågsområdet är det vanligt med ett \(Q\) i storleksordningen 30--100.
Ju högre Q är, desto mindre är bandbredden.

När kretsen är i resonans gäller sambandet
%
\[Q = \frac{f_{\it res}}{b}\]
%
Bandbredden ökar (avstämningsskärpan minskar) vid ökande frekvens på grund av de
större kretsförlusterna.

\smallfig{images/cropped_pdfs/bild_2_3-21.pdf}{Bandbredd i parallellkrets}{fig:BildII3-21}

\subsection{Bandbredd}
\harecsection{\harec{a}{3.2.6}{3.2.6}}
\index{bandbredd}

Bild \ssaref{fig:BildII3-21} visar med en kurva vilket impedansvärde kretsen har
vid olika frekvenser.
Impedansens högsta värde är vid frekvensen \(f_{\it res}\) och avtar vid frekvenser
som är högre eller lägre.
Vid frekvenserna \(f_1\) och \(f_2\) är impedansvärdet till exempel 70~\% av
maximalvärdet.
Med bandbredden \(b\) förstås skillnaden mellan impedansvärdena i ett sådant
frekvenspar, det vill säga \(b = f_2 - f_1\).

% Avsnitt 3.2 Filter
\section{Filter}
\harecsection{\harec{a}{3.2}{3.2}, \harec{a}{3.2.9}{3.2.9}}
\index{filter}
\index{filter!frekvensfilter}
\index{frekvensgång}
\index{filter!frekvensgång}
\index{frequency response}
\label{filter}

Frekvensfilter, eller mer allmänt \emph{filter}, används inom radiotekniken för
många olika ändamål, till exempel för att
\begin{itemize}
  \item eliminera störande signaler
  \item öka avstämningsskärpan (selektiviteten) i mottagare och sändare
  \item framhäva eller dämpa ett sidband i en AM-signal med mera.
\end{itemize}

\emph{Frekvensgången} (eng. \emph{frequency response}) är ett mått på ett
filters förmåga att släppa igenom olika mycket av olika frekvenser.
Frekvensgången presenteras i allmänhet som en kurva med amplitud av genomsläppt
sinussignal som funktion av frekvensen.

Beroende på frekvensgången indelas filtren i olika typer, varav de vanligaste
presenteras här.

Beroende på det tekniska utförandet finns dels så kallade passiva filter vilka
använder extern energi för sin funktion, och dels aktiva filter vilka i princip
är förstärkare som likaledes använder passiva kretsar.
Här presenteras för enkelhets skull passiva filter.

Man skiljer även mellan analoga filter och digitala filter.
Vi beskriver här först några olika typer av klassiska analoga filter.

\subsection{Högpassfilter (HP)}
\harecsection{\harec{a}{3.2.8b}{3.2.8b}, \harec{a}{3.2.9}{3.2.9a}}
\index{högpassfilter}
\index{filter!högpass (HP)}
\index{highpass filter}
\index{HP}

\mediumtopfig{images/cropped_pdfs/bild_2_3-22.pdf}{Högpassfilter}{fig:BildII3-22}


Ett \emph{högpassfilter} (eng. \emph{highpass filter, HP}),
bild~\ssaref{fig:BildII3-22}) släpper igenom signaler med höga frekvenser och
dämpar de med låga frekvenser.

\paragraph{Exempel} En frekvensberoende spänningsdelare som LC-högpassfilter.

Vid låga frekvenser är \(X_C\) stor och \(X_L\) liten.
Över \(X_L\) uppstår då ett litet spänningsfall -- en låg utgångsspänning \(U_a\).
Resultatet blir att låga frekvenser dämpas.

Vid höga frekvenser är \(X_C\) liten och \(X_L\) stor.
Över \(X_L\) uppstår då ett stort spänningsfall -- en hög utgångsspänning
\(U_a\).
Resultatet blir att höga frekvenser släpps igenom.

\(X_L\) kan bytas ut mot en resistor \(R\), men då blir passbandskurvan inte lika brant.

\paragraph{Gränsfrekvens}

Gränsfrekvensen \(f_g\) beror av kapacitansen \(C\), induktansen \(L\) samt
resistansen \(R\).

\paragraph{LC-högpass}
\begin{gather*}
  f_g = \frac{1}{2\pi \sqrt{LC}} \\
  f_g\ \text{[Hz]} \quad L\ \text{[H]} \quad C\ \text{[F]}
\end{gather*}

\paragraph{RC-högpass}
\begin{gather*}
  f_g = \frac{1}{2\pi RC}\\
  f_g\ \text{[Hz]} \quad R\ [\unit{\ohm}] \quad C\ \text{[F]}
\end{gather*}

\paragraph{Räkneexempel}
\begin{enumerate}
\item \(L = 4\ \text{H} \quad C = 1\ \unit{\micro\farad} \quad f_g =\ ?\)
  \[
  f_g = \frac{1}{2\pi \sqrt{4 \cdot 10^{-6}}} = \frac{500}{2\pi }
  = 79,6\ \text{Hz}
  \]
\item \(R = \qty{1}{\kilo\ohm} \quad C = 10\ \text{nF} \quad f_g =\ ?\)
  \[
    f_g = \frac{1}{2\pi  \cdot 1 \cdot 10^3 \cdot 10 \cdot 10^{-9}}
    = \frac{10^5}{2\pi } = \qty{15,9}{\kilo\hertz}
  \]
\end{enumerate}


\subsection{Lågpassfilter (LP)}
\harecsection{\harec{a}{3.2.8a}{3.2.8a}, \harec{a}{3.2.9}{3.2.9b}}
\index{lågpassfilter}
\index{filter!lågpass (LP)}
\index{lowpass filter}
\index{LP|see {lågpassfilter}}
\label{lågpassfilter}

\mediumtopfig{images/cropped_pdfs/bild_2_3-23.pdf}{Lågpassfilter}{fig:BildII3-23}

Om induktor och kondensator respektive resistor och kondensator i ett
högpassfilter byter plats, som i bild~\ssaref{fig:BildII3-23}, så får man i
stället ett LC-lågpassfilter respektive ett RC-lågpassfilter.

Ett \emph{lågpassfilter} (eng. \emph{lowpass filter, LP}) släpper igenom
signaler med låga frekvenser och dämpar de med höga frekvenser.

\paragraph{Exempel} En frekvensberoende spänningsdelare som LC-lågpassfilter.

Vid låga frekvenser är \(X_C\) stor och \(X_L\) liten.
Över \(X_L\) uppstår då ett litet spänningsfall -- en hög utgångsspänning
\(U_a\).
Resultatet blir att låga frekvenser släpps igenom.

Vid höga frekvenser är \(X_C\) liten och \(X_L\) stor.
Över \(X_L\) uppstår då ett stort spänningsfall -- en låg utgångsspänning
\(U_a\).
Resultatet blir att höga frekvenser dämpas.

\paragraph{Gränsfrekvens}

Samma formler används vid beräkning av gränsfrekvensen både i lågpass- och
högpassfilter, således

\paragraph{LC-lågpass}
\begin{gather*}
  f_g = \frac{1}{2\pi \sqrt{LC}} \\
  f_g\ \text{[Hz]} \quad L\ \text{[H]} \quad C\ \text{[F]}
\end{gather*}

\paragraph{RC-lågpass}
\begin{gather*}
  f_g = \frac{1}{2\pi {RC}} \\
  f_g\ \text{[Hz]} \quad R\ [\unit{\ohm}] \quad C\ \text{[F]}
\end{gather*}

\subsection{Bandpassfilter (BP)}
\harecsection{\harec{a}{3.2.7}{3.2.7}, \harec{a}{3.2.8c}{3.2.8c}, \harec{a}{3.2.9}{3.2.9c}}
\index{bandpassfilter}
\index{filter!bandpass (BP)}
\index{bandpass filter}
\index{BP}

\mediumtopfig{images/cropped_pdfs/bild_2_3-24.pdf}{Bandpassfilter}{fig:BildII3-24}

Ett \emph{bandpassfilter} (eng. \emph{bandpass filter}) släpper igenom signaler
bara inom ett visst frekvensområde medan signaler utanför detta frekvensområde dämpas.

Bandpassfiltret består i enklaste fall av två resonanskretsar av LC-typ, vilka
är avstämda till angränsande frekvenser. Kretsarna är kopplade induktivt,
kapacitivt eller galvaniskt så som illustreras i bild~\ssaref{fig:BildII3-24}.

Beroende på kopplingsgrad eller dämpning skiljer man mellan underkritisk
koppling (lös koppling), kritisk koppling och överkritisk koppling
(fast koppling).

I bild~\ssaref{fig:BildII3-24} visas hur passbandet påverkas bland annat av kopplingsgraden.
Lös koppling liten bandbredd.
Kritisk koppling -- större bandbredd.
Fast koppling -- stor bandbredd.

\mediumtopfig[0.6]{images/cropped_pdfs/bild_2_3-25.pdf}{Passfilter}{fig:BildII3-25}
\mediumtopfig{images/cropped_pdfs/bild_2_3-26.pdf}{Bandspärrfilter}{fig:BildII3-26}
\mediumherefig{images/cropped_pdfs/bild_2_3-27.pdf}{Spärrfilter (2 sorter)}{fig:BildII3-27}
\newpage

\subsection{Passfilter}
\harecsection{\harec{a}{3.2.9}{3.2.9d}}
\index{passfilter}
\index{filter!bandpass (BP)}
\index{pass filter|see {passfilter}}
\index{BP}

Passkretsen eller passfilter stäms av till en viss frekvens och erbjuder där
en mycket låg impedans så som illustreras i bild~\ssaref{fig:BildII3-25}.
Passkretsen kopplas i serie med signalvägen och låter signaler med
frekvenser inom filtrets passband att passera.


\subsection{Bandspärrfilter}
\harecsection{\harec{a}{3.2.8d}{3.2.8d}, \harec{a}{3.2.9}{3.2.9e}}
\index{bandspärrfilter}
\index{filter!bandspärr (BR)}
\index{band reject filter (BR)}
\index{BR}

Om serie- och parallellkretsarna i ett bandpassfilter byter plats får man
i stället ett bandspärrfilter så som illustreras i bild~\ssaref{fig:BildII3-26}.
Ett sådant spärrar signaler inom ett visst frekvensområde, men släpper igenom
signaler utom detta område.

\newpage
\subsection{Spärrfilter}
\index{bandspärrfilter}
\index{filter!bandspärr (BR)}
\index{band reject filter (BR)}
\index{BR}
\index{spärrkrets}
\index{sugkrets}


\subsubsection{Spärrkrets}
Spärrkretsen stäms av till en viss frekvens och erbjuder där en mycket hög
impedans.
Spärrkretsen kopplas i serie med signalvägen och spärrar en signal med samma
frekvens som resonansfrekvensen, så som illustreras i bild~\ssaref{fig:BildII3-27}.

\subsubsection{Sugkrets}
Sugkretsen stäms av till en viss frekvens och erbjuder där en mycket låg
impedans.
Sugkretsen kopplas parallellt med signalvägen och kortsluter (suger bort) en
signal med samma frekvens som resonansfrekvensen, så som illustreras i bild
\ssaref{fig:BildII3-27}.

\mediumfig[0.8]{images/cropped_pdfs/bild_2_3-30.pdf}{Mekaniskt filter}{fig:BildII3-30}

\subsection{Kvartskristall}

\harecsection{\harec{a}{3.2.11}{3.2.11}}
\index{kvartskristall}
\index{quartz crystal}
\index{crystal}
\index{Q-värde}
\index{resonator}

\smallfig{images/cropped_pdfs/bild_2_3-28.pdf}{Kvartskristall}{fig:BildII3-28}

En \emph{kvartskristall} (eng. \emph{quartz crystal} eller \emph{crystal}),
egentligen en slipad skiva av kvarts, kan fungera som en
elektromekanisk svängningskropp (resonator), vars egenskaper liknar dem i en
LC-krets.
Detta illustreras i bild~\ssaref{fig:BildII3-28}.

Den låga inre resistansen gör att Q-värdet i en kvartskristall är bättre än
10000.
Som jämförelse är Q-värdet i en LC-krets oftast sämre än 1000.

Många moderna kvartskristaller kan uppvisa olastat Q-värde på 100000.

\vspace{12pt} % Undgår brytning av nästa titelrad

\subsection{Bandfilter med kvartskristaller}
\index{kristallfilter}
\index{crystal filter}
\index{keramiska resonatorer}
\index{ceramic resonators}
\label{bandfilter_kristall}

\smallfigpad{images/cropped_pdfs/bild_2_3-29.pdf}{Bandfilter med kvartskristaller}{fig:BildII3-29}

Bild~\ssaref{fig:BildII3-29} visar hur kvartskristaller kan kombineras till
filter, ofta refererade till som \emph{kristallfilter} (eng.
\emph{crystal filter}), med önskad bandbredd.
Även utföranden med \emph{keramiska resonatorer} (eng.
\emph{ceramic resonators}) finns.
Resonatorerna är avstämda till var sin bestämda frekvens och hela komplexet
bidrar på så sätt till att bilda passband eller andra egenskaper på samma sätt
som med sammankopplade LC-kretsar.

\subsection{Mekaniska filter}
\index{mekaniskt filter}
\index{mechanical filter}
\index{mekanisk resonator}
\index{resonator!mekanisk}

Med en elektromekanisk givare kan man få en kropp (resonator) att svänga på sin
resonansfrekvens.
Med ännu en elektromekanisk givare kan man känna av svängningarna och
åter omvandla dem till elektriska signaler.
Bild~\ssaref{fig:BildII3-30} illustrerar ett sådant arrangemang.
Hela anordningen fungerar som en \emph{elektromekanisk resonator} (eng.
\emph{mechanical resonator}), vars egenskaper liknar dem i en LC-krets.

% 

Resonatorerna kan kombineras till filterkomplex med önskad bandbredd där
resonatorerna är avstämda till var sin bestämd frekvens.
Hela komplexet bidrar på så sätt till att bilda ett passband på samma sätt som
med sammankopplade LC-kretsar.
Beroende på tillämpningen finns olika frekvenslägen i intervallet
\SIrange{60}{600}{\kilo\hertz}.

\emph{Mekaniska filter} (eng. \emph{mechanical filter}) användes mest förr som
mellanfrekvensfilter i högvärdiga radioutrustningar, men har numera till stor
del ersatts av bandfilter med kvartskristaller där arbetsområdet kan ligga
avsevärt högre i frekvens.

\newpage % layout
\subsection{Kavitetsfilter}
\index{kavitetsfilter}
\index{cavity filter}
\index{filter!kavitet}

\smallfig[0.3]{images/cropped_pdfs/bild_2_3-31.pdf}{Kavitetsfilter}{fig:BildII3-31}

Resonanskretsars dimensioner minskar med ökande frekvens.
Vid mycket hög frekvens kan induktorns varvtal i en LC-krets ha minskat till
ett enda varv samtidigt som kapacitansen inom detta enda varv kan räcka för
önskad resonansfrekvens.

En sådan resonanskrets kan bland annat ha formen av en ledare mitt inne i En
elektriskt ledande kavitet, så som illustreras i bild~\ssaref{fig:BildII3-31}.
Ledarens längd tillsammans med kavitetens insida bildar induktorn.
Mellan ledaren och kavitetens insida råder en kapacitans, som kan
kompletteras/justeras med en extra kondensator.

%% k7per: Remove explicit formatting.
%% \newpage % layout

Inkommande och utgående signaler ansluts till filtrets mittledare över
induktionsslingor, kondensatorer eller direkt galvaniskt.
\emph{Kavitetsfilter} (eng. \emph{cavity filter}) kan kopplas ihop för att till exempel 
bilda bandfilter eller frekvensdelare.

Kavitetsfilter används ofta på sändare eftersom de med sina låga förluster kan
hantera stora effekter samt åstadkomma djupa utsläckningar.
Dessa egenskaper gör dem oerhört lämpliga som duplexfilter till repeatrar.

\subsection{Helixfilter}
\index{helixfilter}
\index{filter!helix}

När ett kompakt kavitetsfilter behövs kan man öka reaktansen i mittledaren
både induktivt och kapacitivt genom att utforma den som en spiral (helix).
Detta sker dock på bekostnad av Q-värdet.
Flera kavitetsfilter kan kopplas ihop för att bilda till exempel bandfilter eller spärrfilter.

\subsection{Pi-filter}
\harecsection{\harec{a}{3.2.10a}{3.2.10a}}
\index{Pi-filter}
\index{filter!Pi-filter}

\smallfigpad[0.35]{images/cropped_pdfs/bild_2_3-32.pdf}{Pi-filter}{fig:BildII3-32}

För att överföra HF-signaler med bästa verkningsgrad är det viktigt med god
impedansanpassning mellan de olika kretsarna.
Om anslutningsimpedansen är lika i båda kretsarna behövs inga extra åtgärder.
Om impedanserna däremot är olika behövs korrigeringsnät (filter).

Ofta är nätet Pi-format så som bild~\ssaref{fig:BildII3-32} visar och består av
induktanser och kapacitanser.
Ett Pi-format nät kan sägas bestå av två L-formade nät ställda mot varandra, där
den seriella delen är gemensam (på bilden en induktor).

%% k7per
% \newpage % layout
\subsection{T-filter}
\harecsection{\harec{a}{3.2.10b}{3.2.10b}}
\index{T-filter}
\index{filter!T-filter}

\smallfig[0.35]{images/cropped_pdfs/bild_2_3-33.pdf}{T-filter (två varianter)}{fig:BildII3-33}
\smallfigpad[0.3]{images/cropped_pdfs/bild_2_3-34.pdf}{Halvledardioder}{fig:BildII3-34}

Ett nät kan också vara T-format, som bild~\ssaref{fig:BildII3-33} visar, och bestå
av induktanser och kapacitanser.
Ett sådant nät kan sägas bestå av två L-formade nät ställda ''rygg mot rygg''.
Då är den parallella delen gemensam.
På bilden visas två alternativ.

När den parallella delen är kapacitiv, blir huvudkaraktären ett lågpassfilter.
När den parallella delen i stället är induktiv blir huvudkaraktären ett högpassfilter.

Ett Pi- eller T-filter kan fungera som
\begin{itemize}
  \item resonanskrets
  \item impedanstransformator (anpassning)
  \item neutralisering av reaktans
\end{itemize}

\subsection{Icke-ideala komponenter}
\harecsection{\harec{a}{3.2.12}{3.2.12}}

I verkligheten är alla analoga komponenter även behäftade med oönskade
egenskaper, även kallade parasitiska egenskaper.

Ett motstånd uppvisar inte enbart en strikt resistiv egenskap, utan för högre
frekvenser kommer även en parasitisk seriekopplad induktans att göra sig påmind.

En kondensator har inte perfekt isolation.
Genom en parasitisk resistans parallellkopplad med kondensatorplattorna flyter
en läckström, som kommer att ladda ur kondensatorn.

En induktor är inte perfekt förlustfri, utan den har en parasitisk serieresistans.

För kondensatorer och induktorer kommer resistansen att påverka deras Q-värde.
Ett högt Q-värde innebär att man har låg förlust.
Förlusterna kommer att göra sig påminda när man bygger kretsar med dessa
komponenter.
Till exempel kommer en LC krets i praktiken alltid att vara en LCR-krets, där
förlusterna i spole och kondensator ger en förlust i kretsen och begränsar hur
högt Q-värde som kan uppnås.
När en resonator belastas ökar förlusten ytterligare och därmed sjunker Q-värdet.

% \mediumminusbotfig{images/cropped_pdfs/bild_2_3-35.pdf}{Halv- och helvågslikriktning}{fig:BildII3-35}

\subsection{Digitala filter}
\harecsection{\harec{a}{3.2.13}{3.2.13}}
\index{digitala filter}

Utvecklingen går mot att allt mer signalbehandling sker digitalt. 
\emph{Digitala filter} kan utnyttjas genom att signalen först konverteras till
digital form och sedan efter filtreringen konverteras tillbaka till analog form.
Digitala filter har många fördelar.
Man kan konstruera komplicerade och skarpa filter som behåller sina egenskaper
över tiden, där klassiska analoga kan behöva trimmas både individuellt vid
tillverkning och över tiden för att upprätthålla sina egenskaper.

För mer om digitala filter se \ssaref{DSP} samt \ssaref{digitala filter}.

% Avsnitt 3.3 Kraftförsörjning
\section{Kraftförsörjning}
\harecsection{\harec{a}{3.3}{3.3}}
\label{kraftaggregat}
\index{kraftförsörjning}
\index{kraftaggregat}
\index{batteri}
\index{ackumulator}
\label{sec:kraftfoersoerjning}

Den elektriska energi, som behövs för elektronikutrustningar, hämtas
från det allmänna elnätet, ett batteri eller en
ackumulator.
Vissa batterityper kan återuppladdas och kallas då ackumulator.

Batterier och ackumulatorer avger en nominell spänning som beror av
de ingående materialen och givetvis av laddningstillståndet.
Moderna utrustningar för amatörradio är utförda för \qty{12}{\volt} likström och
försörjs vanligen från ett nätanslutet kraftaggregat.
På så sätt kan mobila radioutrustningar även försörjas från startackumulatorn
i fordonet.

Handburna radioutrustningar försörjs från en inbyggd ackumulator som laddas
från stationär laddare.

Äldre stationära radioutrustningar drivs nästan alltid med nätanslutna
kraftaggregat med en eller flera transformatorer och likriktare.
Alternativt kan samma transformators sekundärsida vara
försedd med flera lindningar för olika spänningar och strömkretsar.

Det allmänna elnätet i Sverige levererar växelspänning med frekvensen
\qty{50}{\hertz}.
Nätspänningen för hushållsändamål är numera 400/\qty{230}{\volt}.

Tidigare importerade utrustningar i marknaden kan vara utförda för andra
nätspännings- och skyddsjordningssystem än vad som nu tillämpas i Sverige.
Försiktighet med sådan utrustning rekommenderas.

% \smallfig[0.2]{images/cropped_pdfs/bild_2_3-34.pdf}{Halvledardioder}{fig:BildII3-34}

%\newpage % layout
\subsection{Halv- och helvågslikriktning}
\harecsection{\harec{a}{3.1.1g}{3.1.1g}, \harec{a}{3.3.1}{3.3.1}}
\index{likriktning}
\index{rectifier}
\label{likriktning}

\emph{Likriktning} (eng. \emph{rectificiation}) av spänningar och strömmar i en
krets görs med ''elektroniska ventiler'' som släpper igenom ström endast i den
så kallade passriktningen och stoppar i spärriktningen så som illustreras i
bild~\ssaref{fig:BildII3-34}.
En sådan strömventil kallas för diod och kan vara av typen vakuumrör eller
halvledare.
I moderna konstruktioner används uteslutande halvledardioder i
likriktarkopplingar.

\newpage
\mediumfig{images/cropped_pdfs/bild_2_3-35.pdf}{Halv- och helvågslikriktning}{fig:BildII3-35}

\subsubsection{Halvvågslikriktning}
\index{halvvågslikriktning}
\index{likriktning!halvvågs}

Vid \emph{halvvågslikriktning} (eng. \emph{half wave rectification}) släpps
endast varannan halvvåg av en växelspänning igenom.
I den strömkrets som bildas av transformatorns sekundärlindning, dioden och
lasten, flyter därför ström endast under varannan halvperiod, så som
illustreras i bild~\ssaref{fig:BildII3-35}.

\subsubsection{Helvågslikriktning}
\index{helvågslikriktning}
\index{likriktning!helvågs}
\index{Graetz-brygga}
\index{likriktning!Graetz-brygga}

I följande kopplingar med två respektive fyra dioder släpps varje halvvåg av
transformatorns växelspänning igenom så att alla halvvågor får samma polaritet.
Ström flyter genom lasten i samma riktning under varje halvperiod.
Följande sätt att anordna \emph{helvågslikriktning}
(\emph{full wave rectification}) är vanliga:
\begin{itemize}
\item Med två dioder och mittuttag på transformatorns sekundärlindning.
  Den ena dioden och ena lindningshalvan släpper igenom ström till lasten
  under ena halvperioden.
  Den andra dioden och andra lindningshalvan under den följande halvperioden.
  Detta illustreras i bild~\ssaref{fig:BildII3-35}, delfigur a.

\item Med fyra dioder (s.k. Graetz-brygga) och inget mittuttag på
  transformatorns sekundärlindning släpper dioderna 1 och 3 igenom
  ström under den ena halvperioden.
  Dioderna 2 och 4 släpper igenom ström under den följande halvperioden.
  Detta illustreras i bild~\ssaref{fig:BildII3-35}, delfigur b samt 1:a och 2:a
  halvvågen.
\end{itemize}

% \newpage % layout
\subsection{Glättningskretsar}
\harecsection{\harec{a}{3.3.2}{3.3.2}}
\index{glättning}
\index{kraftaggregat!glättning}
\index{säkerhetsresistor}
\index{bleeder}
\label{glättningskretsar}

\mediumfig{images/cropped_pdfs/bild_2_3-36.pdf}{Glättning av likspänning}{fig:BildII3-36}
\mediumherefig{images/cropped_pdfs/bild_2_3-37.pdf}{Likriktarkoppling med spänningsdubbling}{fig:BildII3-37}

Efter likriktningen har växelspänningen omvandlats till en pulserande
likspänning som kan ''glättas''.
Efter likriktarna ansluts då ett filter som utför \emph{glättning}.
Glättningsfiltret kan till exempel bestå av laddningskondensatorn \(C_L\),
induktansen \(L\) och glättningskondensatorn \(C_S\) så som bild~\ssaref{fig:BildII3-36}
illustrerar.
Parallellt över denna kondensator ligger för elsäkerhetens skull en
urladdningsresistor \(R\) med hög resistans alltid inkopplad.

\emph{Säkerhetsresistorn} (eng. \emph{bleeder}) ska ladda ur kondensatorerna,
när kraftaggregatet är obelastat och inte anslutet till strömförsörjningen på
primärsidan.
Säkerhetsresistorn ska vara av trådlindad typ och kunna tåla fyra gånger sin
egen effektförbrukning.

I obelastat tillstånd är spänningen över laddningskondensatorn \(\sqrt{2}\)
gånger större än effektivvärdet på transformatorns sekundärspänning.
När en transformator i tomgång har ett effektivvärde av \qty{230}{\volt} över
sekundärlindningen blir spänningen över säkerhetsmotståndet
\(230\cdot\sqrt{2} \approx \qty{325}{\volt}\).

\subsubsection{Spänningshöjande likriktarkopplingar}
\index{kraftaggregat!spänningshöjning}
\index{kraftaggregat!spänningsdubbling}

Vid likriktning av växelspänningar enligt någon av ovanstående metoder behövs
en sekundärspänning från transformatorn av minst samma storlek som den önskade
likspänningen.
Önskas en högre likspänning, till exempel den dubbla, men med samma sekundärspänning
på transformatorn, så kan en speciell likriktarkoppling användas.

%\mediumfig{images/cropped_pdfs/bild_2_3-37.pdf}{Likriktarkoppling med spänningsdubbling}{fig:BildII3-37}

Bild~\ssaref{fig:BildII3-37} visar en spänningsdubblande koppling.
Under 1:a halvvågen laddas kondensator \(C_1\) upp.
Under 2:a halvvågen laddas kondensator \(C_2\) upp.
Kondensatorerna är kopplade i serie och den ena kondensatorn hinner inte bli
urladdad under tiden som den andra kondensatorn blir uppladdad.
Följden blir att belastningen ser kondensatorernas spänningar som seriekopplade
och därmed har en fördubbling av spänningen erhållits.
Det finns även kopplingar för flerdubbling av spänningar, vilka bland annat
brukade användas för att alstra accelerationsspänningen för TV-bildrör.

\smallfig{images/cropped_pdfs/bild_2_3-38.pdf}{Spänningsstabilisering}{fig:BildII3-38}

\subsection{Spänningsstabilisering}
\harecsection{\harec{a}{3.3.3}{3.3.3}}
\index{spänningsstabilisering}
\index{kraftaggregat!spänningsstabilisering}
\label{spänningsstabilisering}

Utspänningen från ett kraftaggregat tillåts i många fall endast att variera
mellan vissa värden, även om inspänningen och strömuttaget varierar mycket.
Ett vanligt sätt att hålla konstant spänning är att efter glättningsfiltret
anordna en stabiliseringskrets, som visas i bild~\ssaref{fig:BildII3-38}.

Glimlampan och zenerdioden har egenskapen att spänningsfallet över dem är i det
närmaste konstant inom ett visst strömområde.
Glimlampor arbetar på högre spänningar och används i utrustningar med
elektronrör.
Zenerdioder arbetar på de lägre spänningar som används i dagens elektronik.

Stabiliseringen tillgår så att till exempel zenerdioden får ingå som aktiv del
i en spänningsdelare, som består av en resistor i serie med belastningen och
zenerdioden parallellt med den.
Zenerdioden tar upp variationerna i belastningsströmmen, varvid spänningen över
spänningsdelarens uttag blir stabiliserad.
Vid större strömuttag kan zenerdioden inte ensam ta upp hela den effekt som den
reglerar bort.
I stället tas effekten upp av en eller flera transistorer som i sin tur
regleras av zenerdioden.

I vissa fall behövs i stället en reglerad utström från kraftaggregatet.
Även för detta ändamål används kopplingar med zenerdioder och transistorer.

Färdiga stabiliseringskretsar i form av integrerade kretsar är numera vanligare
än sådana som är uppbyggda av diskreta komponenter.
Exempel är linjära spänningsregulatorer som 7805 för \qty{5}{\volt} och 7912 för
\qty{-12}{\volt}.

\subsection{Switchaggregat}
\harecsection{\harec{a}{3.3.4}{3.3.4}}
\index{switchaggregat}
\index{kraftaggregat!switchaggregat}

Senare utvecklingsformer är så kallade switchade aggregat.
I sådana regleras spänningen eller strömmen genom sönderhackning (switching).
Genom att förändra förhållandet mellan till- och frånslagstiderna kan man skapa
det önskade medelvärdet.
Metoden ger hög verkningsgrad.
Switchfrekvensen är i storleksordningen \qty{20}{\kilo\hertz} eller högre.
Sådana kraftaggregat kan emellertid ge upphov till radiofrekventa störningar,
varför effektiv avstörning behövs.

Kraftaggregat som omvandlar från nätspänning till likspänning använder
den primärswitchade principen.
I ett primärswitchat aggregat likriktas nätspänningen och switchas på
primärssidan av transformatorn.
Eftersom frekvensen är relativt hög och inte riskerar mätta transformatorns
kärna på samma sätt som vid nätfrekvensen (\qty{50}{\hertz}), behöver kärnan
inte vara så stor.
På sekundärsidan likriktas sedan spänningen och glättning kan ske med relativt
små kondensatorer tack vare den höga frekvensen.
Genom att återkoppla spänningen till primärsidan kan utspänningen regleras i
primärswitchningen istället för att behöva stabiliseras på
sekundärsidan.
Därigenom kan förluster i stabiliseringskretsen undvikas.
Ett primärswitchat aggregat måste ha nätfilter för att klara EMC-kraven.

En annan kategori av switchade aggregat används för likspänningsomvandling, så
kallad DCDC-om\-vand\-ling.
Exempel på sådana är kallade drop-omvandlare, som kan används för att sänka
spänningen.
Andra omvandlare kan höja spänningen eller byta polaritet på den.
Dessa omvandlare arbetar inte sällan med frekvenser på \qty{200}{\kilo\hertz}
till \qty{2}{\mega\hertz}.
Likspänningsomvandlare har inte alltid galvanisk isolation mellan in- och utgång.

Numera finns även switchade ersättare med lägre effektförlust än i de äldre
linjära regulatorerna i 78- och 79-serien.
Problemet med dessa är att de kan generera störningar som behöver tas hänsyn
till.

Switchade kraftaggregat och spänningsstabilisatorer är nu vanliga eftersom
effektförlusterna kan hållas mycket lägre än i gamla linjära aggregat.
Switchningen innebär dock att störningar kan läcka ut både på ingång och utgång
såväl som genom direktutstrålning från själva aggregatet.
På en del switchade nätaggregat kan switchfrekvensen justeras manuellt med en
ratt.
På så sätt kan man flytta störningarna till en frekvens där deras inverkan
minskar.
Störningar kan uppträda både i differentiell och gemensam mode, vilket man måste
ta hänsyn till vid avstörning.

% Avsnitt 3.4 Förstärkare
\newpage
\section{Förstärkare}
\harecsection{\harec{a}{3.4}{3.4}}
\index{förstärkare}
\index{amplifier}
\index{blandning}
\index{mixing}

\subsection{Allmänt}
\label{förstärkarsteg_allmänt}

\mediumfig{images/cropped_pdfs/bild_2_3-40.pdf}{Från elektronrör till transistor}{fig:BildII3-40}

Elektronrör och transistorer, se bild~\ssaref{fig:BildII3-40}, är de
\emph{aktiva komponenter} (eng. \emph{active components}) som används i
oräkneliga elektroniska kopplingar för alstring av signaler, för
\emph{förstärkning} (eng. \emph{amplification}) och \emph{blandning} (eng.
\emph{mixing}) av signaler, för multiplicering av signalfrekvenser etc.

Transistorn presenteras i avsnitt~\ssaref{transistorn} och elektronröret i
avsnitt~\ssaref{elektronrör}.

Först förekom endast elektronrör.
Dessa har emellertid nästan helt ersatts av transistorer.
Elektronrör används dock fortfarande i viss mån, då främst i effektförstärkare
för sändare.
Det finns därför skäl att här behandla såväl elektronrör som transistorer.

\subsection{Huvudegenskaper hos förstärkare}
\subsubsection{LF- och HF-förstärkare}
\harecsection{\harec{a}{3.4.1}{3.4.1}, \harec{a}{3.4.2}{3.4.2}, \harec{a}{3.4.3}{3.4.3}}

\mediumfig{images/cropped_pdfs/bild_2_3-41.pdf}{Principen för förstärkare med elektronrör respektive transistor}{fig:BildII3-41}

Bild~\ssaref{fig:BildII3-41} visar principen för förstärkare med både
elektronrör och transistor.

\index{LF-förstärkare}
\index{förstärkare!LF}
Med \emph{LF-förstärkare} menas förstärkare som arbetar med signaler i det lägre
frekvensområdet, typiskt upp till cirka \qty{100}{\kilo\hertz}.
LF-förstärkare är mycket vanliga såväl i mottagare som sändare.
Utöver de aktiva komponenterna (transistorer, elektronrör) är kondensatorer
och resistorer de viktigaste passiva.

\index{HF-förstärkare}
\index{förstärkare!HF}
Med \emph{HF-förstärkare} menas förstärkare som arbetar med signaler
med högre frekvenser än dem i LF-området.
Även HF-förstärkare är mycket vanliga såväl i mottagare som sändare.
De används till exempel i mottagarnas ingångs- och mellanfrekvenssteg, liksom i
sändarnas oscillatorer, drivsteg och slutsteg.
Utöver de komponenter, som även finns i LF-förstärkare, används kombinationer
av frekvensberoende komponenter såsom induktorer och kondensatorer.

\paragraph{Förstärkning}
\index{förstärkning}
\index{förstärkare!förstärkning}
\index{gain}

Med \emph{förstärkning} (eng. \emph{gain}) avses här kvoten mellan amplituden i
utgående och inkommande signal, varvid frekvensgången har inverkan.

\paragraph{Frekvensgång}
\index{frekvensgång}
Frekvensgången anger hur förstärkningen varierar för olika frekvenser inom
förstärkarens bandbredd.

\paragraph{Bandbredd}
\index{bandbredd}
\index{förstärkare!bandbredd}
\index{bandwidth|see {bandbredd}}

%% k7per: definiera fulla data.
Det frekvensområde där förstärkaren arbetar med fulla data kallas
\emph{bandbredd} (eng. \emph{bandwidth}).
Bandgränserna uttrycks som en nedre och övre gränsfrekvens, där signalnivån
avviker från ett givet värde, vanligen med högst \qty{3}{\decibel}.

För LF-förstärkare för amatörradiobruk är kravet på bandbredd litet; inom ett
band av \qty{300}{\hertz} till \qty{3}{\kilo\hertz} uppnås godtagbar
återgivningskvalitet för tal.
Bandbredden bestäms främst av kondensatorer i kretsen avsedda för överföring
och avkoppling.

HF-förstärkare används för signaler med hög frekvens, typiskt
\qty{100}{\kilo\hertz} och däröver.
Det finns så kallade bredbandiga förstärkare för ett stort frekvensområde, men
även avstämda förstärkare för smala frekvensband.

\subsection{Grundkopplingar för förstärkarsteg}
\harecsection{\harec{a}{2.6.4.1}{2.6.4.1}, \harec{a}{2.6.4.2}{2.6.4.2}, \harec{a}{2.6.4.3}{2.6.4.3}, \harec{a}{2.6.4.4}{2.6.4.4}}
\label{förstärkare_grundkoppling}

\mediumfig{images/cropped_pdfs/bild_2_3-42.pdf}{Grundkopplingar för elektronrör och NPN-transistor}{fig:BildII3-42}

I det föregående har redan visats att en av polerna i ingången respektive
utgången i en förstärkare är gemensam.
I övre delen av bild~\ssaref{fig:BildII3-42} är rörförstärkarens katod den gemensamma
polen -- därav namnet katodkoppling.
På liknande sätt är NPN-transistorns emitter gemensam
-- därav namnet emitterkoppling.

På ett liknande sätt kan någon annan pol vara gemensam.
Man får då i stället en baskoppling eller kollektorkoppling.

Beroende av kopplingssätt fås olika egenskaper.
I bild~\ssaref{fig:BildII3-42} visas tre olika grundkopplingar för ett elektronrör (triod)
respektive en NPN-transistor.

I praktiken känns en grundkoppling igen på vilken elektrod som är
avkopplad till nollpotential över en kondensator.

\index{emitterkoppling}
\index{förstärkare!emitterkoppling}
\emph{Emitterkoppling} används för LF och HF när hög förstärkning eftersträvas.
Eftersom effektförstärkningen är produkten av spännings- och
strömförstärkningen, fås en effektförstärkning mellan 200 och 50000
gånger.
Nackdelen med denna koppling är den ibland låga ingångsimpedansen och den
relativt låga gränsfrekvensen.

\index{baskoppling}
\index{förstärkare!baskoppling}
\emph{Baskoppling} används för HF-förstärkare på grund av sin höga
gränsfrekvens och goda isolation mellan in- och utgång.

\index{kollektorkoppling}
\index{förstärkare!kollektorkoppling}
\emph{Kollektorkoppling} används när hög ingångsimpe\-da\-ns och låg
utgångsimpedans önskas.
Denna koppling har emellertid ingen spänningsförstärkning, men kan användas för
impedansomvandling.

%% k7per: Kollektorkoppling, utresistansen borde vara 50ohm, inte 50kohm.
\begin{table*}[!ht]
  \begin{center}
    \caption{Grundkopplingarnas typiska egenskaper vid NPN-transistor}
    \begin{tabular}{p{0.15\textwidth}|p{0.22\textwidth}|p{0.18\textwidth}|p{0.20\textwidth}}
      \bf Egenskap & \bf Emitterkoppling & \bf Baskoppling & \bf Kollektor\-koppling \\
      \(Z_{in}\) & medel \quad \qty{1}{\kilo\ohm} & liten \quad \qty{50}{\ohm} & stor \quad \qty{100}{\kilo\ohm} \\
      \(Z_{ut}\) & medel \quad \qty{10}{\kilo\ohm} & stor \quad \qty{100}{\kilo\ohm} & liten \quad \qty{50}{\ohm} \\
      Förstärkning & & & \\
      \quad Ström- & stor \quad 100 ggr & <1 \quad 0,9 ggr & stor \quad 100 ggr \\
      \quad Spänning- & stor \quad 100 ggr & stor \quad 100 ggr & <1 \quad 0,99 ggr \\
      \quad Effekt- & mycket stor 10000 ggr & stor \quad 100 ggr & stor \quad 100 ggr \\
      Fasläge & motfas \quad \ang{180} & medfas \quad \ang{0} & medfas \ang{0} \\
    \end{tabular}
  \end{center}
\end{table*}

\subsection{Stabilisering av arbetspunkten}

För att en förstärkare ska kunna arbeta på avsett sätt måste
arbetspunkten, det vill säga arbetsströmmens vilovärde, ställas rätt.

Det gör man genom att placera en förspänning över den styrande
elektroden i elektronröret eller transistorn i fråga.

I en katodkopplad rörförstärkare innebär det att styrgallret ska
ges en viss negativ spänning i förhållande till katoden.
Det kan man göra till exempel med en separat spänningskälla eller vanligare med
en avkopplad resistor mellan katod och minuspolen (jord).

I en emitterkopplad transistorförstärkare innebär det att basen ska
ges en viss positiv spänning i förhållande till emittern.
Det kan man göra till exempel med en separat spänningskälla eller vanligare med
en avkopplad resistor mellan emittern och minuspolen samt en resistiv
spänningsdelare mellan plus- och minuspolen.

\subsection{Klass A-, B- och C-förstärkare}
\harecsection{\harec{a}{3.4.4}{3.4.4}}
\label{klassabc}

\subsubsection{Arbetspunkt}
\index{arbetspunkt}
\index{distorsion}

\emph{Arbetspunkten} för förstärkare väljs olika beroende på önskat arbetssätt.
En olämpligt vald arbetspunkt resulterar i förvrängning av utsignalens form i
förhållande till insignalens form, så kallad \emph{distorsion}.
Distorsion uppstår även vid överstyrning, det vill säga när insignalens
amplitud är för stor för att kunna återges med oförändrad form, även om
arbetspunkten är rätt vald.

Med avseende på arbetspunktens läge klassas därför förstärkare på sätt
som framgår av följande diagram för transistorer eller elektronrör.
En emitterjordad NPN-transistor får anses mest motsvara
elektronrörskopplingen här nedan.
Anodströmmen \(I_a\) motsvaras då närmast av kollektorströmmen
\(I_C\) och styrgallerspänningen \(U_{gi}\) av spänningen
\(U_{BE}\).
Den stora skillnaden är att styrgallerspänningen i dessa
fall alltid är negativ medan bas/emitterspänningen är positiv.
Styrspänningens relativa läge (arbetspunkten) mellan olika
arbetsklasser är emellertid lika.

\tallfig[0.35]{images/cropped_pdfs/bild_2_3-44.pdf}{Förstärkare i klass A}{fig:BildII3-44}

\subsubsection{Klass A}
\index{klass A}

Bild~\ssaref{fig:BildII3-44} illustrerar klass A, vilket är ett arbetssätt i linjära
LF- och HF-förstärkarsteg, till exempel i mottagare samt AM- och SSB-modulerade
sändare.
Vilovärdet på strömmen i huvudkretsen, den så kallade arbetspunkten, placeras i
mitten på den rakaste delen av styrkaraktäristiken (\(I=0,5\cdot I_{max}\)).
Därmed fås låg distorsion.
Verkningsgraden är upp till 50~\%.

\subsubsection{Klass AB}
\index{klass AB}

Klass AB är ett godtagbart linjärt arbetssätt för AM- respektive SSB-modulering,
men med en lägre viloström.
Arbetspunkten ligger mellan den för klass A och B.
Ett linjärt arbetssätt enligt klass A är visserligen önskvärt vid SSB, men
verkningsgraden är lägre.
Klass AB är en kompromiss med bättre verkningsgrad utan en alltför stor
distorsion.

\smallfig{images/cropped_pdfs/bild_2_3-45.pdf}{Förstärkare i klass B}{fig:BildII3-45}

\subsubsection{Klass B}
\index{klass B}

Klass B (bild~\ssaref{fig:BildII3-45}) är ett olinjärt arbetssätt med en låg
viloström i förhållande till \(I_{max}\), det vill säga att arbetspunkten ligger
i nederkant av styrkaraktäristikans nedre krökta del.
Verkningsgraden är upp till 67~\%.
Trots det används klass B i linjära LF-och HF-förstärkarsteg till exempel i
SSB-sändare.

Om klass B skulle tillämpas i ett slutsteg med endast ett rör eller en
transistor skulle större delen av uteffekten förloras i splatter,
det vill säga som förvrängda signaler långt vid sidan om den egentliga
nyttosignalen.
Ett sätt att undvika det är att använda en avstämd utgångskrets med högt
Q-värde.
Linjär förstärkning kan också erhållas med två mottaktkopplade rör eller
transistorer i klass B.
Utgångskretsen behöver då inte vara avstämd av linjäritetsskäl.

\smallfigpad{images/cropped_pdfs/bild_2_3-46.pdf}{Förstärkare i klass C}{fig:BildII3-46}

\subsubsection{Klass C}
\index{klass C}

Bild \ssaref{fig:BildII3-46} visar klass C som används i HF-förstärkar\-steg i
FM- och CW-sändare.
Arbetssättet är kraftigt olinjärt.
Viloströmmen är noll, det vill säga arbetspunkten ligger på den negativa delen
av styrkaraktäristikan.
Endast toppen av den ena halvvågen av insignalen återges och i starkt
förvrängd form.
Verkningsgraden är upp till 80~\%.
En resonanskrets med högt Q-värde dämpar övertoner och behövs som utgångskrets
varvid amplituddistorsion inte framstår som besvärande vid CW och FM.
Med hjälp av resonanskretsen kan frekvensmultiplicering utföras med
förstärkare i klass C.
(På följande tre bilder är \(I_R\) = anodviloström.)

\subsection{Frekvensmultiplicering}
\index{frekvensmultiplicering}
\index{frequency multiplication|see {frekvensmultiplicering}}

\mediumfig[0.8]{images/cropped_pdfs/bild_2_3-47.pdf}{Frekvensmultipliceringskedja}{fig:BildII3-47}

\emph{Frekvensmultiplicering} (eng. \emph{frequency multiplication}) kan
användas för att skapa en högre frekvens än den som avges av oscillatorn.
Bild \ssaref{fig:BildII3-47} visar hur oscillatorn följs av ett eller flera
frekvensmultiplicerande förstärkarsteg som arbetar i klass C.

I utgången av ett frekvensmultiplicerande steg måste finnas en resonanskrets som
är avstämd till önskad frekvens, dvs. till en av insignalens övertoner.
Denna överton förstärks i efterföljande förstärkarsteg, vilket också kan vara
frekvensmultiplicerande.

Ju högre multiplikationsfaktorn är desto högre förspänning krävs för
att resonanskretsen i utgången ska svänga obehindrat.
Med hög multipliceringsfaktor i ett enda steg dämpas signalen så mycket att
en hög förstärkning behövs i efterföljande steg.
I praktiken anordnas därför en kedja av frekvensdubblande och
frekvenstripplande steg.
Den totala multipliceringsfaktorn är faktorerna för vardera steget
multiplicerat med varandra.

Som exempel visar bilden blockschemat för en VHF-sändare med
oscillatorkristaller i \qty{8}{\mega\hertz}-området.
Som räkneövning kan andra kristallfrekvenser sättas in för beräkning av den
slutliga sändningsfrekvensen.
I frekvensmultiplicerande sändare kan även slutsteget arbeta i klass C, vilket
är vanligt i sändare för telegrafi eller FM-telefoni.
För att förhindra utsändning av alla de övertoner som alstras i
förstärkarkedjan förses slutstegets utgång med en resonanskrets som är
avstämd till sändningsfrekvensen.
Övertonsdämpningen kan förbättras ytterligare med ett efterföljande
lågpassfilter.
Övertoner för \qty{144}{\mega\hertz} är \qty{288}{\mega\hertz},
\qty{432}{\mega\hertz} och så vidare.

Frekvensmultiplicering behöver nödvändigtvis inte göras med ett förstärkarsteg
i klass C.
En diod har nämligen olinjär karaktäristik och därmed alstras övertoner i
de strömmar som passerar genom den.
En av dessa övertoner kan filtreras fram och förstärkas.
Till exempel finns det frekvenstripplingssteg byggda kring en speciell typ av
kapacitansdiod -- varaktordiod.
Vanliga frekvensområden för så kallad varaktortripplare är
144/\qty{432}{\mega\hertz} och 432/\qty{1296}{\mega\hertz}.

Såväl signalen från en kristalloscillator som den från en VFO kan
multipliceras till en högre frekvens.

Förr täckte VFO i amatörradiosändarna oftast frekvensområdet
\SIrange{3,5}{3,8}{\mega\hertz}.
Med en så vald VFO-frekvens kunde alla upplåtna frekvensband för
amatörradio nås med frekvensmultiplicering.
De ursprungliga amatörradiobanden i KV-området ligger fortfarande harmoniskt
relaterade av detta skäl. Således:
%%
\begin{align*}
  3,5 \cdot 2 & = \qty{7}{\mega\hertz} \\
  3,5 \cdot 2 \cdot 2 & = \qty{14}{\mega\hertz} \\
  3,5 \cdot 2 \cdot 3 & = \qty{21}{\mega\hertz} \\
  3,5 \cdot 2 \cdot 2 \cdot 2 & = \qty{28}{\mega\hertz}
\end{align*}
%%
Vid frekvensmultiplicering flerfaldigas inte bara oscillatorfrekvensen utan
även variationerna i den.
Om till exempel VFO-frekvensen i området \qty{3,5}{\mega\hertz} ändras med
\qty{50}{\hertz}, ändras utfrekvensen i området \qty{28}{\mega\hertz} med
\(2 \cdot 2 \cdot 2 \cdot 50 = \qty{400}{\hertz}\).
Alla frekvenser i signalen multipliceras på detta sätt.
Amplitudmodulerad telefoni kan därför inte överföras genom en
frekvensmultipliceringskedja utan att talet förvrängs.

% \newpage % layout
\subsection{Sändarslutsteg}
\harecsection{\harec{a}{2.8.2}{2.8.2}}
\index{slutsteg}
\index{förstärkare!slutsteg}

\subsubsection{Slutsteg med en transistor}

\smallfig{images/cropped_pdfs/bild_2_3-48.pdf}{Slutsteg med en transistor}{fig:BildII3-48}

Transistorslutsteg för HF byggs vanligen emitterkopplade på grund av den
högre effektförstärkningen.
Moderna LDMOS-transistorer kan lämna en kilowatt.

Bild \ssaref{fig:BildII3-48} visar ett sådant förstärkarsteg.
Kollektorbelastningen består av en resonanskrets.
För att anpassa transistorns kollektorimpedans till resonanskretsens
impedans, har kollektorn anslutits till ett uttag på kretsens spole.

Drosseln \(Dr\) och kondensatorn \(C\) fungerar som en HF-mässig avkoppling av
strömförsörjningen.
Uteffekten tas ut från resonanskretsen över en kopplingslindning med samma
impedans som belastningen.
För linjär återgivning krävs drift i klass A eller möjligen klass AB.

\subsubsection{Slutsteg med två transistorer}
\index{mottaktskopplat slutsteg}
\index{slutsteg!mottaktskopplat}
\index{push-pull amplifier}

\mediumfig{images/cropped_pdfs/bild_2_3-49.pdf}{Mottaktskopplat slutsteg med transistorer}{fig:BildII3-49}

Bild \ssaref{fig:BildII3-49} visar ett \emph{mottaktkopplat} (eng.
\emph{push-pull amplifier}) förstärkarsteg i klass B, vilket har god verkningsgrad
samtidigt som det är nöjaktigt linjärt för SSB i amatörradio.
I ett slutsteg med endast en transistor skulle denna behöva klara fyra gånger
så stor förlusteffekt.

På grund av de låga impedansvärdena i transistoriserade förstärkarsteg används
transformatorer, vilka inte är frekvensselektiva och därför inte dämpar
övertoner.
Med mottaktkopplingen alstras dock inte jämna övertoner.
För övertonsdämpning används fast avstämda bandpassfilter, ofta ett per
frekvensband, mellan drivsteg och slutsteg samt mellan slutsteg och antenn.

För noggrann anpassning till antennen behövs en antennanpassare --
så kallad matchbox -- med ett \(\pi \)-, T- eller L-kopplat LC-filter.

Att ett slutsteg är ''bredbandsavstämt'' är således en fråga om definitioner.

\subsubsection{Högeffektslutsteg med en tetrod}

\mediumfig{images/cropped_pdfs/bild_2_3-50.pdf}{Högeffektslutsteg med en tetrod}{fig:BildII3-50}

Bild \ssaref{fig:BildII3-50} visar ett effektslutsteg för HF med ett elektronrör,
en så kallad tetrod, i katodkoppling.
Man kan även använda en triod eller en pentod.

Med LC-kretsen i styrgallerkretsen filtreras (selekteras) den önskade
signalfrekvensen ut ur signalerna från föregående steg.

Drosslarna \(Dr\) spärrar HF och kondensatorerna \(C_1\), \(C_2\) och
\(C_3\) kortsluter (avkopplar) HF till jord,
allt för att hindra HF att komma in i kraftaggregatet.

HF-förstärkare kan råka i oönskad självsvängning.
Orsakerna kan vara många, bland annat dålig avkoppling av matningsspänningar,
induktiv och/eller kapacitiv återkoppling i kretsarna med mera.

Återkopplingsvägar både före och efter röret kan bilda oavsiktliga
resonanskretsar som genererar självsvängning, ofta på mycket höga
frekvenser till exempel i VHF-området.
Sådana så kallade parasitsvängningar kan stoppas/dämpas med UHF-drosslar (UHF
Dr) omedelbart intill röranslutningarna.

En åtgärd mot självsvängning i elektronrör är en motkopplingsväg från anod till
styrgaller över en trimningsbar så kallad neutraliseringskondensator \(C_N\).
Slutstegets utgångskrets kan utformas på olika sätt.
Bilden visar ett numera vanligt sätt, det så kallade \(\pi\)-filtret (utläses
pi-), som fungerar som
\begin{itemize}
  \item en resonanskrets som är avstämd till sändningsfrekvensen
  \item ett övertonsdämpande lågpassfilter
  \item anpassning mellan rörets utgångsimpedans och antenntilledningen.
\end{itemize}

\subsection{Högeffektslutsteg med två gallerjordade trioder (elektronrör)}

\mediumfig{images/cropped_pdfs/bild_2_3-51.pdf}{Högeffektslutsteg med två trioder}{fig:BildII3-51}

Bild \ssaref{fig:BildII3-51} visar en gallerjordad koppling med två trioder.
Gallerjordad koppling innebär att elektronrörets styrgaller ligger på
HF-mässig nollpotential medan styrsignalen matas in på katoden.
Likspänningen mellan katod och styrgaller väljs så att rörets arbetspunkt blir
den avsedda.

Gallerjordad koppling passar särskilt för slutsteg med höga effekter,
men fordrar en högre styreffekt än andra kopplingar.
I gengäld ''överförs'' styreffekten till utgången via röret och ingår där i
uteffekten.
I gallerjordad koppling är kapacitansen låg mellan katod och anod, det vill
säga mellan in- och utgång.
Därmed är risken för självsvängning betydligt mindre än i ett katodjordat steg.

Uteffekten kan ökas genom att parallellkoppla två eller flera rör, som då ska
ha så lika data som möjligt.
Uteffekten står i direkt proportion till antalet rör.

Flera parallellkopplade rör medför emellertid ökade totala rörkapacitanser och 
ökade kapacitanser i kopplingsledningarna med mera, vilket är till nackdel vid
höga frekvenser.

Ett enda slutrör för hela effekten är emellertid dyrare än flera små med
tillsammans jämförbar effekt.
Mottaktkoppling av två rör (eng. ''push-pull'') i stället för parallellkoppling
har en fördel i högre förstärkning, men nackdelar i mer komplicerad
bandomkoppling av resonanskretsar med mera.
I moderna rörutrustade slutsteg för amatörradio förekommer därför endast ett
slutrör eller flera parallellkopplade.
Utgångskretsen är i regel ett \(\pi \)-filter med manuell eller automatisk
avstämning.

\subsection{Slutsteg med elektronrör jämfört med transistoriserade slutsteg}

Ett slutsteg med transistorer är kompakt och skaktåligt och använder
bara klenspänningar.
Det är därför särskilt väl lämpat för portabelt och mobilt bruk.

Men transistorer är känsliga för överbelastning.
Redan ytterst kortvarig överbelastning eller överspänning kan förstöra dem.
Transistorer är också känsliga för termisk överbelastning.
Särskilt vid höga effekter i trånga utrymmen är det nödvändigt med god kylning,
eventuellt med fläkt.

Ett slutsteg med elektronrör är inte så skaksäkert, men är mycket okänsligare
i övriga avseenden.
En nackdel är att det behövs extra effekt för uppvärmning av rörens katoder
samt höga anodspänningar, som är farliga vid ovarsamhet.
På grund av behovet av flera olika spänningar är även strömförsörjningen för
ett slutsteg med elektronrör mer komplicerad och omfångsrik.

\newpage
\subsection{Toppvärdeseffekt PEP}
\harecsection{\harec{a}{1.9.6}{1.9.6}}
\index{toppvärdeseffekt}
\index{effekt!toppvärdes}
\index{PEP}
\index{effekt!PEP}
\index{Peak Envelope Power (PEP)}
\index{effekt!Peak Envelope Power (PEP)}
\index{PEP-effekt}
\index{förstärkare!PEP-effekt}
\label{PEP-effekt}

Uteffekten från en sändare kan mätas över en konstlast (eng. \emph{dummy load}).
En konstlast är en resistor som kan omsätta sändarens hela effekt till värme.

Med en HF-mätprob och en detektordiod eller en HF-voltmeter kan man mäta
effektivvärdet på spänningen över konstlasten och beräkna uteffekten med
formeln
%
\[P_{ut} = \dfrac{U^2}{R}\]
%
\(U\) = HF-spänningens effektivvärde och 
\(R\) = resistansen i konstlasten

\paragraph{Beräkning och mätning av PEP-effekt}

Uteffekten definierad som \emph{Peak Envelope Power (PEP)} \cite[1.157]{ITU-RR}
är ''den medeleffekt som matas in i en antennmatarledning under det högsta
effektvärdet inom en frekvenscykel och mätt under normal drift''.
%
\[P_{PEP} = \left[\dfrac{\hat{u}}{\sqrt{2}}\right]^2\dfrac{1}{R} =  \dfrac{\hat{u}^2}{2R}\]
%
där \(\hat{u}\) är momentanspänningen på den största modulationstoppen.

\mediumfig{images/cropped_pdfs/bild_2_3-52.pdf}{Bestämning av PEP-effekten}{fig:BildII3-52}

På grund av SSB-signalens karaktär kan man inte mäta effektivvärdet av
uteffekten från en SSB-sändare.

Bild \ssaref{fig:BildII3-52} visar hur modulationen för ett \emph{aaah} ser ut
på ett oscilloskop.

Moduleringsspänningens topp-till-topp\-värde \(\hat{u}_{t-t}\) mäts lämpligen med ett
oscilloskop när slutsteget är kopplat till en konstlast.

Med topp-till-toppvärdet känt kan man med följande formler beräkna
toppvärdet (amplituden)
%
\[
\begin{array}{lll}
\text{toppvärdet (amplituden)} &  \quad \hat{u}= & \dfrac{\hat{u}_{t-t}}{2}\\
& & \\
\text{effektivvärdet} &\quad  U =& \dfrac{\hat{u}}{\sqrt{2}}
\end{array}
\]%
%
Effekten vid moduleringstopparna, så kallad \emph{Peak Envelope Power (PEP)},
kan beräknas med följande formler.
%
\[
P_{PEP} = \dfrac{U^2}{R} \quad\text{respektive}\quad
P_{PEP} = \dfrac{\hat{u}_{t-t}^2}{8R}
\]

\subsection{Linjäritetskontroll vid SSB}
\harecsection{\harec{b}{7.2.4}{7.2.4}}
\index{SSB!linjäritetskontroll}

\largefig{images/cropped_pdfs/bild_2_3-53.pdf}{Linjäritetskontroll vid SSB}{fig:BildII3-53}

Bild \ssaref{fig:BildII3-53} visar två-tons linjäritetskontroll av SSB.

Linjäriteten i en SSB-sändare kan kontrolleras med ett oscilloskop.
Sändaren moduleras då med två övertonsfria toner.

Slutsteget bör först belastas med konstlast upp till max tillåten effekt.
Resultatet jämförs därefter med antennen som last.

\newpage
\subsubsection{Linjäritetens betydelse i förstärkare}
\harecsection{\harec{a}{3.4.5}{3.4.5}}
\index{förstärkning!linjäritet}

\mediumfig[0.7]{images/cropped_pdfs/bild_2_3-54.pdf}{Linjäritetens betydelse}{fig:BildII3-54}

Bild \ssaref{fig:BildII3-54} visar mer i detalj olinjäritetens inverkan på
signalen.

Förstärkningen bör ske med god verkningsgrad och minsta möjliga förvrängning,
så att det alstras ett minimum av oönskade frekvenser inom minsta möjliga
bandbredd.

Linjär förstärkning innebär att den är lika över hela det aktuella
frekvensområdet.
Frekvensgången måste därför vara så rak som möjligt.
Med tilltagande olinjäritet tillkommer nämligen allt fler oönskade frekvenser.

Det uppstår blandningsprodukter av högre ordning vid olinjär förstärkning.
Genom förvrängning på grund av olinjär förstärkning uppstår ömsesidiga summa-
och skillnadsfrekvenser av de modulerande frekvenserna.

Varje sådan blandningsprodukt blandar sig additivt och subtraktivt med
grundfrekvenserna till ytterligare blandningsprodukter av näst högre ordning.

Dessa är
\begin{itemize}
\item blandningsprodukter i LF-området och deras övertoner, vilka
  undertrycks i efterföljande HF-krets

\item grundfrekvenserna och deras harmoniska övertoner, som alla ner
  till 1:a harmoniska dämpas kraftigt av efterföljande HF-krets

\item alla summa- och skillnadsfrekvenser av de förstnämnda frekvenserna.
\end{itemize}

I området för nyttofrekvenserna kallas dessa produkter för
intermodulationsprodukter och ger talförvrängning.

Utanför nyttofrekvenserna uppfattas intermodulationsfrekvenserna som
störningar och kallas splatter.
På grund av det lilla frekvensavståndet till nyttosignalen kan den
intermodulation, som alstrats i slutsteget inte filtreras bort i efterhand.

Vid linjär drift uppträder grundfrekvensernas övertoner och
intermodulationsfrekvenser endast svagt inom och utom
överföringsbandet och kommer knappast att uppfattas som inkräktande på
annan radiotrafik.
De svaga övertonerna kommer också att dämpas tillräckligt i \(\pi \)-filtret
och eventuella ytterligare övertonsfilter.

\mediumminusbotfig{images/cropped_pdfs/bild_2_3-55.pdf}{Dioddetektorn}{fig:BildII3-55}

\subsection{Utstyrningskontroll av slutsteg}
\index{förstärkning!utstyrningskontroll}
\label{förstärkare_utstyrningskontroll}

Slutstegets linjära utstyrningsområde överskrids om ingångssignalens
amplitud blir för stor.
Då ökar utgångssignalens amplitud inte mycket mer, men utgångssignalens toppar
blir tillplattade (klippta).
Det betyder att slutsteget är överstyrt.

Vid överstyrning uppstår signalförvrängningar som medför intermodulation,
förvrängt tal, splatterstörningar och övertoner.
Den extra effektökning som uppnås med överstyrning förbrukas i stort sett
till signalförvrängning och kommer inte nyttosignalen till godo.
Överstyrning ska därför undvikas.

Drivstegets uteffekt får inte vara så stor att slutsteget blir överstyrt.
Ett slutsteg med jordad katod blir fullt utstyrt redan vid en driveffekt av ett
fåtal watt.
Är uteffekten från drivsteget större än vad som behövs för full utstyrning av
slutsteget och driveffekten inte kan regleras ner, måste en dämpsats
kopplas in mellan drivsteg och slutsteg.
En sådan dämpsats kan behöva ta upp en betydande effekt, från en vanligt
förekommande amatörradiosändare med upp till 100~watt PEP.

Ett slutsteg med jordat galler fordrar en större driveffekt, varvid
risken för överstyrning i slutsteget är något mindre och de
förebyggande åtgärderna inte så omfattande.

Linjära slutsteg innehåller oftast en funktion kallad ALC (Automatic Level
Control), som kontinuerligt känner av driveffektens inverkan på slutsteget.
När driveffekten blir för hög alstras en kontrollspänning i proportion till
utstyrningsgraden.
ALC-spänning återförs till drivsteget och reglerar dess uteffekt så att
överstyrning av slutsteget inte sker.
I transistoriserade slutsteg skapas ALC-spänningen genom likriktning av
slutstegets utspänning.
I rörslutsteg börjar styrgallret dra ström när styrgallerspänningen
blir positiv i signaltopparna, vilket används för att styra ALC-spänningen.
När ALC-regleringen sätter in är överstyrningen således redan ett faktum.
Överstyrning kan ske både på LF- och HF-nivå.

En orsak till övermodulering är för stor amplitud på den modulerande signalen.
Detta kan bland annat avhjälpas med inställning av mikrofonförstärkaren och
riktig mikrofonhantering.

% Avsnitt 3.5 Detektorer -- Demodulatorer
\section{Detektorer -- Demodulatorer}
\harecsection{\harec{a}{3.5}{3.5}}
\label{detektorer}
\index{detektor}
\index{demodulator}

% \mediumminusbotfig{images/cropped_pdfs/bild_2_3-55.pdf}{Dioddetektorn}{fig:BildII3-55}

\subsection{Allmänt}
\label{detektorer_allmänt}

Sändaren omvandlar informationen i lågfrekventa signaler till högfrekvens som
kan strålas ut från en antenn.
I mottagaren återskapas informationen genom att den högfrekventa signalen demoduleras.

Vanligen sker signalbehandlingen i mottagaren i flera steg, där den högfrekventa
radiosignalen först blandas ner till en mellanfrekvens (MF) och sedan
demoduleras till en lågfrekvent signal (LF).

Men det finns även direktblandade mottagare, som blandar ner radiosignalen
direkt till lågfrekvens.

I mottagare som är specialiserade för ett sändningsslag, används bara en typ av
demodulator medan mottagare för flera sändningsslag, AM, SSB/CW, FM et cetera
har flera demodulatorer.
Det finns många typer och namn på demodulatorer, till exempel detektor och
diskriminator.
Här beskrivs några av dem.

\subsection{AM-detektorer}
\index{detektor!AM}
\index{amplitudmodulation!detektor}

\subsubsection{Dioddetektorn AM (A3E)}
\harecsection{\harec{a}{3.5.1}{3.5.1}, \harec{a}{3.5.2}{3.5.2}}
\index{dioddetektor}
\index{amplitudmodulation!dioddetektor}
\index{A3E!dioddetektor}

% \mediumbotfig{images/cropped_pdfs/bild_2_3-55.pdf}{Dioddetektorn}{fig:BildII3-55}

Bild~\ssaref{fig:BildII3-55} visar en superheterodynmottagare där den sista
MF-kretsen är induktivt kopplad till demoduleringsdioden.
Den amplitudmodulerade MF-signalen visas som ett amplitud/tid-diagram.

Dioden klipper antingen de negativa eller positiva halvvågorna,
beroende på hur den är vänd -- polariserad.

LF-signalen filtreras ut ur de högfrekventa pulserna med ett LF-lågpassfilter.

LF-signalen är nu överlagrad på en likspänning.
I talpauserna sänds bara bärvågen och då lämnar AM-demodulatorn bara
likspänning, som skiljs från LF-förstärkaren med en kondensator.
Kondensatorn släpper bara igenom LF-signalen, som förstärks.

Dioddetektorn följer amplituden och är ett exempel på en amplitudformsdetektor.

\mediumminusbotfig{images/cropped_pdfs/bild_2_3-56.pdf}{Produktdetektor för AM (A3E) och CW (A1A)}{fig:BildII3-56}

\subsubsection{Produktdetektorn SSB (J3E)}
\harecsection{\harec{a}{3.5.3}{3.5.3}}
\index{SSB!produktdetektor}
\index{J3E!produktdetektor}
\index{produktdetektor}


Det finns flera metoder att demodulera en SSB-signal, såsom fasningsmetoden,
filtermetoden och den så kallade tredje metoden.
Filtermetoden är numera den allra vanligaste och beskrivs här samt illustreras i
bild~\ssaref{fig:BildII3-56}.
%
En SSB-signal med undertryckt bärvåg består av endast ett sidband.
Det andra sidbandet och bärvågen undertrycks i sändaren.
%
Vid demoduleringen av SSB-signalen alstras i mottagaren en signal som
ersättning för den bärvåg som undertrycktes i sändaren.
Det undertryckta andra sidbandet ersätts inte.

I en mottagare med direktblandning blandas SSB-signalen med VFO-signalen,
varvid en del av blandningsprodukterna faller ut på LF-nivå.

I en superheterodynmottagare däremot, blir SSB-signalen först blandad
med en VFO-signal och som resultat erhålls en mellanfrekvens MF.
Den till MF omvandlade signalen förstärks, filtreras och blandas med en
lokal BFO-signal i ytterligare en blandare, kallad produktdetektor.
Några blandningsprodukterna faller ut på LF-nivå.
Ett lågpassfilter följer efter detektorn för att filtrera ut LF-signalerna.

Numera består produktdetektorn vanligen av en ringblandare, som i ett omvänt
förlopp även kan användas vid DSB-modulering i en sändare.
Bilden visar demoduleringen av en SSB-signal som innehåller tre LF-toner.

\mediumtopfig{images/cropped_pdfs/bild_2_3-57.pdf}{Amplitudbegränsning vid FM-mottagning}{fig:BildII3-57}
\mediumbotfig[0.7]{images/cropped_pdfs/bild_2_3-58.pdf}{Ideal arbetslinje för diskriminator}{fig:BildII3-58}

\subsubsection{CW-/SSB-detektorer CW (A1A)}
\index{CW!detektor}
\index{SSB!detektor}
\index{A1A!detektor}
\index{detektor!CW}
\index{detektor!SSB}
\index{detektor!A1A}

Även telegrafisignaler, även kallat CW, blir demodulerade när MF-signalerna
och BFO-signalen blandas i en produktdetektor.

Till skillnad från SSB är det vid CW inte nödvändigt med en given skillnad
mellan MF- och BFO-frekvenserna.
Frekvensskillnaden påverkar bara överlagringstonens frekvens, men inte
läsligheten av CW-budskapet.

Många moderna mottagare har en fast BFO-fre\-kv\-e\-ns för CW, som ger en
\qty{800}{\hertz}-ton vid rätt frekvensinställning.
I stället för lågpassfiltret för SSB, används ibland ett bandpassfilter, som
bara släpper igenom CW-signaler i frekvensområdet \qty{800}{\hertz} -- en
idealfrekvens för god läsbarhet av morsetecken.

\subsection{FM- och PM-detektorer}
\harecsection{\harec{a}{3.5.4}{3.5.4}, \harec{a}{4.2.4}{4.2.4}, \harec{a}{4.3.5}{4.3.5}}

\index{FM-detektor}
\index{FM!detektor}
\index{detektor!FM}
\index{PM-detektor}
\index{PM!detektor}
\index{detektor!PM}
\index{Automatic Frequency Control (AFC)}
\index{AFC}
\label{fm_detektor}


Vid vinkelmodulering överförs informationen enbart genom frekvens-
eller fasvariationer i bärvågen.
De amplitudvariationer som kan uppstå före demoduleringen är ej önskvärda i
detta sändningsslag.
Av den anledningen finns i FM-mottagare en amplitudbegränsare (eng.
\emph{limiter}) före diskriminatorn (se bild~\ssaref{fig:BildII3-57}).
Frekvensvariationerna i den FM-modulerade signalen omvandlas därefter av
detektorn till LF-spän\-ning som motsvarar det utsända talet.

% \mediumfig{images/cropped_pdfs/bild_2_3-58.pdf}{Ideal arbetslinje för diskriminator}{fig:BildII3-58}

Demoduleringen ska ske med mottagaren inställd mitt på avsedd sändarfrekvens.
Ett hjälpmedel för det är en indikator, som vid rätt inställning visar värdet
noll.
Positivt eller negativt utslag anger att inställningen är för högt respektive
för lågt i frekvens, som illustreras i bild \ssaref{fig:BildII3-58}.
En sådan indikator fanns i tidiga FM-mottagare.
Nu används i stället en \emph{Automatic Frequency Control (AFC)} som själv
ställer in mottagaren om sändarfrekvensen är tillräckligt nära.

\newpage
\subsubsection{Slope-detektorn -- Diskriminatorn FM (F3E)}
\index{slope-detektorn}
\index{detektor!slope-detektorn}
\index{FM!slope-detektorn}
\index{FM-diskriminator}
\index{detektor!FM}
\index{detektor!F3E}
\index{FM!detektor}
\index{F3E!detektor}

\mediumfig[0.8]{images/cropped_pdfs/bild_2_3-59.pdf}{Slope-detektorn}{fig:BildII3-59}

Bild \ssaref{fig:BildII3-59} visar två resonanskretsar som är kopplade induktivt
till den sista MF-kretsen.
Resonansfrekvensen för dessa båda kretsar är något högre respektive något lägre
än mellanfrekvensen.
De signalspänningar som uppträder över resonanskretsarna likriktas och
seriekopplas med varandra med motsatt polaritet.

När de båda resonanskretsarna matas med samma frekvens, kommer
likspänningarna att ta ut varandra.
När frekvensen avviker uppåt i frekvens, kommer kretsen med den högre
resonansfrekvensen i kraftigare svängning än den andra kretsen och avger högre
likriktad spänning.
När frekvensen avviker nedåt i frekvens, skiftar de båda kretsarna roller,
och den resulterande likriktade spänningen skiftar till motsatt polaritet.

Vid växelvisa frekvensändringar i MF, över och under vilofrekvensen, blir
resultatet en växelspänning ut från likriktarnas utgångsfilter, som är
LF-signalen.

\subsubsection{Foster-Seeley-diskriminatorn}
\index{Foster-Seeley-diskriminator}
\index{detektor!Foster-Seeley-diskriminator}
\index{FM!Foster-Seeley-diskriminator}

\smallfig{images/cropped_pdfs/bild_2_3-60.pdf}{Foster-Seeley diskriminator}{fig:BildII3-60}

Bild \ssaref{fig:BildII3-60} illustrerar en \emph{Foster-Seeley-diskriminator}.
Denna tidiga demodulator har god linjäritet, om den föregås av en god
amplitudbegränsare, men har tämligen dålig känslighet.

Sista MF-förstärkarsteget avslutas med en transformator vars båda
lindningar ingår i resonanskretsar avstämda till MF.
MF-signalen överförs från primär- till sekundärsidan dels med induktion och dels
med en kondensator till mitten av sekundärlindningen.
Signalen delas på så sätt i två grenar med en fasförskjutning av +90\degree~
respektive \ang{-90}.
Signalerna i grenarna likriktas var för sig och sammanlagras i ett RC-nät.

Om MF-signalen inte devierar är LF-spänningen i grenarna lika.
Men eftersom grenspänningarna har motsatt polaritet tar de ut varandra och
LF-signalen blir noll.
När MF-frekvensen devierar av modulering ökar signalamplituden i den ena
grenen och minskar i den andra.
LF-signalens amplitud blir då proportionell mot frekvensdeviationen.

\subsubsection{Räknardiskriminatorn}
\index{räknardiskriminator}
\index{FM!räknardiskriminator}
\index{detektor!räknardiskriminator}

\mediumfig[0.75]{images/cropped_pdfs/bild_2_3-61.pdf}{Räknardiskriminatorn}{fig:BildII3-61}

Bild \ssaref{fig:BildII3-61} visar räknardiskriminatorn.
En monostabil vippa (eng. \emph{monoflop}) påverkas att slå över av
fyrkantspulserna från de amplitudbegränsade FM-signalerna.

En sådan vippa är en digitalkoppling som, när den matas med en godtyckligt lång
spänningspuls, ändå kommer att leverera en spänningspuls med konstant längd.
För varje positiv halvvåg levererar den monostabila vippan en impuls av konstant längd.
Tidsavstånden mellan pulserna kommer att vara proportionella mot FM-signalens frekvens.
Vid varierande frekvens kommer impulserna med varierande tidsavstånd.
Ett lågpassfilter filtrerar ut lågfrekvensen ur signalen och en pulserande
likspänning kvarstår.
Med denna likspänning laddas kondensatorn upp till ett medelvärde.
Vid en högre frekvens av lika långa pulser blir medelvärdet högre än vid en
lägre pulsfrekvens.

De överlagrade svängningarna på likspänningen utgör LF-signalen.
Utan en monostabil vippa med lika långa pulser hade medelvärdet varit konstant.
Man kan säga att FM-signalen blivit omvandlad till en pulslängdmodulerad signal
(PLM-signal).

\subsubsection{PLL-demodulatorn}
\index{PLL-demodulator}
\index{PLL!demodulator}
\index{FM!PLL-demodulator}
\index{detektor!PLL-demodulator}

\mediumfig[0.75]{images/cropped_pdfs/bild_2_3-62.pdf}{PLL-demodulatorn}{fig:BildII3-62}

Bild \ssaref{fig:BildII3-62} visar PLL-demodulatorn.
Den frekvensmodulerade MF-signalen och en VCO-signal matas in i en
fasjämförare.
VCO-frekvensen följer frekvensändringarna hos FM-signalen.
Avstämningsspänningen för VCO är en likspänning.
Den modulerande LF-spänningen är överlagrad på denna likspänning.

LF-frekvenserna är för låga för att kunna reglera VCO-frekvensen, men
via en kondensator kan de styra LF-förstärkaren.

De båda sista metoderna lämpar sig speciellt för demodulering av FM-signaler.
Det finns ytterligare sätt att demodulera FM-signaler.
Gemensamt för alla är att de fungerar bättre ju lägre mellanfrekvensen är.
Därför utförs de flesta FM-mottagare som dubbel- eller trippelsuprar, med låg
MF.

% Avsnitt 3.6 Oscillatorer
\section{Oscillatorer}
\harecsection{\harec{a}{3.6}{3.6}}
\index{oscillator}
\label{oscillatorer}

\subsection{Alstring av svängningar}
\label{svängningar_alstring}
Ordet \emph{oscillare} (lat.) har betydelsen svänga och den företeelse
eller anordning som skapar en svängning kallas oscillator.
Vid alla slags svängningar sker växelverkan mellan olika energiformer.
Svängningar förekommer i olika former.
Det kan till exempel vara vibrationer i en kropp, molekylrörelser i gaser och
vätskor eller elektriska laddningars rörelser.

\mediumfig{images/cropped_pdfs/bild_2_3-63.pdf}{Svängningar}{fig:BildII3-63}

\subsubsection{Dämpad svängning}
\index{dämpad~svängning}
Radiosändningar med telegrafi genomfördes i början av 1900-talet med dämpade
svängningar.
Det vill säga en svängning vars amplitud minskar tills svängningen upphört.

Svängningen skapades av en elektrisk gnista i ett gnistgap.
Gnistgapet kopplades till en avstämningskrets som gjorde att svängningsenergin
koncentrerades till en mer bestämd radiofrekvens.

De dämpade svängningarna orsakade på grund av den stora bandbredden störningar
som begränsade användbarheten för telegrafi.

\subsubsection{Odämpad svängning}
\index{odämpad~svängning}
\index{kontinuerlig~svängning}
\index{continuous~wave}
Begreppet odämpad svängning infördes för att särskilja en sinussvängning med 
konstant amplitud och frekvens från den dämpade svängningen.

Till skillnad mot en dämpad svängning har en odämpad svängning en begränsad
bandbredd och går att använda för flera modulationsformer. Se bild~\ssaref{fig:BildII3-63}.
På engelska döptes svängningen till \emph{Continuous Wave} och förkortningen CW
används fortfarande av radioamatörer som en beteckning för telegrafi.

När fördelarna med odämpade sinusvågor blev tydliga och när oscillatorer med
radiorör blev tillgängliga runt år 1913 började myndigheter efter några år
införa begränsningar för användningen av gnistsändare.
Begränsningarna utökades genom internationella överenskommelser och under
1930-talet förbjöds användning av sändare med dämpade svängningar.

% \newpage % layout
\subsection{LC-oscillatorer}
\harecsection{\harec{a}{3.6.1}{3.6.1}, \harec{a}{3.6.2}{3.6.2}, \harec{a}{3.6.3}{3.6.3}}
\index{LC-oscillator}
\index{oscillator!LC}
\label{svängningar_LC-oscillator}

\subsubsection{Variabel frekvensoscillator (VFO)}
\index{oscillator!VFO}
\index{oscillator!variabel frekvens}

\smallfig{images/cropped_pdfs/bild_2_3-66.pdf}{Oscillator enligt Meissner}{fig:BildII3-66}
\smallfigpad{images/cropped_pdfs/bild_2_3-67.pdf}{Emitterkopplad förstärkare}{fig:BildII3-67}
\smallfig{images/cropped_pdfs/bild_2_3-68.pdf}{Komplett Meissneroscillator}{fig:BildII3-68}

En oscillator med inställbar frekvens kallas för VFO (variabel
frekvensoscillator).
Förutom frekvensstabilitet fordras också att noggrann inställning och
avläsning av frekvensen ska kunna göras.

En LC-oscillator är urtypen för en oscillator med variabel frekvens.
Meissnerkopplingen är lätt att urskilja och används här för att beskriva
grundprincipen för en oscillator i stort.
Bland annat Colpitts- och Clappkopplingarna har emellertid bättre stabilitet
och inställbarhet i återkopplingsledet.

\subsubsection{Meissnerkoppling}
\index{Meissnerkoppling}
\index{oscillator!Meissnerkoppling}

Bild~\ssaref{fig:BildII3-66} visar en Meissneroscillator, som består av en
LC-resonanskrets med återkopplingsspole och en förstärkare.
Magnetfältet mellan induktansen i resonanskretsen och återkopplingsspolen är
polariserat så att en förändring i utsignalen medverkar till självsvängning.
(Motsatsen är motkoppling.)

Förstärkaren kan till exempel vara en emitterkopplad transistorförstärkare
enligt bild~\ssaref{fig:BildII3-67}.
Kopplingskondensatorerna \(C_k\) är nödvändiga för att förhindra kortslutning
av de likspänningar som bestämmer arbetspunkten för transistorn.
Å andra sidan kan växelspänningssignalerna passera till och från transistorn.

Återkopplingsvägen görs i detta fall så, att resonanskretsen kopplas
parallellt över förstärkaringången som visas i bild~\ssaref{fig:BildII3-68}.
Återkopplingsspolen fungerar som förstärkarens kollektorresistor.

\subsection{Självsvängningsvillkoret}
\index{oscillator!självsvängningsvillkoret}

\smallfigpad[0.2]{images/cropped_pdfs/bild_2_3-69.pdf}{Svängningsvillkoret}{fig:BildII3-69}

Självsvängning i en förstärkare uppstår genom återkoppling, som visas i
bild~\ssaref{fig:BildII3-69}.
Signalspänningen \(\hat{U}_{in}\) över ingången blir förstärkt med
faktorn \(A\).
När som i bild~\ssaref{fig:BildII3-68} förstärkaren är emitterkopplad, blir
utsignalen fasvriden \ang{180} i förhållande till insignalen.
Fasvridningen \(\alpha=180\degree\) betecknas här med minustecken, alltså blir
förstärkningen \(-A\).

På förstärkarens utgång fås en signalspänning \(\hat{U}_{ut}\) med sambandet
%
\[\hat{U}_{ut} = -A \cdot \hat{U}_{in}\]
%
En del av utsignalen återförs (återkopplas) till ingången.
I en Meissneroscillator sker återkopplingen med en induktor, som är
induktivt kopplad till resonanskretsens induktor.

Kvoten \(k\) mellan den återkopplade signalspänningen \(\hat{U}_{ut}\) och
signalspänningen out på förstärkarens utgång kallas återkopplingsfaktor.
Den återkopplade spänningen \(\hat{U}_k\) fasvrids så att den kommer
i fas med med insignalen.
För den återkopplade signalen fås då sambandet
%
\[\hat{U}_k = -k \cdot \hat{U}_{ut}\]
%
Tillräcklig signalspänning från utgången måste återföras till ingången
för att det ska uppstå självsvängning.
Det sker när den återkopplade signalspänningen \(\hat{U}_k\) är minst lika stor
som ingångsspänningen \(\hat{U}_{in}\) och är i rätt fasläge, det vill säga i
detta exempel
%
\[
\hat{U}_k \geq \hat{U}_{in}
\quad \text{eller} \quad
-k \cdot \frac{\hat{U}_{ut}}{A}
\quad \text{eller} \quad
k \geq \frac{1}{A}
\]
%
Självsvängningsvillkoret blir
%
\[
k \geq \frac{1}{A}
\quad \text{eller} \quad
k \cdot A \geq 1
\]
%
Ett \(k \cdot A \approx 3\) är önskvärt för att oscillatorn ska svänga igång
snabbt.

\subsubsection{Hartleykoppling}
\index{Hartleykoppling}
\index{oscillator!Hartleykoppling}
\index{Huth-Kügnkoppling}
\index{oscillator!Huth-Kügnkoppling}
\index{Tuned-Grid-Tuned-Platekoppling}
\index{oscillator!Tuned-Grid-Tuned-Platekoppling}

\smallfig{images/cropped_pdfs/bild_2_3-70.pdf}{Hartleykoppling}{fig:BildII3-70}
\smallfigpad{images/cropped_pdfs/bild_2_3-71.pdf}{TPTG-koppling}{fig:BildII3-71}
\smallfig{images/cropped_pdfs/bild_2_3-72.pdf}{Colpittskoppling}{fig:BildII3-72}

Bild~\ssaref{fig:BildII3-70} visar en Harleykoppling.
%
Återkopplingen sker galvaniskt över ett uttag på induktorn i oscillatorns
LC-krets.

Bild~\ssaref{fig:BildII3-71} visar en Huth-Kühn- eller TGTP-kopp\-ling
(tuned grid -- tuned plate)
%
Kopplingen är en förstärkare med LC-kretsar både på in- och utgång.
Båda kretsarna är avstämda till samma frekvens.
Återkopplingen sker över de inre kapacitanserna mellan elektronrörets elektroder
respektive mellan transistorns materialskikt.
Denna koppling är av flera skäl inte särskilt vanlig.

\subsubsection{Colpittskoppling}
\index{Colpittskoppling}
\index{oscillator!Colpittskoppling}

Bild~\ssaref{fig:BildII3-72} visar en Colpittskoppling.
%
Återkopplingen sker över en kapacitiv spänningsdelare, som ingår som
en del av oscillatorns LC-krets.

\smallfigpad{images/cropped_pdfs/bild_2_3-73a.pdf}{Clappkoppling}{fig:BildII3-73a}
\subsubsection{Clappkoppling}
\index{Clappkoppling}
\index{oscillator!Clappkoppling}

Denna koppling är en variant av Colpittskopplingen.
Vridkondensatorn för frekvensinställningen är seriekopplad med spänningsdelarens
kondensatorer.
Clapposcillatorns frekvensstabilitet är god.

Vi utvecklar denna beskrivning vidare med bild~\ssaref{fig:BildII3-73a}.
Vridkondensatorn samt en fast och en trimningsbar kondensator är kopplade
parallellt med varandra.
Alla tre kondensatorerna är i sin tur seriekopplade med den kapacitiva
spänningsdelaren \(C_3/C_4\).
Förstärkarens ingång är kopplad till den övre anslutningen av \(C_3\).
Utgången från oscillatorns förstärkare återkopplas över dämpresistorn
\(R_{ct}\) till mitten av spänningsdelaren \(C_3/C_4\)
(återkopplingskretsen).

\newpage
Bild~\ssaref{fig:BildII3-73b} visar förstärkaren i en Clappkoppling.
Förstärkarens arbetspunkt bestäms av spänningsdelaren \(R_1/R_2\).
Ingen kopplingskondensator behövs eftersom det enbart finns kondensatorer
mellan förstärkaringång och jord.
Kondensatorn \(C_6\) avkopplar kollektorn på transistor \(T_1\) HF-mässigt till
jord.
Förstärkaren är alltså kollektorkopplad.

Kondensatorn \(C_7\) kopplar oscillatorns utsignal till buffertsteget.
För frekvensstabilitetens skull stabiliseras spänningen \qty{8}{\volt} med en
LC-krets som avkopplas HF-mässigt med en kondensator.

%\balance
\subsection{Frekvensinställning och bandspridning}
\index{oscillator!frekvensinställning}
\index{oscillator!bandspridning}

%\smallfig{images/cropped_pdfs/bild_2_3-73a.pdf}{Clappkoppling}{fig:BildII3-73a}
\smallfig{images/cropped_pdfs/bild_2_3-73b.pdf}{Förstärkare i Clappkoppling}{fig:BildII3-73b}
\smallfigpad{images/cropped_pdfs/bild_2_3-74.pdf}{Bandspridning}{fig:BildII3-74}

Bild~\ssaref{fig:BildII3-74} illustrerar stegvis hur man åstadkommer
bandspridning.
Att ställa in frekvensen i en LC-oscillator gjordes förr oftast med en
vridkondensator.
I moderna mottagare och sändare används i stället en kapacitansdiod 
(eng. \emph{varicap}), som styrs med en likspänning.

Med en resonanskrets med endast en induktor och en vridkondensator, skulle
alla amatörradiobanden endast vara smala områden utspridda på en mekanisk
skala, det vill säga över vridkondensatorns hela kapacitansområde, varvid
kapacitansen kan varieras med förhållandet 1:5 eller 1:10, till exempel
\SIrange{10}{50}{\pico\farad} eller \SIrange{10}{100}{\pico\farad}.

För att i stället få vart och ett av amatörradiobanden utspridda över större
delen av skalan kan man ordna med bandomkoppling och så kallad bandspridning.
Man parallellkopplar då en relativt stor fast kapacitans med vridkondensatorns
relativt lilla kapacitans.
Den totala kapacitansvariationen i LC-kretsen blir då liten, trots att
kondensatorns hela kapacitansområde utnyttjas.
Resultatet blir en frekvensskala med större upplösning, det vill säga bättre
avläsningsnoggrannhet.

Bandspridning kan också ordnas med två seriekopplade kondensatorer,
varav den större görs variabel.
Typiskt värde på vridkondensatorn i en kortvågsutrustning är då
\SIrange{100}{500}{\pico\farad} och den fasta kondensatorn mycket mindre än så.

% Avsnitt 3.7 Kristalloscillatorer
% \newpage
% \nobalance
\section{Kristalloscillatorer}
\harecsection{\harec{a}{3.6.4}{3.6.4}}
\index{kristalloscillator}
\index{oscillator!kristalloscillator}
\label{kristalloscillator}

\subsection{Kvartskristaller i oscillator\-kopplingar}
\index{XO}
\index{oscillator!XO}
\index{VFO}
\index{oscillator!VFO}

En LC-oscillators frekvensstabilitet begränsas av de ingående
komponenternas egenskaper.
När mycket bättre stabilitet än så krävs, speciellt inom stora
temperaturområden, är kvartskristallen en svängningskrets med bättre data.
Kvartskristallens höga Q-värde ger också en renare signal.

\tallfig{images/cropped_pdfs/bild_2_3-75.pdf}{Colpittsoscillator med kristall i parallellresonansfallet}{fig:BildII3-75}

I en \emph{kristalloscillator} (eng. \emph{Crystal Oscillator (XO)}) är en
kvartskristall det frekvensbestämmande elementet i stället för en LC-krets.
I övrigt kan samma kopplingsprinciper som för en LC-VFO användas.

Kristallen kan utföras så att den svänger antingen som en serie- eller
parallellresonanskrets.
Märk att en kristall svänger på något olika frekvens beroende på om den fås
att fungera som serie- eller parallellkrets.
Den högre frekvensen är den som vanligen används.

Bild~\ssaref{fig:BildII3-75} visar en Colpittoscillator med en kristall i
parallellresonansfallet.
I parallellresonansalternativet kopplas kristallen parallellt över
oscillatorns återkopplingsled.
Den minsta dämpningen av den återkopplade signalen fås när signalens frekvens
är samma som kristallens resonansfrekvens.
Kristallens reaktans är då som högst.

Parallellt över kristallens inre induktans ligger dess inre
seriekopplade kapacitanser \(C\) och \(C_H\).
Yttre kapacitanser (en trimbar och två fasta kondensatorer i serie) är kopplade
parallellt över den inre anslutningskapacitansen \(C_H\).

Om den trimbara kapacitansen ändras, så påverkas kristallens resonansfrekvens.
Man säger då att man ''drar'' kristallen inom ett litet frekvensområde.
Kristallens och oscillatorns egenskaper avgör hur stort området kan vara.
Om kristaller dras för mycket, så kan resonansfrekvensen bli ostabil.
Den relativa frekvensändringen uppgår till högst \(10^{-4} = 0,01\%\)
enligt följande formel:
%
\[
\text{relativ frekvensändring} =
\frac{\text{absolut ändring}}{\text{resonansfrekvens}}
\]

\subsection{Övertonskristaller}
\index{övertonskristall}
\index{kristall!övertonskristall}

\tallfig{images/cropped_pdfs/bild_2_3-76.pdf}{Colpittsoscillator med kristall i serieresonansfallet}{fig:BildII3-76}
\mediumbotfig[0.8]{images/cropped_pdfs/bild_2_3-77.pdf}{Superheterodyn-VFO}{fig:BildII3-77}

Bild~\ssaref{fig:BildII3-76} visar en Colpittsoscillator med kristall i
serieresonansfallet.
I serieresonansalternativet kopplas kristallen in i serie med
oscillatorns återkopplingsled.
Den minsta dämpningen av den återkopplade utgångssignalen fås när signalens
frekvens är samma som kristallens resonanfrekvens.
Kristallens reaktans är då som lägst.
Så kallade övertonskristaller används för oscillatorfrekvenser över cirka
\qty{20}{\mega\hertz}.

Övertonskristallernas dimensioner är lika grundtonskristallernas, men
snittas ut annorlunda och slipas för att svänga på önskad udda överton.
En övertonskristall har övertonens frekvens instämplad i höljet och kristallen
förutsätts arbeta i oscillatorkopplingar som seriekrets.
Genom att låta kristaller svänga på sin överton undviker man en svår
tillverkningsprocedur, nämligen att slipa mycket tunna kristallskivor.

En övertonsoscillator måste alltid innehålla en resonanskrets som är
avstämd till den överton som anges på kristallen.

\paragraph{Modellförsök}
En instrumentsträng sätts i svängning på sin grundton genom en knäppning mitt
på strängen.
En knäppning på en punkt bort från mitten får strängen att svänga på en överton
i stället.

\subsection{Superheterodyn-VFO}
\label{superVFO}
\index{superheterodyn!VFO}
\index{oscillator!syperheterodyn-VFO}

% \mediumbotfig[0.8]{images/cropped_pdfs/bild_2_3-77.pdf}{Superheterodyn-VFO}{fig:BildII3-77}

Bild~\ssaref{fig:BildII3-77} visar en superheterodyn-VFO.
En enkel LC-VFO är inte tillräckligt frekvensstabil i ett högt frekvensläge,
till exempel \SIrange{144}{146}{\mega\hertz}.
Man kan då använda en speciell koppling, som är en kombination av LC-VFO och
XO, kallad super-VFO.

I en super-VFO blandas en låg variabel frekvens från en VFO med en hög
frekvens från en XO.
Ordet super kommer från superheterodyn = överlagring, blandning.
En VFO arbetar stabilare på låg frekvens medan en XO fortfarande arbetar
stabilt även på högre frekvenser, dock inte så högt som vi behöver här.
I vårt exempel arbetar därför VFO i området \SIrange{8}{10}{\mega\hertz} och XO
på \qty{17}{\mega\hertz}.
VFO-signalen blandas med en fast signalfrekvens, som är XO-signalen
\qty{17}{\mega\hertz} multiplicerat med 8, det vill säga \qty{136}{\mega\hertz}.

Ett bandpassfilter filtrerar fram den önskade blandningsprodukten, som
ligger i frekvensområdet \SIrange{144}{146}{\mega\hertz}.
Resultatet blir en hög frekvens, som är både variabel och stabil.

\paragraph{Fördelar}
Frekvensstabiliteten hos en super-VFO är mycket bättre än hos en enkel VFO,
som arbetar direkt i VHF-området.
En super-VFO är dessutom mycket brusfattigare än en PLL-VFO, vilken
beskrivs här nedan.

\paragraph{Nackdelar}
Vid frekvensblandning uppstår oönskade blandningsprodukter, vilka visserligen
dämpas av bandpassfilter, men som det är omöjligt att undertryckta helt.
Bland annat alstras en svag spegelfrekvens, som vandrar från 128 till
\qty{126}{\mega\hertz}, samtidigt som den önskade blandningsprodukten vandrar
från 144 till \qty{146}{\mega\hertz}.
Risken för att spegelfrekvensen förstärks och sänds ut måste elimineras, vilket
kan göras med effektiva bandpassfilter.
Se vidare i avsnitt~\ssaref{blandare} om frekvensblandning.

% \newpage % layout

\subsection{Oscillatorer med faslåsning (PLL)}
\harecsection{\harec{a}{3.7}{3.7}}
\index{PLL}
\index{Phase Locked Loop (PLL)}
\label{PLL}

En kristalloscillator (XO) arbetar med god frekvensstabilitet.
Dess frekvens är fast och bestäms av styrkristallen.

En LC-oscillator arbetar däremot inom ett frekvensområde (VFO), som bestäms av
en LC-krets.
Dennas frekvens är emellertid mindre stabil än den med styrkristall.

I en \emph{faslåst loop} (eng. \emph{Phase Locked Loop (PLL)}) kan god
frekvenstabilitet och stort frekvensområde förenas.
En PLL är en sluten krets för elektrisk styrning av en oscillator, så att dess
frekvens är både stabil och variabel.

\subsubsection{Spänningsstyrd oscillator (VCO)}
\harecsection{\harec{a}{3.6.5}{3.6.5}}
\index{spänningsstyrd oscillator}
\index{oscillator!spänningsstyrd}
\index{VCO}
\index{oscillator!VCO}
\index{VFO}
\index{oscillator!VFO}
\index{kapacitansdiod}
\index{varicap}

I bild~\ssaref{fig:BildII3-78} jämförs en VFO och en VCO.
En VFO, vars frekvens kan styras med en likspänning, kallas
\emph{spänningstyrd oscillator} (eng. \emph{Voltage Controlled Oscillator
  (VCO)}).
I resonanskretsen i en VCO fyller en kapacitansdiod (eng. \emph{varicap, variable
capacitor}) samma uppgift som den mekaniskt variabla kondensatorn i en VFO.

\smallfig{images/cropped_pdfs/bild_2_3-78.pdf}{VFO och VCO jämförs}{fig:BildII3-78}
\newpage
\smallfig{images/cropped_pdfs/bild_2_3-79.pdf}{Kapacitansdiod -- Varicap}{fig:BildII3-79}
\smallfigpad{images/cropped_pdfs/bild_2_3-80a.pdf}{Analogi Människa-PLL}{fig:BildII3-80a}

Bild~\ssaref{fig:BildII3-79} visar en kapacitansdiod.
När en motriktad spänning läggs på dioden bildas ett spärrskikt i dioden,
så att zonerna med fria laddningsbärare isoleras från varandra likt
kondensatorplattor.
Spärrskiktets tjocklek (ca $1/1000$~\unit{\milli\metre}) beror av spänningen
över dioden.
Vid hög spänning är spärrskiktet tjockt, vilket motsvarar
''stort plattavstånd'' och liten kapacitans.
Vid låg spänning är skiktet tunt, vilket motsvarar ''litet plattavstånd'' och
stor kapacitans.

Med en kapacitansdiod i resonanskretsen i stället för en mekaniskt
variabel kondensator, behövs ytterligare två komponenter.
Drosseln \(D_r\) hindrar högfrekvenssignalen att överlagras på styrkretsens
likspänning, vilket annars skulle skulle försämra resonanskretsens godhetstal
(förlorad HF-energi innebär dämpning).
Omvänt hindrar kondensatorn \(C\) att dioden och spärrspänningen kortsluts
genom induktorn.
Oscillatorfrekvensen ställs in med den variabla likspänningen \(U\).
Av en VFO har det blivit en VCO.

\subsubsection{Oscillator med PLL-styrning}
\harecsection{\harec{a}{3.7.1}{3.7.1}}

Bild~\ssaref{fig:BildII3-80a} visar en manuell frekvensstyrning.
Människan jämför och reglerar förlopp utifrån givna fakta.
Det kan liknas med PLL-kretsens sätt att jämföra det inbördes fasläget mellan
signalen från en VCO (är--värdet) och signalen från en XO (bör--värdet).

Som resultat av jämförelsen justeras styrspänningen så att är- och
börfrekvenserna hålls lika.
En sådan reglerkrets består av digitala komponenter.

\mediumfig{images/cropped_pdfs/bild_2_3-80b.pdf}{Oscillator med PLL-styrning}{fig:BildII3-80b}

Bild~\ssaref{fig:BildII3-80b} illustrerar en oscillator med PLL-styr\-ning.
Fasjämföraren levererar en cykliskt justerad styrspänning till
kapacitansdioden i VCO.
Eftersom denna spänning ändras språngvis, avrundas förloppet så att
frekvensändringarna blir mjuka.
Avrundningen sker med ett RC-filter där kondensatorn antar ett medelvärde av
den pulserande utgångsspänningen från jämföraren.
Om VCO-frekvensen är för låg, levererar jämföraren en positiv spänning.
Styrspänningen på kapacitansdioden stiger då med en hastighet som bestäms av
filtrets tidskonstant.

Kapacitansen i kapacitansdioden minskar med ökande spänning, eftersom
spärrskiktet blir tjockare och frekvensen på VCO stiger.

När signalen från VCO åter är lik referenssignalen från XO, till
fasläge och frekvens, ökar utgångsresistansen i fasjämföraren.
Lågpassfiltrets kondensator behåller då sin laddning och styrspänningen till
VCO ändras inte.
Skulle frekvensen på VCO vara för hög, blir jämförarens utgång lågohmig och
filtrets kondensator urladdas med den hastighet som bestäms av tidskonstanten.
Den sjunkande styrspänningen medför att kapacitansdiodens spärrskikt blir
tunnare, kapacitansen tilltar och VCO-frekvensen sjunker tills en ny fas- och
frekvenslikhet uppnåtts.

\newpage
\subsubsection{PLL-oscillator i kombination med frekvensblandning}

\mediumfig{images/cropped_pdfs/bild_2_3-81.pdf}{PLL-oscillator kombinerad med frekvensblandning}{fig:BildII3-81}

Bild~\ssaref{fig:BildII3-81} visar en PLL-oscillator kombinerad med
frekvensblandning.
Signalen \(f_1\) från en VCO alstrar en sändningsfrekvens i bandet
\SIrange{144}{146}{\mega\hertz}.
Denna blandas med signalen \(f_2\) (\qty{136}{\mega\hertz}), som är en
multiplicerad XO-frekvens.
Blandningsprodukten \(f_1 - f_2\) är en signal i området
\SIrange{8}{10}{\mega\hertz} som filtreras fram och påförs en fasjämförare.
Utsignalen från en VFO, som är variabel inom samma frekvensområde
\SIrange{8}{10}{\mega\hertz}, påförs också fasjämföraren.

Utsignalen från jämföraren är en likspänning som beror av frekvensskillnaden
mellan blandningsprodukt och VFO-signal.
Jämförarens utsignal ändras uppåt eller nedåt, beroende på frekvensfelets
riktning.

VCO-frekvensen bestäms av en likspänningsnivå som styrs av jämförarens
utsignal.
Vid varje frekvensändring i VCO, kommer systemet att sträva mot
frekvensskillnaden noll i fasjämföraren, vilket gör att sändningsfrekvensen
hålls vid rätt värde.

% \newpage % layout
\paragraph{Fördelar med en PLL-oscillator}
Den har samma frekvensstabilitet som en VFO eftersom denna även här arbetar på
en låg frekvens.
Till skillnad mot en super-VFO finns inga sidofrekvenser i PLL-oscillatorn,
eftersom VCO alstrar nyttofrekvensen direkt.

\paragraph{Nackdelar med en PLL-oscillator}
Den har högre brusnivå än en super-VFO.
Frekvensstabiliteten är sämre än den för en PLL-oscillator med XO och
programmerbar frekvensdelare.

\subsubsection{PLL med programmerbar frekvensdelare}
\harecsection{\harec{a}{3.7.2}{3.7.2}}
\index{PLL!programmerbar}

\mediumtopfig{images/cropped_pdfs/bild_2_3-82.pdf}{PLL med frekvensdelare}{fig:BildII3-82}

Bild~\ssaref{fig:BildII3-82} visar en PLL med frekvensdelare.
Med PLL blir frekvensen på utsignalen från en VCO låst till referensfrekvensen
från en XO.
I princip fås en VCO med samma frekvensstabilitet som en XO, men som också är
lika svår att ändra frekvensen på.
Med en frekvensdelare i fasregleringsslingan (PLL) kan emellertid utfrekvensen
ändras, medan XO fortfarande avger samma referensfrekvens.
En frekvensdelare är en digital krets som räknar svängningar eller pulser upp
till ett valt tal för att återställas till 1 och börja om igen.
Vid varje återställning avges en utpuls.
Vid en delning med två avges en utpuls för varannan inpuls.
Vid delning med 15 avges en utpuls för var 15:e inpuls och så vidare.

Genom att välja delningstal i PLL kan arbetsfrekvensen i VCO ställas
in stegvis, där varje steg är så stort som en referensfrekvens.
Signalfrekvensen från VCO delas med det valda delningstalet och resultatet
jämförs med referensfrekvensen från XO.
Varje avvikelse referensfrekvensen kommer att medföra justering av
VCO-frekvensen.

Om man till exempel vill täcka 2-metersbandet i steg om \qty{25}{\kilo\hertz},
väljer man referensfrekvensen \qty{25}{\kilo\hertz}.
I delaren delas sändarens utfrekvens med ett tal 5760, 5761, 5762 och så vidare
upp till 5840.
Om till exempel delningstalet 5820 valts, så kommer jämförarens styrspänning att
styra VCO-frekvensen till \qty{145500}{\kilo\hertz}.
Delarens utfrekvens blir då \(145500/5820 = \qty{25}{\kilo\hertz}\), vilket
motsvarar referensfrekvensen.
I detta exempel styrs alltså sändarens utfrekvens så att den alltid blir i steg
om \qty{25}{\kilo\hertz}.

\subsubsection{För- och nackdelar med PLL-oscillatorn}

PLL-oscillatorn har nästan samma frekvensstabilitet som en
kristalloscillator och frekvensen är inställbar i steg.
Till skillnad mot en VFO med mekaniskt inställbar frekvens, så är den PLL-styrda
VCO-oscillatorns frekvens elektroniskt inställbar.
Detta underlättar utformning och placering av reglage etc.\ för
frekvensinställning, frekvensminne och automatisk frekvensavsökning.

Först när den PLL-styrda oscillatorn kom till användning i handapparater och
mobila apparater blev det möjligt med frekvenstäckning över ett helt band med
bibehållet krav på små dimensioner.
Som jämförelse skulle en inbyggnad av säg 80 till 800 stycken kanalkristaller
i en traditionell kristallstyrd apparat vara en mycket platskrävande, dyrbar
och opraktisk lösning.

Men PLL-oscillatorn brusar förhållandevis starkt jämfört med en VCO och
speciellt jämfört med en XO.
VCO-resonanskretsen har nämligen ett relativt lågt godhetstal eftersom en
kapacitansdiod belastar kretsen mer än en mekaniskt variabel kondensator.

Med det lägre godhetstalet blir resonanskretsen ett mindre bra
filter för dämpning av oscillatorbruset.
Kapacitansdioden tillför dessutom ett elektronbrus.
Därtill kommer det så kallade fasbruset från frekvensdelaren och PLL.

Med resonanskretsens låga godhetstal är frekvensstabiliteten i en VCO inte så
bra som den i en kristalloscillator.
Trots det är långtidsstabiliteten god i en VCO, när den ingår i en PLL,
eftersom frekvensen hålls ständigt efterjusterad.
PLL kan däremot inte åstadkomma en lika bra korttidsstabilitet.
Ett fasjämförelseförlopp omfattar ju redan tiden för en period av
referensfrekvensen, och det kommer att förflyta en multipel av denna kortaste
tid innan styrspänningen kan återställa VCO-frekvensen igen.
Detta beror på att kondensatorn i regleringsslingans lågpassfilter först måste
laddas upp under ett antal perioder innan reglering sker.

Dessa kortvariga frekvensavvikelser är en typ av frekvensmodulation
som leder till fasbrus från PLL-oscillatorn.
Det är dock endast i extrema fall som fasbruset verkar störande eftersom det
i moderna apparater reduceras till en acceptabel nivå genom noggrann
skärmning och filtrering.

\subsection{Faktorer som påverkar frekvensstabilitet}
\index{frekvensstabilitet}
\index{oscillator!stabilitet}

Sändarens frekvens ska hållas så stabil som möjligt.
En ostabil sändare är inte godtagbar och skapar svårigheter inte bara för de
radiostationer som deltar i förbindelsen utan även för radiotrafiken
på närliggande frekvenser.

En frekvensstabil oscillator ska ha följande egenskaper:

\subsubsection{Stabil mekanisk uppbyggnad}

Skakningar från underlaget till exempel vid mobilt bruk, vibrationer från en
transformatorkärna etc. kan försämra oscillatorns frekvensstabilitet.

Frekvensbestämmande komponenter såsom fasta och variabla kondensatorer, spolar
och liknande ska vara stabilt monterade, trimkärnorna i spolarna fixerade och
så vidare.

Förbindningarna får inte tillåtas att böja sig eller vibrera.
Apparatstommen måste vara tillräckligt styv för att inte ändra form och
därigenom medföra frekvensändringar vid hantering och så vidare.

\subsubsection{God elektrisk uppbyggnad och högt Q-värde i resonanskretsarna}

Alla elektriska förbindningar måste vara så korta som möjligt och löd- och
kopplingsställen fullgoda.
Induktorer och kondensatorer i resonanskretsarna måste vara förlustfattiga
och högvärdiga i övrigt så att signalen blir så ren som möjligt från oönskade
sidofrekvenser.

Återkopplingen i oscillatorn ska vara så fast (kraftig) att självsvängningen
är stabil.
Men för att få renast möjliga signal får kopplingen inte vara så fast,
att resonanskretsarna blir alltför belastade och deras godhetstal för lågt.

\subsubsection{Avskärmande kapslingar}

Resonanskretsar ska skärmas från yttre kapacitanstillskott, till exempel från
användarens hand.
Det görs med skiljeväggar och komponentkapslingar av metall.
Skärmningarna förhindrar också oönskad koppling mellan oscillatorn och
efterföljande förstärkare genom elektriska och magnetiska fält.

\newpage
\subsubsection{Stabila drivspänningar}

Ostabila drivspänningar medför frekvensändringar.
I en oscillator med transistorförstärkare beror ostabiliteten på förändringar
mellan skikten i en transistors diodsträcka.
Skikten fungerar nämligen som ''kondensatorplattor'' och spärrskiktet där
emellan som dielektrikum.
Tjockleken av spärrskiktet och därmed ''plattavståndet'' står i förhållande
till den spänning som läggs över transistorn.
Den spänningsberoende kapacitansen i transistorn är ansluten till
resonanskretsen via kopplingskondensatorn.

Eftersom kapacitansen i transistorn är en del av resonanskretsen, påverkar
den resonansfrekvensen.
Denna egenskap kan vara till besvär, men kan även användas för att på ett
enkelt sätt ändra oscillatorns arbetsfrekvens.
Se kapacitansdiod och PLL-oscillatorn.

\subsubsection{Buffertsteg}
\label{buffertsteg}

En oscillator i en radiosändare kan bestå av ett enda förstärkarsteg
som alstrar högfrekventa elektriska svängningar.
Vanligen tas endast små effekter ut från en så enkel sändare, normalt mindre än
en watt.
Utan särskilda åtgärder, som till exempel att använda en styrkristall, är
nämligen frekvensen inte särskilt stabil och olämplig för kommunikationsändamål.

Särskilt varierande belastning över oscillatorns utgång medför frekvensändring.
Oscillatorn bör därför ges en så låg och stabil belastning som möjligt.
Ett buffertsteg med hög ingångsimpedans kopplas därför in efter oscillatorn.
Buffertsteget ska också kunna lämna tillräcklig driveffekt till efterföljande 
förstärkare och bör därför ha låg utgångs impedans.
Det måste dessutom arbeta linjärt (se klass A-drift, bild~\ssaref{fig:BildII3-44})
för att inte alstra övertoner och därmed förvränga oscillatorsignalen.
Bild~\ssaref{fig:BildII3-42} visar ett buffertsteg i kollektorkoppling, vilken har
dessa egenskaper.

\subsubsection{Temperaturkompensation och termostater}

Det alstras alltid förlustvärme i elektriska apparater och även i en oscillator.
Vid uppvärmningen utvidgas spolar och kondensatorer i resonanskretsarna,
vilket leder till frekvensändringar.
Även spärrskiktskapacitansen i transistorerna är temperaturberoende.
Det totala temperaturberoendet kan kompenseras genom ett antal åtgärder.

\newpage
Oscillatorn bör monteras så långt bort som möjligt från övriga
värmealstrande komponenter.
Den avskärmande kapslingen omkring oscillatorn ska vara så tjockväggig och
värmeisolerande som möjligt.
Inbyggnad i en termostatreglerad kapsling är ett ännu bättre alternativ.

Komponenterna bör ha uppnått drifttemperatur innan användningen.
Oscillatorn bör därför värmas upp under åtminstone 15~minuter.

\subsection{Frekvensstabilitet och oscillatorbrus}
\harecsection{\harec{a}{3.6.6}{3.6.6}}

Frekvensstabiliteten i kristalloscillatorer är cirka 100 gånger bättre än
den är i LC-oscillatorer.
Likaså är utgångssignalen från kristalloscillatorer renare från fasbrus (jitter).
Varje oscillator avger nämligen även oönskade signaler med frekvenser som
ligger omkring utgångssignalens nominella frekvens.

Oscillatorn är ju en förstärkare, vars utgångsspänning delvis
återkopplas till ingången i medfas.
Detta innebär att utgångssignalen förstärks lavinartat till ett maximum,
omväxlande med att den dämpas lavinartat till ett minimum.
Utan yttre påverkan befinner sig alltså förstärkaren i ett
självsvängningstillstånd mellan två yttervärden.
I återkopplingsvägen placeras ett filter som frekvensbestämmande
element, till exempel en LC-krets eller en kvartskristall.

Återkopplingen blir starkast på filtrets resonansfrekvens, vilket
medför att oscillatorn svänger bäst där.
Eftersom filtret oundvikligen har en viss bandbredd, kommer även ett
spektrum av andra frekvenser tätt omkring resonansfrekvensen att släppas
igenom.
De oönskade frekvenserna omkring den nominella kallas för brus.

I moderna konstruktioner används oftast PLL-oscillatorer.
På grund av sin funktion pendlar deras frekvens alltid något.
Hur mycket beror bland annat på loopfiltret.
Alltså är frekvensen egentligen ett mycket litet band av flera frekvenser varav
en framträder mest.

\newpage
\paragraph{Försök}

Volymkontrollen i en lågfrekvensförstärkare utan insignal vrids till maximum.
Det kommer att höras ett brus i högtalaren, som huvudsakligen kommer från
ingångsstegets transistorer.
När en mikrofon ansluts måste volymkontrollen vridas ner och då hörs bruset
mindre.
Men bruset finns ändå där på en lägre nivå och överlagras på insignalen från
mikrofonen.

Även i en högfrekvensoscillator överlagras bruset på insignalen.
Men ju högre godhetstalet är i resonanskretsen, till exempel en kristall, desto
smalare är filtrets bandbredd, desto kraftigare blir brusundertryckningen och
desto mer framhävs den önskade signalen.
Tack vare det större godhetstalet i resonanskretsen, och därmed den mindre
bandbredden, brusar alltså en kristalloscillator mindre än en LC-oscillator.

En nackdel med kristalloscillatorn är att dess frekvens inte kan ändras inom 
ett större område.
Önskas flera valbara frekvenser från en kristalloscillator måste flera
kristaller användas tillsammans med någon slags omkopplingsanordning
(kanalväljare).

Komponentmängden i en kristalloscillator är mindre än i en VFO, men i
apparater för flera frekvenser uppvägs denna fördel av merkostnaden
för flera kristaller och kanalväljaren.

Kristalloscillatorn har många användningsområden där en
frekvensstabil och brusfattig signal önskas och där platsbrist,
skakningar med mera utesluter användning av en LC-VFO.

% Avsnitt 3.8 Frekvensblandare
\newpage

\section{Frekvensblandare}
\label{blandare}
\index{blandare}
\index{frekvensblandare}
\index{mixer}

\subsection{Grundprinciper}

En anordning som blandar signaler för att skapa andra kallas som namnet säger
för \emph{blandare} (eng. \emph{mixer}).
Blandare används både i mottagare och sändare och funktionsprinciperna är lika
i båda fallen.
Vad som skiljer i stort är hur de används.

Det finns många blandarkopplingar varav de vanligaste beskrivs här.
Enkla typer med vissa nackdelar ställs mot sådana som är mer
komplicerade, men har fördelar.

\mediumbotfig[0.85]{images/cropped_pdfs/bild_2_3-83.pdf}{Principer för frekvensblandning}{fig:BildII3-83}

Bild \ssaref{fig:BildII3-83} visar principerna för frekvensblandning.
När en linjär förstärkare matas med två signaler så sammanlagras de.
Den resulterande signalen vid varje tidpunkt är den förstärkta summan av de
inmatade signalerna.

När en olinjär förstärkare matas med två signaler så blandas de med varandra.
Förutom ingångssignalerna uppträder genom blandningen ytterligare signaler på
förstärkarutgången, så kallade blandningsprodukter.

Två av blandningsprodukterna är särskilt intressanta, det är summan och
skillnaden av ingångssignalernas frekvenser.
De oönskade övriga blandningsprodukterna filtreras bort med en avstämd krets
eller ett bandpassfilter.

\newpage
\subsection{Obalanserad blandare}
\index{obalanserad blandare}
\index{blandare!obalanserad}

\mediumplustopfig{images/cropped_pdfs/bild_2_3-84a.pdf}{Obalanserad blandare}{fig:BildII3-84a}

Bild \ssaref{fig:BildII3-84a} visar en obalanserad blandare.
Vi kan övertyga oss om att de fyra blandningsprodukterna verkligen uppstår.
Först undersöker vi den enklaste blandaren, som är ett olinjärt element i form
av en diod.

Det finns ingen förstärkare i kopplingen.
Signalspänningarna adderas genom att de två transformatorernas
sekundärlindningar är seriekopplade.
Dioden ''förvränger'' kraftigt summaspänningens kurvform.
Beroende av hur dioden är polariserad (vänd i kopplingen) blir den negativa
eller den positiva halvvågen bortskuren.

\newpage
Signalen på blandarens utgång, alltså efter dioden, innehåller
bland annat frekvenserna \(f_1, f_2, f_2+f_1, f_2-f_1\).
Den lägsta frekvensen \(f_1\) kan lättast påvisas genom att ansluta ett
lågpassfilter till blandarens utgång.

Resultatet kan studeras med ett oscilloskop.
Liksom på bilden ser man då att kondensatorn laddas upp till den positiva
halvvågens toppvärde och med gott närmevärde följer kurvformen på \(f_1\).

\newpage
\mediumtopfig{images/cropped_pdfs/bild_2_3-84b.pdf}{Obalanserad blandare med resonator}{fig:BildII3-84b}

Bild \ssaref{fig:BildII3-84b} visar en obalanserad blandare med en resonator.
En resonanskrets med lämplig bandbredd och som är avstämd till
resonansfrekvensen \(f_2\) ansluts nu till blandarens utgång.
En signal med frekvensen \(f_2\) kan då urskiljas och studeras i oscilloskopet.
Resonanskretsen tillförs energi under de positiva halvvågorna.
Energin i resonanskretsen kompletterar med den negativa halvvågen, varvid en
del av kretsens energi förbrukas.
Därför har de positiva och negativa halvvågorna inte samma amplitud (toppvärde).

\newpage
Det syns i oscilloskopet hur amplituden ''svävar''.
Av detta dras slutsatsen att signalen består av fler frekvenser än \(f_2\).
Signalen är sammansatt av \(f_2, f_2+f_1\) och \(f_2-f_1\).
Signalen \(f_1\) ligger utanför resonanskretsens selektiva område och blir
därför bortfiltrerad (undertryckt).
Blandningsprodukterna \(f_2 + f_1\) och \(f_2 - f_1\) har båda en mindre
amplitud än \(f_2\).

Att det finns olika grundtoner och blandningsprodukter kan bevisas med en ännu
smalare resonanskrets med variabel frekvensavstämning, se bildens nedre del.
Vi har hittills utgått från en obalanserad blandare.
Andra blandartyper som den balanserade blandaren och den dubbelbalanserade
blandaren producerar färre blandningsprodukter.

\newpage

\subsubsection{Balanserad blandare}
\index{balanserad blandare}
\index{blandare!balanserad}

\mediumtopfig{images/cropped_pdfs/bild_2_3-85.pdf}{Balanserad blandare}{fig:BildII3-85}

Bild \ssaref{fig:BildII3-85} visar en balanserad blandare.
Den balanserade blandaren har till skillnad från den obalanserade blandaren två
dioder och HF-transformatorernas ena lindning har mittuttag.
Ingången \(E_1\) ligger på den ena transformatorns primärlindning.
Ingången \(E_2\) 1igger över de båda mittuttagen.
Utgången ligger på den andra transformatorns sekundärlindning.

Ingången \(E_1\) matas med en signal med en låg frekvens \(f\).
Eftersom en av de båda dioderna alltid spärrar, flyter ingen ström.
De streckade pilarna visar endast i vilken riktning strömmen kunde flyta, om de
spärrande dioderna vore öppna.
Men så länge som ingen signal ligger på ingång \(E_2\), uppträder ingen
signal på utgången.

Signalen på \(E_1\) avlägsnas och i stället matas ingången med en hög
frekvens \(F\).
Under den positiva halvvågen är de båda dioderna öppna och genom båda flyter
lika stor ström.
De båda transformatorernas lindningshalvor genomflyts av lika ström i motsatt
riktning och då upphäver magnetfälten i lindningshalvorna varandra och ingen
signal uppträder på utgången.

När signaler läggs båda ingångarna händer följande:

Dioderna öppnar och stänger i takt med signalen på ingång \(E_2\), med
frekvensen \(F\).
Den mycket svagare signalen på ingång \(E_1\), med frekvensen \(f\), kan
alltefter polaritet passera diod \(D_1\) eller \(D_2\).
På återvägen överlagras signalen från \(E_1\) på signalen från \(E_2\).
Strömmarna i lindningshalvorna är olika stora.
Då uppträder en signal på utgången.
Efter blandaren följer ett filter som endast släpper igenom de önskade
blandningsprodukterna \(F + f\) eller \(F - f\).

\subsubsection{Dubbelbalanserad blandare}
\index{ringblandare}
\index{dubbelbalanserad blandare}
\index{blandare!ring}
\index{blandare!dubbelbalanserad}

\mediumtopfig{images/cropped_pdfs/bild_2_3-86.pdf}{Dubbelbalanserad blandare}{fig:BildII3-86}

Bild \ssaref{fig:BildII3-86} visar en dubbelbalanserad blandare.

En dubbelbalanserad blandare (även kallad \emph{ringblandare}) består av fyra
dioder, som är riktade åt samma håll i en ''diodring''.
Ingången matas med en signal med en låg frekvens \(f\).
Till skillnad mot i den balanserade blandaren flyter en ström genom dioderna
\(D_1\) och \(D_4\) respektive \(D_2\) och \(D_3\), men inte genom
utgångstransformatorn.
Ingen signal finns på utgången så länge som signalen \(F\) saknas.

Signalen på \(E_1\) avlägsnas och i stället matas ingången \(E_2\) med
en hög frekvens \(F\).
Till skillnad mot i den balanserade blandaren flyter en ström genom dioderna
\(D_1\) och \(D_2\) respektive \(D_3\) och \(D_4\).
Magnetfälten i transformatorernas lindningshalvor upphäver då varandra.
Ingen signal finns på utgången, så länge som signalen \(f\) saknas.

När signaler läggs på båda ingångarna händer följande:

De fyra dioderna kommer att öppna och stänga parvis.
Som i den balanserade blandaren överlagras strömmen från ingång \(E_1\) på den
ström som dioderna öppnar för.

Här utnyttjas båda halvperioderna av \(F\).
Strömmarna i lindningshalvorna blir olika stora.
På utgången uppträder då en signal.
Efter blandaren följer ett filter som släpper igenom de önskade
blandningsprodukterna.

\clearpage
\subsection{Jämförelse av blandare}

\mediumtopfig{images/cropped_pdfs/bild_2_3-87.pdf}{Jämförelse mellan olika blandare}{fig:BildII3-87}

Bild \ssaref{fig:BildII3-87} visar de tre beskrivna grundkopplingarna och de
jämförs med avseende på frekvensspektrum på utgången.

För den obalanserade blandaren uppträder summafrekvensen \(F + f\) och
skillnadsfrekvensen \(F - f\), vidare ingångsfrekvenserna \(f\) och \(F\),
deras övertoner \(2f\), \(3f\), \(4f\), \ldots respektive  \(2F\), \(3F\),
\(4F\), \ldots liksom deras blandningsprodukter \(F\pm 2f\), \(F\pm
3f\), \ldots och \(2F \pm f\), \(2F \pm 2f\), \(2F \pm 3f\) och så vidare.

För den balanserade blandaren saknas frekvensen \(F\) och dess övertoner.
Vidare bortfaller de jämna övertonerna av frekvensen \(f\).

För den dubbelbalanserade blandaren bortfaller ännu fler icke önskvärda
signaler, nämligen ingångssignalerna \(f\) och \(F\) och alla deras övertoner.
Endast blandningsprodukter av udda övertoner uppträder.

För en obalanserad blandare filtrerar resonanskretsen ut frekvenserna
\(F + f\), \(F - f\), och \(F\).
De balanserade blandarna saknar däremot frekvensen \(F\), den filtrerade
signalen innehåller endast blandningsprodukterna \(F + f\) och \(F - f\).
Om dessa båda blandningsprodukter är väl åtskilda eller resonanskretsen har en
bättre selektionsförmåga, då blir enbart summafrekvensen \(F + f\)
eller skillnadsfrekvensen \(F - f\) framfiltrerad.

Vi har visat tre typer av blandare med passiva komponenter.
Sådana innehåller olinjära dioder (germanium- eller kiseldioder).
Det finns även blandare med aktiva komponenter, det vill säga elektronrör eller
transistorer (bipolära, FET, MOSFET), men det skulle leda för långt
att gå in på alla olika lösningar.
Mer om hur frekvensblandning används för demodulering och modulering finns att
läsa i kapitel \ssaref{mottagare} om mottagare och i kapitel~\ssaref{sändare} om
sändare.

\subsection{Icke önskade övertoner och blandningsprodukter}

Varje olinjärt arbetande funktionssteg alstrar förutom nyttofrekvenser
även icke önskade signaler med andra frekvenser.
Både önskade och icke önskade signaler kan bestå av övertoner eller
blandningsprodukter (skillnads- och summatoner) eller bådadera.

Vissa av signalerna filtreras fram för att utgöra nyttosignaler.
Andra signaler filtreras bort, så att till exempel utsändning inte sker på fel
frekvenser.

\mediumtopfig[0.72]{images/cropped_pdfs/bild_2_3-88.pdf}{Frekvensspektrum från en super-VFO}{fig:BildII3-88}

Bild \ssaref{fig:BildII3-88} visar ett frekvensspektrum från en super-VFO, som vi
beskrivit i avsnitt \ssaref{superVFO}.
Vi ska nu undersöka vilka blandningsprodukter som uppstår i en sådan.
De två mest uppenbara frekvenserna är blandningsprodukterna (summan) i området
\SIrange{144}{146}{\mega\hertz} och (skillnaden) i området
\SIrange{128}{126}{\mega\hertz}.

Ut från blandaren finner vi ingångsfrekvensen \\ \qty{136}{\mega\hertz} och dess
övertoner \qty{272}{\mega\hertz}, \qty{408}{\mega\hertz} och så vidare såväl som
VFO-signalen och dess övertoner.
På bilden är VFO-frekvensen \qty{8}{\mega\hertz} och dess övertoner inritade, det
vill säga \qty{16}{\mega\hertz}, \qty{24}{\mega\hertz}, \qty{32}{\mega\hertz} och
så vidare.

Tyvärr bildar också de båda ingångssignalernas övertoner
blandningsprodukter vilket bilden visar.

Bandpassfiltret släpper igenom nyttofrekvensen, men dämpar alla övertoner och
blandningsprodukter.
Detta är enklare ju längre ifrån nyttosignalen de icke önskade signalerna
ligger.
I vårt exempel faller VFO-signalens övertoner inom bandpassfiltrets passband
på följande sätt:

\begin{align*}
  &15 \cdot 9,6   &= 144 \text{MHz} \quad \text{till} \quad 15 \cdot 9,733 &= 146 \text{MHz} \\
  &16 \cdot 9,0   &= 144 \text{MHz} \quad \text{till} \quad 16 \cdot 9,125 &= 146 \text{MHz} \\
  &17 \cdot 8,471 &= 144 \text{MHz} \quad \text{till} \quad 17 \cdot 8,588 &= 146 \text{MHz} \\
  &18 \cdot 8,0   &= 144 \text{MHz} \quad \text{till} \quad 18 \cdot 8,111 &= 146 \text{MHz} \\
\end{align*}

Eftersom det här handlar om 15:e -- 18:e övertonerna, blir amplituderna så små
att vi kan bortse från dem.

Det är viktigt med goda filter i signalbehandlande funktionssteg.
En god regel är att på ett tidigt stadium filtrera bort oönskade
övertoner och blandningsprodukter -- helst i varje steg -- så att
onödigt komplexa signaler undviks.
Det är också viktigt med frekvensvalet, så att oönskade blandningsprodukter
kommer så långt bort från nyttofrekvensen som möjligt, liksom att endast mycket
höga övertoner med motsvarande små amplituder faller inom det nyttiga
frekvensområdet.

% Avsnitt 3.9 Modulatorer
\mediumtopfig[0.5]{images/cropped_pdfs/bild_2_3-89.pdf}{A3E-modulator}{fig:BildII3-89}

\section{Modulatorer}
\index{modulatorer}

\subsection{Allmänt}
\index{modulation}

När en signal (bärvåg) påverkas så att den överför informationen i en annan
signal, sägs bärvågen bli modulerad.
Detta förlopp kallas modulation.
Vad som då händer behandlas främst i avsnitt~\ssaref{modulation}, med
tillämpningar i kapitel~\ssaref{sändare} och delvis i kapitel~\ssaref{mottagare}.

En anordning för modulation kallas för modulator.
En modulator kan ingå som en funktion i sändare, mottagare med flera system.
Beroende på modulationsmetoden används olika kombinationer av delkretsar som
tillsammans utgör modulatorn.

I detta avsnitt ges exempel på några vanliga modulatorer för sändare.

\subsection{Amplitudmodulatorer}
\index{amplitudmodulator}
\index{amplitudmodulation}

Med en amplitudmodulator påverkas bärvågens amplitud proportionellt
mot den modulerande signalens amplitud.

\index{A1A}
\textbf{Vid sändningsslaget A1A} är amplituden på den modulerande signalen
antingen maximal eller ingen.
Då består modulatorn av en nycklingskrets, som påverkar till exempel ett
drivsteg i sändaren så att bärvågen släpps fram helt eller inte alls.

\index{A3E}
\textbf{Vid sändningsslaget A3E} har den modulerande signalens amplitud
ett analogt förlopp, till exempel tal, med vilket bärvågens amplitud moduleras.
Här beskrivs amplitudmodulation i en förstärkare med ett katodkopplat
elektronrör.
En emitterkopplad transistorförstärkare kan moduleras på ett liknande sätt.
I båda fallen moduleras förstärkarens arbetsspänning (anodspänning respektive 
kollektorspänning) med den modulerande signalen.
Det som då händer är att två signaler blandas på ett sätt som beskrivs i
avsnitt~\ssaref{modulation} med tillämpning på A3E.
I vila är då bärvågsamplituden halva den möjliga inom arbetskurvans linjära del.
Vid modulation kommer bärvågens amplitud att variera mellan noll
och den möjliga amplituden.

% \mediumtopfig[0.5]{images/cropped_pdfs/bild_2_3-89.pdf}{A3E-modulator}{fig:BildII3-89}

Bild~\ssaref{fig:BildII3-89} visar ett sändarslutsteg med en triod.
I serie med tilledningen för anodspänningen finns sekundärlindningen av en
modulationstransformator för LF-signalen.

Den LF-förstärkare som driver transformatorn måste för 100~\% modulationsgrad
kunna avge bärvågens halva effekt.
Eftersom uteffekten från en fullt utmodulerad A3E-sändare är 150~\% av den i
vila, måste slutsteget dimensioneras därefter.
Utöver den egna signalspänningen måste modulationstransformatorn även klara
slutstegets arbetsspänning.

\index{klass A}
\index{klass C}
Om som på bilden anodspänningen i ett förstärkarsteg amplitudmoduleras,
kan förstärkarsteget arbeta olinjärt, till exempel i klass C.
Varje följande förstärkarsteg måste däremot arbeta linjärt, till exempel i klass A.

På grund av den låga verkningsgraden och det stora bandbreddsbehovet används i
dagens amatörradiosändare knappast ''äkta'' amplitudmodulering,
det vill säga A3E.
I stället används i läget ''AM'' nästan alltid H3E, det vill säga enkelt
sidband med full eller reducerad bärvåg (se nästa stycke).
Trots det lägre effektbehovet på grund av endast ett sidband och eventuellt
reducerad bärvågsamplitud kan av dimensioneringsskäl ändå inte de flesta
H3E-sändare avge sin fulla effekt kontinuerligt!

Som redan sagts i avsnitt~\ssaref{modulation}, är det onödigt sända ut två
sidband, eftersom båda innehåller samma information.
Det räcker med ett sidband.
Bärvågen innehåller inte någon information.
Den kan därför undertryckas redan i sändaren för att ersättas i mottagaren.
Därmed uppstår sändningsslaget J3E.

\newpage
\subsection{Sändningsslaget J3E (SSB)}
\index{Single Side Band (SSB)}
\index{SSB}
\index{J3E}

Vid sändningsslaget J3E (SSB) sänds således endast ett sidband.
Det andra sidbandet och bärvågen undertrycks, vilket kan göras på flera sätt.
Numera är den så kallade filtermetoden allra vanligast och den enda som
behandlas här.

\mediumfig{images/cropped_pdfs/bild_2_3-90.pdf}{Alstring av J3E (SSB)}{fig:BildII3-90}

Bild~\ssaref{fig:BildII3-90} visar alstring av J3E (SSB).
\index{Upper Side Band (USB)}
\index{USB}
\index{Lower Side Band}
\index{LSB|see {Lower Side Band}}
Med filtermetoden blandas HF- och LF-signalerna i en balanserad blandare där de
undertrycks medan blandningsprodukterna med deras summa- och
skillnadsfrekvenser blir kvar, det vill säga det övre och undre sidbandet.

För att undertrycka det ena sidbandet före sändningen följs blandaren
av ett bandpassfilter med bandbredd och frekvensläge för avsett sidband.
Den signal som sänds ut innehåller på så sätt endast ett sidband (Single Side
Band).

Valet mellan USB och LSB kan göras på två sätt.
Antingen genom att välja mellan ett separat passbandfilter för respektive
sidband eller genom att använda ett enda filter och flytta HF-signalen från ena
sidan till den andra av det filtret (se bild~\ssaref{fig:BildII1-28} i
avsnitt~\ssaref{modulation}).

En J3E-modulator enligt filtermetoden består således av en balanserad blandare
ofta en så kallad ringblandare (se bild~\ssaref{fig:BildII3-87} i avsnitt
\ssaref{blandare}) samt ett bandpassfilter.
För att SSB-signalen ska få avsedd sändarfrekvens kan ytterligare
frekvensblandning behövas (se kapitel~\ssaref{sändare}).

\subsection{Vinkelmodulation}
\index{vinkelmodulation}

Vinkelmodulation är samlingsnamnet för frekvensmodulation (FM) och
fasmodulation (PM).

\subsection{Frekvensmodulation}
\index{frekvensmodulation}
\index{F3E}

Vid sändningsslaget F3E (även kallat FM) varierar bärvågens frekvens i
takt med den modulerande signalens amplitud.
Bärvågen kommer på så sätt att pendla omkring en nominell frekvens, det vill
säga vara frekvensmodulerad.
Bärvågsamplituden ändras däremot inte vid frekvensmodulation.

Likspänningsnivåer kan således överföras eftersom en frekvensavvikelse
(deviation) i bärvågen endast påverkas av den modulerande signalens amplitud.

Vid F3E påverkas resonansfrekvensen i den resonanskrets i oscillatorn som
bestämmer dess arbetsfrekvens.
Det görs enklast genom att tillföra en kondensator med variabelt
kapacitansvärde, en kapacitansdiod (se avsnitt~\ssaref{varicap}).

\mediumfig[0.8]{images/cropped_pdfs/bild_2_3-91.pdf}{Alstring av F3E (FM)}{fig:BildII3-91}
\mediumfig[0.8]{images/cropped_pdfs/bild_2_3-92.pdf}{Alstring av G3E (PM)}{fig:BildII3-92}

Bild~\ssaref{fig:BildII3-91} visar en LC-resonanskrets där det ingår en
kapacitansdiod som styrs av en likspänning med en överlagrad modulerande LF-signal.
En likspänning tjänar som en ställbar förspänning till kapacitansdioden.
På så sätt kan man påverka arbetsfrekvensen.
Med den överlagrade LF-signalen påverkas arbetsfrekvensen i takt med
signalamplituden.

\subsection{Fasmodulation}
\index{fasmodulation}
\index{phasemodulation (PM)}
\index{PM}
\index{G3E}

Vid sändningsslaget G3E (även kallat PM) varierar bärvågens fasläge i
förhållande till en omodulerad referens.
Bärvågens amplitud ändras däremot inte.
Fasändringen -- deviationen -- är direkt proportionell mot hur snabbt fasläget
ändras och till den totala fasändringen.
Hastigheten på fasändringen är direkt proportionell mot frekvensen på den
modulerande signalen och till dess amplitud.

Det betyder att deviationen vid fasmodulation ökar både med amplituden
och frekvensen på den modulerande signalen.
Ändringar i likspänningsnivåer kan därför överföras endast om en fasreferens
används.

Fasmodulation kan alstras till exempel genom att påverka resonansfrekvensen i
en resonanskrets någonstans efter oscillatorn, det vill säga där
oscillatorfrekvensen inte påverkas.
Denna resonanskrets har i viloläge samma resonansfrekvens som oscillatorn.
När kretsen bringas ur resonans genom modulation -- samtidigt som kretsen
påtrycks oscillatorsignalen -- uppstår i kretsen omväxlande en induktiv och
kapacitiv reaktans -- detta inom tiden för varje halv period.
Reaktansen skapar därvid den fasförskjutning som innebär fasmodulation.
Se även avsnitt~\ssaref{parallellresonans} och \ssaref{serieresonans}, bilderna
\ssaref{fig:BildII3-18} och \ssaref{fig:BildII3-19}.

% \mediumfig[0.8]{images/cropped_pdfs/bild_2_3-92.pdf}{Alstring av G3E (PM)}{fig:BildII3-92}

Bild~\ssaref{fig:BildII3-92} visar alstring av G3E (PM).
Liksom vid frekvensmodulation kan till exempel en kapacitansdiod användas för att
med en modulerande signal påverka resonansfrekvensen i en krets.

% Avsnitt 3.10 Digital signalbehandling
\newpage
\mediumfig[0.9]{images/cropped_pdfs/bild_2_1-37.pdf}{Sampling med ADC, DSP och DAC för att återvinna analog signal}{fig:BildII1-37}
\section{Digital signalbehandling}
\label{sec:DSP}
\harecsection{\harec{a}{3.8}{3.8}}
\index{digital signalbehandling}
\index{digital signal processing (DSP)}
\index{DSP|see {digital signalbehandling}}
\index{Software Defined Radio}
\index{SDR|see {Software Defined Radio}}
\index{FPGA}

\emph{Digital signalbehandling} (eng. \emph{Digital Signal Processing, DSP})
har blivit allt viktigare i vardagen och så även inom amatörradion i och med
att \emph{Software Defined Radio (SDR)} blivit en viktig del i allt fler
radior, liksom användning av vanliga datorer.

I takt med att utvecklingen gjort avancerade kretsar allt billigare har det
blivit allt vanligare med olika former av digital signalbehandling, och dessa
används i varierande grad även i radiodesign.

Ofta sammanfattas det med termen \emph{digital signalbehandling} (eng.
\emph{Digital Signal Processing, DSP}).

I grunden bygger den på att man digitaliserar signalerna, behandlar dem
digitalt i exempelvis en processor eller programmerbar logik (FPGA) och sedan
omvandlar dem till analoga signaler igen.

Om man har en särskilt processor för att göra det kallar man den för en
\emph{digital signalprocessor (DSP)}.
Behandlingen kan även göras av dedikerad logik, alltså logik avsedd för speciellt
ändamål, som inte kan programmeras på normalt sätt som en processor.
Det är fortfarande \emph{digital signalbehandling}, men den används nu mer mest
för de delar av signalbehandlingen där man behöver utföra samma standardiserade
jobb fort och effektivt.
En processor kan i stället utföra de mindre frekventa jobben som därmed kan
tillåtas vara mer komplexa.

Ofta förväxlas det begreppet med Digital Signal Processor (DSP), som kommit att
representera en typ av processorer anpassade för signalbearbetning.
Dock är begreppet vidare än så, och vilken annan form av digital processing är
också en digital signalprocessing.

En GPS-mottagare är ett exempel på en sådan mottagare där dedikerad hårdvara
hanterar många miljoner sampel per sekund, men som reducerar dem till några värden
per millisekund som sedan behandlas vidare i en processor.

För att kunna förstå detta behöver vi gå igenom grunderna i konvertering av
signalerna från analogt till digitalt och tillbaka från digitalt till analogt.

\subsection{Sampling och kvantisering}
\harecsection{\harec{a}{1.10.1}{1.10.1}}
\index{sampling}
\index{samplingstakt}
\index{sample rate}
\index{samplingsperiod}
\index{sample (S)}
\index{enheter!sample (S)}
\index{tidsdiskret}
\index{kvantisering}
\index{quantize}
\index{Pulse Code Modulation (PCM)}
\index{PCM}
\label{sampling}

Analoga signaler är vad vi kallar för kontinuerliga i tid.
Hos dessa varierar spänning och ström kontinuerligt så snabbt att vi kan hantera
det fulla radiospektrumet och mer därtill.
Detta fungerar dock inte så väl i den digitala världen.
Där vill vi dels ha värden i digital form, så vi behöver omvandla våra
spänningar och strömmar till tal, och dels behöver vi göra det i en jämn takt.

\emph{Sampling} är ett engelskt ord som betyder att ta strickprov eller att göra
ett urval.
Vi tar ett stickprov (sampel) då och då, och i detta sammanhang gör vi det i en
jämn takt, \emph{samplingstakten} (eng. \emph{sample rate}).
Denna benämner vi ofta med \(f_S\) och begreppet \emph{samplingsperiodtid}
\(T_S=\frac{1}{f_S}\) används också.
Samplingstakten är alltså den jämna takt varmed vi får värden.
Ibland säger man lite slarvigt att samplingstakten är exempelvis
\qty{1}{\mega\hertz}, men det mer korrekta är att den är 1~MS/s, det vill säga 1
miljon sampel per sekund.

Bild~\ssaref{fig:BildII1-37} illustrerar hur en analog signal samplas och
kvantiseras i en ADC, för att behandlas i en DSP, för att därefter konverteras
till analog signal med DAC och filtreras.

Medan sampling är den process som ger oss \emph{tidsdiskreta} värden istället
för tidskontinuerliga värden så är värdena fortfarande inte representerade som
tal, det vill säga värdesdiskreta istället för värdeskontinuerliga.
För att åstadkomma detta behöver man omvandla värdena till fasta nivåer, en
process som kallas för \emph{kvantisering} (eng. \emph{quantize}).

Vid kvantisering har man ofta ett fixt avstånd mellan stegen på en trappstege
av värden.
Varje steg kallas ibland för kvantiseringssteg och storleken på varje
kvantiseringssteg avgör därmed hur hög upplösning man får.
Har man till exempel ett kvantiseringssteg på \qty{0,1}{\volt} så blir 0 till
\qty{0,1}{\volt} tolkat som 0, \SIrange{0,1}{0,2}{\volt} blir tolkat som 1 och
så vidare.
Bild~\ssaref{fig:BildII1-37} visar hur kvantiseringen sker i ADC-steget.

Denna sista del i att omvandla de kvantiserade talen till värden kallas
\emph{Pulse Code Modulation (PCM)}.
Omvandlingen kan även ske olinjärt, alltså med olika avstånd mellan stegen i
kvantiseringstrappan, vilket nyttjats för datakompression i telefonisystem.

\subsection{Minsta samplingsfrekvensen}
\harecsection{\harec{a}{1.10.2}{1.10.2}}
\index{nyquistfrekvens}
\index{Nyquist-Shannons samplingsteorem}
\label{nyquist}

\begin{tcolorbox}[title=Historia]
Denna frekvens kallas för nyquistfrekvensen efter Harry Nyquist (1889--1976),
från Stora Kil i Värmland, efter hans banbrytande arbete på Bell Laboratories
som han publicerade åren 1924 och 1928. Den ingår i \emph{Nyquist-Shannons
samplingsteorem} (eng. \emph{Nyquist-Shannon sampling theorem}).
\end{tcolorbox}

Vår nya begreppsvärld har några inneboende begränsningar, en av dem är minsta
samplingsfrekvensen.
Den lägsta frekvensen vi kan hantera i vårt samplade material är fasta värden
(eller DC som man oftast säger) medan den högsta är den när man alternerar
mellan två värden, säg \num{-1} \num{+1} \num{-1} \num{+1} vilket ju ger hälften
av samplingstakten \(f_S\), för perioden för sekvensen blir \(T = 2T_S\) och
därmed
%% k7per:
\[f=\frac{1}{T}=\frac{1}{2T_S}=\frac{f_S}{2}\].

%% k7per: This figure is still not in a good spot.
\begin{figure*}
\begin{center}
\includegraphics[width=0.45\textwidth]{images/cropped_pdfs/bild_2_1-38.pdf}
\includegraphics[width=0.45\textwidth]{images/cropped_pdfs/bild_2_1-39.pdf}
\includegraphics[width=0.45\textwidth]{images/cropped_pdfs/bild_2_1-40.pdf}
\includegraphics[width=0.45\textwidth]{images/cropped_pdfs/bild_2_1-41.pdf}
\caption{Sampling av DC; \qty{3,6}{\kilo\hertz}; \qty{12,4}{\kilo\hertz} och \qty{38}{\kilo\hertz} med \qty{40}{kS/s} samplingstakt}
\label{fig:BildII1-38}
\end{center}
\end{figure*}

\subsection{Digitala filter}
\harecsection{\harec{a}{3.8.1}{3.8.1}}
\label{digitala filter}
\index{digitala filter}
\index{filter!digitala}
\index{FIR}
\index{filter!FIR}
\index{IIR}
\index{filter!IIR}

Eftersom en signal så som den representeras för digitala kretsar måste vara
samplad och kvantiserad, så kommer signalen att ofrånkomligen bestå av ett
antal sampel med ett visst antal bitar för dess PCM-värde.

Att ändra nivån på en sådan signal görs genom att multiplicera den med något
värde, det vill säga låta varje enskilt sampel i tur och ordning multipliceras
med samma värde, men det ändrar inga egenskaper i frekvensen.
För att få en påverkan med avseende på frekvens behöver man kombinera värdena
från flera olika tidpunkter i signalen, och ofta väljer man att låta de
vägas samman med olika vikt.
Detta görs genom att helt enkelt fördröja samplen i flera steg,
multiplicera varje fördröjning med sin vikt-konstant och sedan summera
resultatet.

Det filter som man då skapat kallas för ett Finite Impulse Response (FIR)
filter, för skickar man in en puls på ingången så kommer den att fördröjas
stegvis och ge svaret från var och en av multiplikatorerna, i var sitt sampel,
till dess att fördröjningskedjan är slut, varvid den impulsen inte ger något
mer bidrag till utgången.
Den räcka med sampel som kom från impulsen kallas för impulsresponsen, och
eftersom den tar slut är den finit, därav namnet.

Man kan göra en variant av det här där man helt enkelt låter en annan
uppsättning med multiplikatorer väga samma fördröjda sampel, men där det
summerade svaret återmatas till ingången och adderas där innan
fördröjningskedjan.
Detta kallas för Infinite Impulse Response (IIR) filter, för att det i likhet
med FIR-filter har en impulsrespons, men eftersom den återmatar så kan denna
rent teoretiskt pågå i all oändlighet, det vill säga engelskans infinite.
I praktiken designas filter så att de inte pågår i evinnerlig tid utan, så att
säga ringer ut.
Själva arkitekturen är dock väldigt lämplig att använda för många ändamål.

Utöver själva filterstrukturen, det vill säga IIR och FIR, så karakteriseras de
av hur många fördröjningssteg man har, då det representerar hur komplext
filtret är, samt av koefficienterna som ger responsen hos filtret.
Design av filterkoefficienter skiljer markant för IIR och FIR, och det finns
både enkla och avancerade verktyg för det.

Ett specialfall på FIR-filter är när koefficienterna är speglade runt mitten.
Då kan man matematiskt visa att de har egenskapen av linjär fas (eng.
\emph{linear phase filter}), och de har enbart påverkan på amplituden.
En fördel med sådana filter, som är fas-linjära, är att olika frekvensers
signal upplever samma grupp-fördröjning och därmed inte förskjuts i förhållande
till varandra.
Detta brukar bland annat öka taltydligheten.

\subsection{Faltning}
\harecsection{\harec{a}{1.10.3}{1.10.3}}
\index{faltning}
\index{convolution}
\index{konvolution}
\index{linjärt tidsinvariant filter}
\index{linear time-invariant filter}
\index{LTI}

Filtrering i den digitala domänen, eller egentligen den tidsdiskreta domänen,
kan beskrivas som att filtrets impulsrespons appliceras på signalen. Denna
process kallas för \emph{faltning} (ty. \emph{faltung} ''vikning'') eller ibland
\emph{konvolution} (eng. \emph{convolution}).
Man kan se det som att varje enskilt sampel kommer att spela upp hela filtrets
svängning med sin amplitud, och responsen från alla sampel blir därför summan
av alla dessa.
Den matematiskt sinnade kan då använda formeln:
%%
\[y(n) = \sum_{m=0}^{N-1} x(n-m)h(m)\]
%%
där \(x(n)\) är den inkommande sampelströmmen och \(n\) är indexet för det
n:te samplet, \(h(m)\) är filtrets respons och slutligen \(y(n)\) är de utgående
samplen.
Denna summering är densamma som beskrivits ovan och skildrar processen
i tidsplanet, det vill säga när vi uttrycker amplituden som funktion av tid.

Motsvarande process kan även utföras i frekvensplanet, det vill säga när vi 
istället uttrycker amplituden som funktion av frekvens.
Om vi då även har konverterat filtrets egenskaper, gör vi helt enkelt en
multiplikation av signal och filter för varje frekvens:
%%
\[Y(f) = X(f)H(f)\]
%%
Bägge representerar faltning, och är viktiga för förståelsen av \emph{linjära
tidsinvarianta} filter (eng. \emph{linear time-invariant, LTI}) filter,
som är det vi i allmänhet fokuserar på.

\subsection{Antivikningsfilter}
\harecsection{\harec{a}{1.10.4}{1.10.4}}
\index{vikning}
\index{aliasing}
\index{antivikningsfilter}
\index{anti-aliasing filter}

Medan bandbredden vi kan representera är begränsad av nyquistfrekvensen så
är däremot inte frekvensen det.
Själva samplingen ger upphov till \emph{vikning} (eng. \emph{aliasing}),
sådan att spektrumet efter halva samplingsfrekvensen blir vänt så att högre
frekvenser blir lägre.
Denna vikning vänder sedan igen när frekvensen blir den hos
samplingsfrekvensen, och spektrumet upprepar sig.
Detta fenomen uppstår alltid när man går mellan kontinuerlig och diskret tid.

Bild~\ssaref{fig:BildII1-38} visar hur fyra olika signaler, DC, sinus med
\qty{3,6}{\kilo\hertz}, \qty{12,4}{\kilo\hertz} och \qty{38}{\kilo\hertz}, samplas
med samplingstakten 40\,kS/s.
Fallet med DC är uppenbart enkelt, alla punkterna hamnar på samma spänning.
Vid en lågfrekvent sinus, som fallet är med \qty{3,6}{\kilo\hertz} här, får man
punkter spridda över kurvan och de påminner om den ursprungliga sinusen, än mer
om man knyter samman punkterna, vilket antivikningsfiltret i praktiken gör.
En frekvens som är nära nyquistfrekvensen, såsom \qty{12,4}{\kilo\hertz} in i
40\,kS/s och dess \qty{20}{\kilo\hertz} nyquistfrekvens, så är samplingspunkterna
nästa helt alternerande mellan högsta och lägsta läge.
I detta fall är det svårt att se den bakomliggande sinussignalen för ett otränat
öga, men den kan fortfarande rekonstrueras med ett antialiasingfiltret.
Ett ännu svårare fall är \qty{38}{\kilo\hertz}, där punkterna visar en sinus med
\qty{2}{\kilo\hertz}, då frekvensen vikt sig ned runt nyquistfrekvensen, och
eftersom infrekvensen är \qty{18}{\kilo\hertz} över nyquistfrekvensen hamnar den
därför \qty{18}{\kilo\hertz} under nyquistfrekvensen, det vill säga på
\qty{2}{\kilo\hertz} i det här fallet.
Denna vikning är det man försöker undvika med antivikningsfiltret, eftersom
toner kan vika sig ned och bli störningar.
Denna vikning sker både vid själva samplingen och även omvänt när man lägger ut
en signal analogt igen. Därför krävs filtrering i bägge riktningarna.

Vid sampling kan alltså högre frekvenser vika ned sig i spektrumet.
Detta är oftast oönskat, varvid man har ett filter före ingången som
undertrycker oönskade signaler.
För exempelvis talsignaler använder man ett lågpassfilter för att undertrycka de
oönskade signalerna högre upp.
Detta filter kan istället användas för ett visst frekvensband för att
konvertera ned detta band i processen, något som är väldigt populärt i
SDR-sammanhang.
I bägge dessa fall är filtret ett \emph{antivikningsfilter} (eng.
\emph{anti-aliasing filter}).

Omvänt, när man ska konvertera från tidsdiskret till tidskontinuerlig
signal så viker sig signalen uppåt i frekvens, och för att undertrycka dessa
oönskade frekvenser används på samma sätt ett antivikningsfilter.
På samma sätt som förut kan man antingen få de låga frekvenserna som för tal
med ett lågpassfilter eller högre upp i ett band med ett lämpligt
bandpassfilter.

Antivikningsfilter kan många gånger vara relativt branta, för de måste
undertrycka andra delar av spektrumet så att dessa inte blir en störning.

Vid varje fall när man använder en annan frekvens än den lägsta upp till
nyquistfrekvensen får man vara omsorgsfull för att se till att man inte viker
det tänkta bandet.
Ofta kombinerar man därför med en separat mixer för att flytta bandet på ett
behändigt sätt, men det förekommer också att man väljer samplingstakten för att
inte vika bandet.

\subsection{Fouriertransform och FFT}
\harecsection{\harec{a}{3.8.2}{3.8.2}}
\index{Fouriertransform}
\index{Fourier!DFT}
\index{Fourier!FFT}
\index{diskret fouriertransform}
\index{Discrete Fourier Transform (DFT)}
\index{DFT}
\index{FFT}
\index{Fast Fourier Transform (FFT)}

En specifik form av processing som blivit tillgänglig är fouriertransform,
det vill säga förmågan att omvandla från signalstyrka över tid till
signalstyrka över frekvens.
Eftersom processingen sker i diskret tid, det vill säga värden med en viss tid
emellan, så som ofrånkomligt med samplade värden, så är det ett specialfall av
fouriertransform, som därför heter \emph{diskret fouriertransform} (eng.
\emph{Discrete Fourier Transform, DFT}).

DFT kan göras på alla möjliga längder av sekvenser, men är beräkningstungt
om man vill ha alla möjliga frekvenser.
För att reducera beräkningsmängden kan man givetvis beräkna DFT bara för ett
fåtal frekvenser, men när det inte är applicerbart behöver man agera lite
smartare.
Så som DFT är formulerat, så ger matematiken flera genvägar, som gör att man
på flera olika sätt kan slå samman beräkningarna och göra delberäkningar som
kan användas av flera andra steg, och på så sätt minska beräkningsbördan.
Detta kan sedan göras hierarkiskt, så att en rekursiv form kan göras.
Det finns flera metoder att göra detta på, men de sammanfattas med som en snabb
DFT, det vill säga \emph{Fast Fourier Transform (FFT)}, som även den är diskret.
En nackdel med FFT är man ofta hamnar på jämna tvåpotenser i antalet sampel,
till exempel 512, 1024, 2048, 4096 sampel och frekvenser.
Man har därmed offrat lite av DFT:ns generalitet.

Det finns mer avancerade formuleringar av FFT som utnyttjar ett eller annat
trick för att jämna ut till fler storlekar, genom att inte bara göra
kombination om 2 sampel, utan även 3, 5 och så vidare, som sedan kan kombineras
till flera storlekar.
Ett annat trick är att helt enkelt fylla på med bara nollor efter, och köra
med en för stor FFT.

Oavsett hur fourieranalysen görs, medger den att man fort kan få upp ett
spektrum.
Detta används nu mer allt oftare för att få en spektrumplot och genom att
lägga flera av dessa efter varandra kan man få de nu mer allt vanligare
spektrumhistogrammen även kända som vattenfallsplottar då de påminner om ett
vattenfall med sina vertikala streck.

\subsection{ADC/DAC}
\harecsection{\harec{a}{1.10.5}{1.10.5}}
\index{ADC}
\index{DAC}
\label{ADC-DAC}

För att hantera dessa delar använder man analog-till-digital-omvandlare
(eng. \emph{Analog-Digital Conversion, ADC}) samt digital-till-analog-omvandlare 
(eng. \emph{Digital-Analog Conversion, DAC}).
En ADC tar hand om sampling, kvantisering och PCM-kodning medan en DAC
omvandlar PCM-koden till analog spänning.
Ofta behöver man komplettera med analoga filter, men moderna sigma-delta-omvandlare 
har kraftigt reducerat kraven.

ADC och DAC köper man idag som färdiga integrerade kretsar, inte sällan med
flera kanaler och det finns även de som har bägge integrerade i samma krets.
Utvecklingen har gjort att man idag kan köpa 24-bitars 48\,kS/s ADC och DAC med
dynamiskt område bättre än \qty{100}{\decibel} för väldigt låg kostnad.


\subsection{Direct Digital Synthesis (DDS)}
\harecsection{\harec{a}{3.8.3}{3.8.3}}
\index{Direct Digital Synthesis (DDS)}
\index{DDS}
\index{fasackumulator}
\index{phase accumulator (PA)}
\index{PA}
\index{nyquistfrekvens}

En term som kommit starkt på senare år är \emph{Direct Digital Synthesis (DDS)}.
Detta syftar på att man kan istället för som med en PLL indirekt styra en
oscillator direkt syntetisera en vågform, och man kan göra det med väldigt
hög upplösning och ändra den väldigt fort.
Medan det kan göras på många sätt, så är den dominerande principen den att
man gör en oscillator med en så kallad \emph{fasackumulator} (eng. \emph{phase
accumulator, PA}).
En fasackumulator är inget annat än ett adderingssteg följt av en delay-steg.
Det är ett extremfall av ett IIR filter, med enbart en pol, som integrerar,
det vill säga ackumulerande effekt.
Värdet ut från denna representerar oscillatorns fas, där av phase accumulator.
Frekvensen styrs helt enkelt med ett värde som anger hur mycket fasen ska
ökas för varje sampel.
Frekvensen blir därför helt linjär, så när som på steg-upplösningen, och kan
varieras fort och fritt.
Upplösningen avgörs därför av hur många bitar bred som hela ackumulatorn har.
Högsta frekvensen blir \emph{Nyquistfrekvensen}, det vill säga halva
samplingsfrekvensen och lägsta blir den som minst signifikanta biten ger.

Den utgående fasen ur själva fasackumulatorn vågformas sedan om till sinus,
cosinus eller vad man nu önskar.
Det går även att använda en uppslagstabell för att kunna syntetisera godtyckliga
vågformer.

Idag finns det färdiga kretsar som ger väldigt stort frekvensområde med 32, 48,
eller fler bitars upplösning.
Inte helt sällan används DDS i kombination med mer klassiska PLL lösningar
för att få bra egenskaper.

DDS har skapat en enorm frihet i hur radioapparater kan designas, och det har
bidragit enormt till både prestanda och miniatyrisering.

%
%
% Kapitel 4 Isolation och jordning
% Avsnitt 4.1 Isolation
% Avsnitt 4.2 Jordning
% Avsnitt 4.3 Gemensam och diff
\section{Isolation}
\index{isolation}
\index{isolator}
\index{galvanisk isolation}
\index{isolation!galvanisk}

\emph{Isolation} (eng. \emph{isolation}) är ett samlingsbegrepp för att separera
olika signaler.
Den första enkla separationen är den hos en \emph{isolator}, det vill säga ett
material som inte leder ström så bra.
Det är den mest grundläggande formen av isolation som förhindrar elektrisk
ledning mellan ledningar.

Man brukar prata om \emph{galvanisk isolation} (eng. \emph{galvanic isolation})
för en isolation som inte kan leda likström.
Transformatorer används ofta för att åstadkomma galvanisk isolation.

Nu är isolation inte begränsat till enbart likström, utan även växelspänning
kan behöva isoleras.
Hur god isolationen är beror kraftigt på frekvensen, och de åtgärder man gör
bör anpassas för hur god isolation man behöver eller vill ha för olika
frekvenser.
Man kan till exempel vilja ha god isolation vid sändar- och mottagarfrekvensen
\qty{14}{\mega\hertz}, men vill inte ha galvanisk isolation för det gemensamma
\qty{12}{\volt} kraftaggregatet.

\section{Jordning}
\index{jordning}
\index{bonding}
\index{jordnät}
\index{jord!jordnät}
\index{bonding network}
\index{jordpotential}
\index{jord!jordpotential}
\index{blomjord}
\index{jord!blomjord}
\index{skyddsjord}
\index{jord!skyddsjord}
\index{nolla}
\index{jord!nolla}

\emph{Jordning} (eng. \emph{bonding}) eller dagligt tal \emph{jord} (eng.
\emph{ground}, \emph{earth}) är en kopplingsstrategi för att få samma
referenspotential i olika delar av en elektrisk koppling.
Man bygger ett \emph{jordnät} (eng. \emph{bonding network, BN})
\cite[kap 3.2.1]{K27-1991} och \emph{earthing network})
\cite[kap 3.1.3]{K27-1991} för att koppla samman de olika jordpunkterna.

Den engelska termen \emph{bonding} och även \emph{bondning network} ger en
indikation på vad det handlar om, nämligen en metod att knyta samman flera
olika delar av en design eller installation för att få en gemensam
referensspänning.
Det är helt enkelt en galvanisk sammankoppling.

Många gånger kallas den referenspotentialen för \emph{jordpotential} för att
det är väldigt behändigt att använda jorden som referens, helt enkelt gräva ned
en ledare i marken, till exempel jordspett (eng. \emph{earth electrode})
\cite[kap 3.1.2]{K27-1991}, för att den vägen få tillgång till jordpotentialen.

Begreppen jord och jordning är dock ofta missförstådda då det finns en övertro
på att man kan ta ned störningar med enbart jordning.
Det förekommer också att man upplever att man har problem där jorden upplevs
skapa störningar, varvid en del felaktigt bryter jorden, och därmed
\emph{skyddsjorden}, något man inte får göra av elsäkerhetsskäl.

På samma sätt tror många att man kan göra sig av med en stor växelström i
jorden.
Detta kallas ibland lite skämtsamt för \emph{blomjordning}, för att man inte
tagit hänsyn till jordledarens resistans och induktans, vilket gör att en
växelström inte kan ta sig så långt då ledaren motarbetar den.
Man hade lika gärna kunnat lägga ned sin jordanslutning i en blomkruka för där
gör den lika god nytta.

Inom elkraft förekommer även termen \emph{nolla} (eng. \emph{neutral}), den
kan lätt förväxlas med jorden, men ska hanteras separat från skyddsjord utom
där elsäkerhetsföreskrifter föreskriver att de ska vara sammankopplade.
Nollan är den ledare som är returledare för strömmen.
I det vanligaste elsystemet \emph{TN-C}, är nollan sammankopplad med skyddsjorden
i elcentralen, men ut från elcentralen hanteras den som en separat ledare.
Man får inte koppla ihop dem för att spara ledare!
Skyddsjorden ska ha väldigt lite ström på sig, och därmed även ha väldigt
låg spänningsskillnad från jordpotentialen, men i praktiken kommer det ändå
finnas skillnader.

\smalltikz{
  \begin{circuitikz}[american voltages]
    % Ground reference
    \draw (0,0) node[ground]{};
    % Source 1 ground
    \draw (0,0) to [R, l^=$Z_1$] (2,0);
    \draw (2,1) to [short, i^=$I_1$] (2,0);
    \draw (1.75,-1) to [short, i=$I_1+I_2+I_3$] (0.25,-1);
    % Source 2 ground
    \draw (2,0) to [R, l^=$Z_2$] (4,0);
    \draw (4,1) to [short, i^=$I_2$] (4,0);
    \draw (3.75,-1) to [short, i=$I_2+I_3$] (2.25,-1);
    % Source 3 ground
    \draw (4,0) to [R, l^=$Z_3$] (6,0);
    \draw (6,1) to [short, i^=$I_3$] (6,0);
    \draw (5.75,-1) to [short, i=$I_3$] (4.25,-1);
  \end{circuitikz}
}{Seriekopplat jordsystem}{fig:kap4-1}

\subsection{Seriekoppling av jord}

Den enklaste uppkopplingen av jordförbindelse är att seriekoppla jorden
\cite[kap 3]{ott1988} mellan ett antal strömförbrukare.
Detta förekommer till exempel i en serie av eluttag matade från samma säkring
eller flera eluttag i en skarvdosa.

I bild~\ssaref{fig:kap4-1} att vi har tre strömförbrukare som var och en
bidrar med en ström \(I_1\), \(I_2\) och \(I_3\), och att dessa är
seriekopplade till en jordanslutning.
Från jordanslutningen till strömbidraget \(I_1\) har vi impedansen \(Z_1\),
och från den punkten har vi impedansen \(Z_2\) fram till strömbidragen \(I_2\)
och slutligen impedansen \(Z_3\) fram till \(I_3\).

En naiv tolkning är att spänningen \(U_1\) för strömbidraget \(I_1\) blir
\(U_1 = Z_1 I_1\), vidare \(U_2 = (Z_1 + Z_2) I_2\) och
\(U_3 = (Z_1 + Z_2 + Z_3) I_3\) för det blir det ju om varje ström ansluts var
och en för sig, det vill säga normal seriekoppling av impedanserna.
Denna analys är dock för enkel för att ta hänsyn till fallet när strömmarna
ansluts samtidigt, eftersom strömmar och spänningar kommer samverka.

Den totala strömmen genom första impedansen \(Z_1\) blir ju summan av de tre
strömmarna, därför måste också spänningen höjas med det bidraget.
Den första spänningen blir därför \(U_1=Z_1 (I_1 + I_2 + I_3)\).
På liknande sätt beräknas den andra spänningen med de bägge strömmarna \(I_2\)
och \(I_3\) plus spänningen \(U_1\) och därför blir
\(U_2 = U_1 + Z_2 (I_2 + I_3)\).
Slutligen blir den sista spänningen \(U_3 = U_2 + Z_3 I_3\).
Med förenkling får vi
%%
\[
\begin{array}{ll}
U_1 & = Z_1 I_1 + Z_1 I_2 + Z_1 I_3 \\
U_2 & = Z_1 I_1 + (Z_1 + Z_2) I_2 + (Z_1 + Z_2) I_3 \\
U_3 & = Z_1 I_1 + (Z_1 + Z_2) I_2 + (Z_1 + Z_2 + Z_3) I_3
\end{array}
\]
%%
Vi ser då att störningen blir
%%
\[
\begin{array}{ll}
\Delta U_1 & =  Z_1 I_2 + Z_1 I_3 \\
\Delta U_2 & = Z_1 I_1 + (Z_1 + Z_2) I_3 \\
\Delta U_3 & = Z_1 I_1 + (Z_1 + Z_2) I_2
\end{array}
\]
%%
Vilket är ett tydligt exempel på hur strömmarna stör varandras spänningar och
därmed har avsaknad av isolation.

Fördelen med seriekopplad jord är förstås att man får flera korta anslutningar
men däremot kommer summeringen av de olika strömmarna göra att man får dålig
isolation mellan de olika jordströmmarna och hur nollpotentialen upplevs.

\subsection{Parallellkoppling av jord}
\index{stjärnjordning}
\index{jord!stjärn-}
\index{skyddsjord}
\index{nolla}

Om vi istället ansluter våra tre laster med individuella ledare till jord
kommer de olika strömmarna inte att samverka, detta är en parallellkoppling
av jord \cite[kap 3]{ott1988}, se bild~\ssaref{fig:kap4-2}.
Vi har därmed åstadkommit en isolation mellan strömmarna med avseende på
jordanslutningen.

\smalltikz{
    \begin{circuitikz}[american voltages]
      % Ground reference
      \draw (3,0) node[ground]{};
      \draw (1,0) to (5,0);
      % Source 1 ground
      \draw (1,0) to [R, l^=$Z_1$] (1,2);
      \draw (1,3) to [short, i^=$I_1$] (1,2);
      % Source 2 ground
      \draw (3,0) to [R, l^=$Z_2$] (3,2);
      \draw (3,3) to [short, i^=$I_2$] (3,2);
      % Source 3 ground
      \draw (5,0) to [R, l^=$Z_3$] (5,2);
      \draw (5,3) to [short, i^=$I_3$] (5,2);
    \end{circuitikz}
}{Parallellkopplat jordsystem}{fig:kap4-2}

% \noindent
Dock kommer varje strömkälla uppleva en förskjutning i spänningen av
sin jord som beror på dess egen ström och impedansen den har till jord.
För att minska denna effekt kan en minskad strömförbrukning användas
eller oftare en förbättrad jordanslutning.

Givetvis kan även varje strömförbrukare ha två jordar, parallellt.
Elkraftsystemens användning av både \emph{skyddsjord} och \emph{nolla} är
just ett sådant system, där nollan är den som har strömmen och tillåts få
åka runt i spänning, medan skyddsjorden i allmänhet enbart har små strömmar.
Skyddsjordens funktion är också att kunna hantera stora strömmar vid fel,
för att kunna bryta tillförseln.
Skyddsjorden har egentligen det som sitt huvudsyfte, men ger ofta en bra
jordreferens.

I apparater och även inne på kretskort kan man ha parallellkoppling.
Det är även känt som \emph{stjärnjordning} (eng. \emph{star grounding})
eftersom kopplingsschemat ser ut att ha en stjärna från en gemensam punkt.
Det kan vara nyttigt att isolera jord för analoga signaler från digitala eller
rent av reläer, PA med mera.
Man försöker sätta stjärnan direkt vid anslutningen till kraftaggregatet för
att hålla dem så gemensamt som möjligt men med så lite påverkan av
seriejordning som möjligt.
Samma teknik används ofta för själva kraftdistributionen av samma skäl.

\subsection{Sammankoppling av apparater}
\label{sammankopplingavapparater}
\index{jordbrum}
\index{jord!brum}

I ett system där man har gjort parallella jordar i matningen,
bild~\ssaref{fig:kap4-3}, vill man nu koppla samman två apparater för att
överföra en signal.
En första naiv lösning är ju att helt enkelt bara dra en tråd \(Z_{signal}\)
från den ena apparaten över till den andra.
Eftersom de har jordanslutning så har de ju en gemensam jordreferens.

\smalltikz{
    \begin{circuitikz}[american voltages]
      % Ground reference
      \draw (2,0) node[ground]{};
      \draw (1,0) to (3,0);
      % Source ground
      \draw (1,2) to [R, l_=$Z_1$, v^=$U_1$] (1,0);
      \draw (0,2) to [short, i^=$I_1$] (1,2);
      % Source output
      \draw (1,4) to [american voltage source, l_=$U_{ut}$] (1,2);
      % Interconnect and load
      \draw (1,4) to [R, l^=$Z_{signal}$] (3,4)
      to [R, l_=$Z_{load}$, v^=$U_{in}$] (3,2);
      % Destination ground
      \draw (3,2) to [R, l_=$Z_2$, v^=$U_2$] (3,0);
      \draw (4,2) to [short, i_=$I_2$] (3,2);
    \end{circuitikz}
}{Sammankopplat system}{fig:kap4-3}

% \noindent
Problemet är att när strömmen \(I_1\) till den första apparaten går igenom
anslutningsimpedansen \(Z_1\) till jord så ger det en spänning
\(U_1 = Z_1 I_1\) på den jordanslutningen.
På samma sätt kommer den andra apparaten att uppleva jorden med en förskjutning
av jordspänningen på \(U_2 = Z_2 I_2\).
Om den tänka utspänningen är \(U_{ut}\) så kommer den egentliga utspänningen
vara \(U_{ut} + U_1\) i förhållande till jord.
Om vi för stunden antar att det inte går någon anmärkningsvärd ström i ledaren
över till den andra apparaten så kommer den uppleva det som en inspänning
\(U_{in}\) i förhållande till sin jordpotential \(U_2\) det vill säga
\(U_{in} = U_{ut} + U_1 - U_2\).

Vi ser här att skillnaden i jordpotential kommer förskjuta den upplevda
inspänningen \(U_{in}\) från den avsedda spänningen \(U_{ut}\) med skillnaden i
jordpotential, det vill säga \(U_1 - U_2\) som i sin tur beror på
anslutningarnas impedans och strömmarna.
Överföringen kan därför ha problem med sin isolation av \(I_1\) och \(I_2\)
till \(U_{in}\).

Om de bägge strömmarna inte har någon starkt frekvensinnehåll för det
frekvensband som man observerar på mottagaren, så fungerar dock detta fint.
Inte helt sällan råkar dock isolationen bli ett bekymmer antingen direkt eller
genom att det stör funktionen indirekt.

Ett försök att minska störningen är förstås att försöka minska \(Z_1\) och
\(Z_2\) genom att göra motståndet mindre, till exempel genom kortare kablar eller
grövre kablar.
Detta fungerar givetvis, men enbart till en viss praktisk gräns.

Det här illustrerar grunden i hur \emph{jordbrum} (eng. \emph{hum}) brukar
uppstå när man kopplar ihop två apparater.
Själva jordbrummet kommer från kraftaggregaten och då deras strömmar delar
krets med nyttosignalen så kommer \emph{överhörning} (eng. \emph{crosstalk})
göra att brummet blir märkbart.
Det finns givetvis många vägar för brum att störa en signal.

\subsection{Isolerad jordning}
\index{isolerad jordning}
\index{jordning!isolerad}
\index{IBN}
\index{signaljord}
\index{flytande}
\index{jord!flytande}
\index{jordbrum}
\index{jord!brum}
\index{chassijordning}
\index{jord!chassi}
\index{ledningsbunden störning}

En strategi för att skapa isolation från jordvägen är att helt enkelt
isolera signalerna och deras jord från kraftförsörjningens jord, detta kallas
för \emph{isolerad jordning} (eng. \emph{isolated bonding} även \emph{isolated
 bonding network, IBN}) \cite[kap 3.2.4]{K27-1991}.
Man börjar plötsligt prata om \emph{skyddsjord} skilt från \emph{signaljord}
(eng. \emph{signal ground}).

\smalltikz{
    \begin{circuitikz}[american voltages]
      % Ground reference
      \draw (3.5,0) node[ground]{};
      \draw (1,0) to (6,0);
      % Source ground
      \draw (1,0) to [R, l^=$Z_1$] (1,2);
      \draw (0,2) to [short, i^=$I_1$] (1,2);
      % Source output
      \draw (1,4) to [american voltage source, l^=$U_{ut}$] (1,2);
      % Interconnect and load
      \draw (1,4) to [R, l^=$Z_{signal}$] (4,4)
      to [R, l^=$Z_{load}$, v_=$U_{in}$] (4,2);
      \draw (1,2) to [R, l^=$Z_{GND}$] (4,2);
      % Destination isolation
      \draw (4,2) to [R, l^=$Z_{iso2}$, v_=$U_5$] (6,2);
      % Destination ground
      \draw (6,0) to [R, l^=$Z_2$] (6,2);
      \draw (7,2) to [short, i_=$I_2$] (6,2);
    \end{circuitikz}
}{Sammankopplat system med utjämningsledare}{fig:kap4-4}

\noindent
I apparater med växelströmsmatning har man redan en transformator som
tillhandahåller en galvanisk isolation mellan primärsidan (elkraft) och
sekundärsidan (elektroniken).
Genom att helt enkelt hålla signaljorden \emph{flytande} (eng.
\emph{floating}), det vill säga utan någon galvanisk koppling till skyddsjord,
så kan man istället koppla samman signaljord på två apparater med separata
ledare \(Z_{GND}\).
I bild~\ssaref{fig:kap4-4} är isolationen hos mottagande apparat representerad
av \(Z_{iso2}\) där spänningen \(U_5\) representerar spänningen mellan primär
och sekundärsida.
På liknande sätt kan isolationen på den sändande apparatens sida moduleras som
 \(Z_{iso1}\), men för detta resonemang räcker \(Z_{iso2}\).

Den galvaniska åtskillnaden gör att isolationen för likström kan variera från
megaohm till gigaohm, men på grund av den kapacitiva kopplingen mellan primär
och sekundär sida i transformatorn sjunker isolationen med stigande frekvens.
I praktiken kan transformatorn på grund av sin obalans driva spänningen \(U_5\)
och därför så kan man behöva lasta dess kapacitiva källan med ett motstånd,
varvid \(Z_{iso2}\) snarare kan vara i kiloohm.

Om vi återgår till de bägge två apparaterna, så kan vi nu istället för att
använda oss av elnätets skyddsjord låta apparaternas signaljord vara kopplad
med en kabel \(Z_{GND}\) parallell med signalledaren \(Z_{signal}\).
Har vi en förhållandevis låg ström genom den impedans som kabeln har så
kommer det fungera fint och \(U_{ut}\) kommer att representeras hyfsat bra som
spänningen \(U_{in}\) över \(Z_{load}\).

Eftersom \(Z_{GND}\) kan vara några fåtal ohm medan \(Z_{iso2}\) för låga
frekvenser är i storleksordningen megaohm så kommer kabeln att koppla väl.
För högre frekvenser kan vi förvänta oss att induktansen i kabeln ökar
impedansen \(Z_{GND}\) samtidigt som kapacitansen gör att impedansen
\(Z_{iso2}\) sjunker varvid för högre frekvenser kommer \(Z_{load}\) vara mer
kopplad lokalt mot \(Z_{2}\) snarare än \(Z_{1}\).

Det här scenariot liknar till exempel det hos en normal hemmastereo och ändå kan
det uppstå \emph{jordbrum} i denna koppling.
Det finns flera skäl.
Ett skäl är att transformatorer visserligen erbjuder en galvanisk isolation,
men de är även kapacitiva spänningsdelare för den spänning som finns över
primärlindningen, med 230~VAC spänning så behövs bara lite läckage över för att
man ska uppleva att isolationen brister.
Det brukar vara rekommendabelt att helt enkelt lasta denna spänningsdelare med
ett motstånd, så att signaljord och skyddsjord sitter ihop med ett någorlunda
högt motstånd, ofta med en kondensator parallellt, för att se till att reducera
det bidraget utan att få för mycket störningar från den ström som kommer flyta
mellan jordarna.

Ett annat scenario som skapar jordbrum är när man i någon ände råkar hårt
koppla samman signaljord och skyddsjord, typiskt att det blir oavsiktlig
kontakt mot chassi, som ska vara skyddsjordad.
Själva chassit brukar man prata om som \emph{chassijordat}, men det är
egentligen bara skyddsjord på de flesta system.

För att isolationsjordning ska fungera måste alla kontakter vara isolerade från
chassit.
Detta gäller även signaljord som inte får ha kontakt med chassi inuti apparaten.
Man behöver alltså försäkra sig om bra isolationsavstånd, vilket väldigt lätt
kan missas av att man har en skruv som råkar skrapa sig igenom skyddslack till
exempel.

En annan nackdel med isolationsjordning är att den gör det svårare att designa
för god EMC-täthet.
För \emph{ledningsbunden störning} (eng. \emph{conductive emission}) så vill
man helst att kontaktens och kabelns skärm sitter i chassijorden med så låg
impedans (induktans) som möjligt.
Isolationsjordning kräver då att man monterar kondensatorer som kopplar ihop
ledarens jord med chassijord och helst runt om för att få lägsta induktans.

Isolationsjordning rekommenderas inte för större system, då den blir svår
att upprätthålla.

Det förekommer att man för att minska störningarna i ett isolationsjordat
system väljer att koppla bort skyddsjorden, för att på det sättet ha mindre
störningar.
Detta är oftast inte tillåtet göra då man normalt inte bryter mot
elsäkerhetsregler och anläggningen riskerar bli farlig, då personskyddet
sätts ur spel.

\begin{center}
\begin{minipage}{0.19\columnwidth}
\Huge{\fontencoding{U}\fontfamily{futs}\selectfont\char 66\relax}
\end{minipage}
\begin{minipage}{0.7\columnwidth}
Varje gång som skyddsjorden kopplas bort för att lösa ett problem så
har man skaffat sig ett större problem, nämligen signifikant sänkt
elsäkerhet, vilket indikerar att man valt en felaktig lösning.
\end{minipage}
\end{center}

\subsection{Sammankopplad jordning}
\index{sammankopplad jordning}
\index{jord!sammankopplad}
\index{jordloop}
\index{jord!loop}
\index{vagabonderande jordström}
\index{jordbrum}
\index{jord!brum}
\index{chassijord}
\index{skyddsjord}

En annan strategi är \emph{sammankopplad jordning} (eng. \emph{mesh bonding}
och \emph{mesh bonding network, mesh-BN}) \cite[kap 3.2.3]{K27-1991}
där man istället för att isolera satsar på att koppla samman jordarna, hårt.
Varje signalkabel sitter ansluten mot \emph{chassijord} och därmed
\emph{skyddsjord} och man låter därmed jordarna sammankopplas.
Varje apparat har en ordentlig jordanslutning som man ansluter till stativjord
eller jordskenor.
Kablar läggs på kabelstegar som jordas.
I detta system kommer varje extra kabel att koppla samman jordarna hårdare,
eftersom man parallellkopplar många impedanser.
Denna strategi väljs ofta i telekommunikationssystem.

%% k7per: Varifrån ska denna figur refereras?
\smalltikz{
  \begin{circuitikz}[american voltages]
    % Ground reference
    \draw (2.5,0) node[ground]{};
    \draw (1,0) to (4,0);
    % Source ground
    \draw (1,0) to [R, l^=$Z_1$] (1,2);
    \draw (0,2) to [short, i^=$I_1$] (1,2);
    % Source output
    \draw (1,4) to [american voltage source, l^=$U_{ut}$] (1,2);
    % Interconnection
    \draw (1,4) to [R, l^=$Z_{signal}$] (4,4)
    to [R, l^=$Z_{load}$, v_=$U_{in}$] (4,2);
    \draw (1,2) to [R, l^=$Z_{GND}$, v_=$U_{GND}$] (4,2);
    % Destination ground
    \draw (4,0) to [R, l^=$Z_2$] (4,2);
    \draw (5,2) to [short, i_=$I_2$] (4,2);
  \end{circuitikz}
}{Sammankopplat system med utjämningsledare}{fig:kap4-5}

\noindent
I ett system som har sammankopplad jord kommer man ofrånkomligen att behöva
hantera vad man kallar för \emph{jordloop} (eng. \emph{ground loop}) eller även
\emph{vagabonderande jordströmmar}.
Många gånger förklaras det som att man får en loop som agerar antenn för ett
magnetfält.
Det är dock sällan som ett magnetfält är så starkt att det inducerar flera
ampere av vanlig \qty{50}{\hertz} ström.

Om vi går tillbaka till sammankoppling av apparater (kapitel
\ssaref{sammankopplingavapparater}) där vi fick en skillnad av spänning \(U_{GND}\)
mellan jordpunkterna så kommer vi ha den även här, men nu ansluter vi ju en
ledning \(Z_{GND}\) mellan dessa punkter, och då kommer det gå en ström som
försöker utjämna potentialen mellan de bägge jordanslutningarna, som då kommer
närmare varandra.
Det är impedansen \(Z_{GND}\) på kabeln som kommer att avgöra hur stor strömmen
blir och hur nära de kommer varandra spänningsmässigt.
Denna ström kan bli ansenlig och har man då en kabel som har till exempel tunn
skärm så kommer kabeln helt enkelt bli varm.
Det är därför lämpligt att lägga en jordkabel parallellt med signalkabeln, för
att låta den med sin större tvärsnittsarea ta merparten av strömmen och därmed
undviker man värme och ström i signalkabeln.

Med en större kabel mellan kommer spänningen sjunka och den vägen kommer
\emph{jordbrummet} minska.

Fördelen med sammankoppling av jordar är att det blir enklare (och billigare)
att designa ur EMC-perspektiv, då man direkt kopplar jordströmmarna i chassit.
Man har inte heller problem med att man skulle råka jorda eller att man skulle
tappa den enda jordvägen.
Istället försöker man koppla ihop jordarna väl.

Ett vanligt problem är om man låter jordströmmarna gå genom kretskort, vilket
gör att man skapar lokala problem med seriejordning.
Man ska se till att jordströmmarna knyter hårt till chassit, men svagt genom
kortet för att på det sättet få bästa möjliga isolation.
Denna princip är också lämplig för att kunna hantera till exempel ESD-skador.

En annan fördel är att man bygger en vana att jorda allt, och för varje
kompletterande jordning gör man systemet starkare.

En självklar fördel är att man dessutom inte bryter skyddsjord, och därmed inte
sänker elsäkerheten på utrustningen och installationen.

\subsection{Balanserad signal}
\index{balanserad signal}
\index{jordbrum}
\index{jord!brum}
\index{galvanisk isolation}
\index{gemensam spänning}
\index{differentiell spänningen}

För att ytterligare få isolation från jordbrum kan man använda en
\emph{balanserad signal} (eng. \emph{balanced signal}).
Grundprincipen är att man skickar samma signal två gånger, men med omvänt
tecken, och sedan ta emot den och bara titta på skillnaden mellan dem.
Skulle nu en störning introducera sig på dessa ledare gemensamt så påverkar
detta inte skillnaden i spänning mellan dem.

\begin{figure}
  \begin{center}
    \begin{circuitikz}[american voltages]
      % Ground reference
      \draw (4,1) node[ground]{};
      \draw (1,1) to (7,1);
      % Source ground
      \draw (1,1) to [R, l^=$Z_1$] (1,4);
      \draw (0,4) to [short, i^=$I_1$] (1,4) to (2,4);
      % Source diff
      \draw (2,6) to [american voltage source, l^=$U_{ut}$] (2,4)
      to [american voltage source, l^=$U_{ut}$] (2,2);
      % Wires and load
      \draw (2,6) to [R, l^=$Z_{signal+}$] (5,6)
      to [R, l^=$Z_{load}/2$, v_=$U_{in+}$] (5,4)
      to [R, l^=$Z_{load}/2$, v_=$U_{in-}$] (5,2);
      \draw (2,2) to [R, l^=$Z_{signal-}$] (5,2);
      \draw (2,4) to [R, l^=$Z_{GND}$] (5,4);
      % Destination isolation
      \draw (5,4) to [R, l^=$Z_{iso2}$, v_=$U_5$] (7,4);
      % Destination ground
      \draw (7,1) to [R, l^=$Z_2$] (7,4);
      \draw (8,4) to [short, i_=$I_2$] (7,4);
    \end{circuitikz}
  \end{center}
  \caption{Sammankopplat system med utjämningsledare och differentiell signal}
  \label{fig:kap4-6}
\end{figure}

Redan tidigare har vi gjort liknande och försökt efterlikna egenskaperna, för
redan när vi skickade en signal på en enkel ledare så skickar vi en spänning
i förhållande till en referensspänning och vi tittar på den inkommande
spänningen i förhållande till referensspänningen.
Dock har vi haft problem att ha en bra gemensam sådan, och det är uppenbart
att vi egentligen observerar skillnaden i spänning.

Med balanserad signal tar vi steget fullt ut och separerar nollreferens från
signal och skickar en signal som vars summa är en fix spänning medan
skillnaden är nyttosignalen.
Det är som om signalen är neutral.
Ofta är dock signalen av praktiska skäl förskjuten spänningsmässigt.

Den balanserade signalen har jord, \emph{pluspol} och \emph{minuspol}.
\emph{Pluspolen} kallas även +, \emph{positiv polaritet}, \emph{het} (eng.
\emph{positive pole}, \emph{positive polarity} och \emph{hot}) medan
\emph{minuspolen} kallas även -, \emph{negativ polaritet}, \emph{kall} (eng.
\emph{negative pole}, \emph{negative polarity} och \emph{cold}).
Utöver dessa har man oftast en \emph{spänningsreferens} som ofta betäcknas som
\emph{jord} (eng. \emph{ground, GND}) eller \emph{nolla} (eng.
\emph{neutral}).

I bild~\ssaref{fig:kap4-6} visas hur ut-spänningen \(U_{ut}\) är dubblerad och
matar på var sin sida om jordpotentialen som  \(I_{1}\) och \(Z_{1}\) ger.
De bägge utspänningarna är kopplade över var sin ledare \(Z_{singal+}\) och
\(Z_{singal-}\) för att över var sin \(Z_{load}/2\) resultera i \(U_{in+}\)
respektive \(U_{in-}\), som i sin tur sitter mot signaljorden på samma sätt
som tidigare.
Den egentliga in-spänningen är från \(U_{+}\) till \(U_{-}\) det vill säga
\(U_{in} = U_{+} - U_{-} = U_{in+}+U_{in-}\)

Transformatorer passar väl för att både generera och ta emot balanserade
signaler, då de har en \emph{galvanisk isolation} för \emph{gemensam spänning}
men transformerar den \emph{differentiella spänningen}.
Detta kan även göras med aktiv elektronik så som op-ampar men även färdiga
kretsar finns.

Transformatorer har fördelen att man kan få den galvaniska skillnaden genom
att helt enkelt bryta jordförbindelsen på ledaren.
Dock, transformatorer har inte fulländad isolation men kan däremot ofta hantera
ganska stora spänningar, vilket kan krävas i besvärliga sammanhang.
För RF är dock transformatorer inte balanserade och ger dålig isolation.
Förbättrad isolation hos transformatorer kan uppnås med ett eller två
skärmlager mellan lindningarna.
Skärmlagren kan anslutas till respektive sidas jord.
För RF krävs dock en strömbalun/RF-choke för att undertrycka den
gemensamma strömmen.

Aktiv elektronik för balansering har sällan galvanisk isolation, men däremot
kan man upprätthålla hög impedans för den gemensamma spänningen, vilket kan
vara nog så tillräckligt.

Differentiell signal i RF kan uppnås genom att använda en RF-choke som
undertrycker den gemensamma spänningen i RF men inte i likspänning.

\section[Gemensam och diff]{Gemensam och differentiell spänning och ström}

När man har ett treledarsystem som vi har med differentiell matning eller
även om man bara har två ledare men mellan system som har gemensam jord
(gäller också om de bara har RF-koppling en annan väg) så kan man betrakta
de två signalledarna antingen som att de har sin individuella spänning och
ström, eller som att de har gemensam och differentiell spänning och ström.

\subsection{Gemensam och differentiell spänning}
\label{comdiffv}

Gemensam spänning och differentiell spänning är ett alternativt sätt att
betrakta spänning på de bägge ledarna, där man delar upp spänningen i det som
är gemensamt för de bägge spänningarna och det som skiljer dem åt. Man kan
alltså betrakta dem på detta alternativa och oberoende (ortogonala) sättet.

\begin{figure}
  \begin{center}
    \begin{circuitikz}[american voltages]
      % Ground reference
      \draw (3.5,1) node[ground]{};
      \draw (0,1) to (7,1);
      % Source ground
      \draw (0,4) to [american voltage source, l^=$U_{g}$] (0,1);
      \draw (0,4) to (2,4);
      % Source diff
      \draw (2,6) node[anchor=east] {$V_{ut+}$} to [american voltage source, l^=$U_{d}/2$] (2,4)
      to [american voltage source, l^=$U_{d}/2$] (2,2) node[anchor=east] {$V_{ut-}$};
      % Wires and load
      \draw (2,6) to [R, l^=$Z_{signal+}$] (5,6) node[anchor=west] {$V_{in+}$}
      to [R, l^=$Z_{d}/2$, v_=$U_{d+}$] (5,4)
      to [R, l^=$Z_{d}/2$, v_=$U_{d-}$] (5,2) node[anchor=west] {$V_{in-}$};
      \draw (2,2) to [R, l^=$Z_{signal-}$] (5,2);
      \draw (2,4) to [R, l^=$Z_{GND}$] (5,4);
      % Destination isolation
      \draw (5,4) to (7,4);
      % Destination ground
      \draw (7,4) to [R, l^=$Z_{g}$, v_=$U_{g}$] (7,1);
    \end{circuitikz}
  \end{center}
  \caption{Sammankopplat system med utjämningsledare och differentiell signal}
  \label{fig:kap4-7}
\end{figure}

I bild~\ssaref{fig:kap4-7} har man den gemensamma spänningskällan \(U_g\), som
från ersatt de förskjutna jordpunkterna i tidigare exempel.
Den differentiella spänningen \(U_d\), det vill säga den drivande spänningen
mellan \(V_{ut+}\) och  \(V_{ut-}\) är fördelad på två spänningskällor som
levererar halva spänningen var.
%%
\[V_{ut+} = U_g + \dfrac{U_d}{2}\]
\[V_{ut-} = U_g - \dfrac{U_d}{2}\]
%%
Omvänt kan man formulera uttrycken för gemensam spänningen \(U_g\) samt
den differentiella spänningen \(U_d\) som \(V_{ut+}\) och \(V_{ut-}\):
%%
\[U_g = \dfrac{V_{ut+}+V_{ut-}}{2}\]
\[U_d = {V_{ut+}-V_{ut-}}\]
%%
På motsvarande sätt på ingången kan man skriva uttrycken för den gemensamma
mottagna spänningen \(U_{g,in}\) och den mottagna differentiella spänningen
\(U_{d,in}\) baserat på inspänningarna \(V_{in+}\) och \(V_{in-}\), man får då
%%
\[V_{g,in} = \dfrac{V_{in+} + V_{in-}}{2}\]
\[V_{in+} = V_{in+}-V_{in-}\]
%%
Ett sätt att illustrera skillnaden är till exempel med en transformator.
En transformator med 1:1 lindning kopplas in mellan två balanserade signaler.
Transformatorns primärlindning kommer att omvandla den differentiella spänningen
\(V_d\) till en motsvarande spänning på utgången.
Däremot kommer den gemensamma spänningen inte att överföras.
Transformatorn blir då en isolator för den gemensamma spänningen precis som vi
förväntar oss av en galvanisk isolation.

Isolationen för den gemensamma spänningen i en transformator är dock främst ett
likströmsbeteende, så ju högre frekvens desto bättre koppling, det vill säga
sämre isolation.
Detta beror på den kapacitiva kopplingen mellan lindningarna som skapar en
ström, som sammankopplar sidorna och resulterar i att den gemensamma spänningen
ändå går igenom transformatorn.
För högre frekvenser är kopplingen väldigt god och transformatorn gör ingen
nytta för att undertrycka den gemensamma spänningen.

Eftersom nyttosignalen är differentiell kan man ibland medvetet använda den
gemensamma spänningen för att överföra matningsspänning till till exempel en
mikrofon.
Denna form av matningsspänning kallas för \emph{fantommatning}
(eng. \emph{phantom power}).
En vanligt förekommande spänning är \qty{48}{\volt}, som då symboliseras med P48.
Det förekommer även på modern Ethernet-utrustning och kallas då för
\emph{Power over Ethernet (PoE)}.

\subsection{Gemensam och differentiell ström}
\label{comdiffi}
\index{RF-choke}
\index{strömbalun}
\index{current balun}

Precis som för spänning kan man beskriva strömmarna i samma ledare som
gemensam och differentiell ström.
Vi kan därför återanvända formlerna och bara byta ut V mot I genomgående och
får då:
%%
\begin{eqnarray*}
I_+ = & I_g + I_d\\
I_- = & I_g - I_d\\
I_g = & \dfrac{I_+ + I_-}{2}\\
I_d = & \dfrac{I_+ - I_-}{2}
\end{eqnarray*}
%%
Om vi återgår till transformatorexemplet så kommer det vara den differentiella
strömmen på primärlindningen som ger upphov till magnetfältet i transformatorn
och som sedan inducerar en differentiell ström i sekundärlindningen.

Isolationen mellan lindningarna förhindrar att det går en ström mellan dem,
och därför förhindras den gemensamma strömmen vid låga frekvenser.
Vid högre frekvenser kommer dock den kapacitiva kopplingen mellan de två
sidorna att ske varvid en gemensam ström kommer uppstå för högre frekvenser,
det vill säga för högre frekvenser kommer isolationen att bli sämre.

Ett intressant specialfall är om vi sätter en ringkärna på vår kabel, lindar
kabeln flera varv genom den, eller bara lindar den runt luft.
Då kommer strömmen i den ena ledaren inducera en ström i den andra ledaren och
vice versa.
Denna koppling kan liknas vid att vrida en 1:1 transformator 90~grader fel.
Eftersom den inducerade strömmen har motsatt riktning så kommer den motverka
den gemensamma strömmen, men inte den differentiella strömmen. Dessutom kommer
denna koppling bli starkare för högre frekvenser (i den fina teorin) och
därmed skapa en högre isolation för gemensam ström.
Detta kallas för bland annat \emph{RF-choke} (eng. \emph{RF-choke}) och
\emph{strömbalun} (eng. \emph{current balun}).
Den kompletterar isolationen hos en transformator eller löser den nödvändiga
isolationen helt på egen hand.

RF-choke är ett oerhört användbart verktyg för att undertrycka RF-strålning
och det man ofta i EMC sammanhang kallar ledningsbunden strålning, som är en
gemensam ström ut på ledarna.
Att det är den gemensamma strömmen förstås lätt eftersom den differentiella
strömmen från de bägge ledarna kommer att motverka varandra i utstrålat
magnetfält medan den gemensamma strömmen samverkar och därför är det enbart
den som ger ett utstrålat magnetfält.

Det är därför man ofta hittar klumpar som sitter på kablar till till exempel skärmar.
Dessa klumpar är helt enkelt en ringkärna som förstärker kopplingen mellan
ledarna för att undertrycka den gemensamma strömmen för RF och därmed minska
störningen.

\subsection{Generell gemensam och differentiell analys}
\label{comdiffgeneric}
\index{mod!gemensam}
\index{gemensam strömöverföring}
\index{mod!differentiell}
\index{differentiell strömöverföring}
\index{Common Mode (CM)}
\index{CM}
\index{Differential Mode (DM)}
\index{DM}

Efter att ha studerat gemensam och differentiell spänning (kapitel
\ssaref{comdiffv}) och gemensam och differentiell ström (kapitel~\ssaref{comdiffi})
kan vi sammanfattningsvis konstatera att den grundläggande metoden att omvandla
de individuella spänningarna och strömmarna till \emph{gemensam överföring}
(eng. \emph{Common Mode, CM}) och \emph{differentiell överföring}
(eng. \emph{Differential Mode, DM}) är en kraftfull metod både för att
förstå och avhjälpa problem och uppnå isolation.
För spänning har vi ekvationerna
%%
\begin{eqnarray*}
V_+ = & V_{CM} + V_{DM}\\
V_- = & V_{CM} - V_{DM}\\
V_{CM} = & \dfrac{V_+ + V_-}{2}\\
V_{DM} = & \dfrac{V_+ - V_-}{2}
\end{eqnarray*}
%%
För ström har vi ekvationerna
%%
\begin{eqnarray*}
I_+ = & I_{CM} + I_{DM}\\
I_- = & I_{CM} - I_{DM}\\
I_{CM} = & \dfrac{I_+ + I_-}{2}\\
I_{DM} = & \dfrac{I_+ - I_-}{2}
\end{eqnarray*}

\subsection{Gemensam och differentiell impedans}

Precis som man har impedans på ingångar så har man det på ingångar i
treledarsystem.
Det som är den normala impedansen för en transmissionsledare till exempel är
egentligen den differentiella impedansen, det vill säga förhållande mellan den
differentiella spänningen och differentiella strömmen.
Den gemensamma impedansen är på samma sätt förhållandet mellan gemensam
spänning och gemensam ström
%%
\begin{eqnarray*}
Z_{DM} = & \dfrac{U_{DM}}{I_{DM}}\\
Z_{CM} = & \dfrac{U_{CM}}{I_{CM}}
\end{eqnarray*}
%%
Egentligen är det inte så konstigt, om man har en koaxialkabel i ett 50~ohm
system så har sändare och mottagare idealt 50~ohm som differentiell impedans.
I ett system som har isolerad jordning så kan den gemensamma impedansen vara
många megaohm eller högre, eftersom den är isolerad.

\subsection{Obalans}
\index{strömbalun}
\index{obalans}

Så här långt har huvudsakligen antagit att vi har balans, det vill säga att
transformatorer, induktorer med mera är ideala och ger lika bra koppling till
bägge sidor.
Givetvis finns inte detta i verkligheten, och man har en obalans.
Vid obalans får man en signal som är gemensam att läcka över till den som är
differentiell och omvänt att differentiell läcker över till den gemensamma.
Det resulterar dels i minskad isolation och dels i minskad signal.
I allmänhet är den minskade isolationen värre än förlusten av signal, som i
allmänhet är försumbar.

I en transformator ligger lindningarna ofta så att den kapacitiva kopplingen
från ena polen på en spole är starkare än från den andra polen.
Det ger därför en obalans i hur de kopplar kapacitivt.
Genom att lägga ett skärmlager mellan lindningarna kan den kapacitiva
kopplingen jämnas ut, då de kopplar kapacitivt till skärmlagret istället,
som kan lågresistivt hindra koppling.
En ännu bättre lösning är att ha dubbla lager med isolation, för då
kan de kopplas mot respektive sidas jord, och kvar blir bara den kapacitiva
kopplingen mellan jordarna, som oftast är ett mindre problem.
Med dessa metoder fås bättre isolation än vad en oskärmad transformator kan
erbjuda, på grund av just obalans.

Den kapacitiva kopplingen har väldigt hög impedans vid \qty{50}{\hertz}, så man
kan använda relativt höga motståndsvärden för att lasta ned den hårt.
Fördelen är att man kan undvika direkt koppling, vilket kan skapa andra
problem som när man vill ha relativ isolation galvaniskt.

I en strömbalun kan den ena ledaren ha något lite längre varv runt kärnan än
den andra.
Det ger inte en perfekt 1:1 relation i kopplingen och därmed en obalans.

I en transformator med mitt-tapp kan mitt-tappen sitta lite förskjuten från
riktiga mitten, så att anslutningen av mitt-tappen till jord skapar en
obalans.

Dessa exempel på brister i konstruktionen ska man vara medveten om, så att man
inte tillskriver en transformator eller strömbalun att ha egenskaperna av en
perfekt isolation.
Snarare ska man förvänta sig att den inte är perfekt och anpassa sin design
efter det.
Många gånger kan en kombination av åtgärder ge fullgott resultat utan att vara
särdeles dyrt eller klumpigt, men det kräver eftertanke och helhetssyn.

Ett enkelt fall i ljudsammanhang är \qty{50}{\hertz} \qty{230}{\volt} men man
vill hålla störningen mindre än säg \qty{1}{\milli\volt}.
Det kräver mer än \qty{106}{\decibel} isolation mellan \qty{230}{\volt}
differentiellt på primärlindningen och \qty{1}{\milli\volt} gemensamt på
sekundärlindningen.
Så god balans kan vara svår att finna i enskilda komponenter.
Principen återkommer oavsett spänning och frekvens, det är en brist
man behöver lära sig att förstå och hantera.

\subsection{Obalans i antennsystem}
\index{obalans!antennsystem}
\index{mantelström}
\index{obalans!mantelström}
\index{balun}
\index{strömbalun}
\label{obalans_antennsystem}

Obalans kan även förekomma i antennsystem, där en obalanserad antenn omvandlar
den utsända signalen, som är differentiell, till att delvis bli gemensam.
Detta gör att via reflektion från den obalanserade antennen går en ström i
matningsledningen som gör att den strålar.

Detta har traditionellt uttryckts som att strömmen vänder och går på utsidan av
skärmen, men det som hänt är att den differentiella strömmen, som ju motverkar
utstrålning plötsligt får en pålagd gemensam komponent som då kommer stråla.
Man kan uppleva det om man berör ledningen så kan man känna denna som en ström,
vilket man upplever går på utsidan.
Kabeln har då blivit en strålande del av antennen, något som för vissa
antenntyper är en medveten design.

Det är också denna ström som behöver motverkas för att operatören inte ska
skada sig.
Detta görs med en strömbalun, lämpligtvis en kvartsvåg ned från anslutningen
till antennen.
Strömbalunen motverkar den gemensamma strömmen utan att nämnvärt påverka
den differentiella, så det är ett fint exempel på en bra åtgärd.

De allra flesta antenner har en annan impedans i matningspunkten än vad dess
matarledning har.
Detta kräver en impedansanpassning för optimal energiöverföring.
En annan aspekt är att för en koaxial matning så överförs energin enkelsidigt
(single-ended) det vill säga att det är mittledaren i förhållande till
skärm/jord som överför energi.
När vi ansluter denna ledare till en dipolantenn vill vi se till att strömmen
går balanserat ut i de bägge ledarna, så att mittpunkten är nära noll, så att
det inte går en ström med gemensam mod ut i matarledningen.

Vi har alltså dels behovet att omvandla obalanserad signal till balanserad
samt undertrycka gemensam signal i ledaren, och därtill impedanskonvertera den.
Detta brukar man låta en balun (balanced-unbalanced) göra, vilket som namnet
anger bara ger indikation på konverteringen, men den gör alltså flera saker.
Eftersom ingen balun är perfekt designad så kommer den i sig själv ha en
obalans, varvid den ändå kommer ge viss gemensam ström.
För högre effekter kan därför en separat spärr komma att behövas.

Utöver balun finns även unun (unbalan\-ced-un\-balan\-ced) som gör
impedanskonvertering enbart.

Även om man har en bra balun riskerar man att få mantelströmmar, ty antennen
kan vara av en obalanserad typ, till exempel Off-Center-Feed (OCF)/Windom, eller
för att den kopplar olika med miljön som träd och torn med mera.

Att undvika att det går gemensam ström, även kallad \emph{mantelström} kan
krävas av många olika anledningar, och det är viktigt dels för att få ut
energin där den ska, det vill säga radierat ut i luften på ett korrekt sätt,
men även av säkerhetsskäl så att inte utrustnings- eller personskada uppstår.

%
%
% Kapitel 5 Mottagare
\chapter{Mottagare}
\label{ch:mottagare}
\index{mottagare}

Energin i de elektromagnetiska magnetfält, som omger oss, alstrar högfrekventa
strömmar i alla metallföremål.
För att effektivt fånga upp dessa fält används antenner.
Fastän energin i fälten kan få en lampa att lysa om sändarantennen är
tillräckligt nära, så går det ändå inte att uppfatta den information som fälten
också kan innehålla.
För det behövs en radiomottagare för att dels förstärka de oftast mycket svaga
signalerna och dels uttyda informationen i dem.

Lyssna på amplitudmodulerade rundradiosändningar på mellanvåg kan man enklast
göra med hjälp av en detektormottagare.
Speciellt under dygnets mörka timmar vintertid kan man höra utländska sändare
med denna enkla mottagare, låt vara att det hörs mycket svagt.
I detektormottagaren omvandlas fältens energi till elektricitet och sedan till
ljud.
Så länge som ingen förstärkare används, förbrukas ingen annan energi än den som
fångas ur fälten -- radiovågorna.

% Avsnitt 5.1 Raka mottagare
\section{Raka mottagare}
\harecsection{\harec{a}{4.1.2}{4.1.2}}
\index{rak mottagare}
\index{mottagare!rak}

\subsection{Mottagare med kristalldetektor}
\index{kristalldetektor}
\index{mottagare!kristall}

\mediumplusbotfig{images/cropped_pdfs/bild_2_4-01.pdf}{Detektormottagare}{fig:bildII4-1}

Detektormottagaren består av ett mycket litet antal komponenter.
Princip och arbetssätt framgår av bild \ssaref{fig:bildII4-1}.
Samma princip används även i mer komplicerade mottagare, mätinstrument etc.
Antennkretsen består av antenn, jordtag och däremellan en induktor
(kopplingsspole), som överför energin från antennen till en resonanskrets.
Resonanskretsen används för att välja ut (selektera) en bärvåg med önskad
frekvens.
Bärvågen kan naturligtvis inte höras, men av kurvformen på bilden framgår
att bärvågen är amplitudmodulerad med en LF-signal.

För att återvinna LF-signalen utför man en så kallad demodulering med hjälp
av dioden.
Dioden klipper bort antingen de positiva eller negativa halvvågorna i den
mottagna signalen, beroende på hur dioden är vänd, polariserad.
Kondensatorn, som är kopplad parallellt över hörtelefonen, glättar de
högfrekventa spänningstopparna till ett amplitudmedelvärde (jämför med
entaktsblandare i Kapitel \ssaref{detektorer}).
Detta spänningsvärde varierar på ett sätt, som motsvarar den modulerande
spänning i sändaren som kommer av tal, musik etc.
Vi har nu demodulerat bärvågen, återställt LF-signalen och kan höra den i
mottagaren.

Signalspänningen över resonanskretsen är störst när dess resonansfrekvens och
antennströmmens frekvens är lika.

% \newpage
\mediumfig[0.45]{images/cropped_pdfs/bild_2_4-02.pdf}{Selektion i detektormottagare}{fig:bildII4-2}

Överst i bild \ssaref{fig:bildII4-2} ser man att mottagaren är inställd på
samma frekvens som sändare 2.
Även sändare 3 hörs eftersom bandbredden i resonanskretsen är stor.
Nederst i bilden är resonanskretsen inställd på sändare 3, men man hör
också sändare 2 och 4.

Bandbredden i resonanskretsen blir mindre ju mindre den belastas,
det vill säga dämpas.
I bild \ssaref{fig:bildII4-1} består belastningen av antennen (via
kopplingsspolen), hörtelefonen och avkopplingskondensatorn (via dioden).

Mindre belastning kan åstadkommas på två sätt; dels med ''lösare''
koppling mellan antennkrets och resonanskrets och dels med bättre
impedansanpassning mellan resonanskrets och diod.
Båda sätten tillämpas i bild \ssaref{fig:bildII4-3}.
Hur selektionen då förbättras visas i bild \ssaref{fig:bildII4-4}, vilket ska
jämföras med bild \ref{fig:bildII4-2}.

\mediumplusbotfig[0.77]{images/cropped_pdfs/bild_2_4-03.pdf}{Detektormottagare med LF-förstärkare}{fig:bildII4-3}

\mediumplustopfig{images/cropped_pdfs/bild_2_4-05.pdf}{Förbättrade HF-egenskaper i detektormottagare}{fig:bildII4-5}

\subsection{Detektormottagare med förstärkare}

Om man vill höra sändningarna över högtalare, behövs högre effekt än
vad som kan fångas upp genom antennen.
För ändamålet används en LF-förstärkare, som drivs av en annan energikälla,
till exempel ett batteri.
LF-förstärkaren kan även minska belastningen på resonanskretsen.

I bild \ssaref{fig:bildII4-3} har ett LF-lågpassfilter satts in efter
HF-avkopplingskondensatorn.
Det dämpar LF-signaler med högre frekvens än vad som behövs för god mottagning.

\subsubsection{Mottagare med bättre HF-egenskaper}
\label{mottagare_bättre_hf}

Ett sätt att minska bandbredden i en detektormottagare är att koppla
flera resonanskretsar med samma frekvens efter varandra, så som illustreras
i bild \ssaref{fig:bildII4-5}.
Den större dämpningen av fler kretsar kan kompenseras med en HF-förstärkare.

Sådana mottagare används för speciella ändamål, till exempel för övervakning
av en enda frekvens.
I sådana fall är resonanskretsarna fast avstämda.
Kanske utnyttjas till och med en kvartskristall som filter för den speciella
frekvensen.
Se bild \ssaref{fig:bildII4-6} om hög selektion.

\smallfig{images/cropped_pdfs/bild_2_4-04.pdf}{Förbättrad selektion}{fig:bildII4-4}

\smallfigpad{images/cropped_pdfs/bild_2_4-06.pdf}{Hög HF-selektion}{fig:bildII4-6}

\newpage

\smallfig{images/cropped_pdfs/bild_2_4-07.pdf}{CW i detektormottagare}{fig:bildII4-7}

\subsection{Detektormottagare och sändningsslag}

I huvudsak fungerar detektormottagaren endast vid amplitudmodulering.
Det innebär sändningsslagen A3E och A2A, det vill säga amplitudmodulerad
telefoni respektive tonmodulerad telegrafi, båda med full bärvåg.

Däremot fungerar detektormottagaren inte vid A1A, det vill säga telegrafi med
endast bärvåg.
En omodulerad bärvåg alstrar nämligen endast en likström i en
detektormottagare.
Vid nyckling hörs då endast knäppningar i hörtelefonen vid början och
slutet av teckendelarna, så som illustreras i bild \ssaref{fig:bildII4-7}.

Detektormottagaren fungerar inte heller vid J3E, det vill säga SSB och övriga
sändningsslag med undertryckt bärvåg.
Ljud såsom tal förvrängs nämligen kraftigt i en J3E-signal eftersom
bärvågskomponenten saknas.

I båda ovannämnda fall kan talet återställas med tillsats av en bärvåg.
Slutligen kan sändningsslag som innebär frekvens- och fasmodulering i
princip inte demoduleras med detektormottagare.

\newpage
\mediumtopfig{images/cropped_pdfs/bild_2_4-08.pdf}{Mottagare med direkt frekvensblandning}{fig:bildII4-8}

\subsection{Mottagare med direkt frekvensblandning}
\harecsection{\harec{a}{4.2.2}{4.2.2}, \harec{a}{4.3.2}{4.3.2}, \harec{a}{4.3.3}{4.3.3}, \harec{a}{4.3.6}{4.3.6}, \harec{a}{4.3.7}{4.3.7}}
\index{frekvensblandning}
\index{mottagare!direkt frekvensblandare}
\index{svävningston}
\index{BFO}
\index{Beat Frequency Oscillator (BFO)}
\index{A1A}
\index{J3E}

För att demodulera \emph{A1A} och \emph{J3E} i en rak mottagare --
detektormottagare måste den kompletteras med en oscillator som alstrar en
intern bärvåg.
Denna blandas med den mottagna signalen.
Det uppstår då en \emph{svävningston} (eng. \emph{beat frequency}).
Därav namnet \emph{Beat Frequency Oscillator (BFO)}.

Förfarandet har givit mottagartypen sitt namn -- direktblandad mottagare.

Ett sätt att komplettera den raka mottagaren med BFO framgår av bild
\ssaref{fig:bildII4-8}.
När BFO kopplas till och ställs in på en frekvens tillräckligt
nära mottagningsfrekvensen så uppstår en hörbar ton.

Demodulatordioden tillförs alltså två HF-signaler, dels den från antennen och
dels den från BFO.
Dessa båda signaler blandas i dioden och skillnadsfrekvensen är den hörbara
tonen.
Övriga blandningsprodukter dämpas av ett lågpassfilter.

\mediumtopfig{images/cropped_pdfs/bild_2_4-09.pdf}{Demodulering i mottagare med direkt frekvensomvandling -- CW-signaler}{fig:bildII4-9}

\mediumherefig{images/cropped_pdfs/bild_2_4-10.pdf}{Demodulering i mottagare med direkt frekvensomvandling -- SSB-signaler}{fig:bildII4-10}

\newpage
\subsubsection{Mottagning av telegrafi (CW)}
\harecsection{\harec{a}{4.2.1}{4.2.1}}
\index{CW}
\index{telegrafi}
\index{mottagare!CW}

Bild \ssaref{fig:bildII4-9} illustrerar blandning av CW-signal och BFO-signal
för ett antal fall.

Då BFO (VFO) är inställd på frekvensen \(f_2\) = \qty{1831}{\kilo\hertz} och den
mottagna signalen \(f_1\) har frekvensen \qty{1830}{\kilo\hertz} så hörs en
svävningston med frekvensen \qty{1000}{\hertz}.
Samma resultat fås om BFO ställs in på frekvensen \(f_2\) =
\qty{1829}{\kilo\hertz}.

Med BFO på frekvensen \(f_2\) = \qty{1830}{\kilo\hertz} hörs ingenting av
signalen \(f_1\) = \qty{1830}{\kilo\hertz} från sändaren.
Frekvensskillnaden är noll hertz.

Med BFO på frekvensen \(f_2\) = \qty{1849}{\kilo\hertz} hörs nästan ingenting av
signalen \(f_1\) = \qty{1830}{\kilo\hertz} från sändaren, då mixprodukten
\qty{19}{\kilo\hertz} knappt är hörbar.

De flesta föredrar en ton med frekvensen cirka \qty{800}{\hertz} för mottagning
av telegrafi.
BFO-frekvensen skulle i så fall ställas in på 1830,8 eller
\qty{1829,2}{\kilo\hertz} om \(f_1\) vore en telegrafisändning.

\mediumfig{images/cropped_pdfs/bild_2_4-11.pdf}{Selektionen i direktblandade mottagare}{fig:bildII4-11}

\subsubsection{Mottagning av J3E (SSB)}
\harecsection{\harec{a}{4.2.3}{4.2.3}}
\index{J3E}
\index{SSB}
\index{mottagare!SSB}

När en SSB-sändare sägs arbeta till exempel på frekvensen
\qty{1835}{\kilo\hertz}, så innebär det frekvensen på den bärvåg som undertryckts
i sändaren redan före utsändningen.

Vad som uppfattas av mottagarens ingångskretsar är alltså det utsända sidbandet.
När en SSB-signal demoduleras, så blandas den lokala bärvågen i mottagaren med
de mottagna modulationsprodukterna.
Vid blandningen uppstår blandningsprodukter som består dels av LF, dels av
andra högre frekvenser som dämpas i ett lågpassfilter.

Bild \ssaref{fig:bildII4-10} illustrerar en undertryckt bärvåg på
\qty{1835}{\kilo\hertz} och dess lägre sidband LSB som sträcker sig från
\qty{1832}{\kilo\hertz} till \qty{1834,7}{\kilo\hertz}.
Det demodulerade sidbandet sträcker sig från \qty{300}{\hertz} till
\qty{3}{\kilo\hertz}.

Inom amatörradio används för SSB det lägre sidbandet vid frekvenser
under \qty{10}{\mega\hertz}.
Med en frekvens av till exempel \qty{1835}{\kilo\hertz} och ett talspektrum av
\SIrange{300}{3000}{\hertz} kommer det lägre sidbandet att finnas mellan 1834,7
och \qty{1832,0}{\kilo\hertz}.
Tre modulerande frekvenser 300, 1000 och \qty{3000}{\hertz} visas på bilden.

Med en bärvågsfrekvens av \qty{1835}{\kilo\hertz} motsvaras de modulerande
frekvenserna av utfrekvenserna 1834,7; 1834 och \qty{1832}{\kilo\hertz}.
VFO ersätter SSB-sändarens bärvåg och ska ha samma frekvens --
\qty{1835}{\kilo\hertz} -- för att kunna återge 300, 1000 och \qty{3000}{\hertz}.

\newpage
\subsection{Selektionen i direktblandade mottagare}
\index{selektion}
\index{mottagare!selektion}
\label{selektion_direktblandade}

Direktblandade mottagare kan ses som en typ av detektormottagare, även
kallad ''rak'' mottagare.
Begreppet ''rak'' kommer av att HF-signalen från antennen passerar genom en
selektiv krets och en eventuell HF-förstärkare rakt fram till detektorn,
utan att frekvensen omvandlas.

I en detektormottagare är bandbredden oftast rätt stor.
Flera sändare hörs därför samtidigt.

På grund av att blandningsdioden i en direktblandad mottagare även fungerar
som AM-demodulator, så hörs faktiskt alla sändare inom förkretsens bandbredd.
Detta kan undvikas till en del genom att dioden, som fungerar som
entaktsblandare, byts till en mottaktsblandare eller ännu hellre till en
ringblandare.
Sådana blandare undertrycker ingångsfrekvenserna och släpper endast igenom
blandningsprodukter.
Bara den sändarsignal hörs då, vars frekvens tillsammans med VFO-frekvensen
ger blandningsprodukter, som faller inom LF-filtrets passband.
Mottagningsfrekvensen är VFO-frekvensen.
Resonanskretsen fungerar som en ställbar förselektor och LF-lågpassfiltret
ger den egentliga frekvensselektionen.

Vilka HF-signaler bildar blandningsprodukter med VFO-frekvensen och
vilka av dessa passerar sedan genom lågpassfiltret efter blandning ner
till LF-nivå?

\textbf{Exempel:}
En CW-sändare med en \qty{1830}{\kilo\hertz} frekvens tas emot genom att
mottagarens VFO ställs in på frekvensen \qty{1829,2}{\kilo\hertz}.
Från blandarutgången kommer då en ton med frekvensen \qty{800}{\hertz}.

Men sändaren är inte ensam på bandet.
Kommer till exempel SSB-sändaren på 1835, som moduleras med 300, 1000 och
\qty{3000}{\hertz}, att störa mottagningen?
Se bild \ssaref{fig:bildII4-11}.

Förkretsen i mottagaren är så bred att denna sändning passerar.
SSB-sändarens signalfrekvenser i det utsända sidbandet är 1834,7; 1834,0 och
\qty{1832}{\kilo\hertz}.
Dessa frekvenser blandas med mottagarens VFO-frekvens \qty{1829,2}{\kilo\hertz}
och alstrar blandningsprodukterna 5,5; 4,8 och \qty{2,8}{\kilo\hertz}.
Eftersom lågpassfiltret i mottagarens LF-förstärkare har bandbredden
\SIrange{0}{3000}{\hertz}, så kommer endast blandningsprodukten
\qty{2,8}{\kilo\hertz} att vara störande.
För att förbättra CW-mottagningen, så kan lågpassfiltret bytas ut mot ett
bandpassfilter, som endast släpper igenom ett smalt frekvensområde omkring
mittfrekvensen \qty{800}{\hertz}.

\mediumbotfig{images/cropped_pdfs/bild_2_4-12.pdf}{Passbandbredd och spegelfrekvenser i direktblandade mottagare}{fig:bildII4-12}
\subsection{Passband och spegelfrekvenser i direktblandare}
\index{passband}
\index{spegelfrekvenser}
\index{direktblandare}
\index{blandare!spegelfrekvenser}
\label{passband_spegelfrekvens}

I exemplet i förra stycket blev problemet med en störande ton löst med
ett bandpassfilter med annan frekvensgång.
Men vilka frekvenser kan tas emot genom ett lågpassfilter,
\SIrange{0}{3000}{\hertz}, om VFO-frekvensen är till exempel
\qty{1829,2}{\kilo\hertz}?

\textbf{Experiment:}
Ändra frekvensen på en CW-sändare långsamt från 1820 till
\qty{1840}{\kilo\hertz}.
Se bild \ssaref{fig:bildII4-12}

Sändarfrekvensen \qty{1820}{\kilo\hertz} hörs knappast eftersom
blandningsprodukten har frekvensen \qty{9,2}{\kilo\hertz} och den dämpas
kraftigt av lågpassfiltret.
Först när sändarfrekvensen är \qty{1826,2}{\kilo\hertz} hörs en tydlig ton med
frekvensen \qty{3000}{\hertz}.
Fortsätter man att ändra sändarfrekvensen, så sjunker tonens frekvens för att
bli noll (svävningsnoll), när sändarfrekvensen är lika med mottagarens
VFO-frekvens \qty{1829,2}{\kilo\hertz}.
Om man nu fortsätter med att höja frekvens, så blir blandningsproduktens
frekvens åter högre.
Vid sändarfrekvensen 1832,2 är den \qty{3000}{\hertz}.
Vid ännu högre sändarfrekvens dämpas blandningsprodukten igen av lågpassfiltret.

Slutsatsen av experimentet blir följande:
Vid en direktblandande mottagare med VFO-frekvensen \\ \qty{1829,2}{\kilo\hertz}
och ett \qty{3}{\kilo\hertz} lågpassfilter blir varje sändare hörbar, som har en
sändningsfrekvens mellan 1826,2 och \qty{1832,2}{\kilo\hertz}, varvid
blandningsprodukten har frekvenser från \qty{3000}{\hertz}, ner genom noll och
upp till \qty{3000}{\hertz} igen.

% \mediumfig{images/cropped_pdfs/bild_2_4-12.pdf}{Passbandbredd och spegelfrekvenser i direktblandade mottagare}{fig:bildII4-12}

Vår mottagare har bandbredden \qty{6}{\kilo\hertz}.
Varje annan sändare inom denna \emph{passbandbredd} kommer att höras eller --
om man så tycker -- störa mottagningen.

Tillbaka till exemplet med bandpassfiltret.
Vilka frekvenser kan tas emot med ett bandpassfilter \SIrange{700}{900}{\hertz}
(mittfrekvens \qty{800}{\hertz}), om VFO-frekvensen är \qty{1829,2}{\kilo\hertz}?
Jo, vi kan lyssna rätt ostört till vår CW-sändares \qty{800}{\hertz}-ton på
frekvensen \qty{1830}{\kilo\hertz}.
Ändå kan en annan sändare med frekvensen \qty{1828,4}{\kilo\hertz} störa
mottagningen därför att denna är \emph{spegelfrekvens} (eng. \emph{mirror
frequency}) till mottagningsfrekvensen \qty{1830}{\kilo\hertz}.
Vid VFO-frekvensen \qty{1829,2}{\kilo\hertz} uppstår en blandningsprodukt, inte
bara vid sändarfrekvensen \qty{1830}{\kilo\hertz} utan också vid
\qty{1828,2}{\kilo\hertz}.
Även denna andra sändarfrekvens, liksom nyttofrekvensen, släpps igenom
bandpassfiltret.

Spegelfrekvensmottagning är en principiell nackdel i mottagare med
direktblandning.
Nytto- och spegelfrekvens i det senaste exemplet ligger
% k7per: temporary fix until we move away from \SI to \qty
\qty{1,6}{\kilo\hertz} ($2\cdot 800$~\unit{\hertz}) ifrån varandra, alltså dubbla värdet
av bandpassfiltrets mittfrekvens.

Vid SSB-mottagning måste naturligtvis hela LF-området upp till
\qty{3000}{\hertz} kunna släppas igenom.
Utöver det önskade frekvensområdet \SIrange{1832}{1835}{\kilo\hertz}, kommer
även spegelfrekvenser i området \SIrange{1835}{1838}{\kilo\hertz} att kunna tas
emot.

Vid en LF-bandbredd av \qty{3}{\kilo\hertz} har således den direktblandade
mottagaren en bandbredd av \qty{6}{\kilo\hertz}, vilket är en god
avstämningsskärpa i jämförelse med den \qty{300}{\kilo\hertz} breda förkretsen.

\subsection{För- och nackdelar med direkt\-blandare}

Enkel uppbyggnad, men trots det en god känslighet och hygglig avstämningsskärpa.
VFO kan även användas till att styra en sändare.

Spegelfrekvensmottagning är tyvärr oundviklig.
Vidare kan signaler från starka sändare stråla in i den känsliga LF-förstärkaren
och orsaka LF-detektering, om mottagaren är otillräckligt skärmad.
Förbättrad isolering mellan antenn och VFO kan dock fås med en HF-förstärkare.

Entakts diodblandare är olämplig i en direktblandad mottagare.
Den tar emot alla sändare inom förkretsens passband och en del av VFO-signalen
kommer att strålas ut i antennen.
Ingen av dessa nackdelar finns i en mottakts- eller ringblandare.

% Avsnitt 5.2 Superheterodynmottagare
\section{Superheterodynmottagare}
\harecsection{\harec{a}{4.1.1a}{4.1.1a}, \harec{a}{4.3.1}{4.3.1}, \harec{a}{4.3.4}{4.3.4}}
\index{superheterodynmottagare}
\index{mottagare!superheterodyn}
\index{super}
\index{mottagare!super}
\label{superheterodynmottagaren}

Superheterodynprincipen ger mycket större möjligheter, när önskemålet
är en högselektiv mottagare för flera olika frekvenser.

Skillnaden mellan en direktblandad mottagare och en
\emph{superheterodynmottagare}, ofta bara kallad ''super'' eller
''superhetero'', är att blandningsprodukterna i direktblandaren blir till
LF direkt, medan de i supern först bildar en mellanfrekvenssignal MF,
vilken sedan demoduleras och LF-detekteras.

I det följande kallas superheterodynmottagaren enbart \emph{super}.
I supern blandas de mottagna signalerna med signalen från en VFO.
Före blandningen har HF-signalerna passerat ett selektivt försteg, som
dämpar spegelfrekvenser.
För att inte störa mottagningen placeras VFO-frekvensen alltid utanför
det frekvensband, där man vill ta emot signaler.

Alla mottagna signaler blandas med VFO-signalen.
Mottagningsfrekvensen är vanligen skillnaden mellan en fast så kallad
mellanfrekvens MF och VFO-frekvensen.
Mellanfrekvensen är egentligen mittfrekvensen i ett fast passband skapat av
ett antal filter.

\mediumfig{images/cropped_pdfs/bild_2_4-13.pdf}{Superheterodynmottagaren i princip}{fig:bildII4-13}

Bild~\ssaref{fig:bildII4-13} visar en mottagare med mellanfrekvensen
\qty{455}{\kilo\hertz}, som är vanlig i äldre mottagare.
MF-filtret kan i enklaste fall bestå av ömsesidigt magnetiskt kopplade
LC-resonanskretsar.
Bättre avstämningsskärpa fås med resonatorer av keramik eller kvarts eller med
hjälp av elektromekaniska resonatorer.

\textbf{Exempel:}
En sändning på frekvensen \qty{3600}{\kilo\hertz} ska tas emot.
Vi ställer då in VFO-frekvensen till \qty{4055}{\kilo\hertz}, eftersom
mellanfrekvensen är \(4055 - 3600 = \qty{455}{\kilo\hertz}\).
Den mottagna signalen hamnar då mitt i MF-filtrets passband.

Signaler på angränsande frekvenser tas också emot och alstrar
blandningsprodukter.
Med ett mellanfrekvensfilter med till exempel \qty{3}{\kilo\hertz} bandbredd
(\SIrange{453,5}{456,5}{\kilo\hertz}), kan signalfrekvenser mellan 3598,5 och
\qty{3601,5}{\kilo\hertz} passera genom filtret.
En signal med en närliggande frekvens till exempel \qty{3603}{\kilo\hertz}, och
blandad med den inställda VFO-frekvensen \qty{4055}{\kilo\hertz}, kommer att
alstra en skillnadsfrekvens av \qty{452}{\kilo\hertz}.
Denna signal ligger utanför filtrets passband och kommer att dämpas och når
inte detektorn.

VFO-signalen kan givetvis läggas under i stället för över mellanfrekvensen.

\textbf{Exempel:}
VFO-frekvensen \qty{3145}{\kilo\hertz} kan också användas för mottagning av
frekvensen \qty{3600}{\kilo\hertz}, om mellanfrekvensen är \qty{455}{\kilo\hertz}
(\(3600 - 455 = \qty{3145}{\kilo\hertz}\)).
Men för att undvika att eventuella övertoner från VFO-signalen blandas med
mottagna signaler är det lämpligt att placera VFO-frekvensen över
mottagningsfrekvensen.

Efter MF-filtren följer bland annat detektorer för olika sändningsslag samt
LF-förstärkare.
Jämför med bild~\ssaref{fig:bildII4-5} och \ssaref{fig:bildII4-6}.

\subsection{Dubbelsuperheterodynmottagare}
\harecsection{\harec{a}{4.1.1b}{4.1.1b}}
\index{dubbelsuperheterodynmottagare}
\index{mottagare!dubbelsuperheterodyn}

Det är svårt att bygga enkla mellanfrekvensfilter för höga frekvenser,
med liten bandbredd och branta flanker.
Det är fallet för en enkelsuper för kortvåg med en enda mellanfrekvens, till
exempel \qty{9}{\mega\hertz}.

En god närselektion på höga frekvenser är endast möjlig med relativt
dyrbara kristallfilter.
Däremot går det att få god närselektion med enklare medel på lägre frekvenser.

En dubbelsuper, det vill säga en super med dubbel frekvensomvandling,
möjliggör god både när- och förselektion, illustreras i bild
\ssaref{fig:bildII4-14}.
I 1:a blandaren blandas den mottagna signalen med signalen från en 1:a
oscillator (VFO) till en hög mellanfrekvens, till exempel 9 eller
\qty{10,7}{\mega\hertz}.

Därmed kan en god spegelfrekvensdämpning erhållas.
Första MF-filtret kan göras enklare och utan den höga selektivitet som hade
behövts i en enkelsuper.
1:a MF blir sedan blandad ytterligare en gång i 2:a blandaren till en 2:a MF,
till exempel \qty{455}{\kilo\hertz}.
För den andra blandningen används en fast oscillator.
Filtret i 2:a MF kan lättare utföras med en hög selektivitet, på grund av den
lägre frekvensen.

\mediumfig{images/cropped_pdfs/bild_2_4-14.pdf}{Dubbelsuperheteodynen i princip}{fig:bildII4-14}

\textbf{Exempel:}
Trots att MF-filtret inte är en enkel resonanskrets, kan ett ''Q-värde''
beräknas.
Vid en passbandbredd av \qty{6}{\kilo\hertz} och en centerfrekvens av
\qty{455}{\kilo\hertz} kan Q-värdet anses vara
%%
\[ Q = \frac{f_{\textit{res}}}{b} = \frac{455}{6} = 76 \]
%%
I ett MF-filter med centerfrekvensen \qty{9}{\mega\hertz} skulle det behövas ett
nära 20~gånger högre Q-värde för samma bandbredd \qty{6}{\kilo\hertz}
%%
\[ Q = \frac{f_{\textit{res}}}{b} = \frac{9000}{6} = 1500 \]
%%
Ett så högt Q-värde kan endast erhållas med kristallfilter.
För högre mottagningsfrekvenser räcker det, på grund av filterproblematiken,
oftast inte med en dubbel frekvensomvandling.
Om man antar en dubbelsupermottagare för VHF-området
\SIrange{144}{146}{\mega\hertz} enligt bilden, så skulle en 1:a MF med
frekvensen \qty{10,7}{\mega\hertz} inte vara tillräckligt hög.
Vid en mottagningsfrekvens av \qty{146}{\mega\hertz} är nämligen spegelfrekvensen
\(146 + (2 \cdot 10,7) = \qty{167,4}{\mega\hertz}\), alltså endast 1,15 gånger
mottagningsfrekvensen.
Det hade alltså varit lämpligt med en trippelsuper, det vill säga en trefaldig
frekvensomvandling, med en 1:a MF i frekvensområdet \qty{70}{\mega\hertz}.

% Avsnitt 5.3 Jämförelse superheterodyn
\input{koncept/chapter5-4}
% Avsnitt 5.4 Panoramamottagare
\section{Panoramamottagare}
\index{panoramamottagare}
\index{mottagare!panorama}
\index{spektrumanalysator}

%\mediumfig{images/cropped_pdfs/bild_2_4-15.pdf}{Panoramamottagare}{fig:bildII4-15}
\mediumtopfig{images/cropped_pdfs/bild_2_4-15.pdf}{Panoramamottagare}{fig:bildII4-15}
\smallfig[0.4]{images/cropped_pdfs/bild_2_4-17.pdf}{Signal- och svepspänningar}{fig:bildII4-17}
\mediumfig{images/cropped_pdfs/bild_2_4-16.pdf}{Anslutning av panoramamottagare till stationsmottagare}{fig:bildII4-16}

I en \emph{panoramamottagare} (eng. \emph{panorama receiver}) eller
\emph{spektrumanalysator} (eng. \emph{spectrum analyzer}) visas på en
oscilloskopskärm var det finns signaler inom ett frekvensband, som illustreras
i bild~\ssaref{fig:bildII4-15}.
En panoramamottagare är en superheterodyn.
Ofta implementeras de så att de sveper över mellanfrekvensen på en mottagare,
och hjälper därmed till att se vad som finns i angränsande del av bandet innan
det filtrerats för smalt.
Detta hjälper till att identifiera närliggande störkällor så väl som andra
potentiella stationer att köra QSO med.

%\smallfig{images/cropped_pdfs/bild_2_4-17.pdf}{Signal- och svepspänningar}{fig:bildII4-17}

Bild~\ssaref{fig:bildII4-17} illustrerar frekvenssvepet över spektrat.
Mottagaroscillatorn är en VCO (spänningsstyrd oscillator).
Dennas frekvens styrs av en sågtandformad likspänning, som stiger linjärt för
att snabbt falla tillbaka och återupprepas.
VCO sveper då över det önskade frekvensbandet med ett antal gånger per sekund.
Med samma sågtandspänning avlänkas strålen på skärmen utmed x-axeln.

Den mottagna signalen demoduleras och översätts till en likspänning
som skildrar de mottagna signalernas styrka.
Med denna likspänning avlänkas strålen på bildskärmen utmed y-axeln.
Strålens avstånd från x-axeln anger alltså den mottagna stationens styrka
och strålens läge utmed x-axeln anger var stationen ligger i det
frekvensområde som avsöks.
Beroende på hur stort frekvenssving som ges VCO, så kommer ett större eller
mindre frekvensområde att avsökas och visas på skärmen.
Området kan vara så brett som ett amatörband eller mer och ner till några
få \unit{\kilo\hertz}.

Utöver övervakning av ett frekvensband kan en panoramamottagare användas för
studium till exempel av signaler och sidofrekvenser som alstras i den egna
stationen.
För noggranna mätningar behövs emellertid ett hjälpmedel av högre kvalitet,
kallat \emph{spektrumanalysator}.
En sådan arbetar i grunden på samma sätt som en panoramamottagare.

En panoramamottagare kan anslutas till en mottagare för att studera
signalerna inom MF-passbandet, så som visas i bild~\ssaref{fig:bildII4-16}.
Då är mottagningsfrekvensen i bildskärmens mitt.
Stationerna under och över i frekvens visas till vänster respektive höger
om den egna frekvensen.

Vid ändrad mottagningsfrekvens blir denna fortfarande kvar mitt på skärmen.

%\mediumfig{images/cropped_pdfs/bild_2_4-16.pdf}{Anslutning av panoramamottagare till stationsmottagare}{fig:bildII4-16}

% Avsnitt 5.5 Mottagningskonvertern
\section{Mottagningskonvertern}
\index{mottagningskonverter}
\index{konverter}
\index{mottagare!konverter}

\mediumfig{images/cropped_pdfs/bild_2_4-18.pdf}{Mottagningskonverter UHF till KV}{fig:bildII4-18}

Konverter betyder i detta sammanhang frekvensomvandlare.
När det är önskvärt att flytta över alla signalerna inom ett helt frekvensområde
till ett annat, så används en mottagningskonverter där frekvensblandning och
frekvensfilter används, så som illustreras i bild~\ssaref{fig:bildII4-18}.

Konvertern fungerar som tillsats före en mottagare för att denna även
ska kunna användas inom ett annat frekvensområde.
I en konverter är oscillatorfrekvensen fast, medan avsökningen av
frekvensområdet görs med VFO i mottagaren.
Mellanfrekvensfiltret i mottagaren är så brett som hela det frekvensområde
som tas emot av konvertern och avsöks med mottagaren.

\begin{tcolorbox}[enhanced jigsaw,breakable,title=Exempel]
I en KV-mottagare för området \SIrange{28}{30}{\mega\hertz} vill man även kunna
lyssna i området \SIrange{432}{434}{\mega\hertz} (UHF).
Den i konvertern mottagna UHF-signalen förstärks för att sedan blandas med
\qty{404}{\mega\hertz}, en frekvens som multiplicerats upp från en
kristalloscillator (CO) i konvertern.
De blandningsprodukter som filtreras fram kommer att ligga inom området
\SIrange{28}{30}{\mega\hertz} och kan alltså avlyssnas i KV-mottagaren.
Övriga blandningsprodukter blir undertryckta i KV-mottagarens ingångskretsar.
Blandningsfrekvensen \qty{404}{\mega\hertz} i konvertern är beräknad på följande
sätt:

\noindent
Mittfrekvensen i UHF-bandet är
%%
\[\frac{432+434}{2} = \qty{433}{\mega\hertz} = f_1\]
%%
Mittfrekvensen i KV-mottagarens frekvensband är
%%
\[\frac{28 + 30}{2} = \qty{29}{\mega\hertz}\]
%%
Med vilken frekvens \(f_2\) måste \qty{433}{\mega\hertz} blandas för att erhålla
en blandningsprodukt av \qty{29}{\mega\hertz}?
\qty{29}{\mega\hertz} är mindre än \(f_1\) , alltså kan endast
skillnadsfrekvensen komma i fråga (vid summafrekvens skulle blandningsfrekvensen
bli högre än \qty{433}{\mega\hertz}).
Vid användning av skillnadsfrekvensen ges två möjligheter:
%%
\begin{gather*}
  \text{för }f_2 - f_1 = f_2 - 433 = \qty{29}{\mega\hertz}\text{ är }f_2 = \qty{462}{\mega\hertz} \\
  \text{för }f_1 - f_2 = 433 - f_2 = \qty{29}{\mega\hertz}\text{ är }f_2 = \qty{404}{\mega\hertz}
\end{gather*}
%%
Vi bestämmer oss för alternativet \qty{404}{\mega\hertz} av ett speciellt skäl.
Här motsvaras den högsta UHF-frekvensen \qty{434}{\mega\hertz} av \(434 - 404 =
\qty{30}{\mega\hertz}\) och den lägsta UHF-frekvensen \qty{432}{\mega\hertz} av
\(432 - 404 = \qty{28}{\mega\hertz}\).
På så sätt kan kHz-graderingen på KV-mottagarens skala användas direkt utan
omräkning.
\end{tcolorbox}

Fördelen med en konverter är att kostnaden för en sådan är låg jämfört
med den för en komplett mottagare för ett tillkommande band.
Förutsättningen är att en mottagare redan finns.
Nackdelen är att mottagaren inte samtidigt kan användas för sin
ordinarie funktion.

% Avsnitt 5.6 Transvertern
\section{Transvertern}
\index{transverter}
\index{mottagare!transverter}

\mediumfig{images/cropped_pdfs/bild_2_4-19.pdf}{Transverter mellan UHF och KV}{fig:bildII4-19}

En transverter (\emph{trans}sceiver-con\emph{verter}), är en kombinerad
frekvensomvandlare för både sändning och mottagning, som illustreras i
bild~\ssaref{fig:bildII4-19}.
Den förflyttar både mottagnings- och sändningssignaler mellan två
frekvensområden.

Transvertern är ett bra exempel på hur samma teknik kan användas både
i mottagare och sändare.
Om till exempel en KV-transceiver redan finns, kan både mottagning och sändning
ordnas även på andra band med en transverter som tillsats.

\textbf{Exempel:}
En konverter förflyttar de mottagna UHF-signalerna till kortvågsområdet.
Som huvudmottagare används en KV-transceiver i mottagningsläge.
Konvertern kan utökas till att även fungera vid sändning och kallas då
transverter.
Med KV-transceivern i sändningsläge flyttas dess signaler till UHF-området
genom blandning i transvertern av KV-signalen och en multiplicerad signal
från en lokaloscillator (LO).
Den önskade blandningsprodukten i UHF-området filtreras fram och förstärks i
efterföljande driv- och slutsteg.
Samma frekvensmultipliceringskedja efter kristalloscillatorn CO kan användas
för sändning och mottagning.

Fördelen med en transverter är att kostnaden för en sådan är låg jämfört med
den för en komplett transceiver även för det tillkommande bandet.
Förutsättningen är att en transceiver för något band redan finns.
Nackdelen är att den befintliga transceivern inte samtidigt kan användas på
några andra frekvenser än de som används för tillfället.

% Avsnitt 5.7 AGC
\section[AGC]{Automatisk förstärknings\-reglering (AGC)}
\harecsection{\harec{a}{4.3.8}{4.3.8}}
\index{automatisk förstärkningreglering}
\index{AGC}
\index{mottagare!automatisk förstärkningsreglering}
\index{mottagare!AGC}
\label{AGC}

För att mottagaren ska fungera bra för såväl mycket svaga som för mycket starka
insignaler behövs en förstärkningsreglering i signalvägen genom mottagaren.
Signalspänningen på mottagaringången kan vara från delar av en mikrovolt upp
till över 100~millivolt -- ett spänningsförhållande på 1:100~000.
Det motsvarar mer än nio S-enheter, vilket är ett mått på signalstyrkan
(se bilaga~\ssaref{s-enhet}).

Vid mottagning av en stark signal är det inte tillräckligt med att
bara minska LF-förstärkningen.
Förstärkarstegen i HF- och MF-delen blir ändå överstyrda av den starka
insignalen och utsignalen förvrängs om inget ytterligare görs.
Det är därför nödvändigt att minska förstärkningen även i HF- och
MF-förstärkarstegen, ju mer desto starkare insignalen är.
Som hjälpmedel finns oftast ett reglage för HF-förstärkningen (RF gain), och
därutöver en \emph{automatisk förstärkningsreglering} (eng. \emph{Automatic
Gain Control (AGC)}).

En mottagare med god reglering kan arbeta med signalstyrkor mellan mikrovolt
och volt.
Beroende hur den mottagna signalen är modulerad (sändningsslaget), sker AGC
på olika sätt.

Både vid AM och SSB finns informationen i sidbanden.
HF- och MF-stegen måste därför arbeta i det linjära området och de får inte
överstyras.
Förstärkningen i mottagaren måste alltså regleras med hänsyn till detta.

\subsection{AGC vid AM (A3E)}
\index{amplitudmodulation}
\index{AGC}
\index{A3E}

%\largefig{images/cropped_pdfs/bild_2_4-20.pdf}{AGC vid AM-mottagning med superheterodynmottagare}{fig:bildII4-20}
\pagefig{images/cropped_pdfs/bild_2_4-20.pdf}{AGC vid AM-mottagning med superheterodynmottagare}{fig:bildII4-20}
\pagefig{images/cropped_pdfs/bild_2_4-21.pdf}{AGC vid SSB- och CW-mottagning med superheterodynmottagare}{fig:bildII4-21}

Den likspänning som uppstår vid demoduleringen av MF-signalen i en
AM-mottagare används till förstärkningsreglering -- AGC, så som illustreras
i bild~\ssaref{fig:bildII4-20}.
Den LF-spänning som är överlagrad på likspänningen undertrycks i ett
RC-lågpassfilter.
Likspänningen över kondensatorn följer variationerna i den mottagna signalens
styrka med en tidskonstant av cirka 0,1~sekunder.
Likspänningen blir alltså inte påverkad av de betydligt snabbare
spänningsändringarna som kommer av moduleringen.

En stark bärvågssignal alstrar en hög likspänning och en svag signal
en låg likspänning, oberoende av moduleringen.
Denna likspänning återförs till de framförliggande HF- och MF-förstärkarstegen,
vilka är gjorda så att en hög reglerspänning sänker förstärkningen, medan en
låg spänning tillåter en hög förstärkning.

På så sätt kommer signalstyrkan efter de reglerade stegen att hållas
konstant samtidigt som mottagarens ingång inte överstyrs.

Den likspänning som filtrerats fram från detektorn kallas reglerspänning eller
AGC-spänning.
Diodens polarisering är inte viktig för att få ut LF vid demoduleringen, men
däremot för att få rätt polaritet på AGC-spänningen.
I de flesta mottagare används negativ AGC-spänning.

\subsection{AGC vid SSB (J3E)}
\index{SSB}
\index{AGC}

I de flesta utföranden lämnar produktdetektorn en växelspänning utan
överlagrad likspänning.
Reglerspänningen alstras därför genom likriktning av MF-spänningen med hjälp
av en separat demoduleringsdiod eller genom likriktning av LF-växelspänningen,
så som illustreras i bild~\ssaref{fig:bildII4-21}.

Vid SSB alstras det ju ingen MF-spänning under talpauserna, eftersom
ingen bärvåg tas emot då.
Tidskonstanten på lågpassfiltret för reglerspänningen måste därför vara längre
än vid AM, det vill säga 0,5 till 2~sekunder.
En alltför snabb tillbakagång i reglerspänningen på grund av en för kort
tidskonstant skulle leda till mer mottagningsbrus i talpauserna.
I moderna mottagare finns det ofta en omkopplare eller justering för olika
tidskonstanter.

%\newpage % Layout
\subsection{AGC vid CW (A1A)}
\index{CW}
\index{AGC}

Metoden för att alstra AGC-spänning är samma vid CW och SSB.

\subsection{AGC vid FM (F3E)}
\index{frekvensmodulation}
\index{AGC}

FM-mottagare brukar inte regleras av den anledningen att det vid FM
inte finns någon information i signalamplituden, utan finns i stället
i frekvensvariationerna i signalen.

%\mediumfig{images/cropped_pdfs/bild_2_4-21.pdf}{AGC vid SSB- och CW-mottagning med superheterodynmottagare}{fig:bildII4-21}

Helt avsiktligt läggs därför förstärkningen i mottagaren så, att en
sinussignal blir en kantvåg på grund av överstyrning i förstärkarstegen.
Ett eller flera sådana amplitudbegränsande steg, även kallat ''limiter'',
placeras före demoduleringssteget.
Störningar av amplitudvariationer kommer då att klippas bort och inte störa
mottagningen.

Störande signaler inom nyttobandbredden har dock ingen större inverkan
så länge som den önskade signalens styrka är en halv s-enhet större än
den störande signalens styrka.
Likaså försvinner det störande bruset vid mottagning av en FM-sändare mycket
snabbt över denna signalnivå.
Amplitudmodulerade störningar, som till exempel de från tändgnistor i
förbränningsmotorer, har liten påverkan vid tillräckligt stark nyttosignal.

% \newpage % layout
\subsection{Signalstyrkemätare (S-meter)}
\harecsection{\harec{a}{4.3.9}{4.3.9}}
\index{signalstyrkemätare}
\index{mottagare!signalstyrkemätare}
\index{S-meter}
\index{mottagare!S-meter}

AGC-spänningen i en mottagare för AM, CW och SSB kan även styra en
S-meter, som ger besked om hur stark signalen in i mottagaren är.
(Se bilaga \ssaref{s-enhet}.)

\subsection{Brusspärr}
\harecsection{\harec{a}{4.3.10}{4.3.10}}
\index{brusspärr}
\index{mottagare!brusspärr}
\index{squelch}
\index{mottagare!squelch}
\index{repeater}
\index{bärvågsstyrd}
\label{brusspärr}

I en FM-mottagare hörs bara brus när det inte kommer in en tillräckligt stark
signal.
Bruset är genomträngande eftersom FM-mottagare arbetar med hög förstärkning.
En \emph{brusspärr} (eng. \emph{squelch}) är en anordning som stoppar
signalerna till LF-förstärkaren när signalerna ej uppnår en viss nivå.
På så sätt slipper man att höra på bruset.
I mottagare för flera sändningsslag och därför även AGC kan denna funktion
styra brusspärren, men i en ren FM-mottagare arbetar MF-förstärkarna utan AGC.
I det fallet behövs någon annan anordning för att skilja mellan en modulerad
signal och brus.
Ofta finns ett reglage (squelch) för hur stark signalen ska vara innan spärren
öppnar.

För en \emph{repeater} används brusspärren även för att starta sändaren när den
är \emph{bärvågsstyrd}, det vill säga att man låter signalstyrkan på den
mottagna signalen slå på även sändaren om den inkomna signalen är tillräckligt
stark.
Nuförtiden anses bärvågsstyrd repeater inte lämplig, då den kan okynnesöppna av
störningar, varvid störningen förstärks.
Istället bör tonöppning eller subton användas.

\subsection{Tonöppning}
\index{tonöppning}
\index{1750 Hz}
\label{tonöppning}

Ett alternativ till bärvågsbaserad brusspärr är \emph{tonöppning} (eng.
\emph{1750 Hz tone-burst}) som i allmänhet är att man lägger ut en
\qty{1750}{\hertz} tonskur för att öppna en repeater.
En del äldre amatörer har lärt sig att vissla rätt ton för att öpnna repeatern
då de en gång i tiden inte hade handapparater med inbyggd 1750~Hz-knapp.

\subsection{Subton}
\index{subton}
\index{CTCSS}
\label{subton}

Ett problem med bärvågsöppning och \qty{1750}{\hertz} tonöppning är när
närliggande repeatrar med samma in- och utfrekvens har sådana konventioner att
bägge hör handapparatens öppningston, då kommer bägge repeatrarna riskera att
öppna och då störa varandras sändning.

%% k7per
% \newpage % layout
Ett alternativ är därför att använda \emph{subton} (eng. \emph{subtone}) som är
en frekvens under \qty{300}{\hertz}, i allmänhet 60 till \qty{250}{\hertz}.
Ett existerande subton-system är \emph{Continuous Tone Coded Squelch System
  (CTCSS)} där sändaren lägger ut en kontinuerlig subton.
Mottagaren detekterar en vald subton, och enbart när den tonen har tillräcklig
styrka så öppnar squelchen och håller den öppen så länge som det finns en
subton.

För repeatrar så används det för att även starta sändaren, så därför måste man
välja den subton som repeaterns mottagare är inställd på för att öppna
repeatern.
Det förekommer också att repeatern i sin tur skickar ut subton, oftast den som
används för att öppna den.
Detta öppnar i sin tur handapparaterna för den valda repeatern.
Detta kan också användas av handapparater för att hitta repeatrar och lära sig
den subton som öppnar den.

Skulle flera repeatrar höra samma källa så kan därför olika subton användas
för att undvika öppning av fel repeater.
SSA:s repeateransvarig har tilldelat CTCSS subtoner för de olika regionerna,
och inom varje region finns flera subtoner för att kunna separera inom
regionen.

\subsection{DTMF}
\index{DTMF}

Ett system för att skicka styrkommandon och även öppna repeatrar är
\emph{DTMF (Dual Tone Multi Frequency)}, som bygger på principen att man
skickar två samtidiga toner.
De två tonerna väljs som en av fyra i två olika serier.
Detta ger \(4 \cdot 4 = 16\) kombinationer, varav 10 representerar siffrorna 0--9.
DTMF kommer ursprungligen från telefonisystem, men fungerar även bra över
vanliga radiokanaler.
DTMF kan användas för att styra repeatrar, som att slå på och stänga av dem,
samt styra andra egenskaper.

% Avsnitt 5.8 Egenskaper i mottagare
\section{Egenskaper i mottagare}
\harecsection{\harec{a}{4.4}{4.4}}
\index{mottagare!egenskaper}

\subsection{Närliggande kanaler}
\harecsection{\harec{a}{4.4.1}{4.4.1}}

Närliggande kanaler kan skapa störningar när de läcker in.
Därför gäller det att mottagaren kan undertrycka dem, även när de är starkare
än den valda kanalen, så att man får så god läsbarhet på den valda kanalen.
Närliggande kanaler kan därför anses vara störkällor.
Moderna mottagare medger att flytta både över och undre gräns för att
undertrycka allt för närgående kanaler.
Även begreppet roofing filter förekommer för filter som hjälper till att
filtrera med branta flanker och god undertryckningsförmåga.
Detta är en del av många att ha goda så kallade stor-signal-egenskaper.

% \newpage % layout
\subsection{Selektivitet}
\harecsection{\harec{a}{4.4.2}{4.4.2}}
\index{selektivitet}
\index{mottagare!selektivitet}
\index{förselektering}
\index{spegelfrekvenser}
\label{selektivitet}

Med \emph{selektivitet} (eng. \emph{selectivity}) menas en mottagares förmåga
att skilja ut önskade signaler och undertrycka övriga.
Summariskt beskrivet kallas avståndet mellan yttergränserna för det önskade
frekvensområdet för bandbredd.

När det gäller superheterodynmottagare finns två selektivitetsbegrepp:
\begin{itemize}
  \item Det ena är \emph{förselektering} för att dämpa de \emph{spegelfrekvenser} som
uppstår i samband med blandning av mottagna signaler och oscillatorfrekvenser i
mottagaren.
  \item Det andra är selektiviteten i en superheterodynmottagares MF-steg som används
för att utskilja den önskade signalen efter blandningsförloppen.
\end{itemize}

\subsection{Frekvensstabilitet}
\harecsection{\harec{a}{4.4.4}{4.4.4}}
\index{frekvensstabilitet}
\index{temperaturkompenserad}
\index{TCXO}
\index{ugnskompenserad}
\index{OCXO}
\index{lokaloscillatorfrekvens}
\index{PLL}
\index{DDS}
\index{Direct Digital Synthesis (DDS)}

\emph{Frekvensstabilitet} (eng. \emph{frequency stability}) är viktigt för
mottagare är viktigt för att kunna hitta sändare på angiven frekvens fort,
kunna stanna på den frekvensen utan att glida ifrån signalen och dessutom
undvika att glida in i närliggande signaler som stör.

Frekvensstabilitet tillgodoses i moderna mottagare med kristalloscillatorer,
och man kan ofta köpa till \emph{temperaturkompenserad} (\emph{TCXO}) eller
\emph{ungskompenserad} (\emph{OCXO}) kristalloscillator för att få en högre
frekvensstabilitet i hela mottagaren.

För att få bäst nytta så genereras alla \emph{lokaloscillatorfrekvenser} låsta
till samma kristalloscillator, något som med modern PLL och DDS teknik blivit
inte bara möjligt utan både kompakt, billigt och med hög prestanda.

%\pagefig[0.9]{images/cropped_pdfs/bild_2_4-22.pdf}{Enkelsuper med låg MF och ingen förselektion}{fig:bildII4-22}
%\pagefig[0.9]{images/cropped_pdfs/bild_2_4-23.pdf}{Enkelsuper med låg MF och med förselektion}{fig:bildII4-23}
%\pagefig[0.5]{images/cropped_pdfs/bild_2_4-24.pdf}{Enkelsuper med hög MF och med förselektion}{fig:bildII4-24}
%\pagefig{images/cropped_pdfs/bild_2_4-25.pdf}{Samtidig för- och närselektion i superheterodynmottagare}{fig:bildII4-25}
%\pagefig{images/cropped_pdfs/bild_2_4-26.pdf}{MF-bandbredd vid AM (A3E)}{fig:bildII4-26}
%\pagefig{images/cropped_pdfs/bild_2_4-27.pdf}{MF-Bandbredd och passbandtuning vid SSB (J3E)}{fig:bildII4-27}

\newpage
\subsection{Spegelfrekvensproblemet vid mottagning}
\harecsection{\harec{a}{4.4.5}{4.4.5}}
\index{spegelfrekvenser}
\index{mottagare!spegelfrekvenser}
\index{närselektion}
\index{mottagare!närselektion}
\index{förselektion}
\index{mottagare!förselektion}
\label{spegelfrekvens_mottagare}

\mediumplustopfig{images/cropped_pdfs/bild_2_4-22.pdf}{Enkelsuper med låg MF och ingen förselektion}{fig:bildII4-22}

\textbf{Exempel:}
I bild \ssaref{fig:bildII4-22} ska en sändning på \qty{3600}{\kilo\hertz} ska tas
emot och VFO-frekvensen är \qty{4055}{\kilo\hertz}.
Mellanfrekvensfiltret undertrycker sändningar på så närliggande frekvenser
som till exempel 3603 och \qty{3597}{\kilo\hertz}.
Denna egenskap kallas för \emph{närselektion}.

Men tyvärr kan en sändning på så avlägsen frekvens som \qty{4510}{\kilo\hertz}
ändå störa mottagningen, den goda närselektionen till trots.
Avståndet från \qty{4510}{\kilo\hertz} till vår mottagningsfrekvens
\qty{3600}{\kilo\hertz} är \qty{910}{\kilo\hertz}.
Frekvensen \qty{4510}{\kilo\hertz} och VFO-signalen bildar också en
blandningsprodukt, som har frekvensen \qty{455}{\kilo\hertz}.
Vid en VFO-frekvens av \qty{4055}{\kilo\hertz} och en mottagningsfrekvens av
\qty{3600}{\kilo\hertz} benämns \qty{4510}{\kilo\hertz} som
\emph{spegelfrekvensen}.
Avståndet mellan spegelfrekvens och mottagningsfrekvens är dubbla värdet av
mellanfrekvensen -- i detta exempel \(2 \cdot \qty{455}{\kilo\hertz} =
\qty{910}{\kilo\hertz}\).

%% k7per
% \newpage % layout
Signaler på mottagningsfrekvensen och spegelfrekvensen alstrar båda
blandningsprodukter med VFO-frekvensen, som har mellanfrekvensens värde.
Mellanfrekvensfiltret kan därför inte undertrycka en främmande signal på
spegelfrekvensen.

\mediumplustopfig{images/cropped_pdfs/bild_2_4-23.pdf}{Enkelsuper med låg MF och med förselektion}{fig:bildII4-23}

Däremot kan en mottagaringång med \emph{förselektering} (eng.
\emph{preselection}) undertrycka den.
I bild \ssaref{fig:bildII4-23} finns en selektiv krets före blandaren släpper
igenom ett smalt frekvensband med mittfrekvensen \qty{3600}{\kilo\hertz}, men
dämpar till exempel frekvensen \qty{4510}{\kilo\hertz} på grund av den stora
frekvensskillnaden.
En förselektion har alltså tillförts som komplement till den närselektion som
erhålls med mellanfrekvensfiltret.

\mediumminustopfig{images/cropped_pdfs/bild_2_4-24.pdf}{Enkelsuper med hög MF och med förselektion}{fig:bildII4-24}

Ju längre ifrån varandra nyttofrekvens och spegelfrekvens ligger, desto bättre
är förselektionen.
Med en mellanfrekvens av \qty{455}{\kilo\hertz} är alltså detta avstånd
\qty{910}{\kilo\hertz}.
I långvågs- och mellanvågsområdet är det tillräckligt för att man med enkla
medel ska kunna skapa tillräckligt selektiva filter.

\textbf{Exempel:}
Vid den högsta mottagningsfrekvensen på mellanvåg \qty{1605}{\kilo\hertz} är
spegelfrekvensen \qty{2515}{\kilo\hertz}, som ligger 1,57 gånger högre i frekvens
och med ett avstånd av \qty{910}{\kilo\hertz}.
I kortvågsområdet dämpas inte en spegelfrekvens på avståndet
\qty{910}{\kilo\hertz} tillräckligt kraftigt.
Vid den högsta mottagningsfrekvensen på kortvåg \qty{30}{\mega\hertz} ligger
nämligen spegelfrekvensen \qty{30,910}{\mega\hertz} endast 1,03 gånger högre i
frekvens.
Med antagandet, att förselektionskretsen har ett Q-värde av 30, blir
bandbredden \qty{53,5}{\kilo\hertz} vid frekvensen \qty{1605}{\kilo\hertz}.

Med samma Q-värde blir bandbredden \qty{1000}{\kilo\hertz} vid frekvensen
\qty{30}{\mega\hertz}, vilket innebär att förkretsen inte längre kan dämpa så
närliggande spegelfrekvenser på ett effektivt sätt.

\newpage
I mottagare för högre frekvenser används därför högre mellanfrekvens
för att öka avståndet till spegelfrekvensen, som illustreras i bild
\ssaref{fig:bildII4-24}.
I moderna kortvågsmottagare är det vanligt med en mellanfrekvens av
\qty{9}{\mega\hertz} eller högre.
Vid en mottagningsfrekvens av \qty{30}{\mega\hertz} och en mellanfrekvens av
\qty{9}{\mega\hertz} är spegelfrekvensen \qty{48}{\mega\hertz}, vilket är
1,6~gånger mottagningsfrekvensen.
Detta möjliggör förselektionsfilter med tillräcklig dämpning av
spegelfrekvensen.

\mediumminustopfig{images/cropped_pdfs/bild_2_4-25.pdf}{Samtidig för- och närselektion i superheterodynmottagare}{fig:bildII4-25}

Bilden~\ssaref{fig:bildII4-25} visar hur när- och förselektion kompletterar
varandra i ett frekvensspektrum.
Märk, att passbandbredden \(b\) i förselektionskretsen anger avståndet mellan
de frekvenser där signalamplituden dämpats till 70~\% av toppvärdet.
I exemplet här ovan har antagits att förkretsen för kortvågsmottagning har
samma Q-värde som förkretsen för mellanvågsmottagning.

\newpage
Vid högre frekvenser, i VHF- och UHF-området, kan inte önskat Q-värde
erhållas i sådana kretsar som används i KV-området och lägre.
Andra lösningar blir då nödvändiga, till exempel kavitetsfilter och helixfilter.

\newpage % layout
\subsubsection{MF-bandbredd vid AM (A3E)}
\index{amplitudmodulation}
\index{mottagare!AM}
\index{MF-bandbredd}
\index{mottagare!MF-bandbredd}

\mediumfig{images/cropped_pdfs/bild_2_4-26.pdf}{MF-bandbredd vid AM (A3E)}{fig:bildII4-26}

Bild~\ssaref{fig:bildII4-26} visar en amplitudmodulerad signals frekvensspektrum
består av bärvågen och två sidfrekvenser -- eller sidband om sidfrekvenserna
är många.

Bandbredden i MF-kretsarna måste vara minst så stor att sidofrekvenserna
längst bort från bärvågen kan passera.
Dessa frekvenser motsvarar de högsta modulerande tonerna.
Vid rundradiosändningar på mellanvåg utsänds alla frekvenser upp till
\qty{4,5}{\kilo\hertz}.
Detta motsvarar en bandbredd av \qty{9}{\kilo\hertz}.
För enbart talöverföring är en bandbredd av \qty{6}{\kilo\hertz} tillräcklig,
vilket motsvarar en LF-gränsfrekvens av \qty{3}{\kilo\hertz}.

Ett för smalt MF-filter skär bort de yttre delarna av sidbanden.
LF-signalerna kommer då att förlora de höga tonerna (diskanten).
Om däremot filtret är för brett, kommer närliggande utsändningar också att
höras.

I vissa mottagare kan MF-bandbredden anpassas till förhållandena.
Det är alltså en fråga om en kompromiss mellan bättre ljudkvalitet och
mindre störd mottagning.

\subsubsection{MF-bandbredd vid SSB (J3E)}
\index{SSB}
\index{mottagare!SSB}
\index{MF-bandbredd}
\index{mottagare!MF-bandbredd}
\index{snedstämning}
\index{MF-skift}
\index{passband-tuning}

\mediumfig{images/cropped_pdfs/bild_2_4-27.pdf}{MF-Bandbredd och passbandtuning vid SSB (J3E)}{fig:bildII4-27}

Mellanfrekvensfiltret för SSB-mottagning ska endast släppa igenom
ett av de två sidbanden, så som illustreras i bild~\ssaref{fig:bildII4-27},
vars bredd är skillnaden mellan högsta och lägsta överförda LF-frekvens.
Inom amatörradio är detta \(\qty{3}{\kilo\hertz} - \qty{0,3}{\kilo\hertz} =
\qty{2,7}{\kilo\hertz}\), alltså något mindre än hälften av bandbredden vid AM.

Ett alltför brett MF-filter skulle också släppa igenom oönskade
signaler från angränsande frekvenser.
Å andra sidan skulle ett för smalt MF-filter skära bort signaler i det
önskade frekvensregistret och försvåra mottagningen.
Smala filter kan å andra sidan utnyttjas för att dämpa signaler, till exempel från
en för nära liggande sändare eller en som har för stor bandbredd.

När närliggande sändare stör mottagningen ges följande möjligheter:
\begin{itemize}
\item \emph{Snedstämning.}
  Att göra en liten snedavstämning, uppåt eller nedåt i frekvens.
  Därigenom ändras frekvensläget på det mottagna talet, men vid små
  frekvensavvikelser blir förvrängningen liten.
  Läsligheten blir sämre, men mottagningen på det hela taget bättre.

\item \emph{MF-skift.}
  Som just beskrivits kan en liten snedavstämning göras.
  I vissa mottagare är det ordnat så att också BFO-frekvensen kan förskjutas
  så att frekvensläget på talet blir återställt igen.
  Därmed blir MF-passbandet skenbart förflyttat uppåt eller nedåt i frekvens
  (MF-skift, IF-shift).
  Det verkliga frekvensläget mellan nyttosignal och BFO behålls.
  I alla händelser blir basen eller diskanten på nyttosignalen avskuren,
  beroende på var denna ligger i frekvens.

% \newpage % layout
\item \emph{Passband-tuning.}
  Om det finns störande sändare både över och under i frekvens, går det inte
  att skära bort störningarna med ett enkelt MF-skift, eftersom antingen den
  ena eller den andra störande sändaren ändå skulle höras.
  För det fallet erbjuder några moderna mottagare möjligheten att flytta
  MF-passbandets övre och undre frekvensgräns oberoende av varandra (bandpass
  tuning m.m.).
  Detta förutsätter, att mottagaren är en trippelsuper med branta filter i
  varje MF-steg.
  Vidare måste VFO, 1:a BFO och 2:a BFO kunna ställas in var för sig.
  Frekvensläget på MF I och/eller MF II kan då förskjutas över respektive
  filters passband, oberoende av varandra.
  Därigenom uppstår skenbart effekten att filterkurvorna skjuts emot varandra.
  Samma effekt skulle fås om kristallfiltren gick att avstämma, vilket ju inte
  är möjligt.
  Moderna SDR mottagare kan göra motsvarande genom att justera de digitala
  MF-filtren.
\end{itemize}

\subsubsection{MF-bandbredd vid CW (A1A)}
\index{CW}
\index{mottagare!CW}
\index{MF-bandbredd}
\index{mottagare!MF-bandbredd}

En CW-signal har som bekant inte bandbredden noll hertz, utan det handlar i
grunden om en amplitudmodulerad signal.
Vid en nycklingshastighet av 60 tecken per minut är bandbredden cirka
\qty{100}{\hertz} och vid i 120 tecken per minut den dubbla, cirka
\qty{200}{\hertz}.

I vissa mottagare används ett SSB-filter även för mottagning av CW.
En vanlig bandbredd på ett SSB-filter är \qty{2,7}{\kilo\hertz} och då kommer
även stationer på närliggande frekvenser att höras, detta illustreras i bild
\ssaref{fig:bildII4-28}.
Låt vara att de flesta av dessa stationer hörs med olika frekvens.

\mediumfig{images/cropped_pdfs/bild_2_4-28.pdf}{Olika MF-bandbredder vid CW (A1A)}{fig:bildII4-28}

Fler än 20 CW-stationer får plats inom en bandbredd motsvarande en SSB-kanal.
Den mänskliga hjärnan, kan med någon övning koncentrera sig på en av dessa
signaler medan övriga uppfattas som störande.

Det tidigare nämnda LF-bandpassfiltret skulle emellertid åstadkomma en
bättre selektion och bekvämare avlyssning.
Men om en annan station inom passbandet är mycket starkare än den station
som är av intresse, då blir MF-förstärkaren antingen överstyrd av den
starkare signalen eller AGC reglerar ner förstärkningen så att den svagare
signalen inte längre kan höras trots det smala LF-filtret.
Selektionen i en mottagare bör därför sitta ''så långt fram som möjligt''.
I det skildrade exemplet skulle ett smalt filter i MF vara till bättre nytta
vid CW-mottagning.
Bandbredden på ett sådant filter är \SIrange{250}{500}{\hertz}, således endast
något bredare än CW-signalen.

Med ett ännu smalare CW-filter kan, på grund av bristande frekvensstabilitet hos
sändare och/eller mottagare, svårigheter uppstå att finna den önskade signalen.
Välutrustade mottagare har passband-tuning även för CW, steglös
bandbreddsreglering eller stegvis valbara filterbandbredder.
Då kan mottagaren ställas in på den önskade signalen med en stor bandbredd
som därefter minskas.
För mottagning av RTTY (radiofjärrskrift) med \qty{170}{\hertz} skift mellan de
två frekvenserna, kan ett \qty{500}{\hertz}-filter användas.
Smalare filter går däremot inte så bra.

\subsubsection{Bandbredd vid FM (F3E)}
\index{frekvensmodulation}
\index{mottagare!FM}
\index{MF-bandbredd}
\index{mottagare!MF-bandbredd}
\index{frekvensdeviation}
\index{bandbredd}
\label{bandbredd_fm}

%% k7per: modulerande LF-moduleringsfrekvensen -> LF-moduleringsfrekvensen
En FM-sändare med frekvensdeviationen \(\Delta f_{max}\) och högsta
modulerande LF-moduleringsfrekvensen \(f_{LF_{max}}\) har bandbredden
%%
\[ b = 2 \cdot (\Delta f_{max} + f_{LF_{max}}) \]
%%
Inom amatörradio är det brukligt med en maximal deviation av \qty{3}{\kilo\hertz}
och en övre gränsfrekvens av \qty{3}{\kilo\hertz}, vilket motsvarar en bandbredd
av \qty{12}{\kilo\hertz}.

Fullgod mottagning är möjlig endast om MF-filtren i mottagaren har
minst den bandbredd, som sändaren har.
Men vid för stor mottagarbandbredd kan även stationer på närliggande frekvenser
uppfattas.
Sedan 1996 är det av IARU Region~1 rekommenderade kanalavståndet
\qty{12,5}{\kilo\hertz} vid FM-trafik på VHF- och UHF-amatörradiobanden.

Det är vanligare med för stor deviation på FM-sändaren än att
mottagaren är alltför smal.
En för stor deviation, avsaknad av deviationsbegränsare och för hög
LF-gränsfrekvens medför en onödigt stor bandbredd på sändaren.
Motstationen får då mottagningssvårigheter och stationer på angränsande
kanaler blir också störda.

Det blir allt vanligare med \qty{12,5}{\kilo\hertz} kanalavstånd även för
repeatrar, varför det är viktigt att alla sändare är rätt inställda.

\subsection{Signalkänslighet och brus}
\harecsection{\harec{a}{4.4.3}{4.4.3}}
\index{signalkänslighet}
\index{mottagare!signalkänslighet}
\index{brus}
\index{mottagare!brus}
\index{SN}
\index{SINAD}
\label{signalkänslighet_brus}

Om man ställer in mottagaren på en ledig frekvens, så hör man vid full
förstärkning ett brus likt det från ett vattenfall.

Bruset kommer från de svaga växelspänningar som uppstår när
laddningsbärarna rör sig genom de material som strömkretsen består av.
Beroende av bruskällan sträcker sig frekvensspektrum från noll
till nära nog oändligt.
På grund av egenskaperna skiljer man mellan en rad specifika bruskällor:

\begin{itemize}
\item Resistorbrus, även kallat ''vitt brus'', som uppstår i resistiva
  komponenter.
  Bruset sträcker sig över hela det mätbara frekvensområdet varvid
  energifördelningen är lika över hela området.

\item Kretsbrus, som uppstår i resistanser i resonanskretsar.

\item Antennbrus, som är sammansatt av bruset från antennens
  strålnings- och förlustresistanser samt av det galaktiska brus som
  antennen tagit emot.

\item Transistorbrus uppstår av laddningsbärarnas rörelser i
  halvledarmaterial.
\end{itemize}

Mer information om brus i komponenter finns i avsnitt~\ssaref{termisktbrus}

Det bildas en sammanlagd brusspänning som kan bestämmas.
Man talar om ett brustal, som är ett mått på mottagningssystemets egenbrus.
Detta ska ställas mot styrkan på den mottagna signalen.
Man talar om ett förhållande mellan signaleffekt och bruseffekt.
Det finns flera metoder att mäta och uttrycka detta förhållande som kallas
S/N (signal to noise ratio).
För att uppfatta den information som kommer ur en mottagares LF-utgång måste
nyttosignalen vara ett antal gånger starkare än bruset.
Den lägre gränsen för att uppfatta tal i kortvågsmottagare är ett brusavstånd
i storleksordningen \qty{10}{\decibel}.

%% k7per: Where should these be referenced?
\smallfig{images/cropped_pdfs/bild_2_4-29.pdf}{S/N-värde}{fig:bildII4-29}

\smallfig{images/cropped_pdfs/bild_2_4-30.pdf}{SINAD-värde}{fig:bildII4-30}

I en broschyr på en kortvågsmottagare kan man till exempel läsa
''Sensitivity SSB, CW: less than 0,25~\(\mu V\) for 10~dB S/N''

Termen S/N betyder Signal/Noise, det vill säga styrkeförhållandet signal/brus
uttryckt i \unit{\decibel}.
Det innebär att en signal kan läsas vid \qty{25}{\micro\volt} signalnivå och ett
S/N av mindre än \qty{10}{\decibel}.
Utöver brusnivån i mottagaren spelar också distorsionen en roll.
%%
\begin{align*}
  \begin{array}[b]{l}
    \text{Signalbrus-} \\
    \text{förhållande}
  \end{array} &= \frac{S + N + D}{N} \text{ dB} \\
  \text{där} \quad S &= \text{Signalnivå} \\
  N &= \text{Brusnivå} \\
  D &= \text{Distorsionsnivå} \\
\end{align*}
%%
I en broschyr på en VHF-mottagare kan man till exempel läsa
''Sensitivity FM: Less than 0,18~\(\mu V\) for 12~dB SINAD''

Termen SINAD betyder Signal, Noise and Distorsion.
Vid denna definition tar man även hänsyn till distorsionsprodukter som orsakas
av den modulerande signalen.
%%
\[
\text{SINAD} = \frac{S+N+D}{N+D}\text{ dB}
\]

\subsection{Intermodulation, korsmodulation}
\harecsection{\harec{a}{4.4.6}{4.4.6}, \harec{a}{4.4.7}{4.4.7}}
\label{intermodulation}
\index{intermodulation}
\index{korsmodulation}
\index{blocking}
\index{storsignalegenskaper}

Utöver att en bra modern mottagare bör ha tillräcklig frekvensstabilitet,
känslighet och selektivitet bör den även ha goda så kallade
\emph{storsignalegenskaper}.

%% k7per???
Med storsignalegenskaper menar man hur bra en relativt svag nyttosignal på
mottagaringången motstår påverkan av starka, frekvensnära signaler med hög
fältstyrka.
Störningar av detta slag uppstår genom icke linjära förlopp i komponenter i
mottagarens ingångssteg, varvid mottagna signaler med stor amplitud blir
förvrängda.

Korsmodulation och intermodulation är två begrepp som är förknippade
med storsignalegenskaperna.
Båda kan visserligen definieras och bestämmas entydigt, men de förväxlas ändå
ofta.

En för stark signal klipper dessutom i mixrar och detta gör att allt mindre
signal kan detekteras varvid känsligheten sjunker och till slut kommer den
tänkta signalen vara helt undertryckt, så kallad \emph{blocking}.

\subsubsection{Korsmodulation}
\harecsection{\harec{a}{4.4.8}{4.4.8}}
\index{korsmodulation}
\index{mottagare!korsmodulation}

Med korsmodulation menas, att den inkommande nyttosignalen amplitudmoduleras
med modulationsprodukter från en annan frekvensnära amplitudmodulerad signal,
varvid korsmodulationen uppstår i olinjära komponenter i mottagaringången
(försteg, blandare).
När man med mottagaren i AM-läge ställt in den på någon bärvåg så hörs också
andra starka, frekvensnära stationer.

Det måste alltså alltid finnas en nyttosignal på den inställda frekvensen för
att det ska uppstå korsmodulation.
När nyttosignalen försvinner så försvinner även korsmodulationen.

För dåligt fasbrus hos mottagaren kan vara en orsak till att starka
grannkanaler mixas in och detekteras.

%% k7per
%% \newpage % layout
\subsection{Intermodulation}
\index{intermodulation}
\index{mottagare!intermodulation}

Vid så kallad intermodulation blandas två starka inkommande signaler i olinjära
komponenter i mottagaringången.
Deras blandningsprodukter faller på mottagningsfrekvensen så att den störs,
vare sig det finns en nyttosignal där eller inte.

\subsection{Frekvensstabilitet}

Se avsnitt \ssaref{oscillatorer}.

%
%
% Kapitel 6 Sändare och transceivers
% Avsnitt 6.1 Sändare
% Avsnitt 6.2 Egenskaper i sändare
\chapter[Sändare]{Sändare och transceivers}

\section{Sändare}
\label{sändare}
\index{sändare}

\subsection{Blockschema}
\index{blockschema}
\label{sändare_blockschema}

\emph{Blockschema} är sätt att göra en översiktlig principskiss över en
apparats design, där de övergripande principerna för funktionen är illustrerade
i en förenklad form där varje block representerar en grundläggande funktion.
Denna förenkling gör att man kan snabb få en översikt utan att fastna i
konstruktionsdetaljerna av enstaka block.

Hela apparaten kan ses som ett antal funktionsblock.
Hur de samverkar framgår i stort av blockschemat.
Där återfinns oscillatorer, blandare, förstärkare etc.
I schemat kan även finnas uppgifter om frekvenser och spänningar med mera.

Det finns olika slags funktionsblock -- kretsar.
Kombinationen av block ger apparater med olika egenskaper.
Exempel är så kallade raka sändare med samma frekvens genom hela sändaren,
superheterodynsändare där frekvensblandning används,
frekvensmultiplicerande sändare etc.

\smallfig{images/cropped_pdfs/bild_2_5-01.pdf}{Enstegs sändare}{fig:bildII5-1}

\mediumbotfig{images/cropped_pdfs/bild_2_5-02.pdf}{Flerstegs rak sändare}{fig:bildII5-2}

\subsection{Rak sändare}
\harecsection{\harec{a}{5.1.1b}{5.1.1b}, \harec{a}{5.2.1}{5.2.1}, \harec{a}{5.3.3}{5.3.3}}
\index{rak sändare}
\index{sändare!rak}

En \emph{rak sändare}, som illustreras i bild~\ssaref{fig:bildII5-1}, är det
enklaste sändarkonceptet.
Då är oscillatorns frekvens samma som sändningsfrekvensen och ingen
frekvensomvandling sker i signalvägen.
Om en antenn kopplas till oscillatorn så blir den en enkel enstegs sändare.

I flerstegs raka sändare följs oscillatorn av ytterligare funktioner på samma
frekvens som oscillatorn.
Buffertsteg, drivsteg och slutsteg kan vara sådana funktioner.

Bild~\ssaref{fig:bildII5-2} visar en rak sändare, som består av oscillator +
buffertsteg 1 + buffertsteg 2 + drivsteg + effektförstärkare.

Oscillatorn följs av ett avlastande buffertsteg 1.
På så sätt blir oscillatorns frekvensstabilitet bättre.
Buffertsteg 2 avlastar ytterligare och matar dessutom ett effekthöjande
drivsteg, som ger driveffekt till slutsteget, samt slutsteget där den slutliga
effekthöjningen sker.

Raka sändare kan användas för CW, FM, PM och AM, men inte DSB och SSB.
Fördelen med raka sändare är enkelheten.
Nackdelen är att alla steg arbetar på samma frekvens, varvid risken för
återverkan på ett föregående funktionssteg är större.
Oönskad återkoppling kan då bli följden, som kan ge ostabila egenskaper hos
sändaren.
Genom att i första hand bygga in VFO och buffertstegen i metallkapslingar,
så kallade skärmar, så minskas denna risk då högre isolation åstadkoms.

\subsection{Sändare med frekvensmultiplicering}
\harecsection{\harec{a}{5.3.2}{5.3.2}, \harec{a}{5.3.5}{5.3.5}, \harec{a}{5.3.6}{5.3.6}, \harec{a}{5.3.8}{5.3.8}, \harec{a}{5.3.9}{5.3.9}, \harec{a}{5.3.11}{5.3.11}}
\index{frekvensmultiplicering}
\index{sändare!frekvensmultiplicering}
\index{utgångsfilter}
\index{sändare!utgångsfilter}

Helst väljer man en arbetsfrekvens för oscillatorn där den är mest
frekvensstabil.

Om högre frekvens önskas på nyttosignalen, så kan man
till exempel multiplicera oscillatorfrekvensen, detta kallas för
\emph{frekvensmultiplicering} (eng. \emph{frequency multiplication}).
I olinjära kretsar alstras övertoner, som ofta utnyttjas i detta syfte.

Endast när kravet på frekvensstabilitet är lågt används den frekvens,
som VFO eller CO arbetar på, även för nyttosignalen.

\mediumplustopfig{images/cropped_pdfs/bild_2_5-03.pdf}{FM-sändare med frekvensmultiplicering}{fig:bildII5-3}
\mediumtopfig{images/cropped_pdfs/bild_2_5-04.pdf}{2-bands CW-sändare med frekvensblandning}{fig:bildII5-4}

Oscillatorn svänger här på en låg frekvens, som multipliceras i
olinjära förstärkarsteg till en hög sändningsfrekvens.
Oftast multipliceras frekvensen två eller tre gånger i vart och ett av
förstärkarstegen.

Bild~\ssaref{fig:bildII5-3} visar ett blockschema för en FM sändare för
\qty{435}{\mega\hertz} (\qty{70}{\centi\metre}-bandet).
Oscillatorfrekvensen är \qty{8,056}{\mega\hertz}.
I fyra av de efterföljande förstärkarna multipliceras frekvensen 2, 3, 3
respektive 3~gånger, alltså totalt 54~gånger.
Sändningsfrekvensen blir då \(8,056 \cdot 54 = \qty{435}{\mega\hertz}\).

Variationer i oscillatorfrekvensen blir också multiplicerade.
I detta exempel blir sändningsfrekvensens deviation 54~gånger större än
oscillatorfrekvensens deviation.
En deviation av max \qty{3000}{\hertz} från den nominella sändningsfrekvensen
motsvaras av följande deviation från oscillatorfrekvensen,
%%
\[\Delta f = \frac{3000}{54} = 55,6\text{ [Hz]}\]
%%
FM-sändare för VHF, UHF och SHF utförs ofta med
frekvensmultiplikation.
Jämfört med en rak sändare är komponentbehovet större, men i stället ger
den låga oscillatorfrekvensen god frekvensstabilitet, vilket är en fördel.
Risken för oönskade självsvängningar är mindre i en frekvensmultiplicerande
än i en rak sändare, eftersom in och utgångsfrekvenserna i flera av stegen är
olika.

Genom att ersätta frekvensmodulatorn med en fasmodulator så kan samma
sändare även användas för fasmodulerad signal.

De frekvensmultiplicerande stegen i bild~\ssaref{fig:bildII5-3} arbetar i klass C,
det vill säga olinjärt, vilket medför amplituddistorsion.
Vid frekvens- och fasmodulering saknar emellertid detta betydelse, eftersom
amplituden i det fallet inte är informationsbärande.

Övertoner i nyttosignalen bör dock filtreras bort, något som sker med ett
filter på utgången, det så kallade \emph{utgångsfiltret}
(eng. \emph{output filter}).


\newpage
\subsection{Sändare med frekvensblandning -- superheterodynsändare}
\harecsection{\harec{a}{5.1.1a}{5.1.1a}}
\label{sändare_frekvensblandning}

\subsubsection{Telegrafisändare (CW) för kortvåg}
\index{CW}
\index{sändare!CW}

En VFO är mest stabil på låga frekvenser medan en CO har god
stabilitet även på högre frekvenser.
När signalerna från dessa blandas, bildas blandningsprodukter som är
skillnaden och summan av signalernas frekvenser.
Bild~\ssaref{fig:bildII5-4} visar en telegrafisändare där detta fenomen används för
sändning inom området \SIrange{14,0}{14,5}{\mega\hertz} eller
\SIrange{3,5}{4,0}{\mega\hertz} beroende på passbandet i filtret efter
blandaren.

Resultatet är en superheterodyn-VFO med både variabel och stabil
signal.
På bilden har valts ett filter med passband för det övre av dessa
frekvensområden.

\subsubsection{Telefonisändare (SSB) för kortvåg}
\harecsection{\harec{a}{5.2.2}{5.2.2}, \harec{a}{5.3.1}{5.3.1}, \harec{a}{5.3.4}{5.3.4}, \harec{a}{5.3.10}{5.3.10}, \harec{a}{5.3.12}{5.3.12}}
\index{telefoni}
\index{SSB}
\index{sändare!SSB}

\mediumplustopfig{images/cropped_pdfs/bild_2_5-05.pdf}{2-bands SSB-sändare med frekvensblandning}{fig:bildII5-5}
\mediumplustopfig{images/cropped_pdfs/bild_2_5-06.pdf}{Flerbands SSB-sändare med frekvensblandning}{fig:bildII5-6}

Bild~\ssaref{fig:bildII5-5} visar en SSB-sändare för två kortvågsband och
bygger på sändaren i bild~\ssaref{fig:bildII5-4}.
Filtermetoden är den mest använda för att bereda en SSB-signal.
Oscillatorsignalen amplitudmoduleras i en balanserad blandare.
I en sådan undertrycks bärvågen medan de båda sidbanden släpps fram.
Det ena sidbandet undertrycks med ett bandpassfilter, ofta implementerat med
ett kristallfilter för att få god undertryckning av oönskat sidband.
Denna SSB-signal flyttas till avsett frekvensband
genom ännu en frekvensblandning och ytterligare filtrering.

I exemplet är CO-frekvensen \qty{9}{\mega\hertz}.
VFO har frekvensområdet \SIrange{5,0}{5,5}{\mega\hertz}.
Vid blandningen fås blandningsprodukter inom frekvensområdena
\SIrange{14,0}{14,5}{\mega\hertz} och \SIrange{4,0}{3,5}{\mega\hertz}.
Genom att välja bandpassfilter kan man sända i ett av dessa frekvensområden.
Efterföljande driv- och slutsteg utförs för att arbeta i detta frekvensband,
antingen utan särskild avstämning -- så kallad bredbandigt utförande -- eller
genom avstämning på en viss frekvens, vilket ger renaste signalen.

Bild~\ssaref{fig:bildII5-6} visar en SSB-sändare som liknar den i
bild~\ssaref{fig:bildII5-5}.
Den stora skillnaden är att signalfrekvensen kan flyttas till flera olika band
med hjälp av ännu en frekvensblandning.
Därför används fler valbara bandpassfilter.

I en SSB-signal ligger all information i amplituden, till skillnad
från en FM-signal där all information ligger i frekvensen.
En SSB-signal får alltså inte förvrängas.
Det innebär att förstärkarstegen i SSB-sändare måste arbeta linjärt, det vill
säga en utsignal ska vara proportionell mot insignalen i varje moment.

\subsection{PLL-styrda sändare}
\index{faslåstloop}
\index{Phase Locked Loop (PLL)}
\index{PLL}
\index{sändare!PLL}

PLL-styrning är inte ett sändarkoncept.
Det är ett sätt att styra frekvensen i en oscillator och hålla den stabil med
hjälp av en likspänning från en \emph{faslåst loop} (eng.
\emph{Phase Locked Loop, PLL}) vilket är en digitalt styrd krets.

En PLL kan användas till exempel i raka sändare och heterodynsändare.
I det första fallet (bild~\ssaref{fig:bildII5-2}) kan frekvensen i den enda
oscillatorn styras av en PLL.
I det andra fallet (bild~\ssaref{fig:bildII5-6}) kan frekvensen i
någon av oscillatorerna styras av en PLL.
En närmare beskrivning av PLL-styrning av dessa två sändarkoncept följer här.

\subsubsection{PLL-styrd FM-sändare för 144--146~MHz}
\harecsection{\harec{a}{5.2.3}{5.2.3}}

\mediumplustopfig{images/cropped_pdfs/bild_2_5-07.pdf}{PLL-styrd FM-sändare för FM}{fig:bildII5-7}

Bild~\ssaref{fig:bildII5-7} visar en PLL-styrd rak sändare med en VCO
(spänningsstyrd oscillator) och ett PA (effektförstärkare).

VCO ingår som det frekvensstyrda elementet i en PLL.
Utfrekvensen från VCO (är-värdet) avläses och delas periodiskt med talet 10
och matas in i en programmerbar frekvensdelare.
Eftersom frekvensområdet för VCO är \SIrange{144}{146}{\mega\hertz}, kommer
infrekvensen till den programmerbara delaren att ligga i området
\SIrange{14,4}{14,6}{\mega\hertz}.
Delningstalet i denna delare kan programmeras in i steg om 1 mellan
talen 5760 och 5840.

Med den första delarens divisor 10 och den andra delarens divisor
inställd till exempel på 5760, så avges ur delarkedjan en puls varje gång som
VCO har genomfört 57600 svängningar.
Vid en VCO-frekvens av \qty{144}{\mega\hertz} (\qty{144000}{\kilo\hertz}) motsvaras
divisorn 57600 (= \(10 \cdot 5760\)) av en pulsfrekvens av \qty{2,5}{\kilo\hertz}
ut från räknarkedjan.
På samma sätt kommer en VCO-frekvens av \qty{144025}{\kilo\hertz} och divisorn
57610 (= \(10 \cdot 5761\)) också att ge en pulsfrekvens av
\qty{2,5}{\kilo\hertz}, likaså \qty{146}{\mega\hertz} och divisorn 58400 och så
vidare.

VCO-frekvensen låses alltså i intervall om \qty{25}{\kilo\hertz} till närmaste
delningstal, för att uppnå en pulsfrekvens av \qty{2,5}{\kilo\hertz}.
Om VCO-frekvensen (är-värdet) avviker från det inställda delningstalet
(bör-värdet), så kommer pulsfrekvensen att bli högre eller lägre än
\qty{2,5}{\kilo\hertz}.

Pulsfrekvensen jämförs i en så kallad fasjämförare med en kristallstyrd
referensfrekvens som efter en delning med 10 också är \qty{2,5}{\kilo\hertz}.
Utspänningen från jämföraren är en likspänning, som intar ett
medelvärde då infrekvenserna är lika, men ett högre eller lägre värde
när de skiljer.
Denna likspänning används för att kontinuerlig styra VCO-frekvensen
till likhet med börvärdet.
Regleringsförloppets hastighet bestäms av tidskonstanten i ett
lågpassfilter, det så kallade loop-filtret.

Sändningsfrekvensen regleras alltså med styrspänningen.
Med samma spänning går det också att frekvensmodulera oscillatorn.
Det görs så, att LF-signalen från modulatorn överlagras på styrspänningen genom
additiv blandning (se kapitel~\ssaref{blandare}) via en kondensator.
De variationer i reglerspänningen som kommer av talet är snabbare än
loopfiltrets tidskonstant.
Variationerna av talet hinner därför inte uppfattas som frekvensavvikelser och
blir därför inte utreglerade.
Drosseln efter loop-filtret förhindrar att moduleringssignalen
kortsluts av filtrets kondensator.

Frekvensinställningen, det vill säga programmeringen av delaren, kan utföras
på flera sätt. Exempel är tumhjuls-omkopplare,
logikkretsar i kombination med en knappsats och så vida\-re.

\subsubsection{PLL-styrd sändare för kortvåg}

\mediumfig[0.9]{images/cropped_pdfs/bild_2_5-08.pdf}{PLL-styrd SSB-sändare för kortvåg.}{fig:bildII5-8}

Bild~\ssaref{fig:bildII5-8} visar ett avancerat koncept för en kortvågssändare.
SSB-signalen alstras på frekvensen \qty{9}{\mega\hertz} och blandas med
\qty{61}{\mega\hertz} i 1:a blandaren.

Summafrekvensen \qty{70}{\mega\hertz} filtreras fram som mellanfrekvens.
Den önskade sändningsfrekvensen fås genom att blanda \qty{70}{\mega\hertz} MF
med frekvensen från VCO och därefter filtrera fram skillnadsfrekvensen.

VCO i detta exempel täcker frekvensområdet \SIrange{40}{69,5}{\mega\hertz}.
Således blir sändarens täckningsområde \SIrange{1,5}{30}{\mega\hertz}.
För att filterfunktionen ska bli optimal, kan den delas upp på flera
valbara filtersektioner, till exempel ett per amatörband.
Valet kan ske automatiskt och styrt av frekvensläget på VCO.

Den absoluta ändringen mellan de två extrema sändningsfrekvenserna är
så stor som \qty{28,5}{\mega\hertz} eller 1:20.
Frekvensändringen i VCO är \qty{29,5}{\mega\hertz}, men där är
ändringsförhållandet mellan de extrema frekvenserna endast 1:1,74, vilket kan
täckas av en enda VCO.
Vid en lägre 2:a MF-frekvens skulle det behövas flera omkopplingsbara
VCO för att täcka hela frekvensområdet

\textbf{Exempel:} Vid en MF på \qty{9}{\mega\hertz} behöver VCO-funktionen täcka
\SIrange{9,5}{39}{\mega\hertz}, det vill säga 1:4,11, vilket är för mycket för
en VCO.

SSB-signalen efter 2:a blandaren är inte lämplig att använda i
regleringsslingan i PLL.
Anledningen är att bärvågen är undertryckt i denna signal och att därför
HF-frekvenserna i det resterande sidbandet varierar i takt med de
modulerande LF-frekvenserna.

I konceptet på bilden rekonstrueras bärvågen i en 1:a kontrollblandare,
genom blandning av de två CO-frekvenserna 9 och \qty{61}{\mega\hertz}.
Den framfiltrerade bärvågen med frekvensen \qty{70}{\mega\hertz} blandas med
VCO-frekvensen i 2:a kontrollblandare och ur denna signal
framfiltreras den rekonstruerade bärvågen.
Denna stämmer perfekt med den undertryckta bärvågens frekvens och
innehåller inga LF-signaler.
Bärvågsfrekvensen delas i en programmerbar frekvensdelare och jämförs
med frekvensen från en kristallstyrd referensoscillator CO.
Ur fasjämföraren erhålls en likspänning som styr VCO via ett loop-filter.
Frekvensen ställs in genom att programmera delaren i PLL.

I moderna sändare finns även mikroprocessorer, som erbjuder
frekvensinställning, minnen och avsökning av frekvenser, m.m.

Det beskrivna konceptet är avancerat.
Frekvensen i alla oscillatorer styrs av samma referensoscillator.
Frekvensstabiliteten beror alltså enbart på referensoscillatorns stabilitet.

Omkopplingen mellan LSB och USB kan göras antingen genom att behålla SSB-filtret
och ändra frekvensen \qty{9}{\mega\hertz} med ett värde så att filtret blir
verksamt i det motsatta sidbandet eller genom att behålla frekvensen
\qty{9}{\mega\hertz} och byta till ett SSB-filter som är verksamt i det motsatta
sidbandet.

En PLL-styrd sändare har både kristalloscillatorns stabilitet och variabel
frekvens över ett stort frekvensområde trots ett litet antal styrkristaller.
En sådan sändare kan relativt enkelt styras digitalt.

En principiell nackdel med alla sändare med PLL-oscillator är fasbruset.
En annan nackdel är den stora komponentmängden
(se kapitel~\ssaref{superheterojämförelse}).

\section{Egenskaper i sändare}
\harecsection{\harec{a}{5.4}{5.4}}

Sändare har många olika egenskaper som man ska vara uppmärksam på, dels för
att ha en effektiv sändare, dels för att få bra kvalitet på sändning och dels
för att inte störa grannkanaler eller på andra band.

\subsection{Frekvensstabilitet}
\harecsection{\harec{a}{5.4.1}{5.4.1}}
\index{frekvensstabilitet}

\emph{Frekvensstabiliteten} (eng. \emph{frequency stability}) är en
grundläggande egenskap, eftersom en sändare som inte är frekvensstabil nog
kommer bli svår för en mottagare att följa och uppfatta.
Dessutom riskerar man att störa grannkanaler.
Mindre avdrift i frekvens kan tolereras, men helst ska den uppfattas som
helt stabil.

I gamla tider så var resonatorerna LC-kretsar, och både mekanik och elektronik
kunde driva betänkligt.
Med modernare kristallstyrda sändare, där man använder PLL eller DDS synteser så
kan frekvensstabiliteten härledas till en enskild kristalloscillator.
Denna är typiskt en okompenserad kristall, med man brukar kunna välja en
temperatur-kompenserad kristalloscillator -- TCXO eller en ugnskompenserad
kristalloscillator -- OCXO.
Det kan även förekomma att man kan låsa på en extern referensfrekvens,
ofta \qty{10}{\mega\hertz}.

%% \newpage % layout
\subsection{RF-bandbredd}
\harecsection{\harec{a}{5.4.2}{5.4.2}}
\index{RF bandbredd}
\index{splatter}
\index{spegelfrekvenser}

\emph{RF bandbredden} (eng. \emph{RF bandwidth}) är den bandbredd som den
modulerade signalen har när den kommer ut ur sändaren.
Det är viktigt att den är begränsad så att den håller sig inom de gränser
som finns för signaltypen, så att sändaren inte stör grann-kanalerna.
Till exempel kan en sändare anpassad för FM \qty{25}{\kilo\hertz} kanaldelning
modulera för starkt för NFM \qty{12,5}{\kilo\hertz} kanaldelning och helt enkelt
störa grann-kanalerna.

Det är ofta svårt att begränsa RF-bandbredden direkt på utgången av sändaren,
eftersom den förväntas kunna byta kanal.
Istället så begränsar man bandbredden på mellanfrekvens direkt vid modulatorn,
och innan frekvens-skiftningen upp till rätt frekvens.
Detta kräver dock att efterföljande steg är linjära nog att inte skapa oönskade
sidband i så kallat splatter eller tar upp spegelfrekvenser.

\newpage
\subsection{Sidband}
\harecsection{\harec{a}{5.4.3}{5.4.3}}
\index{sidband}
\index{övre sidband}
\index{sidband!övre}
\index{Upper Side Band (USB)}
\index{USB}
\index{undre sidband}
\index{sidband!undre}
\index{Lower Side Band}

När man sänder skapas \emph{sidband} (eng. \emph{side band}).
För AM och SSB så skapas bägge, \emph{övre sidbandet}
(eng. \emph{Upper Side Band, USB}) eller \emph{undre sidbandet}
(eng. \emph{Lower Side Band, LSB}) av den modulerade signalen.
För SSB undertrycks även bärvågen.
För FM skapas bredare sidband som behöver filtreras.

\subsection{Ljudbandbredd}
\harecsection{\harec{a}{5.4.4}{5.4.4}}
\index{ljudbandbredd}
\index{audio bandwidth}

Bandbredden på ljudsignalen, den så kallade \emph{ljudbandbredden} (eng.
\emph{audio bandwidth}) in kan vara väldigt stor, och det är därför viktigt
att sändaren begränsar den bandbredden så att sändaren inte råkar modulera
utanför sin kanal, något som främst påverkar bandbredds begränsning uppåt,
oftast \qty{3}{\kilo\hertz} för amatörradio.
Bandbredden kan också behöva begränsas nedåt vid \qty{300}{\hertz} för att inte
råka störa till exempel signalering med subtoner.
Denna nedre begränsning kan dock ibland behöva sättas ur spel för att
kunna skicka ut subtonssignaler, men även för andra former av signaler.

\subsection{Olinjaritet}
\harecsection{\harec{a}{5.4.5}{5.4.5}}
\index{olinjaritet}
\index{nonlinearity}
\index{splatter}

\emph{Olinjaritet} (eng. \emph{nonlinearity}) i ett sändarsteg ger dels
övertoner som behöver begränsas, ofta genom ett filter på utgången, men man
försöker även begränsa hur olinjärt steget tillåts bli.
För tal kommer olinjäritet även att påverka intermodulationen mellan flera
olika frekvenser i tal, vilket dels skapar störningar inom bandet men även
utanför och därmed breddar det.
Detta kallas för splatter och är en oönskad egenskap.
God linjäritet även vid höga effekter är därför eftersträvansvärt.
Ibland kan man ha så mycket olinjäritet att taltydligheten blir låg, det
kan därför vara lämpligt att dra ned något på effekten så taltydligheten går
upp, vilket då ger bättre signalrapport än när intermodulationen är för hög.

\subsection{Utgångsimpedans}
\harecsection{\harec{a}{5.4.6}{5.4.6}}
\index{utgångsimpedans}
\label{utgångsimpedans}

\emph{Utgångsimpedansen} (eng. \emph{output impedance}) är förstärkarens
drivegenskaper och de ska oftast vara anpassade till kabel.
Oftast är det 50~ohm, men för förstärkare som har inbyggd matchbox,
automatisk eller ej, så kan förstärkarens utimpedans anpassas
för att kunna driva en antennsystem med större avvikelser i impedans.
%% k7per: Matchning?
En god matchning i impedans krävs för att få en bra energiöverföring av den
tillförda energin utan att för mycket studsar tillbaka.
Många sändare har skyddskretsar som drar ned uteffekten vid för stor
reflekterad energi, för att skydda slutsteget, och det gör att ett
impedansmatchfel ger ännu större reduktion i utsänd effekt än vad själva
impedansfelet i sig skulle motivera.

\subsection{Uteffekt}
\harecsection{\harec{a}{5.4.7}{5.4.7}}
\index{uteffekt}

\emph{Uteffekten} (eng. \emph{output power}) är den effekt som sändaren är
kapabel att sända, på ett visst band, vid god utgångsmatchning.
Ofta är den mätt i \pep för att matcha kraven från övervakande myndigheter.
Det kan gå att få högre faktisk effekt ur en sändare, men då kommer den vara
så pass olinjär att den inte förväntas klara krav på splatter.

\subsection{Effektivitet}
\harecsection{\harec{a}{5.4.8}{5.4.8}}
\index{effektivitet}

\emph{Effektiviteten} (eng. \emph{efficiency}) på en sändare eller slutsteg
är den utsända effekten i förhållande till den tillförda effekten.
Effektiviteten varierar med uteffekt och frekvens.

\subsection{Frekvensdeviation}
\harecsection{\harec{a}{5.4.9}{5.4.9}}
\index{frekvensdeviation}

\emph{Frekvensdeviationen} är den maximala avvikelsen från bärvågen som
tillåts vid frekvensmodulation.

\subsection{Modulationsindex}
\harecsection{\harec{a}{5.4.10}{5.4.10}}
\index{modulationsindex}
\index{modulationsdjup}

\emph{Modulationsindex} (eng. \emph{modulation index}), eller även
\emph{modulationsdjupet}, anger hur djup modulation av bärvågen är.
För hög modulations undertrycker bärvågen och kan göra det svårt för
mottagaren att detektera.
För låg modulation ger svaga sidband att förmedla tal, och för mycket
av energin går till att sända enbart bärvåg.

\subsection{CW-klickar}
\harecsection{\harec{a}{5.4.11}{5.4.11}}
\label{cw-klickar}

Vid CW kan för snabb stig och falltid på bärvågen ge onödig bandbredd och
uppfattas som klickar eller chirpar.
Då detta är störande ska bandbredden begränsas genom att filtrera bärvågens
till- och frånslag.

\subsection{SSB övermodulation och splatter}
\harecsection{\harec{a}{5.4.12}{5.4.12}}
\index{övermodulation}
\index{SSB!övermodulation}
\index{splatter}
\label{splatter}

Övermodulation vid SSB ger intermodulation och splatter, vilket ger dels en
signal om är svår att läsa och dels en för bred signal.

\subsection{RF-spurioser}
\harecsection{\harec{a}{5.4.13}{5.4.13}}
\index{spurios}

Utöver den förväntade bärvågen kan en sändare skicka ut frekvenser som vare
sig tillhör bärvåg och dess sidband.
Harmoniska undertoner samt helt andra orelaterade frekvenser ska vara
undertryckta.
Detta regleras i EMC standarden för radioutrustning, i det här fallet för
amatörradio.

\subsection{Chassistrålning}
\harecsection{\harec{a}{5.4.14}{5.4.14}}

En sändare förväntas kunna leverera en stor effekt ut på antennutgången, men
från själva inneslutningen, chassit, och övriga anslutningar ska sändaren
inte sända bärvåg, sidband eller några andra signaler.

\subsection{Fasbrus}
\harecsection{\harec{a}{5.4.15}{5.4.15}}
\index{fasbrus}

\emph{Fasbrus} (eng. \emph{phase noise}) är en egenskap hos alla oscillatorer,
som ger en fasmodulation av bärvågen.
Alla steg i en sändare bidrar med brus och ger sammanlagt det totala fasbruset.
En sändares fasbrus kan sträcka sig långt utanför den normala modulerade
bandbredden, och speciellt för repeatrar så kan sändarens fasbrus höja
brusgolvet för mottagaren om inte korrekt trimmade duplexfilter
används för att undertrycka sändarens fasbrus på mottagarens ingångsfrekvens.

% Avsnitt 6.3 Transceiver
\newpage
\smallfig{images/cropped_pdfs/bild_2_5-09.pdf}{Separat sändare och mottagare}{fig:bildII5-9}

\smallfig{images/cropped_pdfs/bild_2_5-10.pdf}{Transceiver med samma VFO}{fig:bildII5-10}

\section{Transceiver}
\index{transceiver}
\index{sändtagare|see {transceiver}}

En \emph{transceiver} -- transmitter receiver -- är både en sändare och
mottagare med delvis gemensamma funktioner.
Dessa kan till exempel vara oscillatorer, signalbehandlingskretsar, filter,
strömförsörjning och så vidare, vilket innebär besparing av ingående
komponenter, men också vissa funktionella begränsningar.

Transceiverkoncept är numera vad som används allra mest av radioamatörer.
Eftersom man på olika vis önskar sig så många sändar- och mottagarfunktioner
som möjligt inom samma skal, så kan det vara svårt att undvika kompromisser.
Så kan till exempel en specialiserad, separat mottagare ha bättre eller fler
egenskaper än i en transceiver.

\subsection{Jämförelse mellan stationskoncept}

Bild~\ssaref{fig:bildII5-9} visar i stort en station med skilda sändar- och
mottagarfunktioner, men att antennen är gemensam.
Bild~\ssaref{fig:bildII5-10} visar i stort en transceiver där VFO och antenn är
gemensamma, men i övrigt med skilda funktioner.
Bild~\ssaref{fig:bildII5-11} visar samma transceiver, men med ett mer detaljerat
blockschema.

\mediumfig{images/cropped_pdfs/bild_2_5-11.pdf}{Direktblandad transceiver med gemensam VFO}{fig:bildII5-11}

\subsection{Simplex}
\index{simplex}

En station som växelvis sänder och tar emot på en frekvens använder
trafikmetoden \emph{simplex}.
Detta är den vanligaste trafikmetoden på kortvåg.

\subsection{Halvduplex}
\index{halvduplex}
\index{split}

En station som växelvis sänder och tar emot på två skilda frekvenser använder
trafikmetoden \emph{halvduplex}.
Trafikmetoden används oftast vid trafik via repeater men även vid \emph{pile up}
på kortvåg, metoden kallas då \emph{split}.
Se vidare avsnitt \ssaref{cq dx och split}.

\subsection{Duplex}
\index{full duplex}
\index{duplex}
\index{duplexfilter}
\index{utsläckning}
\index{notch}
\label{duplex}

En stations sägs sända \emph{duplex} eller \emph{full duplex} när den kan
samtidigt sända och ta emot på två olika frekvenser.

Duplex-operation kräver i allmänhet stor isolation mellan sändare och mottagare,
något som ofta åstadkoms med kavitetsfilter kopplade mellan sändare och antenn
och mottagare och antenn.
Om gemensam antenn används, så kopplas dessa kavitetsfilter ihop till vad som
kallas \emph{duplexfilter}.

För en lyckad duplex-operation krävs i allmänhet mer än \qty{100}{\decibel}
isolation mellan sändare och mottagare.
Mottagarens kavitetsfilter trimmas så att det får en djup \emph{utsläckning}
(eng. \emph{notch}) vid sändarens frekvens, men med så lite förlust som möjligt
på mottagarens frekvens.
Sändarens kavitetsfilter trimmas så att det får en djup utsläckning/notch vid
mottagarens frekvens, för att på så sätt minimera att sändarens fasbrus höjer
brusgolvet för mottagaren, men med så liten förlust som möjligt på sändarens
frekvens.

\subsection{CW-transceiver med direktblandare}
\index{CW}
\index{transceiver!CW}
\index{direktblandare}
\index{Receiver Incremental Tuning (RIT)}
\index{RIT}
\index{keyed operated xmitter (KOX)}
\index{KOX}

Bild~\ssaref{fig:bildII5-11} visar en enkel transceiver för telegrafi.
Sändaren är en rak sändare och mottagaren arbetar med direktblandning.
För 1-kanaltrafik räcker det med en gemensam VFO för sändning och mottagning.
Om motstationen svarar exakt på sändningsfrekvensen, vilken ju är
VFO-frekvensen, så erhålls svävningsnoll i mottagaren.
För att få hörbara morsetecken är mottagaren utrustad med
\emph{Receiver Incremental Tuning (RIT)}, som ändrar VFO-frekvensen med
cirka \qty{800}{\hertz} vid mottagning.

I konstruktionen finns en anordning kallad \emph{Key Operated Xmitter (KOX)}.
Denna kopplar om transceivern till sändning när telegrafnyckeln trycks ner och
till mottagning igen efter en viss tid sedan nyckeln har släppts upp.
Telegrafnyckeln styr också en tongenerator som ljuder i takt med de sända
morsetecknen, så kallad medhörning.

Denna transceiver är utförd för endast ett frekvensband och i övrigt
mycket enkel.

\subsection{Kristallstyrd FM-transceiver för VHF}
\index{frekvensmodulation}
\index{transceiver!FM}
\index{dubbelsuperheterodyn}

Bild~\ssaref{fig:bildII5-12} visar en kristallstyrd FM-sändare med
frekvensomkopplare för kanalval inom \SIrange{144}{146}{\mega\hertz}-bandet.

En kristallfrekvens av cirka \qty{12}{\mega\hertz} multipliceras 12 gånger i en
kedja av förstärkarsteg för att ge sändningsfrekvensen.
Bilden visar räkneexempel för två frekvenskanaler.
Det frekvenssving i oscillatorn, som alstras av modulatorn,
multipliceras också med 12.
För ett sving av \qty{3}{\kilo\hertz} på bärvågen är svinget på oscillatorn bara
\qty{250}{\hertz}.

Efter mikrofonförstärkaren följer en amplitudbegränsare, som ska
hålla deviationen inom ett givet maxvärde, oavsett signalstyrkan från
mikrofonen.
Därefter följer ett lågpassfilter, som dels dämpar de övertoner som
uppstår vid amplitudbegränsningen och dels begränsar de höga frekvenserna
i den modulerade signalen.
Båda åtgärderna begränsar bandbredden.

Mottagaren är en \emph{dubbelsuperheterodyn}, ofta kallad för dubbelsuper.
Den mottagna signalen passerar genom ett förselektionsfilter och en
HF-förstärkare för att i 1:a blandaren blandas med en lokal signal.

\mediumfig{images/cropped_pdfs/bild_2_5-12.pdf}{Kristallstyrd 6-kanals FM-transceiver för VHF}{fig:bildII5-12}

En kristallstyrd lokaloscillator med efterföljande
frekvensmultipliceringssteg alstrar denna signal.

Lokaloscillatorkedjans utfrekvens läggs \qty{10,7}{\mega\hertz} över eller under
mottagningsfrekvensen och mellanfrekvensen efter den 1:a blandningen blir då
\qty{10,7}{\mega\hertz}.
Skilda oscillatorer används vid sändning respektive mottagning varför
styrkristallerna för sändning respektive mottagning på en given kanal får
olika frekvens.
Vid omkoppling till en annan kanal väljs ett annat kristallpar, vilket
lämpligen sker med samma omkopplare.

Den relativt höga 1:a mellanfrekvensen \qty{10,7}{\mega\hertz} ger ett så stort
avstånd till spegelfrekvensen, att bandbredden i förselektionsfiltren
är tillräckligt smal för att undertrycka spegelfrekvensen.
Av samma skäl bör 1:a mellanfrekvensen i en UHF-mottagare väljas ytterligare
tre gånger högre.
Den relativt låga 2:a mellanfrekvensen medger en god närselektering
redan med enkla bandfilter.
En eventuell MF-förstärkare ger tillräcklig signalstyrka till FM-demodulatorn.

För denna lösning behövs det två styrkristaller för varje
frekvenskanal, vilket av kostnadsskäl kan vara en nackdel.

\subsection{PLL-styrd FM-transceiver för VHF}
\index{PLL}
\index{frekvensmodulation}
\index{transceiver!PLL}
\index{transceiver!FM}
\index{dubbelsuperheterodyn}
\index{simplex}

\mediumfig{images/cropped_pdfs/bild_2_5-13.pdf}{PLL-styrd FM-transceiver för VHF}{fig:bildII5-13}

Den PLL-styrda sändare som redan beskrivits i bild~\ssaref{fig:bildII5-7} har här
kompletterats med en svingbegränsare och ett lågpassfilter i modulatorn.
Liksom i den station med kanalkristaller, som beskrivits i
bild~\ssaref{fig:bildII5-13}, är mottagaren även i detta fall en dubbelsuper.

VCO används även som lokaloscillator i mottagaren.
Eftersom sändaren och mottagaren ska användas på samma frekvens
(simplextrafik), måste i detta koncept VCO-frekvensen vara olika vid
sändning och mottagning.
Eftersom mottagarens mellanfrekvens MF är \qty{10,7}{\mega\hertz} måste nämligen
VCO ligga \qty{10,7}{\mega\hertz} högre eller lägre vid mottagning än vid
sändning.
Vid sändning däremot, är VCO-frekvensen densamma som sändningsfrekvensen.

Den programmerbara frekvensdelaren i PLL-kret\-sen arbetar därför med
olika delningstal vid sändning respektive mottagning, se tabell~\ssaref{tab:delningstal}.
Inställningen av divisorn kan ske med kanalomkopplare, tumhjulssats,
knappsats eller ''VFO-ratt'' + digitalräknare och så vidare.
PLL-styrningen ger dessutom möjligheter, till exempel att ordna en automatisk
avsökning över ett önskat frekvensområde -- så kallad scanning.

\begin{table}[ht]
\begin{center}
  \begin{tabular}{ll|lll}
    \multicolumn{2}{l|}{Sändning} &
    \multicolumn{3}{l}{Mottagning} \\
    QRG & Deln.- & QRG & VCO & Deln.- \\
    MHZ & tal    & MHz & MHz & tal \\
    \hline
    \multicolumn{2}{p{7em}|}{Simplexkanaler, exempel} & & & \\
    144,000 & 5760 & 144,000 & 154,700 & 6188 \\
    144,025 & 5761 & 144,025 & 154,725 & 6189 \\
    & & & & \\
    \multicolumn{2}{p{7em}|}{Repeaterkanaler, exempel} & & & \\
    145,000 & 5800 & 145,600 & 156,300 & 6252 \\
    145,025 & 5801 & 145,625 & 156,325 & 6253 \\
  \end{tabular}
\end{center}
\caption{Exmpel på användning av olika delningstal vid simplex- och repeaterkanaler.}
\label{tab:delningstal}
\end{table}

VCO-frekvensen är lika vid sändning och mottagning medan delningstalet
bestämmer arbetsfrekvensen.

\newpage
\subsection{Kortvågstransceiver för SSB och CW}
\index{SSB}
\index{transceiver!SSB}
\index{CW}
\index{transceiver!CW}
\index{talstyrd sändning}
\index{Voice Operated Xmitter (VOX)}
\index{VOX}

\mediumfig{images/cropped_pdfs/bild_2_5-14.pdf}{SSB-transceiver för kortvåg}{fig:bildII5-14}

Vi har redan beskrivit en KV-sändare och KV-mot\-tag\-are för SSB.
I det koncept på en kortvågstransceiver, som visas här i
bild~\ssaref{fig:bildII5-14}, ingår en super-VFO i signalberedningen.
VFO-signalen (\SIrange{5}{5,5}{\mega\hertz}) blandas med signalen från en
kristallstyrd CO, vars frekvens är valbar med en bandomkopplare.
Samtidigt kopplas ett bandpassfilter in efter blandaren i super-VFO, som svarar
till det aktuella frekvensbandet.

Till exempel i \qty{21}{\mega\hertz}-bandet är VFO-filtrets passband
\SIrange{12}{12,5}{\mega\hertz}.
När en VFO-signal \SIrange{12}{12,5}{\mega\hertz} blandas med en
\qty{9}{\mega\hertz} SSB modulerad signal erhålls en frekvens i området
\SIrange{3}{3,5}{\mega\hertz} och en frekvens i området
\SIrange{21}{21,5}{\mega\hertz}.
Den önskade av dessa frekvenser filtreras fram med omkopplingsbara
bandpassfilter, vilket sker med den bandkopplare som nämnts tidigare.

I den enkla kortvågssändare som beskrivits tidigare är det
tillräckligt med en enda sats av omkopplingsbara bandpassfilter.
Det större antalet filter i den här beskrivna utrustningen behövs för att
även kunna använda super-VFO som en del i mottagaren, vilken arbetar
som enkelsuper.
Eftersom en MF på \qty{9}{\mega\hertz} används även i mottagaren kommer
mottagning och sändning att kunna ske på samma frekvens.

Mottagaren beskrivs inte närmare.
Med lämpliga omkopplingsanordningar kan vissa funktionsblock i
transceivern användas både vid mottagning och sändning.
Bild~\ssaref{fig:bildII5-14} visar en SSB-transceiver där passbandfilter i
förkretsar, MF-filter och kristalloscillatorer har dubbel användning.
Funktionsblocken visas inplacerade i sina alternativa
funktioner, däremot inte omkopplingsanordningarna.

Vid sändning och mottagning av CW förbikopplas den balanserade modulatorn och
kristallfiltret i signalbehandlingskretsarna för \qty{9}{\mega\hertz}.
För mottagning av CW ändras BFO-frekvensen i mottagaren så att
det hörs en svävningston när en bärvåg tas emot.
Utan denna frekvensändring skulle endast bärvågsbruset höras.

Även en RIT och en \emph{talstyrd sändning} (eng.
\emph{Voice Operated Xmitter (VOX)}) är inritade.


\newpage
\mediumplustopfig{images/cropped_pdfs/bild_2_5-15.pdf}{PLL-styrd SSB-transceiver för kortvåg}{fig:bildII5-15}

\subsection{PLL-styrd kortvågstransceiver}
\index{PLL}
\index{transceiver!PLL}

En modern transceiver i den högre prisklassen finns i bild~\ssaref{fig:bildII5-15},
i så kallad ''all-mode''-utförande, erbjuder många funktionella möjligheter.
Flera av dem kommer emellertid endast till användning i speciella situationer.
Konceptet för en sådan transceiver beskrivs här i stort.
Huvudprincipen för signalbehandlingen kan beskrivas som en PLL-styrd
dubbelsuper.
SSB-signalen bereds på \qty{9}{\mega\hertz}-nivån och flyttas därefter upp till
\qty{70}{\mega\hertz}-nivån genom frekvensblandning och filtrering.
De möjliga sändningsfrekvenserna mellan 0,5 och \qty{30}{\mega\hertz} skapas
genom att blanda den fasta SSB-signalen med en variabel frekvens från VCO.
Den steglösa frekvenstäckningen som innefattar mellanvågs- och kortvågsområdet
är emellertid endast avsedd för mottagningsfunktionen i transceivern.
För sändningsfunktionen kan tillkomma blockeringskretsar, som förhindrar
sändning utanför tillåtna frekvensband.

Denna förenklade beskrivning omfattar inte kristalloscillatorerna för 9 och
\qty{61}{\mega\hertz} i fasregleringskretsen och inte heller SSB-modulatorn,
FM-modulatorn och anordningarna för CW-sändning.

Mottagaren är en dubbelsuper med hög 1:a MF-frekvens.
Mottagare för höga frekvenser kan till och med utföras som en trippelsuper.
Samma bandpassfilter, blandare och kristallfilter används både vid sändning
och mottagning.

Genom lämplig programmering av frekvensdelaren kan sändning och
mottagning ske på samma frekvens eller på skilda frekvenser
(split-trafik).

En extra VFO-funktion kan åstadkommas genom att frekvensdelaren
programmeras med delningstal som hämtas från ett digitalt minne.
Den extra VFO-funktionen kan sedan efterjusteras genom att ändra
delningstalet med frekvensratten.
Minnet blir ännu mer användbart, om det förutom frekvenser också kan lagra
uppgifter till exempel om sändningsslag och andra inställningar.

\newpage % layout
\subsection{Sammanfattning}

Till skillnad från den raka sändaren är den här beskrivna PLL-styrda
transceivern mycket komplicerad.
Den tekniska utvecklingen går fort.
Nya, bättre och mer invecklade apparater utvecklas ständigt.
Men det är inte alls nödvändigt att använda det senaste och mest
avancerade inom apparattekniken för att utöva amatörradio.
Det går mycket bra att börja med enkla medel och med liten ekonomisk insats.

% \newpage % layout 
Det finns ett stort utbud av begagnade apparater som i olika avseenden
är konkurrenskraftiga med senare konstruktioner.
Det ligger i amatörradions traditioner att ta tillvara tillgänglig
utrustning och förbättra denna efter bästa förmåga.

Ytterst beror resultatet och framgången mest på radiooperatörens
skicklighet, val av frekvens, antenn och tillfälle.

%
%
% Kapitel 7 Antennsystem
\chapter{Antennsystem}
\index{antenn}
\label{antenner_allmänt}

Aldrig så förnämliga radioapparater kommer inte till sin fulla rätt utan ett
effektivt antennsystem.
Det är en huvudförutsättning för framgångsrik radiokommunikation.

Antennen omsätter elektrisk energi från sändaren till elektromagnetiska fält
som strålas ut, det vill säga radiovågor.

Vid mottagning fångar antennen upp radiovågorna och omsätter dem till
elektriska signaler som förs till mottagaren.

Antennsystemet består av den egentliga antennen och transmissionsledningen
mellan denna och sändaren respektive mottagaren.
I antennsystemet ingår även impedansanpassningar, antennkopplare med mera.

% Avsnitt 7.1 Allmänt
\section{Allmänt}
\label{sec:antennsystem-allmaent}

\subsection{Våghastighet}
\label{ljushastigheten}
\index{våghastighet}
\index{ljushastighet}
\index{dielektricitetskonstant}
\index{permeabilitetetskonstant}
\index{frekvens}

I vakuum breder elektromagnetiska vågor ut sig med hastigheten \(c_0\), vilken
mest kallas ljushastigheten.
Den är i SI-systemet~\cite{SIbrochure8} fastställd till
%%
\[c_0 = 299\,792\,458 \approx 300 \cdot 10^6 \text{ [m/s]}\]
%%
I andra media än vakuum har samma vågor utbredningshastigheten \(c\).
Formeln är då
%%
\[c = \frac{c_0}{\sqrt{\mu_r \cdot \varepsilon_r}}\]
%%
där \(\mu_r\) är relativa permeabilitetskonstanten och \(\varepsilon_r\) är
relativa dielektricitetskonstanten för det medium som vågorna passerar igenom.
För enkelhetens skull sätts här \(\mu_r\) och \(\varepsilon_r\) till 1,
alltså \(c_0 = c\).

Sambandet mellan våghastigheten i vakuum, frekvensen och våglängden är förenklat:
%%
\[ c = \lambda \cdot f \quad c\text{ [m/s] }\quad f\text{ [Hz] } \quad \lambda\text{ [m]}\]
%%
och våglängden således
%%
\[ \lambda = \frac{c}{f} \quad \text{[m]} \]

\subsection{Antennlängd}

\subsubsection{Elektriska längden}
\index{elektrisk längd}
\index{antenn!elektrisk längd}

Längden för en resonant, ideal antenn som är en våglängd lång kan beräknas med
ovanstående formel.
Vi kallar denna längd för den \emph{elektriska längden}.
Således \(l_e = \lambda\).

Elektriska längden (\(l_e\)) för en halvvågsantenn (\(\lambda/2\)) är hälften
av den elektriska längden för en helvågsantenn (\(\lambda\)):
%%
\[l_e = \frac{\lambda}{2} \quad \text{[m]}\]

\subsubsection{Mekaniska längden}
\index{mekanisk längd}
\index{antenn!mekanisk längd}
\index{förkortningsfaktor}
\index{antenn!förkortningsfaktor}
\index{resonansfrekvens}
\index{antenn!resonansfrekvens}

Man skiljer på antennens elektriska och mekaniska längd.
Av flera orsaker blir den mekaniska antennlängden (\(l_m\)) för samma
frekvens kortare än den elektriska (\(l_e\)).
Det beror bland annat på våghastighet och ledningsförmåga i de material som
ingår samt övriga elektriska egenskaper beroende på antennens mekaniska
utförande, påverkan från jordplan och omgivning med mera.

Ett förhållande mellan längd och tjocklek av 10000 ger till exempel en cirka 2~\%
mekaniskt kortare antenn.
Förhållandet 30 ger en cirka 5~\% kortare antenn.
Det första värdet kan passa för en \qty{2}{\milli\metre} tjock halvvågsantenn
för \qty{7}{\mega\hertz}.
Det andra värdet för en \qty{3,5}{\milli\metre} tjock halvvågsantenn för
\qty{145}{\mega\hertz}.
Diagram för den så kallade förkortningsfaktorn finns i de flesta
antennhandböcker.

I följande formel har den mekaniska längden (\(l_m\)) för en fritt upphängd
trådantenn valts 2~\% kortare än den elektriska längden.
%%
\[l_m = \frac{\lambda}{2} \cdot 0,98 \quad \text{[m]}\]
%%
\begin{exempelbox}
Beräkna den elektriska och mekaniska längden på en halvvågsantenn med
resonansfrekvensen \(f = 7\)~MHz.
%%
\[
c = \lambda \cdot f
\quad c\ \text{[m/s]} \quad f\ \text{[Hz]} \quad \lambda \text{[m]}
\]
\tcblower
Elektriska våglängden för \qty{7}{\mega\hertz} är:
%%
\[
\lambda = \frac{c}{f} = \frac{300 \cdot 10^6}{7 \cdot 10^6} = 42,86
\quad \text{[m]}
\]
%%
Antennen är en halvvågsantenn, således är elektriska längden:
%%
\[
l_e = \frac{\lambda}{2} = \frac{42,86}{2} = 21,43 \quad \text{[m]}
\]
%%
och mekaniska längden:
%%
\[
l_m = \frac{\lambda}{2} \cdot 0,98 = \frac{42,86}{2}\cdot 0,98 = 21
\quad \text{[m]}
\]
\end{exempelbox}

\newpage
\subsection{Ström och spänning i en halvvågs\-antenn}
\harecsection{\harec{a}{6.2.1}{6.2.1}}
\index{halvvågsantenn}
\index{antenn!halvvågs}
\index{antenn!ström och spänning}

\smallfig{images/cropped_pdfs/bild_2_6-01.pdf}{Spänning och ström i en halvvågsantenn}{fig:bildII6-1}

När en halvvågsantenn matas med HF-energi på grundfrekvensen, så uppstår en
stående våg med ett typiskt utseende.

Bild~\ssaref{fig:bildII6-1} visar att i vardera änden av antennen uppnår spänningen
\(U\) ett maximum (en spänningsbuk), i mitten uppnår strömmen \(I\)
ett maximum (en strömbuk).
Antennen strålar mest där strömbuken finns.

Tag till exempel en \qty{40}{\metre} lång metalltråd som antenn.
Frekvensen för grundresonansen är cirka \qty{3,5}{\mega\hertz}, men den är även
i resonans på de harmoniska övertonerna (7, 14, 21, \qty{28}{\mega\hertz} osv.).

Bild~\ssaref{fig:bildII6-3} visar ström- och spänningsfördelningen på antennen
vid de respektive övertonerna.

För \qty{80}{\metre} (\qty{3,5}{\mega\hertz}) är matningspunkten ett i
spänningsminimum (en spänningsnod) och ett strömmaximum (en strömbuk).
Strömmen är hög därför att matningspunkten har låg impedans.

Samma antenn på \qty{40}{\metre}, \qty{20}{\metre}, \qty{15}{\metre},
\qty{10}{\metre} (7, 14, 21, \qty{28}{\mega\hertz}) har ett spänningsmaximum
(spänningsbuk) och ett strömminimum (strömnod) i matningspunkten, som då har hög
impedans.

Ur horisontaldiagrammet för antennen kan utläsas att ytterligare
strålningskäglor (strålningslober) utvecklas för varje överton i den
påmatade frekvensen.
Samtidigt blir strålningen alltmer till riktad längs med antennen.

\subsection{Impedansen i antennens matningspunkt}
\harecsection{\harec{a}{6.2.2}{6.2.2}}
\index{impedans!antenn}
\index{antenn!impedans}
\index{halvvågsantenn}
\index{antenn!halvvågs}
\index{överton}
\label{antenner_impedans}

\smallfig{images/cropped_pdfs/bild_2_6-02.pdf}{Matningsimpedansen i en halvvågsantenn}{fig:bildII6-2}

\tallfig[0.5]{images/cropped_pdfs/bild_2_6-03.pdf}{Halvvågsdipol matad med harmoniska övertoner}{fig:bildII6-3}

Impedansen \(Z\) för varje punkt på en antenn kan beräknas med Ohms lag
\(Z = \frac{U}{I}\).

Bild~\ssaref{fig:bildII6-2} visar matningsimpedansen i en halvvågsantenn.
På grundfrekvensen för en halvvågsantenn är impedansen \(Z\) i antennens
mittpunkt låg då spänningen är låg och strömmen hög i mittpunkten.
I halvvågsantennens yttre punkter är det tvärt om, impedansen är hög eftersom
strömmen är låg och spänningen är hög.

När antennen, mätt i våglängder, befinner sig mycket högt över jordytan, det
vill säga utan nämnvärd påverkan från omgivningen är impedansen i mittpunkten
\qty{73}{\ohm} på grundfrekvensen.
I praktiken kan impedansen avvika mycket från detta värde.

Antenn och matningskabel måste vara impedansanpassade till varandra
för att det inte ska uppstå vågreflektion i anslutningen.

Märk, att halvvågsantennen är i resonans inte bara på grundtonen utan även på
övertoner.
För 2:a, 4:e etc. harmoniska övertonen har matningspunkten hög impedans.
Vid matning med en lågohmig koaxialkabel uppstår då en kraftig missanpassning i
anslutningen mellan antenn och kabel, vilket måste åtgärdas på något sätt.
Se avsnitt~\ssaref{transmissionsledningar} i detta kapitel.

\subsubsection{Matningsimpedansen i några antenner}
\index{W3DZZ-antenn}
\index{antenn!W3DZZ}

Med W3DZZ-antennen (se avsnitt~\ssaref{W3DZZ}) löses hjälpligt anpassningsproblemet
med mittmatade partier på 2:a harmoniska övertonen, det vill säga dubbla
grundfrekvensen.
På 80- och \qty{40}{\metre}-banden är antennens matningsimpedans cirka
\qty{60}{\ohm} och på de högre banden cirka \qty{100}{\ohm}.
En kompromiss är att mata denna antenn med en \qty{75}{\ohm}-kabel för att inte
få alltför stor missanpassning på något band.

\paragraph{Den omvikta dipolen (folded dipole):}
\index{omvikt dipol}
\index{antenn!omvikt dipol}
\index{folded dipol}
\index{antenn!folded dipol}

Matningsimpedansen är cirka \qty{240}{\ohm}.
En bandkabel med impedansen \qty{300}{\ohm} kan användas alternativt en
koaxialkabel med impedansen 50 eller \qty{75}{\ohm} över en transformator med
impedansomsättningen 4:1.

\paragraph{Jordplanantennen (GP-antennen):}
\index{jordplanantenn}
\index{antenn!jordplan}
\index{GP-antenn}
\index{antenn!GP}

Matningsimpedansen är \SIrange{30}{60}{\ohm}.
När jordplanets spröt inte riktas horisontellt, utan snett nedåt, erhålls en
matningsimpedans av \qty{50}{\ohm}, vilket passar bra för en koaxialkabel med
\qty{50}{\ohm} impedans.

\paragraph{Yagi- och Quad-antenner:}
\index{yagiantenn}
\index{Yagi-Uda}
\index{antenn!yagi-}
\index{antenn!Yagi-Uda}
\index{quadantenn}
\index{antenn!Quad}

En anpassningsanordning för anslutning av \qty{50}{\ohm} koaxialkabel ingår
oftast i fabriksgjorda riktantenner.
En \qty{50}{\ohm} koaxialkabel kan då anslutas direkt till antennens
matningspunkt.

\subsubsection{Reaktansen i en icke-resonant antenn}
\harecsection{\harec{a}{6.2.3}{6.2.3}}
\index{reaktans!antenn}
\index{antenn!reaktans}

Den elektriska resonanskretsen behandlas i avsnitt~\ssaref{oscillatorer}.
Där framställs resonanskretsens grundegenskaper resistans \(R\),
induktans \(L\) och kapacitans \(C\) som koncentrerade till komponenter kallade
resistor, induktor respektive kondensator.

Även en enkel tråd har dessa egenskaper, men fördelade över hela tråden.
Denna kan därför ses som ett stort antal komponenter, som tillsammans bildar en
resonanskrets, vilken naturligtvis kan fungera som antenn.

När antennen matas med växelström med samma frekvens som antennens
resonansfrekvens, så svänger antennen med minsta impedansen.
Resonansfallet kan i korthet beskrivas så att den induktiva och kapacitiva
reaktansen i antennen tar ut varandra medan resistansen kvarstår.

Impedansen är vektorsumman av resistansen och de kapacitiva och
induktiva reaktanserna.
I resonans är antennens impedans lika med resistansen, vilket är ett
specialfall.

Om sändningsfrekvensen är en annan än antennens resonansfrekvens, så händer
endera av följande:

\begin{description}
\item[När antennströmmen har lägre frekvens] än antennens resonansfrekvens, så blir
den resulterande reaktansen negativ (kapacitiv), det vill säga \(X_C\) är större
än \(X_L\).

\item[När antennströmmen har högre frekvens] än antennens resonansfrekvens,
så blir den resulterande reaktansen positiv (induktiv), det vill säga \(X_L\)
är större än \(X_C\).
\end{description}

\smallfig{images/cropped_pdfs/bild_2_6-04.pdf}{Elektrisk förlängning och förkortning av antenner}{fig:bildII6-4}

\subsection{Elektrisk ''förlängning'' och ''förkortning''}
\label{elektrisk förlängning}
\index{elektrisk förlängning}
\index{antenn!elektrisk förlängning}
\index{elektrisk förkortning}
\index{antenn!elektrisk förkortning}

Om sändarfrekvensen, avviker mycket från antennens resonansfrekvens,
så kan reaktansen i antennen behöva elimineras eller åtminstone
minskas för en bättre impedansanpassning mellan antenn och matarledning.
Den enklaste åtgärden är då att försöka ändra antennlängden.

Om detta inte låter sig göras, så kan man i serie med en ''för kort'' antenn
sätta in en induktor -- en så kallad elektrisk förlängning.
Om i motsatt fall antennen är ''för lång'', så kan man sätta in en
kondensator en så kallad elektrisk förkortning.
Bild~\ssaref{fig:bildII6-4} visar elektriska förlängningar och förkortningar av
antenner.

Vid användningen av amatörradio ändras sändarfrekvensen ofta, varför
antennsystemet bör kunna stämmas av från marken/operatörsplatsen.
Då kan en antennkopplare med nödvändiga reaktiva komponenter behövas.
Se längre fram i kapitlet.

\subsection{Anpassning till sändarens impedans}
\harecsection{\harec{a}{5.3.7}{5.3.7}}
\index{avstämningsanordning}
\index{match}
\index{sändare!match}
\index{impedansanpassning}
\index{sändare!impedansanpassning}
\index{\(\pi \)-filter}
\index{sändare!\(\pi \)-filter}
\index{ståendevåg}
\label{antenner_ståendevåg}

Ett sändarslutsteg med elektronrör är vanligen utrustat med en
\emph{avstämningsanordning} (eng. \emph{matching network} och \emph{match})
vid HF-utgången.
Syftet är att kunna anpassa sändarens utgångsimpedans till impedansen i
antennledningen.
I moderna sändare består denna anordning mycket ofta av ett så kallat
\(\pi \)-filter, vars utgångsimpedans kan variera mellan cirka
\SIrange{30}{150}{\ohm}.

Ett transistoriserat slutsteg är oftast utfört för en fast utgångsimpedans av
\qty{50}{\ohm} och är alltså i behov av en avstämningsanordning, om inte
antennsystemet inom vissa gränser håller samma impedans.
Toleransgränsen för felanpassning brukar vara ett SVF av storleksordningen 2:1
innan sändarens skyddskretsar automatiskt minskar uteffekten.

Vid lika impedans i sändarutgång, matarledning och antennanslutning
uppträder ingen stående våg på matarledningen och mesta möjliga effekt
överförs från sändaren till antennen.

\subsection{Antennens strålningsdiagram}
\index{strålningsdiagram}
\index{antenn!strålningsdiagram}

\mediumfig{images/cropped_pdfs/bild_2_6-05.pdf}{Vertikaldiagram för halvvågsantenn}{fig:bildII6-5}

En antenns strålningsbild beskrivs bäst i tre dimensioner.
Bild~\ssaref{fig:bildII6-3} visar bland annat ett horisontaldiagram för en
halvvågsantenn.

Bild~\ssaref{fig:bildII6-5} visar strålningen i vertikalplanet som funktion av
antennhöjden för samma antenn.
Vertikaldiagrammet kan ha mycket olika utseende beroende på antennens utförande,
dess elektriska höjd över mark och omgivningens elektriska egenskaper.
För att överbrygga stora avstånd, måste antennen ha en flack utstrålning
relativt markplanet.

En horisontellt upphängd antenn med en längd av \(\lambda/2\) har övervägande
flack utstrålning när den placeras på en höjd av \(\lambda/2\), \(\lambda\),
\(3\lambda/2\), \(2\lambda\), osv. över mark.

När en horisontell antenn däremot placeras \(\lambda/4\), \(3\lambda/4\),
\(5\lambda/4\) osv. över mark, är utstrålningen övervägande vertikal, vilket
inte ska förväxlas med polarisationen, som i detta fall är horisontell.

Samma diagram gäller både för en sändar- och mottagarantenn.
Styrkan på en utstrålad signal motsvaras av styrkan på mottagen signal.

\subsection{Antennvinst}
\harecsection{\harec{a}{6.2.5}{6.2.5}}
\index{antennvinst}
\index{antenn!antennvinst}
\index{antennförstärkning}
\index{antenn!antennförstärkning}
\index{isotropisk antenn}
\index{antenn!isotropisk}
\index{symbol!\(G_{ant}\) antennförstärkning}
\index{dBi isotropiskt gain}
\index{antenn!dBi isotropiskt gain}
\index{isotropisk förstärkning (dBi)}
\index{antenn!isotropiskt förstärkning (dBi)}
\index{dBd dipol gain}
\index{antenn!dBd dipol gain}
\index{dipol gain (dBd)}
\index{antenn!dipol gain (dBd)}
\label{antenner_antennvins}

Med \emph{antennvinst} eller \emph{antennförstärkning} \(G_{ant}\) (eng.
\emph{antenna gain}) menas förhållandet mellan effekten \(P_f\) i
huvudstrålningsriktningen (framriktningen för en antenn med osymmetriskt
utstrålad effekt) och effekten från en definierad referensantenn.

Förmågan att ha högre antennförstärkning i en riktning i förhållande till andra
riktningar kallas för direktivitet (eng. \emph{antenna directivity}).

En referensantenn som tänks vara oändligt liten och som strålar med exakt
samma effekt \(P_f\) i alla riktningar kallas \emph{isotropisk antenn}.
En isotropisk antenn är emellertid endast teoretisk definierbar.

Med effekten \(P_i\) från den isotropiska antennen som referens blir
antennförstärkningen
%%
\[G = 10 \log\frac{P_f}{P_i} \quad \text{[dBi]}\]
%%
En i praktiken definierbar referens är halvvågsdipolen, vars
huvudstrålning är vinkelrätt ut från dipolen och runt omkring den.
Referenseffekten är då \(P_d\) och antennförstärkningen
%%
%% k7per: Varför Bild här? Tog bort den...
\[G = 10 \log\frac{P_f}{P_d} \quad \text{[dBd]}\] %% Bild \ssaref{fig:bildII6-6}}
%%
Bild~\ssaref{fig:bildII6-6} visar antennförstärkningen med dBd i effekt och
bild~\ssaref{fig:bildII6-7} dBd i spänning.

\smallfig{images/cropped_pdfs/bild_2_6-06.pdf}{Antennförstärkning dBd i effekt}{fig:bildII6-6}

\smallfigpad{images/cropped_pdfs/bild_2_6-07.pdf}{Antennförstärkning dBd i spänning}{fig:bildII6-7}

Antennförstärkning kan också definieras som förhållandet mellan den elektriska
fältstyrkan \(U_f\) i huvudstrålningsriktningen och referensfältstyrkan (dipol).
Jämfört med \(\lambda/2\)-dipol är antennförstärkningen
%%
%% k7per: Varför Bild här? Tog bort den.
\[G = 20 \log\frac{U_f}{U_d} \quad \text{[dBd]}\] %% Bild \ssaref{fig:bildII6-7}}\]
%%
Man använder uttrycket dBi när antennförstärkningen anges i förhållande till
en isotrop antenn och dBd i förhållande till en halvvågsantenn.
Se kapitel~\ssaref{decibel} om decibelbegreppet.

\vspace{1ex}
\noindent\textbf{Exempel på beräkning av antennförstärkning:}
%%
\begin{align*}
  U_f &= \qty{40}{\micro\volt} \quad U_d = \qty{20}{\micro\volt} \quad G =\ ? \\
  G &= 20 \log\frac{U_f}{U_d} = 20 \log\frac{40}{20} \\
  &= 20 \log 2 = 20\cdot 0.3 = 6 \quad \text{[dBd]}
\end{align*}
%%
\qty{6}{\decibel} antennförstärkning motsvarar en fördubblad fältstyrka
[\unit{\volt\per\metre}], det vill säga 1~S-enhets ökning vid den mottagande
stationen, liksom att \qty{6}{\decibel} antennförstärkning motsvarar en 4~dubblad
sändareffekt [\unit{\watt\per\square\metre}].

\noindent
Ungefärlig antennförstärkning för olika antenner med en isotrop antenn som
referens:
\vspace{1ex}
\begin{center}
\begin{tabular}{l|ll|ll}
  & \multicolumn{2}{l|}{\(\lambda/2\)-dipol} &
  \multicolumn{2}{l}{Isotrop} \\
  \hline
  Isotrop antenn       & \num{-2,1} & dBd & 0   & dBi \\
  GP, \(\lambda/4\)    & \num{-1,8} & dBd & 0,3 & dBi \\
  Dipol, \(\lambda/2\) & 0    & dBd & 2,1 & dBi \\
  GP, \(5/8\lambda\)   & 1,2  & dBd & 3,3 & dBi \\
  & & & & \\
  Dipol, \(1/1\lambda\) & 1,8 & dBd & 3,9  & dBi \\
  2-elements yagi       & 5   & dBd & 7,1  & dBi \\
  2-elements quad       & 6   & dBd & 8    & dBi \\
  3-elements yagi       & 8   & dBd & 10,1 & dBi \\
\end{tabular}
\end{center}

\vspace{1ex}
\noindent
Antenner har förluster, det gör att olika antenner kan ha olika
effektivitet, därför finns måttet \emph{antenneffektivitet}
(eng. \emph{antenna efficiency}) \(\eta\) som beror på relationen mellan
utstrålningsresistansen \(R_R\) och förlustresistansen \(R_L\):
%%
\[\eta = \frac{R_R}{R_R+R_L}\]
%%
Antennens effektivitet, verkningsgrad, kan även beräknas med hjälp av antennens
förlusteffekt och den tillförda effekten.
Observera att verkningsgraden för en antenn alltid är mindre än 1 då det är lätt
att förväxla de olika effekterna i beräkningen.

\subsection{Effektivt utstrålad effekt}
\harecsection{\harec{a}{6.2.7}{6.2.7}}
\index{effektivt utstrålad effekt (ERP)}
\index{antenn!effektivt utstrålad effekt (ERP)}
\index{Effective Radiated Power (ERP)}
\index{antenn!Effective Radiated Power (ERP)}
\index{ERP}
\index{antenn!ERP}
\index{ekvivalent isotropiskt utstrålad effekt (EIRP)}
\index{antenn!ekvivalent isotropiskt utstrålad effekt (EIRP)}
\index{Equivalent Isotropically Radiated Power (EIRP)}
\index{antenn!Equivalent Isotropically Radiated Power (EIRP)}
\index{EIRP}
\index{antenn!EIRP}

Effektivt utstrålad effekt (ERP) (eng. \emph{effective radiated power}) är den
effekt som sändarantennen strålar ut i sin bästa strålningriktning.
ERP beräknas som den effekt som tillförs själva antennen, multiplicerat med
antennvinsten relativt en halvvågsdipol.
Förlusterna på vägen från sändaren till antennen räknas bort före beräkningen av
ERP.

Ekvivalent isotropiskt utstrålad effekt (EIRP)
(eng. \emph{Equivalent isotropically radiated power}).
EIRP beräknas relativt en teoretisk antenn (\emph{isotropisk antenn}) som
strålar lika mycket i alla riktningar.
Vid beräkningen används den effekt som tillförs själva antennen, multiplicerat
med antennvinsten relativt en isotrop.
På samma sätt som vid beräkningen av ERP ska förlusterna på vägen från sändaren
till antennen räknas bort före beräkningen av EIRP.

\subsection{Fram/backförhållande (antennvinst)}
\harecsection{\harec{a}{6.2.8}{6.2.8}}
\index{fram/backförhållande}
\index{antenn!fram/backförhållande}

Med fram/backförhållande (F/B) för en riktantenn menas förhållandet mellan den
utstrålade effekten i framriktningen \(P_f\) och effekten i backriktningen
\(P_b\). Se bilderna~\ssaref{fig:bildII6-8} och \ssaref{fig:bildII6-9}.
%%
\[ F/B = 10 \log\frac{P_f}{P_b} \quad \text{[dB]} \]
%%
\smallfig{images/cropped_pdfs/bild_2_6-08.pdf}{F/B-förhållande i effekt}{fig:bildII6-8}

Fram/backförhållandet kan också definieras som förhållandet mellan elektriska
fältstyrkan \(U_f\) i framriktningen och referensfältstyrkan \(U_b\) i
backriktningen
%%
\[ F/B = 20 \log\frac{U_f}{U_b} \quad \text{[dB]} \]
%%
\smallfigpad{images/cropped_pdfs/bild_2_6-09.pdf}{F/B-förhållande i spänning}{fig:bildII6-9}
%%

%% k7per
%% \newpage % layout
\textbf{Exempel 1:}
%%
\begin{align*}
  U_f &= \qty{40}{\micro\volt} \quad U_b = \qty{4}{\micro\volt} \quad F/B =\ ? \\
  F/B &= 20 \log\frac{U_f}{U_b} = 20 \log\frac{40}{4} \\
  &= 20 \log 10 = 20 \cdot 1 = 20 \quad \text{[dB]}
\end{align*}
%%
\(F/B = \qty{20}{\decibel}\) betyder att fältstyrkan \(U_f\) i huvudriktningen är
10 gånger så hög som fältstyrkan i backriktningen \(U_b\).

\noindent\textbf{Exempel 2:}
%%
\begin{align*}
  U_f &= \qty{15}{\micro\volt} \quad U_b = \qty{15}{\micro\volt} \quad F/B =\ ? \\
  F/B &= 20 \log\frac{U_f}{U_b} = 20 \log\frac{15}{15} \\
  &= 20 \log 1 = 20 \cdot 0 = 0 \quad \text{[dB]}
\end{align*}
%%
\(F/B = \qty{0}{\decibel}\) betyder att \(U_f = U_b\), det vill säga att
fältstyrkorna i fram- och backriktning är lika stora, vilket inträffar för en
dipol.

\newpage
\subsection{Halvvärdesbredd}
\index{halvvärdesbredd}
\index{antenn!halvvärdesbredd}

Studera diagrammet för den horisontella strålningen från en riktantenn.

Antennen avger sin största utstrålade effekt \(P_f\) i huvudriktningen.
Effekten avtar utanför huvudriktningen.
Fältstyrkan \(U_f\) förhåller sig på liknande sätt.

Med effekthalvvärdesbredd menas den vinkel inom vilken nyttoeffekten
är minst hälften så stor som i huvudriktningen.
Bild~\ssaref{fig:bildII6-10} visar halvvärdesbredder.
%%
\begin{gather*}
  \text{Observera, att } \frac{P_f}{2} \text{ motsvarar }
  \sqrt{\frac{1}{2}}U_f \\
  ( \approx 0,7 U_f \text{ motsvarande \qty{3}{\decibel}})
\end{gather*}
%%
Med spänningshalvvärdesbredd menas den vinkel inom vilken spänningen
(fältstyrkan) är minst hälften så stor som den största nyttospänningen \(U_f\).
Spänningshalvvärdesbredden för en dipol är ungefär \ang{90}.

\mediumfig{images/cropped_pdfs/bild_2_6-10.pdf}{Halvvärdesbredder}{fig:bildII6-10}

\subsection{Antennarea}
\harecsection{\harec{a}{6.2.6}{6.2.6}}

\emph{Antennarea} (eng. \emph{capture area}) är den area som parabolantenner
och horn har.
Antenngainet (G) beror antennarean (\(A_{phy}\)), effektiviteten (\(e_a\)) och
våglängden (\(\lambda\)) enligt:
%%
\[ G = \frac{4\pi A_{phy}e_a}{\lambda^2} \]

% Avsnitt 7.2 Polarisation
\section{Polarisation}
\harecsection{\harec{a}{6.2.4}{6.2.4}, \harec{a}{6.2.9}{6.2.9}}
\index{polarisation}
\index{polarisation!vertikal}
\index{polarisation!horisontell}
\index{polarisationsdämpning}

Se även i kapitlen \ssaref{vågpolarisation} och \ssaref{radiovågornasegenskaper}.

En elektromagnetisk våg är sammansatt av ett magnetiskt och ett
elektriskt fält, vinkelrätt orienterade mot varandra.

Polariseringsriktningen för en elektromagnetisk våg definieras av riktningen på
dess elektriska fält.
Är det elektriska fältet vertikalt blir polarisationen vertikal respektive
horisontell om det elektriska fältet är horisontellt.

Polarisationsriktningen på de utsända radiovågorna beror i främst på
sändarantennens utförande.

\subsection{Polarisation på HF -- Kortvåg}
\label{polarisation_hf}

För bästa mottagning ska en antenn ha samma polarisationsriktning som i den
infallande vågen.
På kortvåg är det nödvändigtvis inte samma riktning som den från
sändarantennen, eftersom de utsända vågorna oftast har reflekterats i
jonosfären.
Det kan då uppstå en polarisationsvridning som inte kan förutses.
Att då kunna växla mellan mottagarantenner med olika polarisation kan vara en
fördel.
Riktantenner för kortvåg monteras nästan alltid med horisontella element --
horisontell polarisation.

\subsection{Polarisation på VHF/UHF/SHF}
\label{polarisation_vhf}

\smallfig{images/cropped_pdfs/bild_2_6-11.pdf}{Inverkan av polarisation}{fig:bildII6-11}

I de högre frekvensområdena används både horisontell, vertikal eller
cirkulär polarisation.

Polarisationsriktningen ändras inte spontant under överföringen så
länge som vågorna inte reflekterats på vägen.
Vilken polarisation man väljer är av mindre betydelse än att den är lika
för både sändar- och mottagarantennen.

För cirkulärt polariserade antenner, där polarisationen vrider sig
omkring utbredningsaxeln, gäller att överföringen är bäst, när
vridningens riktning är lika både i sändar- och mottagarantennen.

Om en sändare som i det nedre delen av bild \ssaref{fig:bildII6-11} har vertikal
polarisation och mottagaren horisontell polarisation så dämpas den mottagna
signalstyrkan kraftigt.
Räknat i \unit{\decibel} kan dämpningen i det olämpliga arrangemanget vara mer än
\qty{30}{\decibel}.

% Avsnitt 7.3 Antenner för kortvåg
\section{Antenner för kortvåg}

\subsection{Mittmatad halvvågsantenn}
\harecsection{\harec{a}{6.1.1}{6.1.1}}
\index{mittmatad halvvågsantenn}
\index{halvvågsantenn}

Föregående avsnitt illustrerar \emph{mittmatad halvvågsantenn}

\subsection{Ändmatad halvvågsantenn}
\harecsection{\harec{a}{6.1.2}{6.1.2}}
\index{ändmatat halvvågsantenn}
\label{ändmatad_halvvågsantenn}

Utstrålningen från en halvvågsantenn är i princip lika hur den än matas.
En \emph{ändmatad halvvågsantenn} har därför ett strålningsdiagram som är lika
det för en mittmatad antenn.
Vid längre antenner blir strålningskaraktären däremot en annan.

Skillnaden mellan änd- och mittmatade halvvågsdipoler är att
anslutningsimpedansen är mycket högre i ändarna än i mitten.
För att mata antennen längst ut i ena änden behövs en anpassningskrets som
transformerar koaxialkabelns låga impedans till antennelementets höga.

En sådan anpassning, \emph{transformation}, kan göras med en \(\lambda/4\) lång
dubbel transmissionsledning.
Matningen sker i den ena ändan av ledningen och i den andra ändan ansluts
ledningens ena part till antennen och den andra parten lämnas fri.

En sådan antenn med \(\lambda/4\) transmissionsledning och \(\lambda/2\)
antennelement kallas Zepp-antenn och användes först hängandes under ballonger
och luftskepp, så kallad zeppelinare.

\emph{J-antennen} (eng. \emph{J-pole}) \cite[J-antenne]{Rothammel2001} är
elektriskt lika Zepp-antennen men den tillverkas oftast av metallrör, eller
för portabelt bruk, av bandkabel och kallas då ofta för \emph{Slim-Jim}.

Antennen kan byggas i flera olika varianter och de vanligaste behöver en balun
vid matningspunkten för att undvika att matande koaxialkabel blir en del av
antennen och därför försämrar antennens verkningsgrad och strålningsdiagram.

Enstaka utföranden av J-antennen kan byggas så att antennelementet direkt kan
jordas för att minska risken för skador på ansluten radioutrustning vid åska.

\smallfig{images/cropped_pdfs/bild_2_6-12.pdf}{Omvikt dipol}{fig:bildII6-12}

\subsection{Omvikt dipol (folded dipole)}
\harecsection{\harec{a}{6.1.3}{6.1.3}}
\index{omvikt dipol}
\index{folded dipol}

Bild~\ssaref{fig:bildII6-12} visar en omvikt dipol som kan ses som två eller
flera parallella element, som är sammankopplade i ändarna.
Mittpunkten på ett av elementen är ansluten till antennledningen.

Matningsimpedansen för en omvikt \(\lambda/2\)-dipol med två element är
cirka fyra gånger högre än den för en enkel dipol, det vill säga
\SIrange{200}{300}{\ohm}.
Den omvikta dipolen, som endast fungerar på grundfrekvensen och på
dess udda övertoner, är relativt bredbandig.
Matningsimpedansen kan ändras med sinsemellan olika diametrar på de ingående
elementen samt med antalet parallellkopplade element.

\smallfig{images/cropped_pdfs/bild_2_6-13.pdf}{GP-antenn}{fig:bildII6-13}

\subsection{Jordplanantenn}
\harecsection{\harec{a}{6.1.4}{6.1.4}}
\index{jordplanantenn}
\index{antenn!jordplan}
\index{GP-antenn}
\index{antenn!GP}
\label{jordplanantenn}

Bild~\ssaref{fig:bildII6-13} visar en \emph{jordplanantenn} eller
\emph{GP-antennen} (eng. \emph{Ground plane antenna}) som består av en
lodrät strålare som den ena polen och flera sammankopplade
\(\lambda/4\)-radialer eller markplanet som den andra polen.

GP-antennen är rundstrålande och har vertikal polarisering.
Dess relativt flacka utstrålning, i jämförelse med en horisontell antenn,
gör den lämpad för långa distanser.
Av mekaniska skäl används den mest på högre frekvenser (\qty{14}{\mega\hertz}
och högre).

Med horisontella radialer som jordplan är matningsimpedansen cirka
\qty{35}{\ohm}.
För att få god impedansanpassning, till exempel till en \qty{50}{\ohm}
koaxialkabel som matarledning, görs radialerna sluttande nedåt i en lämplig
vinkel.

Koaxialkabelns innerledare ansluts till antennen och kabelskärmen till
radialerna.

Om antennen placeras omedelbart ovan markytan, kan marken användas som
jordplan, särskilt om dess elektriska ledningsförmåga är god.

Om antennelementet inte har en elektrisk längd av \(\lambda/4\), kan
längden anpassas elektriskt på liknande sätt som beskrivits tidigare i
kapitel~\ssaref{elektrisk förlängning} för dipolantenner.
Bild~\ssaref{fig:bildII6-14} illustrerar detta.

\mediumtopfig{images/cropped_pdfs/bild_2_6-14.pdf}{GP-antenner med elektrisk längdanpassning}{fig:bildII6-14}
\mediumbotfig{images/cropped_pdfs/bild_2_6-15.pdf}{SVF-kurvor för flerbands GP-antenn}{fig:bildII6-15}

\newpage
\subsection{Flerbands GP-antenner}
\index{GP-antenn}
\index{antenn!GP}

En GP-antenn kan fås att fungera på flera band genom inbyggnad av en spärrkrets
i antennelementet för tillkommande band och av jordplansradialer med anpassad
längd eller med spärrkretsar även i jordplanet för de banden.
Detta illustreras i bild~\ssaref{fig:bildII6-15} för fem olika band.

Antennen fungerar som \(\lambda/4\) GP-antenn åtminstone på de lägsta banden.
Den mekaniska längden på en flerbands GP för kortvåg blir kort, 4 à 6,5~meter,
vilket på de lägre banden innebär dålig verkningsgrad och liten bandbredd.
Jämför med SVF-kurvorna på bild~\ssaref{fig:bildII6-15}.
Flerbands GP-antenner för upp till sju kortvågsband tillverkas.

\clearpage
\subsection{Flerbands halvvågsantenner}
\harecsection{\harec{a}{6.1.7}{6.1.7}}
\label{W3DZZ}

\mediumminustopfig{images/cropped_pdfs/bild_2_6-16.pdf}{W3DZZ-antennen}{fig:bildII6-16}

W3DZZ-antennen är en vanligt förekommande flerbandsantenn (namnet
efter konstruktörens anropssignal) som visas i bild~\ssaref{fig:bildII6-16}.
Det är en horisontellt upphängd dipolantenn för 80, 40, 20, 15 och
10~m-banden.

W3DZZ-antennen är cirka 33,6~meter lång och har två spärrkretsar,
symmetriskt utplacerade omkring matningspunkten.
Matningen sker med koaxialkabel och balun.

Antennen har en matningsimpedans av cirka \qty{60}{\ohm} på 80- och
40-metersbanden.
På de högre banden är anpassningen inte optimal -- matningsimpedansen stiger
där upp till cirka \qty{100}{\ohm}.
Många använder bland annat av den anledningen inte W3DZZ-antennen på höga
kortvågsband utan föredrar där en flerbandig GP-antenn eller en riktantenn
(yagi, quad m.fl.).
W3DZZ-antennens arbetssätt:
\begin{itemize}
\item 80 m-bandet \\ Hela antennen fungerar som en \(\lambda/2\)-dipol med
  resonansfrekvensen \qty{3,7}{\mega\hertz}.
  Den mekaniska längden är \(2 \cdot 16,8\) meter och förlängs elektriskt med
  induktanserna i spärrkretsarna, vilka f.ö. är ur resonans på detta band.

\item 40 m-bandet \\ Spärrkretsarna är i resonans och ''kopplar bort''
  antenndelen utanför dem.
  Delen där innanför fungerar som en \(\lambda/2\)-dipol med resonansfrekvensen
  \qty{7,05}{\mega\hertz}.

\item 20 m-bandet \\ Hela antennen fungerar som \(3\lambda/2\)-dipol
  med resonansfrekvensen \qty{14,1}{\mega\hertz}.

\item 15 m-bandet \\ Hela antennen fungerar som \(5\lambda/2\)-dipol
  med resonansfrekvensen \qty{21,2}{\mega\hertz}.

\item 10 m-bandet \\ Hela antennen fungerar som \(7\lambda/2\)-dipol
  med resonansfrekvensen \qty{28,4}{\mega\hertz}.
\end{itemize}

% Avsnitt 7.4 Riktantenner för kortvåg
\section{Riktantenner för kortvåg}

\subsection{Riktbar dipolantenn}
\index{riktbar dipolantenn}
\index{antenn!riktbar dipol}

\mediumminustopfig{images/cropped_pdfs/bild_2_6-17.pdf}{Riktbar dipolantenn}{fig:bildII6-17}

En dipolantenn av måttlig mekanisk storlek kan göras vridbar så att
utstrålningen kan riktas, så som illustreras i bild~\ssaref{fig:bildII6-17}.
Men eftersom en ensam dipol strålar i många riktningar, låt vara mest vinkelrätt
ut från antennen, så kan energin i de flesta riktningarna ses som ''förlorad''.

När ett passivt antennelement -- en reflektor -- placeras bakom det aktiva
elementet kan emellertid bakåtstrålningen delvis vändas framåt och man får i
stället en viss riktverkan.
För att det ska fungera ska de båda elementen ha ett visst inbördes förhållande
mellan elementens längd och avståndet mellan elementen.


\subsection{Yagiantenner}
\harecsection{\harec{a}{6.1.5}{6.1.5}}
\index{yagiantenn}
\index{antenn!yagi-}
\index{monobandbeam}
\index{antenn!monobandbeam}
\index{multibandbeam}
\index{antenn!multibandbeam}

\mediumminustopfig{images/cropped_pdfs/bild_2_6-18.pdf}{Flerbands yagiantenner}{fig:bildII6-18}

Med ytterligare passiva antennelement -- så kallade direktorer -- framför det
aktiva elementet, blir riktverkan ännu bättre.
Reflektorn är alltid elektriskt längre än det aktiva elementet och direktorerna
är alltid elektriskt kortare.
Direktorernas längd blir kortare på längre avstånd från det drivna elementet.
Läs mer om riktantenner i kapitel~\ssaref{riktantenn}.

En sådan antenn är Yagi-Uda-antennen, döpt efter sina japanska upphovsmän.
Den kallas oftast enbart för \emph{yagiantenn}.
Den är ursprungligen avsedd för ett enda frekvensband, en så kallad
\emph{monobandbeam}.

Om alla element förses med lämpliga spärrkretsar, med W3DZZ-antennen som
förebild, fås en riktantenn som är användbar på flera frekvensband, en så
kallad \emph{multibandbeam}.
De vanligaste antennerna för flera band har två till tre element och är
konstruerade för frekvensbanden \qty{10}{\metre}, \qty{15}{\metre} och
\qty{20}{\metre}.
Bild~\ssaref{fig:bildII6-18} visar flerbands yagiantenner med 2, 3 respektive 5
element samt deras strålningsdiagram i horisontalplanet.
Matningen sker oftast med en koaxialkabel med den karakteristiska impedans
\qty{50}{\ohm}.
Eftersom matningsimpedansen för själva riktantennen nästan aldrig är
\qty{50}{\ohm}, så behövs oftast en impedansanpassning mellan antenn och kabel.

\subsection{Cubical Quad-antenner}
\index{cubical quadantenn}
\index{quadantenn}

\mediumfig{images/cropped_pdfs/bild_2_6-19.pdf}{Cubical Quad-antenner}{fig:bildII6-19}

Bild~\ssaref{fig:bildII6-19} visar en \emph{cubical quad-antennen} som är en
kvadratisk helvågsstrålare med en sidlängd av \(\lambda/4\), det vill säga
totalt \(1\,\lambda\).

En 2-elements quad-antenn består av en strålare och en reflektor på ett inbördes
avstånd av 0,15--0,2\,\(\lambda\).
Det finns även 3 och 4-elements quad-konstruktioner med beaktansvärda
dimensioner.
Antennen görs lämpligen vridbar och bör monteras åtminstone $3/4\,\lambda$
över mark.

Matningen sker oftast med en koaxialkabel och beroende på elementavståndet
varierar matningsimpedansen mellan 50 och \qty{70}{\ohm}.
Beroende på hur matningspunkten placeras är det möjligt att välja mellan
horisontell eller vertikal polarisering.

Det finns två utföranden av quad-antenner, det ena med en bärande bom
med spridare för att bära upp antennelementen och det andra med bara
spridare från ett centralt fäste, den så kallade spider quad (spindel).

Quad-antenner byggs vanligen för 10-, 15- och \qty{20}{\metre}-banden.
Spiderprincipen är att föredra vid utförande för flera band, eftersom ett
optimalt element\-avstånd kan väljas för varje band utan att antalet spridare
behöver ökas.

Genom den flacka strålningsvinkeln är quad-\-an\-tennen en utmärkt DX-antenn.
En två-elements quad anses kunna ge samma resultat som en 3-elements yagiantenn.

För kortvågsbruk finns många antenntyper, såsom longwire-, zepp-,
windom-, romb-, delta loop-, quad- loop-antenner etc.
För mer information hänvisas till antennlitteratur.

% Avsnitt 7.5 Antenner för VHF/UHF/SHF
\clearpage
\section{Antenner för VHF/UHF/SHF}
\index{antenn!VHF}
\index{antenn!UHF}
\index{antenn!SHF}

\subsection{Allmänt}
\label{antenner_vhf_allmänt}

Alla antenner fungerar efter samma principer.
Principerna för kortvågsantenner kan därför tillämpas även för antenner för
högre frekvenser.
Byggmåtten på en VHF/UHF-antenn är betydligt mindre än för en motsvarande
HF-antenn.
Då våglängden \(\lambda\) vid \qty{145}{\mega\hertz} är cirka \qty{2}{\metre}
jämfört med cirka \qty{80}{\metre} vid \qty{3,5}{\mega\hertz} är möjligt att
bygga riktantenner med rimliga dimensioner för VHF/UHF, även om många
antennelement används.

Om man bortser från rundstrålande vertikalantenner för trafik på korta avstånd
och mobil trafik, så används riktantenner främst på grund av den större
räckvidden.
En riktantenns egenskaper uttrycks i första hand i storheterna strålningsvinkel,
antennvinst, fram/backförhållande och halvvärdesbredd.

Eftersom polarisationsvridning sällan förekommer vid högre frekvenser, är det
viktigt att sändar- och mottagarantenner har samma polarisationsriktning.
Horisontell polarisation anses vara bättre lämpad för långa distanser, eftersom
vågor med horisontell polarisation böjer av bättre över horisontella
formationer (bergryggar etc.).
Även passage genom skogspartier går bättre med horisontellt polariserade vågor.
Antenner med horisontell polarisation används därför ofta för SSB- och CW-trafik
på långa avstånd och utmed markytan.
Sådan trafik sker i allmänhet från fasta stationer.

För både mobil trafik och lokal fast trafik används oftast antenner med vertikal
polarisation.
Vertikala antenner ger de önskvärda rundstrålande egenskaperna för mobil trafik
och är bäst lämpade att montera på fordon.

\subsection{Riktantenner}
\index{antenn!riktantenn}
\index{riktantenn!elementlängd}
\label{riktantenn}

En \(\lambda/2\)-antenn strålar vinkelrätt ut från antennledaren och
runt omkring den.
Placeras ett reflektorelement (längd \(\approx\lambda/2 + 5\%\)) bakom antennen
på ett avstånd av \(\approx \lambda/5\) så reflekteras den bakåtriktade
strålningen delvis framåt.
En större del av energin kommer då att samlas i en riktning.
Med ett direktorelement (längd \(\approx\lambda/2 - 5\%\)) framför det
strålande elementet på ett avstånd av \(\approx\lambda/10\) så kommer
utstrålningsvinkeln att bli mindre.

\subsection{Yagiantenner}
\index{yagiantenn}
\index{antenn!yagi-}
\label{antenner_vhf_yagi}

\largefig{images/cropped_pdfs/bild_2_6-20.pdf}{Strålningsdiagram för horisontell yagiantenn}{fig:bildII6-20}

Den typ av riktantenn, som består av en strålare, en passiv reflektor
samt ett antal passiva direktorer, kallas \emph{yagiantenn} och illustreras i
bild~\ssaref{fig:bildII6-20}.
Observera att vertikaldiagrammet visar strålningsdiagrammet med antennen
placerad nära jord.
För VHF, UHF och SHF placeras antennen ofta så högt över mark att antennen kan
ses som placerad i fri rymd.
Strålningsdiagrammet blir då annorlunda genom att antennens huvudlob sänks jämfört
med det i bilden och hamnar med centrum symmetriskt runt horisontalplanet.

Yagiantennen kan utföras med olika antal direktorelement i kombination med
olika längd.

Det finns tre sätt att optimera en riktantenn, nämligen maximal
riktverkan, minimala sidolober eller maximalt fram/backförhållande.
Dessa egenskaper är, emellertid ej möjliga att uppnå samtidigt.
Ökas till exempel antalet element, så ökar den så kallade antennvinsten genom
att öppningsvinkeln på strålningen blir mindre, men samtidigt minskar
matningsimpedansen och den användbara bandbredden.

Som tumregel kan man konstatera att det är inte mängden element som
avgör antennvinsten, utan den dominerande faktorn är längden på bommen.
Mängden element påverkar antennvinsten med \SIrange{1}{2}{\decibel} från optimal
till medioker.
Egenskaper som fram/back förhållande eller minimala sidolober beror mer
på mängden, placering och storlek på direktorerna.

\subsection{Gruppantenner}

Ordnas flera riktantenner vid sidan av och/eller över varandra så
erhålls en så kallad gruppantenn.
Ett sådant arrangemang av så kallade stackade antenner ger en ännu mindre
öppningsvinkel på strålningen vertikalt och/eller horisontellt.
Därigenom erhålls ytterligare antennvinst.

\subsection{Parabolantenner}
\harecsection{\harec{a}{6.1.6}{6.1.6}}
\index{parabolantenn}
\index{antenn!parabol}
\index{antenn!SHF}

Särskilt på frekvenser i mikrovågsområdet och högre har radiovågorna i
stort sett samma utbredningsegenskaper som ljusets.
Behöver stor riktverkan uppnås på dessa höga frekvenser, används ofta en
parabolisk yta som spegel bakom själva antennen, tillsammans kallas det dock
för \emph{parabolantenn}.
Jämför med reflektorn i en ficklampa.

Den egentliga antennen (den s.k. mataren), vars strålning är riktad
mot parabolen för att reflekteras, kan vara utformad på många sätt.
Eftersom parabolens storlek står i omvänd proportion till frekvensen, så
används av praktiska skäl inte paraboliska reflektorer på låga frekvenser.

\subsection{Övriga antenntyper}

Rundstrålande antenner: Ground plane, \(\lambda/4\)-, \(\lambda/2\)-,
\(5\lambda/8\)-antenner m.fl.
Riktantenner: Quad-, HB9CV-, helical-, parabol- och hornantenner med flera.

% Avsnitt 7.6 Transmissionsledningar
\section{Transmissionsledningar}
\label{transmissionsledningar}
\index{transmissionsledning}

En matarledning ska med så små förluster som möjligt överföra den
högfrekventa energin från sändaren fram till sändarantennen.
Omvänt ska den energi som fångats upp av mottagarantennen transporteras
till mottagaren med så små förluster som möjligt.

\subsection{Avstämd matarledning}
\index{transmissionsledning!avstämd}
\label{avstämd_matarledning}

\smallfig{images/cropped_pdfs/bild_2_6-21.pdf}{Spänningskopplad $\lambda/2$-dipol}{fig:bildII6-21}

Bild~\ssaref{fig:bildII6-21} visar en \(\lambda/2\)-dipol som kopplas till
sändarutgången via en \(\lambda/4\) matarledning.
För tydlighetens skull visas ledningen som en bandkabel.

Vid sändning uppstår en stående våg på matarledningen och på dipolen.
Även matarledningen svänger med och är avstämd till resonans
-- därav namnet avstämd matarledning.

Vi följer ström- och spänningsfördelningen bakåt från dipolen till
sändaren och finner följande:

I vardera änden av \(\lambda/2\)-dipolen uppträder en spänningsbuk (streckade
linjer) och i mitten av dipolen uppträder en strömbuk (heldragna linjer).
Den stående vågen, med strömbuken på dipolens mitt, fortsätter ner på
\(\lambda/4\)-matarledningen.
I nedre änden av matarledningen vid sändarutgången har det uppstått en strömnod
och en spänningsbuk, vilket innebär att matarledningen ska spänningskopplas
till sändaren.

\tallfig{images/cropped_pdfs/bild_2_6-22.pdf}{Strömkopplad $\lambda/2$-dipol}{fig:bildII6-22}

Om matarledningen i stället är \(\lambda/2\) lång, så uppstår i
stället en spänningsnod och en strömbuk i nedre änden av ledningen,
vilket innebär att matarledningen ska strömkopplas till sändaren,
vilket visas i bild~\ssaref{fig:bildII6-22}.

Ström- och spänningsfördelningen kan ritas upp för en \(\lambda\)-dipol
respektive \(\lambda/2\)-dipol i kombination med matarledningar med längderna
\(n \cdot \lambda/4\) (med \(n\) = 1, 2, 3 \dots).
Med hjälp av teckningen kan man avgöra om ström- eller spänningskoppling måste
användas.

\mediumfig{images/cropped_pdfs/bild_2_6-23.pdf}{Samma $\lambda/2$-dipol på grundfrekvensen respektive 1:a övertonen}{fig:bildII6-23}

Bild~\ssaref{fig:bildII6-23} visar en \(\lambda/2\)-dipol för 80~m-bandet
ansluts till en avstämd matarledning med längden \(\lambda/2 = 40\)~m.

Önskar man använda denna dipol för 80~m-bandet på 40-, 20- och 10~m-banden
måste en så kallad antennkopplare anslutas mellan sändaren och matarledningen.
Kopplaren har alltid strömmatad ingång och valmöjlighet för ström- respektive
spänningsmatad utgång.
Se om antennkopplare sist i detta kapitel.

\subsection{Oavstämd matarledning}
\index{transmissionsledning!oavstämd}
\label{oavstämd_matarledning}

Begreppet ''oavstämd'' syftar på ledningslängden, som under vissa
bestämda förutsättningar kan vara godtyckligt lång.
I motsats till den avstämda matarledningen behöver ledningslängden på en
oavstämd matarledning inte stå i förhållande till våglängden \(\lambda\).
Som matarledning kan användas en koaxialkabel eller öppen transmissionsledning.

\textbf{Fördelar:}
Enkel uppbyggnad, mindre kritisk kabeldragning och längden kan väljas godtyckligt.

\textbf{Nackdelar:}
Sändaren, matarledningen och antennen måste alltid vara impedansanpassade till
varandra.
Dessutom måste antenn- och kabelströmmarna balanseras.
I det följande visas hur dessa krav kan uppfyllas.

Som matarledning upp till mikrovågsområdet är koaxialkabeln vanligast.

\subsection{Koaxialkabel}
\harecsection{\harec{a}{6.3.2}{6.3.2}}
\index{koaxialkabel}
\index{transmissionsledning!koaxial}

\smallfig{images/cropped_pdfs/bild_2_6-24.pdf}{Koaxialkabel}{fig:bildII6-24}

Koaxialkabelns uppbyggnad framgår av bild~\ssaref{fig:bildII6-24}.
I en koaxialkabel bildas ett radiellt elektriskt fält mellan mittledaren och
insidan av ytterledaren.
Av strömmen bildas också ett magnetiskt koncentriskt fält mellan inner- och
ytterledaren.
Resultatet blir ett elektromagnetiskt fält, som breder ut sig i kabeln som en
TEM-våg (TE-våg~=~transversell elektrisk, TM-våg~=~transversell magnetisk och
TEM-våg~=~transversell elektromagnetisk våg).

Koaxialkabeln består av en isolerad innerledare omgiven av en ytterledare, vars
insida är kabelns andra strömledare.
Ytterledaren förhindrar dessutom HF-utstrålning och inkommande störningar.
I motsats till den symmetriskt uppbyggda bandkabeln, tillhör
koaxialkabeln de osymmetriska ledningarna.

Vanliga karaktäristiska impedanser för koaxialkabel är 50 och \qty{75}{\ohm}.

\subsection{Bandkabel}
\harecsection{\harec{a}{6.3.1}{6.3.1}}
\index{bandkabel}
\index{transmissionsledning!bandkabel}

\smallfig{images/cropped_pdfs/bild_2_6-25.pdf}{Bandkabel}{fig:bildII6-25}

Som framgår av bild~\ssaref{fig:bildII6-25} består bandkabeln av två parallella
ledare med samma dimensioner.
Kabelns isolering håller samtidigt ledaravståndet rätt.
I ett kraftigare utförande övergår denna ledningstyp till att bestå av ett
ledarpar med isolerade spridare på jämna avstånd.
Den kommer att likna en stege det ursprungliga utförandet på en matarledning.

Vanliga karaktäristiska impedanser för bandkabel är 300 och \qty{450}{\ohm}.

\subsection{Vågledare}
\harecsection{\harec{a}{6.3.3}{6.3.3}}
\index{vågledare}
\index{transmissionsledning!vågledare}

Inom mikrovågsområdet är den vanligaste typen av matarledning så kallade
vågledare som saknar mittledare.
I en vågledare matas energin fram enbart som speciella elektriska och
magnetiska fält (TEM) i mönster som kallas moder.

\subsection{Hastighetsfaktor}
\harecsection{\harec{a}{6.3.5}{6.3.5}}
\index{transmissionsledning!hastighetsfaktor}

Vid bestämning av den mekaniska längden på en matarledning måste hänsyn tas
till att våghastigheten längs ledningen är lägre än ljushastigheten.
Man talar om en hastighetsfaktor relativt ljushastigheten.
Hastighetsfaktorn beror på ledningens utförande och ingående material.

En koaxialkabel har hastighetsfaktorn \(v = \frac{1}{\sqrt{\varepsilon}}\),
där \(\varepsilon\) är den relativa dielektricitetskonstanten i
isolationsskiktet.
Ett vanligt förekommande isolationsmaterial i koaxialkablar är polyetylen med
dielektricitetskonstanten \(\varepsilon = 2,25\).
Hastighetsfaktorn \(v\) (velocity factor) blir då
%%
\[
v = \frac{1}{\sqrt{\varepsilon}} = \frac{1}{\sqrt{2,25}} = \frac{1}{1,5} = 0,666
\]
%%
1~meter av en sådan koaxialkabel är \(1/0,666 = 1,5\) meter för en HF-signal.
Även bandkablar har naturligtvis en hastighetsfaktor, vanligen 0,7--0,85.

%% k7per
%% \newpage %layout

\subsection{Karaktäristisk impedans Z i led\-ningar}
\harecsection{\harec{a}{6.3.4}{6.3.4}}
\index{transmissionsledning!impedans}

Antag att en HF-sändare har kopplats till en oändligt lång ledning.
Om man undersöker kvoten mellan spänning och ström på godtyckliga ställen
utmed ledningen, så kommer man att finna samma kvot överallt.
Denna konstant uttrycks i ohm, om spänning och ström uttrycks i volt
respektive ampere.
Konstanten kallas vågimpedans eller karaktäristisk impedans.

Oändligt långa ledningar är ju orealistiska och då kan man i stället bestämma
vågimpedansen genom ledningens geometriska uppbyggnad, dielektricitetskonstant
och dess induktivitet och kapacitet per längdenhet.

\noindent\textbf{Exempel:}
Vi undersöker elektriska karaktäristika i en kabel av typ RG-213/U.
På en provbit med längden 1~meter mäter vi en kapacitans av \qty{97}{\pico\farad}
mellan inner och ytterledaren.
När kabelns ena ände kortsluts mäter vi en induktans av \qty{262}{\nano\henry}.

Den uppmätta kapacitansen och induktansen bestämmer kabelns karaktäristiska
impedans \(Z\), också kallat våg motstånd, som är oberoende av ledningens längd.
Med ovanstående uppmätta värden blir impedansen:
%%
\begin{align*}
  Z &= \sqrt{\frac{L}{C}} \quad L\text{ [H]} \quad C\text{ [F]} \quad
  Z [\unit{\ohm}] \\
  Z &= \sqrt{\frac{262000\cdot 10^{-12}}{97\cdot 10^{-12}}} =
  \sqrt{\frac{262000}{97}} = \qty{52}{\ohm}
\end{align*}
%%
Den karaktäristiska impedansen för en matarledning, bestäms av ledningens
dimensioner och av isolationsmaterialets dielektricitetskonstant.

\noindent
För en bandkabel är:
%%
\begin{align*}
Z = & \frac{276}{\sqrt{\varepsilon_r}}\cdot\log\frac{2a}{d} \quad \unit{\ohm} \\
[&a = \text{centrumavståndet mellan ledarna i mm}] \\
[&d = \text{ledardiametern i mm}] \\
[&\varepsilon_r = \text{dielektricitetskonstanten, överslagsvärde 1,5}] \\
[&\varepsilon_r \text{ för luft} = 1,0]
\end{align*}
%%
För en koaxialkabel är:
%%
\begin{align*}
Z = & \frac{138}{\sqrt{\varepsilon_r}}\cdot\log\frac{D}{d} \quad \unit{\ohm} \\
[&D = \text{ytterledarens innerdiameter i mm}] \\
[&d = \text{innerledarens ytterdiameter i mm}]
\end{align*}
%%
Data, impedansdiagram och formler för beräkning av transmissionsledningar
finns bland annat i antennhandböcker.

\mediumbotfig{images/cropped_pdfs/bild_2_6-26.pdf}{Stående våg på ledning}{fig:bildII6-26}

\subsection{Stående vågor}
\index{stående~våg}
\index{transmissionsledning!stående~våg}
\label{stående_vågor}

Både när sändarens och matarledningens anslutningsimpedans är olika
liksom när matarledningens och antennens anslutningsimpedans är olika,
så uppstår så kallad missanpassning som hindrar energitransporten.

Antag att matarkabelns och antennens anslutningsimpedans är olika.
En del av HF-energin kommer då att strålas ut från antennen, men resten
reflekteras tillbaka i matarledningen.
På kabeln finns alltså en framåtgående våg mot antennen och samtidigt en
reflekterad våg tillbaka mot sändaren.

Den spänning och ström som man då kan mäta var som helst på kabeln, är den
algebraiska summan av amplituden hos den framåtgående och den reflekterade
vågen.

Flyttar vi mätpunkten stegvis utmed kabeln, så kommer spänningen och
strömmen att stiga och sjunka på ett regelbundet sätt.

Den tillbakagående vågens spänning \(U_b\) och den framåtgående vågens
spänning \(U_f\) överlagras på varandra.
Kvoten för ström och spänning är därmed inte konstant utmed matarledningen,
utan får ett vågformat förlopp -- en stående våg.

Punkterna för maxima och minima beror av belastning relativt
vågresistansen och av frekvensen.

Stående vågor uppträder inte bara i antennkablar utan även i fasta material
(trådar o.d.), i luft (ljud), i ljus (t.ex. laser), i elektromagnetiska fält
och så vidare.

Bild~\ssaref{fig:bildII6-26} visar stående våg på ledning.
Spänningen utmed kabeln varierar regelbundet mellan
%%
\[U_{max} = U_f + U_b \quad \text{och} \quad U_{min} = U_f - U_b\]

%% \newpage % layout

\subsection{Ståendevågförhållande (SVF)}
\harecsection{\harec{a}{6.3.6}{6.3.6}}
\index{SVF}
\index{transmissionsledning!SVF}

(se även SWR = Standing Wave Ratio i avsnitt\ssaref{SVF}).

Med ståendevågförhållandet SVF menas förhållandet mellan \(U_{max}\)
och \(U_{min}\) eller mellan \(I_{max}\) och \(I_{min}\).

\begin{align*}
  \text{SVF} &= \frac{U_{max}}{U_{min}} = \frac{U_f + U_b}{U_f - U_b} \quad
  \text{eller} \\
  \text{SVF} &= \frac{I_{max}}{I_{min}}
\end{align*}
%%
Ståendevågförhållandet SVF kan även anges med hjälp av impedanserna i
matarledningen (\(Z\)) och i antennens matningspunkt (\(Z_a\)).
%%
\begin{align*}
  \text{SVF} &= \frac{Z}{Z_a} \quad \text{där } Z > Z_a \quad \text{eller} \\
  \text{SVF} &= \frac{Z_a}{Z} \quad \text{där } Z > Z_a
\end{align*}
%%
Ståendevågmätning beskrivs i avsnitt~\ssaref{mäta_ståendevåg}.

\mediumfig{images/cropped_pdfs/bild_2_6-27.pdf}{SVF-problemet förenklad bild}{fig:bildII6-27}

Bild~\ssaref{fig:bildII6-27} visar en förenklad bild av SVF-problemet och vad en
SVF-meter visar beroende på var den kopplas in i kedjan sändare -- ledning --
antennkopplare -- ledning -- antenn.

Vid ett högre SVF-tal än 2:1 till 3:1 vid sändarutgången, bör en antennkopplare
sättas in efter sändaren för att skydda den från (överhettning och) överslag.
Antennkopplare har även andra benämningar, till exempel matchbox,
antennavstämningsenhet och så vidare.
Bäst är att göra sådana impedansanpassningar i alla led, att antennkopplaren
blir onödig.

\clearpage
\subsection{Effektförluster}
\harecsection{\harec{a}{6.3.7}{6.3.7}}
\index{transmissionsledning!effektförlust}

I varje matarledning uppstår förluster, dels av resistansen i ledarna
och dels i isolationsmaterialet (dielektrikum) mellan ledarna samt i
någon mån av fältutstrålning från dem.
De mest påtagliga effektförlusterna i en ledning beror av förlusterna per
längdenhet och därmed även av längden.
Vidare beror förlusterna av ståendevågförhållandet på ledningen på grund av
dålig impedansanpassning.

Ett högt SVF-förhållande ger större ledningsförluster eftersom den
reflekterade effekten då pendlar fler gånger på ledningen.
Den reflekterade effekt som återvänder till ledningens början är mindre när
ledningen har stora förluster än om den inte hade det.
Det medför att det verkliga SVF-förhållandet i ledningens slut är
större än vad som syns på ett instrument i början.

Förlusterna i en transmissionsledning stiger med ökad frekvens och anges av
tillverkarna i datablad som dämpningen i dB per 100~m eller dB per 30~m ledning.

I tabell~\ssaref{Kabeldämpning} visas kabeldämpningen, effektförlusten, i dB per
\qty{30}{\metre} för några vanliga typer av koaxialkablar.

\begin{table*}[!ht]
  \begin{center}
\begin{tabular}{|l|l|c|c|c|c|c|c|c|} \hline
	\text{Kabeltyp} & \text{Impedans} & 30 & 50 & 100 & 145 & 150 & 440 & 450 \\
	 & & \text{MHz} & \text{MHz} & \text{MHz} & \text{MHz} & \text{MHz} & \text{MHz} & \text{MHz}\\ \hline
	\text{RG8X} & 50 \text{ohm} & 2,0 & 2,1 & 3,0 & 4,5 & 4,7 & 8,1 & 8,6 \\ \hline
	\text{RG58A/U} & 50 \text{ohm} & 2,5 & 4,1 & 5,3 & 6,1 & 6,1 & 10,4 & 10,6 \\ \hline
	\text{RG59} & 75 \text{ohm} & & 2,4 & 3,5 & & & 7,6 & \\ \hline
	\text{RG174} & 50 \text{ohm} & 5,5 & 6,6 & 8,8 & 13,0 & & 25,0 & \\ \hline
	\text{RG213} & 50 \text{ohm} &  & 1,5 & 2,1 & 2,8 & 2,8 & 5,1 & 5,1 \\ \hline
	\text{RG214} & 50 \text{ohm} & 1,2 & 1,6 & 1,9 & 2,8 & 2,8 & 5,1 & 5,1 \\ \hline
\end{tabular}
\caption{Kabeldämpning per 30 m}
\label{tab:kabeldaempning}
\end{center}
\end{table*}

\subsection{Baluner -- Balansering -- Transformering}
\harecsection{\harec{a}{6.3.8}{6.3.8}}

\subsubsection{Balansering}
\index{balun}
\label{antenner_balansering}

Man skiljer mellan symmetriska ledningar (bandkabel m.fl.) och osymmetriska
(koaxialkabel), där dessutom den ena ledaren (skärmen) ofta är jordad.

På samma sätt finns det symmetriska antenner (dipol, W3DZZ m.fl.) och
osymmetriska (ground plane, Marconi m.fl.).

Vill man ansluta en symmetrisk (mittmatad) antenn till en osymmetrisk
ledning (koaxialkabel), så måste en strömbalansering göras i övergången.
Om inte, så kommer matarledningen att stråla, vilket kan medföra störningar på
radio och TV.
Utan balansering kommer dessutom dipolens strålningsbild inte att vara
symmetrisk.

En balansering måste också göras i övergången mellan en bandkabel (symmetrisk)
och sändaren när den har anslutning för koaxialkabel (osymmetrisk).
Balansering av impedans och därmed ström sker med en
anordning kallad BAL UN (av de engelska orden BALanced-UNbalanced).

Baluner kan utföras på flera sätt.
Grundläggande har balunen lika in- och utgångsimpedans,

\noindent
\textbf{Exempel:}
\begin{itemize}
\item Ringkärnebalun 1:1 för balansering.
\item Koaxialledare anordnad som balun 1:1.
\end{itemize}

\subsubsection{Transformering}

\tallfig{images/cropped_pdfs/bild_2_6-28.pdf}{Balansering -- transformering}{fig:bildII6-28}

I samband med balanseringen kan en impedanstransformering behövas och det finns
baluner (transformatorer) som både balanserar och transformerar impedanser.

Bild~\ssaref{fig:bildII6-28} visar en transformator med osymmetrisk ingång och
symmetrisk utgång.
Om båda lindningarnas varvtal är lika så sker ingen impedanstransformering.
Om förhållandet mellan varvtalen är 1:2 så blir förhållandet mellan
impedanserna 1:4.
Se vidare i avsnitt~\ssaref{sec:transformator}.

Bilden visar också att matarledningens impedans \(Z\) transformeras om så
att den blir lika antennens anslutningsimpedans \(R_a\).
Denna transformering kan ske induktivt eller kapacitivt.

\noindent\textbf{Exempel:}
\begin{itemize}
\item Ringkärnebalun 1:4.
\item Koaxialledare anordnad som balun 1:4.
\end{itemize}

\subsection{Ringkärnebalun}
\index{balun!ringkärna}

Bild~\ssaref{fig:bildII6-29} visar en \emph{ringkärnebalun} som är en form av
transformator.
I den finns en ringkärna av hårt sammanpressat järnpulver av en legering, som
tillsammans med lindningarnas utförande gör att frekvensbandbredden blir stor.

\smallfig{images/cropped_pdfs/bild_2_6-29.pdf}{Ringkärnebalun}{fig:bildII6-29}

\subsection{Koaxialledare som balun}
\index{balun!koaxial}

Balansering kan även göras med ett ett koaxialkabelarrangemang, som i
så fall är starkt frekvensberoende.
Bild~\ssaref{fig:bildII6-30} visar tre utföranden, som alla arbetar enligt
principen för en matarledning med en elektrisk längd av \(\lambda/4\) och
kortsluten i ena änden.

Den mekaniska längden är \(k\cdot\lambda/4\), varvid \(k\) är hastighetsfaktorn
för våghastigheten i kabeln.
För de vanligaste koaxialkablarna RG-58 och RG-213 är \(k\) cirka 0,66.
\(\lambda/4\)-ledningen i den översta figuren fungerar som en
parallellresonanskrets med mycket hög impedans \(Z\) i den öppna övre änden.

I den mellersta figuren är den översta delen av matningskabeln en
\(\lambda/4\) lång parallellresonanskrets tillsammans med parallellt
ansluten ledare (i detta fall en koaxialkabel som kortslutits i båda ändar).
Den nedersta högra figuren i bilden visar den kortslutna
\(\lambda/4\)-ledningen i tre varianter.
I samtliga fall uppstår HF-mässigt en strömbalanserande effekt mellan
dipolhalvorna.

Dessutom hindras även antennströmmar från att komma ner på utsidan av
matningskabelns skärm.

\multifig[0.45]{images/cropped_pdfs/bild_2_6-30_1.pdf,images/cropped_pdfs/bild_2_6-30_2.pdf,images/cropped_pdfs/bild_2_6-30_3.pdf}{Koaxialledare som balun}{fig:bildII6-30}

\mediumfig{images/cropped_pdfs/bild_2_6-31.pdf}{Sätt att ansluta en matningsledning}{fig:bildII6-31}

\newpage
% k7per: Rewrite all these uppbyggnad nd funktion paragraphs, so the text flows better.
\subsection{Sätt att ansluta en matningsledning}
\harecsection{\harec{a}{6.3.9}{6.3.9}}
\index{antenn!anpassning}

Bild~\ssaref{fig:bildII6-31} visar flera sätt att ansluta en matningsledning.

\subsubsection{T-, delta- och gamma-anpassning}

Funktion:
En mittmatad halvvågsdipol har i fria rymden en impedans av cirka \qty{73}{\ohm}.

Flyttas matningspunkten bort från mitten, åt det ena eller andra hållet, så är
impedansen högre än i mitten.

Det finns alltid två symmetriskt liggande punkter på antennen där
impedansen är precis lika stor.

T-, delta- och gamma-anpassning är användbar när matarkabelns är högre
än antennens mittpunktsimpedans.
Matningsledningen kan anslutas till de punkter på antennen som har samma
impedans som matarledningen.
T-anpassning används för symmetriska matarledningar, gamma-anpassning för
osymmetriska ledningar och delta-anpassning för båda ledningstyperna.

\subsubsection{\(\lambda/4\)-anpassningsledning -- stub}

Uppbyggnad: Antennen ansluts till en \(\lambda/4\) anpassningsledning
och matarledningen i sin tur till anpassningsledningen.

Funktion: Anpassningsledningen består av en öppen \(\lambda/4\)-matarledning.
Den har teoretiskt impedansen \(Z = 0\) den ände som är ansluten till antennen
och \(Z = \infty\) den andra.
Utmed anpassningsledningen finns alltid en impedans som är lika matarledningens
impedans.

\subsubsection{$\lambda/2$-fasningsledning}

\smallfig{images/cropped_pdfs/bild_2_6-32.pdf}{$\lambda/2$-fasningsledning}{fig:bildII6-32}

Bild~\ssaref{fig:bildII6-32} visar en $\lambda/2$-fasningsledning.

Funktion: När till exempel en omvikt dipol med matningsimpedansen \qty{240}{\ohm}
ska anslutas till en \qty{50}{\ohm}-kabel, behövs en impedanstransformering med
förhållandet 4:1.
En \(\lambda/2\) lång fasningsledning kan användas för detta ändamål.
Fasningsledningen har dessutom en strömbalanserande verkan.

Observera: Med en \(\lambda/2\)-fasningsledning enligt bilden kan
impedanstransformering endast göras i förhållandet 4:1.

\subsection{Transmissionsledningen}
\index{transmissionsledning}

En transmissionsledning för radiofrekvent energi består av två elektriska
ledare.
Den enklaste formen av en sådan ledning är två parallella ledare.
En annan form av transmissionsledning är koaxialkabeln, där den ena ledaren
löper inuti den andra.

Försök: Koppla en parallelledning till utgången på en VHF-sändare --
till exempel med induktiv koppling.
Ge ledningen passande längd och mata ut högfrekvent energi på ledningen.
Nu kan fördelningen mellan spänning och ström på olika punkter utmed ledningen
undersökas.
När det finns en spänning mellan de två ledarna i ledningen alstras det ett
elektriskt fält mellan dem.

Eftersom en glimlampa lyser när den omges av ett elektriskt fält kan den
användas som en enkel spänningsindikator.
När en elektriskt ledande krets -- en induktionsslinga -- omges av ett
varierande magnetiskt fält alstras det en ström i slingan.
Med en glödlampa inkopplad i slingan kan den användas som en enkel
strömindikator.

\subsubsection{Öppen transmissionsledning}
\index{transmissionsledning!öppen}

\mediumminustopfig{images/cropped_pdfs/bild_2_6-33.pdf}{Förlopp i öppen $\lambda/4$ transmissionsledning}{fig:bildII6-33}

Bild~\ssaref{fig:bildII6-33} visar Förlopp i öppen $\lambda/4$
transmissionsledning.
Håll glimlampan nära intill ledningen.
Glimlampan tänds med jämna mellanrum när den flyttas utmed ledningen.

När i stället en induktionsslinga med glödlampa hålls nära intill ledningen,
kommer glödlampan att lysa mitt emellan de ställen där glimlampan lyser.
Där glimlampan tänder har det bildats spänningsmaximum och där glödlampan lyser
har det bildats strömmaximum.
Det har bildats en stående våg på ledningen.

Bilden visar ström- och spänningsfördelningen för en öppen transmissionsledning
med längden \(l = n\cdot\lambda/4\) med udda \(n = 1, 3, 5 \dots\).
För bilden har valts $n = 5$.

Utmed ledningen uppstår omväxlande elektriska och magnetiska fält allt
efter som svängningen fortsätter.
Med en serie om fyra figurer visas förloppet av en svängning, en period.
Skillnaderna i den elektriska fältstyrkan framställs som olika långa fältlinjer.
Observera fältlinjernas riktning.

Skillnaderna i den magnetiska fältstyrkan kan också utläsas ur
bilderna i form av antalet symboler ''\(\cdot\)'' respektive ''\(\times\)''.
Båda tecknen betecknar elektromagnetiskt fält, ''\(\cdot\)'' i riktning ut ur
papperet och ''\(\times\)'' in i papperet.
För tydlighetens skull skildras endast den elektromagnetiska fältstyrkan
mellan ledarna och inte utanför ledarparet.

\subsubsection{Kortsluten transmissionsledning}
\index{transmissionsledning!kortsluten}

\mediumminustopfig{images/cropped_pdfs/bild_2_6-34.pdf}{Förlopp i kortsluten $\lambda/4$ transmissionsledning}{fig:bildII6-34}

På bild~\ssaref{fig:bildII6-34} visas såväl ström- och spänningsförhållandena
som fältlinjeförloppen på en avstämd, kortsluten transmissionsledning med
längden \(l = \lambda/4\) med jämna \(n = 2, 4, 6, 8 \dots\).
För bilden har valts \(n = 6\).

\subsection{$\lambda/4$-ledning som resonanskrets}

% \largefig{images/cropped_pdfs/bild_2_6-35.pdf}{$\lambda/4$ transmissionsledning som resonanskrets}{fig:bildII6-35}

Bild~\ssaref{fig:bildII6-35} visar ström- och spänningsfördelningen för en öppen
respektive en kortsluten transmissionsledning med längden \(l = \lambda/4\).

Den öppna \(\lambda/4\)-ledningen har en strömbuk i ingångsänden.
En sådan ledning måste således strömkopplas, det vill säga den anslutande
impedansen måste vara låg.

Den kortslutna \(\lambda/4\)-ledningen har en spänningsbuk i ingångsänden.
En sådan ledning måste spänningskopplas, det vill säga den anslutande
impedansen måste vara hög.

En öppen \(\lambda/4\)-ledning kan ses som en seriekopplad LC-krets.
När ledningen är i resonans flyter en hög ström i ingången, medan spänningen
där är låg.

En kortsluten \(\lambda/4\)-ledning kan ses som en parallellkopplad LC-krets.
När ledningen är i resonans är spänningen hög över ingången, medan strömmen där
är låg.

\largefig{images/cropped_pdfs/bild_2_6-35.pdf}{$\lambda/4$ transmissionsledning som resonanskrets}{fig:bildII6-35}

\newpage
\subsection{Antennkopplare}
\index{antenn!anpassning}

Bild~\ssaref{fig:bildII6-36} visar en antennkopplare för bandkabel av olika
längder.
Storleken på kondensatorerna: \(C_1 = C_2 = \qty{500}{\pico\farad}, C_3 =
\qty{300}{\pico\farad}\).

\subsubsection{Avstämning vid spänningskoppling}
\index{antenn!avstämning}

\(C_1\) och \(C_2\) helt invridna eller kortslutna, \(C_3\) avstäms
för resonanstillstånd (parallellresonans).

\subsubsection{Avstämning vid strömkoppling}

\(C_3\) helt utvriden, \(C_1\) och \(C_2\) avstäms för
resonanstillstånd (serieresonans), med maximal och lika ström i båda ledarna.

Matarledningen kan förlängas elektriskt med induktanser när den är för
kort för att kunna avstämmas.

Märk, att en antennkopplare mycket väl även kan utformas för
koaxialkabelutgång.

\subsection{För- och nackdelar med avstämd matarledning}

När en matarledning är rätt avstämd transporterar den energi utan att
stråla själv.

När dipolen kopplas till en avstämd matarledning, kan den med hjälp av
en antennkopplare arbeta på flera amatörradioband.
Detta är en anledning till varför en avstämd matarledning gärna används för
portabla installationer (t.ex. för field days).
Injusteringen mot sändaren blir enklare.

Inom amatörradion används numera nästan uteslutande koaxialkabel som
matarledning i stället för bandkabel.
Detta är av flera skäl:

\begin{itemize}
\item En bandkabel måste hängas upp så fritt som möjligt och den får
  inte komma för nära murutsprång, takrännor osv.
  Vidare måste den isoleras väl vid genomföringar i väggar.

\item De flesta sändaramatörer har inte plats med långa matarledningar
  (\(n\cdot\lambda/4\) med \(n = 1, 2, 3 \dots\)).

\item Vid tvära bockar på ledningen kan det upp stå oönskad
  utstrålning och därmed risk för störningar på radio och TV med mera.
\end{itemize}

\tallfig[0.35]{images/cropped_pdfs/bild_2_6-36.pdf}{Antennkopplare}{fig:bildII6-36}

%
%
% Kapitel 8 Vågutbredning
\chapter{Vågutbredning}
\label{vågutbredning}
\index{vågutbredning}

Elektromagnetisk vågutbredning är energitransport och förutsättningen för all
radiokommunikation.
Radiovågornas utbredning på vägen mellan sändare och mottagare påverkas
emellertid på många sätt.
Med vetskap om radiovågornas utbredningssätt kan man mer metodiskt försöka uppnå
önskade radioförbindelser.

% Avsnitt 8.1 Kraftfält antenner
\section[Kraftfält antenner]{Kraftfälten omkring antenner}
\index{kraftfält}
\index{antenner!kraftfält}

För att sända ut och ta emot radiovågor behövs antenner.
Mycket förenklat är en antenn en elektrisk krets, som består av en induktor
och en kondensator som illustreras i bild~\ssaref{fig:BildII7-01}.

Med kondensatorns elektroder helt isärdragna och förminskade har
resonanskretsen fått ett mycket annorlunda mekaniskt utseende.
Sedan induktorn i LC-kretsen tagits bort, så återstår mekaniskt sett endast
en enkel ledare, men elektriskt sett finns kretsen ändå kvar.
Ledaren med sin utsträckning är fortfarande en induktor och ytorna på dess
motstående halvor är fortfarande elektroderna i kondensatorn med
omgivningen som dielektrikum.

En elektrisk ledare, en stång, tråd etc. är alltså en elektrisk
resonanskrets, vars resonansfrekvens mest bestäms av längden och
tjockleken. Ledaren (antennen) kan kallas dipol -- den har två poler,
detta är grunden för alla typer av antenner.

\mediumplusbotfig{images/cropped_pdfs/bild_2_7-01.pdf}{Från sluten LC-resonanskrets till antenn}{fig:BildII7-01}
\mediumfig{images/cropped_pdfs/bild_2_7-02.pdf}{Pendlingen mellan E-fält och H-fält}{fig:BildII7-02}

Det finns vissa likheter mellan en mekanisk pendel och en elektrisk
resonanskrets.
% k7per: Tar bort denna titel...
%\subsubsection{Mekanisk pendel}
%
Energin i en mekanisk pendel växlar mellan två ytterlighetstillstånd.
Det ena är när pendeln just vänder i ett ytterläge.
Då innehåller den enbart lägesenergi och ingen rörelseenergi.
När pendeln rör sig mot mittläget, så omvandlas lägesenergin till rörelseenergi.
I mittläget, som är det andra ytterlighetstillståndet, innehåller pendeln enbart
rörelseenergi och ingen lägesenergi etc.

\subsubsection{Elektrisk resonanskrets}
\index{kraftfält!elektrisk resonanskrets}
\index{elektrisk resonanskrets}
\index{Maxwell}

Den elektriska resonanskretsen kan jämföras med den mekaniska pendeln där det
hela tiden pågår en pendling eller omvandling mellan lägesenergi och
rörelseenergi.
Se bild~\ssaref{fig:BildII7-02}.

När strömmen i den elektriska resonanskretsen just upphört för att vända så
innehåller kondensatorn mest laddning, det vill säga, det starkaste elektriska
fältet mellan elektroderna.
Detta fält kan jämföras med pendelns lägesenergi.
Den utjämningsström som följer från den ena elektroden över till den andra
omges av ett magnetiskt fält som kan jämföras med pendelns rörelseenergi.

Förloppet visas i bild~\ssaref{fig:BildII7-02}, där det framgår att dipolen omges
av det starkaste elektriska fältet vid tidpunkten \(t=0\) samt vid
\(t=1/2T\) med omvänd polaritet, där T är periodtiden.
Vidare att dipolen omges av det starkaste magnetiska fältet vid tidpunkten
\(t=1/4T\) samt vid \(t=3/4T\) med omvänd strömriktning och fältpolaritet.


\smallfig{images/cropped_pdfs/bild_2_7-03.pdf}{Elementär dipol}{fig:BildII7-03}

Med förklaringen av E- och H-fälten som bakgrund följer nu en enkel
framställning av hur radiovågor uppstår ur dessa fält.

Maxwell påvisade i sina ekvationer bland annat sambandet mellan elektroner
i rörelse i en ledare och elektromagnetiska vågor i rummet.
Vidare, att elektroner som rör sig med avtagande eller tilltagande hastighet
avger elektromagnetisk energi.

Hur energi strålar från en ledare kan förklaras med en (tänkt)
elementär dipol, som genomflyts av växelström (Bild~\ssaref{fig:BildII7-03}).

Dipolen består av två lika stora elektriska laddningar med motsatt polaritet.
När den matas med en växelström, så rör sig laddningarna ständigt,
omväxlande emot respektive ifrån varandra.
Tänk på två kulor i var sin ände av en spiralfjäder.
Avståndet mellan laddningarna ändras i takt med styrkan och riktningen på
strömmen.
Systemet är alltså under ständig hastighetsändring (ökning respektive
minskning), vilket är förutsättningen för att energi ska strålas ut.

Först är laddningarna nära varandra på grund av liten laddning.
Vid ökande ström ökar avståndet mellan laddningarna och det byggs upp ett
mer utbrett och energirikt E-fält.
Samtidigt byggs även ett H-fält upp omkring dipolen, vinkelrätt mot E-fältet
och så vidare.
Detta gäller både för en elementär dipol och en elektrisk ledare med många fria
elektroner (verklig antenn).

Formeln för det resulterande S-fältet är \(\overline{S} =
\overline{E}\times\overline{H}\), vilket visar att den lagrade energin
i dipolens närmaste omgivning ökar när avståndet (potentialen) mellan
dipolens laddningar ökar.

Bild~\ssaref{fig:BildII7-04} visar hur ett E-fält byggs upp omkring en dipol och
avskiljs från den.
De visade kraftlinjerna är E- fältet.
H-fältet visas inte, men ligger vinkelrätt mot E-fältet, i cirklar omkring
antennen. Se bild~\ssaref{fig:BildII7-05}.

\mediumfig{images/cropped_pdfs/bild_2_7-04.pdf}{Ett självständigt E-fält skapas}{fig:BildII7-04}

\mediumfig{images/cropped_pdfs/bild_2_7-05.pdf}{E-, H- och S-fälten omkring en antenn (förenklad framställning)}{fig:BildII7-05}

När dipolens laddningar ändrar riktning och åter börjar att röra sig
emot varandra, börjar det E-fält som byggts upp att också byta riktning.
Men det kommer inte att falla tillbaka till dipolens mitt
utan sluts till ett eget kretslopp -- Maxwells första ekvation.
Jämför med en såpbubbla som lämnat blåsröret.
Omkring dipolen har det nu bildats ett självständigt E-fält, som sin tur
alstrar ett eget H-fält.

En period av en elektromagnetisk våg (ett S-fält) har alstrats och
fortsätter att utvidga sig.
För varje följande period alstras ett nytt E-fält, som separeras från antennen
och bildar ett H-fält och så vidare.
Varje gång bildas alltså en ny ''fältbubbla'' inne i den föregående, vilken
håller på att utvidgas.
Resultatet är ett elektromagnetiskt fält, det vill säga en radiovåg.

Som nämnts består en radiovåg av ett högfrekvent elektromagnetiskt fält (S).
Det är i sin tur sammansatt av två andra fält, det elektriska E-
och det magnetiska H-fältet.
Energin i S-fältet fördelas lika mellan E-fältet och H-fälten,
vars krafter korsar varandra vinkelrätt.
S-fältet ligger i plan med både E- och H-fälten och breder ut sig vinkelrätt
mot dem.
S-fältets riktning beror av den inbördes riktningen på E- och H-fälten.

När E-fältet är vertikalt, sägs vågen vara vertikalt polariserad.
När samma fält är horisontellt sägs vågen vara horisontellt polariserad.
När E-fältet roterar i vågfrontens plan, och därmed även H-fältet, sägs vågen
vara cirkulärt polariserad.

Fälten framställs i text och bild som så kallade kraftlinjer med pilar som
föreställer kraftriktningen.
Linjernas längd föreställer fältets styrka.
Bild~\ssaref{fig:BildII7-06} visar ett avsnitt av en vågfront S med vertikal
polarisation.

\smallfig{images/cropped_pdfs/bild_2_7-06.pdf}{E-, H- och S-fält}{fig:BildII7-06}

% Avsnitt 8.2 Radiovågornas egenskaper
\input{koncept/chapter8-2}
% Avsnitt 8.3 Jonosfärskikten
\section{Jonosfärskikten}
\harecsection{\harec{a}{7.3}{7.3}}
\index{jonosfär}
\index{jonosfärsskikt}
\label{vågutbredning_jonosfärskikten}

På höga höjder kan atomer och molekyler färdas långa sträckor utan att
kollidera, de skiktas då genom gravitationens inverkan så att de lättare
atomerna lägger sig över de tyngre.
Kraftig solinstrålning slår loss elektroner från atomerna så att det bildas
positivt laddade atomkärnor och fria elektroner \emph{jonisering}.

Dessa joniserade skikt som delvis består av elektriskt ledande gas har
gett namn åt \emph{jonosfären}.

När en radiovåg passerar genom ett joniserat skikt i atmosfären, kan vågen
ändra riktning, vilket kallas för refraktion.
För att refraktion ska uppstå måste i första hand två villkor uppfyllas, det
första är tillräckligt tät jonisering och det andra är tillräckligt
lång våglängd.
Under ''gynnsamma'' omständigheter kan vågorna till och med böjas av
ner mot jorden, vilket är den viktigaste förutsättningen för långväga
radioförbindelser på kortvåg.

Joniseringen av atmosfären är emellertid oregelbunden och varierar
bland annat med höjden över jordytan, solinstrålning, tidpunkt på dygnet,
årstiden med mera.
Ett antal joniserade skikt kan definieras.
Se bild~\ssaref{fig:bildII7-7}.

\subsection{D-skiktet}
\index{jonosfärsskikt!D-skiktet}
\label{d-skiktet}

\emph{D-skiktet} förekommer under den ljusa delen av dygnet på en höjd av cirka
\SIrange{50}{90}{\kilo\metre}.
På \SIrange{70}{90}{\kilo\metre} höjd orsakas joniseringen huvudsakligen av
röntgenstrålar från solen, medan den kosmiska strålningen har störst påverkan på
\SIrange{50}{70}{\kilo\metre} höjd.
D-skiktet dämpar de infallande radiovågorna, med största verkan i
kortvågsområdets lågfrekventa del och under de ljusaste timmarna under sommaren.
D-skiktet har dålig reflektionsförmåga och verkar hindrande på
långdistansförbindelser.

\subsection{Mögel-Dellinger-effekten}
\index{Mögler-Dellinger-effekten}
\index{jonosfärsskikt!black out}

Strålning från gasutbrott på solytan kan jonisera D-skiktet så kraftigt, att
alla radiovågor med frekvenser över cirka \qty{1}{\mega\hertz} dämpas helt,
detta kallas för \emph{Mögel-Dellinger-effekten}.
Radiotrafik som baseras på vågutbredning via jonosfären är då
omöjlig att genomföra under en tidsrymd av ett antal minuter upp till
flera timmar -- det blir ''black out''.

\subsection{E-skiktet}
\index{jonosfärsskikt!E-skiktet}
\label{e-skiktet}

\emph{E-skiktet} (Kenelly-Heaviside-skiktet) är det lägsta reflekterande
jonosfärskiktet.
Det förekommer på en höjd av cirka \SIrange{90}{140}{\kilo\metre} och är mest
koncentrerat på cirka \qty{110}{\kilo\metre} höjd.
E-skiktet alstras av att ultraviolett strålning joniserar syreatomer.
Skiktet reflekterar vågor bäst i kortvågsområdets lågfrekventa del och är
kraftigast under den ljusa delen av dygnet.
På grund av D-skiktets dämpande verkan under de ljusaste timmarna är E-skiktet
mest användbart under grynings- och skymningstimmarna.

Ett säsongmaximum i reflektionsförmågan inträffar under sommaren.
Förbindelseavstånd på upp till \qty{2000}{\kilo\metre} är möjliga.

\subsection{Sporadiska E-skiktet}
\index{jonosfärsskikt!sporadiska E-skiktet}
\index{sporadiska E-skiktet}
\label{sporadiskt_e}

Den starkare solinstrålningen under sommaren orsakar en kraftigare
jonisering i den lägre jonosfären än under vintern.
Inom E-skiktet bildas då sporadiska tunna molnlika partier med mycket hög
joniseringsgrad och stor reflektionsförmåga, det så kallade \emph{sporadiska
E-skiktet} (\(E_s\)).
Vågutbredningen via \(E_s\) är mycket olika på olika latituder och är bäst
omkring 40:e latituden.
Mycket goda långväga förbindelser kan uppnås.

\mediumfig{images/cropped_pdfs/bild_2_7-07.pdf}{Jonosfärskikten}{fig:bildII7-7}

\subsection{F-skiktet}
\index{jonosfärsskikt!F-skiktet}
\index{F-skiktet}
\index{F1@\(F_1\)}
\index{F2@\(F_2\)}

\emph{F-skiktet} är det högst liggande jonosfärsskiktet.
Det förekommer såväl dag- som nattetid på en höjd av
\SIrange{140}{500}{\kilo\metre}.
Den nedre del av skiktet, \SIrange{140}{200}{\kilo\metre}, uppvisar andra
variationer än den övre delen.
F-skiktet beskrivs därför som två skikt, \(\mathrm{F_1}\) upp till cirka
\qty{200}{\kilo\metre} höjd och \(\mathrm{F_2}\) över denna höjd.

Liksom E-skiktet, påverkas \(\mathrm{F_1}\)-skiktet kraftigt av
instrålningen från solen.
Det når sin högsta joniseringsgrad ungefär en timme efter högsta lokala
solståndet och förekommer endast under sommaren.
Under natten förenar sig \(\mathrm{F_1}\)- och \(\mathrm{F_2}\)-skikten till
ett enda F-skikt.

\(\mathrm{F_2}\)-skiktet är det skikt som varierar mest i tiden och rummet.
Den högsta joniseringsgraden inträffar vanligen sent efter högsta lokala
solståndet, ibland under aftontimmarna.
Skiktets maximala jonisering är på \SIrange{250}{350}{\kilo\metre} höjd på
mellanlatituder och på \SIrange{350}{500}{\kilo\metre} höjd vid ekvatorn.
På mellanlatituder ligger den största elektrontätheten i skiktet högre under
natten än under dagen.
Vid ekvatorn är förhållandet omvänt.

Reflektioner i \(\mathrm{F_2}\)-skiktet möjliggör att stora
avstånd kan överbryggas (1 hopp = \SIrange{3000}{4000}{\kilo\metre}).

\mediumfig{images/cropped_pdfs/bild_2_7-08.pdf}{Jonosfärsutbredning.}{fig:bildII7-8}

\subsection{Höjd till reflekterande skikt}
\index{jonosfärsskikt!höjd till skikt}

När en radiovåg, som riktas rakt uppåt, träffar jonosfären kan den antingen
\begin{itemize}
  \item absorberas -- sugas upp
  \item reflekteras
  \item tränga igenom.
\end{itemize}

Vilket som inträffar beror på den använda frekvensen.
Ju högre frekvensen är på den uppåtriktade radiovågen, desto högre upp i ett
atmosfärskikt kommer avböjningen tillbaka att inträffa.
Höjden till skiktet beräknas ur radiovågens utbredningshastighet och
utbredningstid fram och åter mellan skiktet och jordytan.

\subsection{Kritisk frekvens}
\harecsection{\harec{a}{7.4}{7.4}}
\index{kritisk frekvens}

Vid en viss övre frekvens upphör reflektionen i atmosfärskiktet och
vågen går ut i rymden i stället för ner till jordytan.
Denna frekvens kallas den \emph{kritiska frekvensen}, som varierar med
joniseringsgraden i atmosfären.
Den kritiska frekvensen är högst vid högt solfläckstal, såväl i E- som i
F-skikten, eftersom joniseringsgraden då är störst.
Den kritiska frekvensen för E-skiktet varierar mellan cirka
\SIrange{1}{4}{\mega\hertz} beroende på tidpunkt i solfläckscykeln och tid på
dagen.
Den kritiska frekvensen för F-skiktet varierar med tid på dagen, årstid och
skede i solfläckscykeln.
Den kan variera från \SIrange{2}{3}{\mega\hertz} natten under ett
solfläcksminimum till \SIrange{12}{13}{\mega\hertz} på dagen under ett
solfläcksmaximum.

\subsection{Kritisk vinkel}
\index{kritisk vinkel}

Rymdvågen måste träffa ett joniserat atmosfärskikt med en tillräckligt
flack vinkel för att reflekteras, den så kallade kritiska vinkeln.
Denna vinkel är frekvensberoende.
Allt eftersom den utsända frekvensen ökas ytterligare över den kritiska
frekvensen, måste vågen träffa atmosfärskiktet i en allt flackare vinkel för att
vågen ska reflekteras mot jordytan.
Genom att sända ut vågen i mycket flack vinkel mot \(\mathrm{F_2}\)-skiktet kan
långa distanser överbryggas vid frekvenser som är upp till 3,5 gånger den
kritiska frekvensen.

Så snart den kritiska frekvensen är högre än frekvensområdet för ett
amatörband är det alltså möjligt att kommunicera över rymdvåg på detta band.
Det kan ske över alla avstånd, allt ifrån skipavståndet till det som avgörs av
utbredningsförlusterna.

\subsection{Högsta användbara frekvens (MUF)}
\harecsection{\harec{a}{7.6}{7.6}}
\index{högsta användbara frekvens}
\index{MUF|see {högsta användbara frekvens}}
\index{Maximum Usable Frequency (MUF)|see {högsta användbara frekvens}}

Radiovågorna vandrar från sändaren till en avlägsen mottagare genom att
reflekteras en eller flera gånger i jonosfären och på jordytan.
Se bild~\ssaref{fig:bildII7-8}.
För att detta ska ske kan frekvensen inte vara högre än den
\emph{högsta användbara frekvensen} (eng. \emph{Maximum Usable Frequency, MUF})
för en viss överföringssträcka.

MUF är högst mitt på dagen eller tidig eftermiddag.
Allra högst är MUF under perioder av högt solfläckstal och kan då komma
upp till över \qty{30}{\mega\hertz}.
Under tidiga morgontimmar sjunker MUF ofta under \qty{5}{\mega\hertz} och ibland
ännu lägre särskilt vintertid.

De jonosfäriska förlusterna är lägst nära MUF och ökar snabbt under
dagtid för lägre frekvenser.
Aktuella MUF-data publiceras periodiskt i olika media, men kan också
överslagsberäknas med hjälp av speciella datorprogram.

\subsection{Optimal trafikfrekvens (FOT)}
\index{optimal trafikfrekvens}
\index{FOT}

I praktiken är det av intresse att veta det frekvensområde där
kommunikation bäst kan genomföras.

Rekommenderad övre frekvensgräns för en tillförlitlig radioförbindelse
kallas \emph{optimal traffic frequency (FOT)} och väljs något under
MUF som marginal för oregelbundenheter och turbulens i jonosfären,
liksom för korttidsavvikelser från det förutsagda månatliga
medianvärdet för MUF.
FOT är vanligen ungefär \qty{15}{\percent} lägre än MUF.

\newpage
\subsection{Lägsta användbara frekvens (LUF)}
\index{lägsta användbara frekvens}
\index{LUF}
\index{Lowest Usable Frequency (LUF)}

Ju lägre sändningsfrekvens som väljs, desto mer dämpas vågorna i
jonosfären, intill den frekvens då de inte kan uppfattas.
Den \emph{lägsta användbara frekvensen}
(eng. \emph{Lowest Usable Frequency, LUF}) är den
frekvens som ger tillfredsställande kommunikation för en viss
utbredningsväg och vid en viss tidpunkt.

Vid frekvenser under LUF är mottagning inte möjlig eftersom brusnivån
då är för hög.
Ju mer frekvensen höjs över LUF, desto bättre blir signal-brus-förhållandet.

Till skillnad från MUF, som endast påverkas av de jonosfäriska
förhållandena, kan LUF till en del påverkas genom utsänd effekt och bandbredd.
Generellt kan LUF sänkas cirka \qty{2}{\mega\hertz} för varje
\qty{10}{\decibel} ökning av \erp.

\begin{figure*}[t]
  \begin{center}
    \includegraphics[width=0.6\textwidth,angle=270]{images/cropped_pdfs/bild_2_7-09.pdf}
    \caption{Radioprognos för amatörradiobanden på kortvåg}
    %% k7per: konverted from fig macro?
    %% {fig:bildII7-9}
    \label{fig:bildII7-9}
  \end{center}
\end{figure*}

\subsection{Vågutbredningsförutsägelser}
\index{vågutbredningsförutsägelser}
\index{vågutbredning!förutsägelser}
\index{vågutbredning!propagation forecasts}
\index{ursigram}
\index{URSI}
\index{Union Radio-Scientifique Internationale (URSI)}

Det görs regelmässiga förutsägelser av de jonosfäriska förhållandena.
Fortlöpande fysiska observationer, statistisk och matematisk bearbetning ligger
till grund för förutsägelserna, vilka bland annat utnyttjas för att planera
radiotrafiken.
Vågutbredningsförutsägelser (eng. \emph{propagation forecasts}) som ger
upplysningar om de lämpligaste frekvenserna och tiderna för olika
förbindelsesträckor görs av både civila och militära institutioner.
Tidigare meddelades sådana förutsägelser i offentliga publikationer samt i
amatörradions tidskrifter och bulletiner.

Idag finns i stort sett alla vågutbredningsförutsägelser, data över
solaktiviteten och information om det geomagnetiska fältet fritt tillgängligt på
internet.

En sökning på nätet efter \emph{propagation forecast ham radio} ger träffar på
många webbplatser.
SSA:s webbplats har i högermarginalen information om \emph{solar terrestrial data}
och \emph{calculated conditions}.

Förr var vågutbredningsförutsägelser endast tillgängliga som \emph{Ursigram} som
skickades per telex eller brev från \emph{Union Radio-Scientifique
Internationale (URSI)}.

Ursigrammen kunde erhållas genom ett dyrt årsabonnemang och de innehöll aktuella
mätvärden såsom solfläckstal \(R\), \qty{10}{\centi\metre} solflux \(F\),
magnetiskt index \(K\) och gränsdämpningsvärden.
De kunde även innehålla anvisningar om särskilda händelser (flares,
magnetstormar, polarkalott absorption, Mögel-Dellinger-effekter och liknande).

Bild~\ssaref{fig:bildII7-9} visar en radioprognos för juni 1997 ur SSA:s
medlemstidning QTC.
Tabellen i bilden visar sannolikheten i procent för att få förbindelse på de
olika kortvågsbanden från Sverige till andra länder och världsdelar.

Trots sin höga ålder ger tabellen en bild över hur möjligheten att få en
förbindelse på kortvåg varierar med frekvens och tid på dygnet.
Observera även det låga solfläckstalet SSN (Sun Spot Number).


%%\largefig{images/cropped_pdfs/bild_2_7-09.pdf}{Radioprognos för amatörradiobanden på kortvåg}{fig:bildII7-9}

% Avsnitt 8.4 Solens inverkan
\input{koncept/chapter8-4}
% Avsnitt 8.5 Vågutbredning på kortvåg
\section{Vågutbredning på kortvåg}
\harecsection{\harec{a}{7.7}{7.7}}
\index{vågutbredning!kortvåg}
\index{vågutbredning}

Bild~\ssaref{fig:bildII7-11} illustrerar vågutbredning på kortvåg.

\subsection{Markvåg}
\index{markvåg}
\index{vågutbredning!markvåg}
\label{subsec:markvaag}

\emph{Markvågen} (eng. \emph{ground wave}) breder ut sig längs jordytan utan
kontakt med atmosfären genom reflektion eller refraktion.

Markvågen har vertikal polarisering och en vertikal vågfront när jordplanets
ledningsförmåga är god.
Vid sämre ledningsförmåga lutar vågfronten framåt.

Markvågens räckvidd står i förhållande till den använda frekvensen,
sändareffekten och jordplanets ledningsförmåga.

Vid frekvenser under cirka \qty{10}{\mega\hertz} är jordytan är en tämligen god
ledare.
Markvågsutbredning utnyttjas därför mest vid låga frekvenser, till exempel för
rundradio i lång- och mellanvågsbanden då räckvidden kan vara i
storleksordningen \qty{1000}{\kilo\metre}.
På kortvåg är markvågsräckvidden i 80-metersbandet cirka \qty{100}{\kilo\metre}
och i 10-metersbandet cirka \qty{15}{\kilo\metre}.

\subsection{Rymdvåg}
\index{rymdvåg}
\index{vågutbredning!rymdvåg}
\label{rymdvåg}

Under vissa förutsättningar reflekteras radiovågorna mot joniserade
atmosfärsskikt och når åter jordytan på stort avstånd från utsändningspunkten,
då kan man använda \emph{rymdvåg} (eng. \emph{space wave}).
Rymdvågsutbredning utnyttjas mellan platser på jordytan med stort avstånd.

För att bäst uppnå den önskade reflektionen måste man dels välja
lämplig tidpunkt och frekvens och dels utforma antennen så att den har
sin huvudriktning i en bestämd vinkel mot det reflekterande skiktet.

Jonosfären är den del av atmosfären på cirka 50 till \qty{350}{\kilo\metre} höjd,
där instrålningen från solen skapar fria elektroner och joner i en sådan mängd
att det bildas skikt med god elektrisk ledningsförmåga.
Under vissa villkor reflekterar dessa skikt radiovågorna, men kan under
andra villkor även absorbera dem.

\mediumfig{images/cropped_pdfs/bild_2_7-11.pdf}{Vågutbredning på kortvåg}{fig:bildII7-11}

När vågorna från jordytan reflekterats mot de joniserade skikten, kan
de återträffa jordytan på ett avstånd av upp till \qty{4000}{\kilo\metre} från
utsändningspunkten, beroende på frekvens och polarisering.
Därefter kan de åter reflekteras mot jordytan och upp i jonosfären
och så vidare (flerstegshopp).
Under gynnsamma förhållanden når rymdvågen mycket långt genom växelvisa
reflektioner mellan jordytan och jonosfären.

\subsection{Död zon (skip zone) och skip-avstånd}
\index{kritisk vinkel}
\index{vågutbredning!kritisk vinkel}
\index{död zon}
\index{vågutbredning!död zon}
\index{skip zone}
\index{vågutbredning!skip zone}
\index{skipavstånd}
\index{vågutbredning!skipavstånd}

Rymdvågorna böjs tillbaka mot jorden när de träffar jonosfären i en
vinkel som är flackare än den så kallade \emph{kritiska vinkeln}.
När vågorna träffar jonosfären med en brantare vinkel än den kritiska vinkeln
sker det ingen avböjning utan vågorna passerar genom jonosfären och rakt ut
i rymden.
Beroende på den kritiska vinkeln för tillfället, kommer därför reflekterade
rymdvågor inte att höras förrän på ett visst avstånd bort från sändaren.
Detta avstånd kallas för \emph{skip-avstånd}.

Men sändarens markvåg har också ett visst täckningsområde och mellan
detta och zonen där rymdvågen kan höras bildar en skymningszon, en
\emph{död zon} (eng. \emph{skip zone}).

\subsection{Grålinjeutbredning -- grayline}
\index{grålinjeutbredning}
\index{grayline}
\index{vågutbredning!grayline}

Med \emph{grålinjeutbredning} (eng. \emph{gray line}) menas det smala bälte på
jordytan där det för tillfället råder gryning eller skymning.

Tidintervallet för gray line varierar med stationens latitud.
Vid ekvatorn är det \(\pm\) 5~minuter och i Skandinavien \(\pm\) cirka
\(1\frac{1}{2}\) timme omkring tidpunkten för solens uppgång
respektive nedgång.

När åtminstone en av två stationer befinner sig inom gray line kan
kortvågsförbindelse erhållas över ett mycket större avstånd än annars.

Kommunikation längs med gray line går bäst på låga frekvenser, till exempel på
\qty{3,5}{\mega\hertz} amatörband, under det tidsintervall då D-skiktet just
har börjat byggas upp (gryning) respektive nästan har brutits ned (skymning).
Då är joniseringen av D-skiktet liten och en rymdvåg som
träffar skiktet kommer då snarare att böjas av i D-skiktet än att helt dämpas.
Vågutbredningen sker då både genom refraktion i D-skiktet och reflektion i
E-skiktet.

\subsection{Fädning eller signalbortfall}
\index{fädning}
\index{vågutbredning!fäding}
\harecsection{\harec{a}{7.8}{7.8}, \harec{a}{7.9}{7.9}}
\label{subsec:faedning}

Fältstyrkan på de mottagna vågorna kan variera kraftigt från ett ögonblick till
ett annat.
Fenomenet kallas \emph{fädning} (eng. \emph{fading}, uttalas fejding).

Sådana interferensfenomen uppstår när vågorna samtidigt vandrat flera
vägar fram till mottagarantennen, så kallad flervägsutbredning.
När de träffar mottagarantennen kan de vara tidsförskjutna sinsemellan, med
utsläckningseffekter som följd (interferensförluster).

\newpage % layout
\paragraph{Andra typer av fädning är när}
\begin{itemize}
\item polariseringriktningen ändras p.g.a. oregelbundenheter i
  jonosfären (polariseringsförluster)
\item överföringsvägen dämpar vågorna tidsmässigt oregelbundet
  (absorbtionsförluster)
\item vågutbredningsriktningen ändras genom reflektioner mot hus,
  bergväggar etc. (reflektionsförluster, vid t.ex. mobil radiotrafik).
\end{itemize}

\subsection{Om amatörradiobanden på kortvåg}
\label{om_kortvågsbanden}

En mer omfattande analys finns i \cite{ARRLHDB2015}.

\subsubsection{1,8 MHz (160 m)}
\index{vågutbredning!1,8 MHz}

Bandet kallas även ''top-band''.
Räckvidden är normalt relativt liten, nattetid under vintern cirka
\qty{1200}{\kilo\metre} och i bästa fall några tusen km.
Men under solfläcksminimum kan räckvidden vara mycket större nattetid.

\subsubsection{3,5 MHz (80 m)}
\index{vågutbredning!3,5 MHz}

Under dagtid är räckvidden cirka \qty{500}{\kilo\metre} och under kvällstid
\SIrange{1000}{1500}{\kilo\metre}.
Tidigt på morgonen under vintermånaderna, särskilt under solfläcksminimum, är
räckvidden tillräcklig för interkontinentala förbindelser (DX = long distance).
Under sommarmånaderna har bandet hög atmosfärisk brusnivå.
Döda zoner förekommer normalt inte.

\subsubsection{7 MHz (40 m)}
\index{vågutbredning!7 MHz}

Detta band har större räckvidd än 80\,m-bandet.
Under dagtid har det en räckvidd av \SIrange{1000}{2000}{\kilo\metre}.
Under natten, särskilt under vintern, kan hela världen nås.
Döda zoner är \qty{100}{\kilo\metre} under dagen och \qty{1000}{\kilo\metre}
under natten.

\subsubsection{10 MHz (30 m)}
\index{vågutbredning!10 MHz}

Detta band är unikt då det delar egenskaper med dag och nattband.
Kommunikation upp till \qty{3000}{\kilo\metre} via \(F_2\) reflektion är i
allmänhet möjlig.

\subsubsection{14 MHz (20 m)}
\index{vågutbredning!14 MHz}

20\,m-bandet är ett säkert DX-band för stora avstånd.
Under kvällarna ökar räckvidden på ett rymdvågshopp upp till cirka
\qty{4000}{\kilo\metre}.
Särskilt gynnsam vågutbredning erhålls vid kontakt genom en skymningszon, det
vill säga där den ena parten har dag och den andra har natt.
Döda zoner uppträder nästan alltid.

\subsubsection{18 MHz (17 m)}
\index{vågutbredning!18 MHz}

Detta band är i stora delar likt 20\,m-bandet, men \(F_2\) fluktuationerna är
kraftfullare.

\subsubsection{21 MHz (15 m)}
\index{vågutbredning!21 MHz}

Vågutbredningen i 15\,m-bandet är bäst vid högt solfläckstal.
Under solfläcksmaximum är bandet nästan ständigt öppet för DX-förbindelser.

Under solfläcksminimum är bandet i bästa fall öppet kortare perioder på dagtid
under sommarmånaderna.

Bandet är dött nattetid. Vid reflektioner via sporadiskt E-skikt kan avstånd av
mer än \qty{2000}{\kilo\metre} överbryggas.

\subsubsection{24 MHz (12 m)}
\index{vågutbredning!24 MHz}

Detta band kombinerar fördelarna med \qty{10}{\metre} och \qty{15}{\metre} banden.
Det är huvudsakligen ett dagband, men kan även vara öppet efter solnedgången.

\subsubsection{28 MHz (10 m)}
\index{vågutbredning!28 MHz}

Bandet är lämpat för närkontakter upp till \qty{50}{\kilo\metre} nattetid och
för DX-kontakter dagtid, dock ej dagar då E-skiktet är kraftigt joniserat och
skärmar av F-skiktet.
Vågutbredningsvägen för DX är på den sida av jorden som har dagsljus.
Döda zoner på upp till \qty{4000}{\kilo\metre} kan uppstå.
Förbindelser över stora avstånd är möjliga med låg effekt.

Under solfläcksminimum är bandet inte användbart för DX-kontakter.
Då är endast kortvariga förbindelser på avstånd upp till \qty{2000}{\kilo\metre}
möjliga genom reflektioner via sporadiska E-skikt (short skip).

Bandet har i många fall VHF-karaktär och man kan ha kontakter via Aurora och
andra liknande utbredningsformer såsom Aurora-E och dubbelt hopp på
Auroraringen.

% Avsnitt 8.6 Vågutbredning på VHF-EHF
\section[Vågutbredning på VHF-EHF]{Vågutbredning på VHF, UHF, SHF och EHF}
\label{vågutbredning_vhf}
\index{vågutbredning!VHF}
\index{vågutbredning!UHF}
\index{vågutbredning!SHF}
\index{vågutbredning!EHF}
\index{vågutbredning}
\index{troposfären}
\index{sked}
\index{VHF}
\index{UHF}
\index{SHF}
\index{EHF}
\index{UKV}

\subsection{Allmänt}
Frekvensområdet \SIrange{30}{300000}{\mega\hertz} delas upp i följande mindre
avsnitt som kallas:

\begin{itemize}
  \item Very High Frequency (VHF), \SIrange{30}{300}{\mega\hertz}
  \item Ultra High Frequency (UHF), \SIrange{300}{3000}{\mega\hertz}
  \item Super High Frequency (SHF), \SIrange{3}{30}{\giga\hertz}
  \item Extremely High Frequency (EHF), \SIrange{30}{300}{\giga\hertz}.
\end{itemize}

På VHF och högre frekvenser (tidigare UKV) förekommer sällan någon
vågutbredning via jonosfären annat än under tiden för maximal solaktivitet.
I stället utnyttjas den lägre delen av atmosfären och
knappast högre än \SIrange{4}{5}{\kilo\metre} över jordytan.
Denna del av atmosfären kallas för troposfär och vågutbredningen benämns därför
troposfärisk vågutbredning.

All vågutbredning i troposfären förutsätter i princip optisk sikt.
Emellertid förekommer en viss vågavböjning utmed jordytan, varför den praktiska
räckvidden utmed siktlinjen är något längre än till den optiska horisonten.
Man talar om radiohorisont.

På de högre frekvenserna är det på grund av vågutbredningen ofta svårare att
få radiokontakt med andra radioamatörer.
En metod för att veta att det finns någon som lyssnar i andra ändan är att komma
överens om frekvens, riktning och tidpunkt i förväg.
En sådan överenskommelse kallas \emph{sked} förkortat av
(eng. \emph{Scheduled QSO} via \emph{sched}) och är vanlig även på
kortvågsfrekvenser.

\emph{Brytningsindex} i atmosfären är en viktig faktor för vågutbredning bortom
frisiktsavståndet, speciellt vid frekvenser över \qty{100}{\mega\hertz}.
Även den splittring av vågorna som uppstår när de träffar oregelbundenheter i
atmosfären kan utnyttjas för kommunikation på avstånd som är flera gången
frisiktsavståndet.

Vid högre frekvenser begränsas emellertid räckvidden av atmosfärens
dämpande inverkan.
Likaså förloras vågenergi i den topografi, vegetation och bebyggelse som ligger
i siktlinjen mellan sändare och mottagare.
I gynnsamma fall är det dock möjligt att överbrygga avstånd på upp till
\qty{1000}{\kilo\metre} genom troposfären.
Sådana avstånd kallas för överräckvidd.

\subsection{Troposfären -- Troposcatter}
\harecsection{\harec{a}{7.10}{7.10}}
\index{vågutbredning!troposcatter}
\index{troposfären}

När en kallfront nära jordytan stöter samman med en varmfront uppstår turbulens
i luften med elektriska uppladdningar i gränsskiktet som följd.

Under sådana väderförhållanden kan radiovågor i VHF-området och däröver att
brytas eller splittras upp när de träffar det laddade gränsskiktet --
\emph{troposcatter}.
Då kan oväntade radiokontakter uppnås.

\subsection{Temperaturinversion}
\harecsection{\harec{a}{7.12}{7.12}}
\index{temperaturinversion}
\index{vågutbredning!temperaturinversion}
\index{duct}

När ett varmt luftskikt lägger sig över ett kallare luftskikt uppstår
en så kallad \emph{temperaturinversion}.

Vågor på VHF och UHF bryts då mot gränsskiktet och böjs av mot jordytan.
Om det finns två inversionsskikt samtidigt, så kan de bilda
en slags vågledare, så kallad \emph{dukt} (eng. \emph{duct} = ledning).
En räckvidd på \SIrange{600}{1300}{\kilo\metre} kan uppnås.
Denna typ av vågutbredning förekommer ofta vid högt atmosfärstryck under
sommaren.

\subsection{Reflektion mot E\raisebox{-.4ex}{s} (sporadiskt E)}
\harecsection{\harec{a}{7.13}{7.13}}
\index{sporadiskt E}
\index{vågutbredning!sporadiskt E}

Vid stark solinstrålning bildas, på de lägre latituderna, joniserade gasmoln på
en höjd av cirka \qty{120}{\kilo\metre} och med en oregelbunden fördelning.

Den kritiska frekvensen är hög för \(\mathrm{E_s}\)-skiktet och det kan även
reflektera vågor på VHF och UHF så effektivt att avstånd av
\SIrange{1000}{4000}{\kilo\metre} kan överbryggas.

\subsection{Aurorareflektion}
\harecsection{\harec{a}{7.14}{7.14}}
\index{aurora}
\index{vågutbredning!Aurora}
\index{norrsken}
\index{polarsken}

Soleruptioner (flares) utstrålar stora mängder ultraviolett ljus och kastar ut
elektriskt laddade partiklar, som efter 1--2 dagar fångas upp av jordens
magnetosfär och tränger ner i polarzonerna.
När partiklarna kolliderar med atmosfären bildas det polarsken i form av lysande
''draperier'' -- \emph{Aurora borealis} kallat norrsken på norra halvklotet
eller \emph{Aurora australis} kallat sydsken på södra halvklotet -- samtidigt
som atmosfären joniseras.
Aurora är joniserade skikt i samma plan som jordens magnetfält och speciellt
vågor med frekvenser över \qty{30}{\mega\hertz} reflekteras emot dessa.

VHF- och UHF-kommunikation kan ske med hjälp av aurorareflektion.
De signaler som reflekteras av Aurora är kraftigt distorderade och har förlorat
all ton.
Den reflekterade signalen blir bred i frekvens, vilket emellertid gynnar
kommunikation med telegrafi när signalerna är svaga.
Oftast är endast telegrafiförbindelser i långsam takt möjliga.
Vid starkare Aurora går också SSB att använda.

\subsection{Reflektion mot meteorer -- Meteorscatter}
\harecsection{\harec{a}{7.15}{7.15}}
\index{meteorscatter}
\index{vågutbredning!meteorscatter}

Radiovågor på VHF och UHF reflekteras mot joniserade spår efter det
meteorgrus som faller in i jordatmosfären.
Detta fenomen kan utnyttjas för radioförbindelser.

Joniseringen sker när partiklarna passerar genom E-skiktet och brinner upp.
Eftersom joniseringen har en varaktighet av endast 0,1--10~sekunder måste
\emph{MS-förbindelser} planeras och förberedas väl.
Förbindelserna begränsas vanligen till utbyte av anropssignaler och
signalrapporter med höghastighetstelegrafi med en hastighet av
300--3000~tecken per minut.
Under de större meteorskurarna kan kontakter uppnås utan överenskommelser på
förhand (''sked''), både på telegrafi (CW) och telefoni (SSB).

\subsection{EME-förbindelser}
\harecsection{\harec{a}{7.16}{7.16}}
\index{EME}
\index{vågutbredning!EME}
\index{månstuds}
\index{vågutbredning!månstuds}

Radioförbindelse från en punkt på jorden till en annan kan åstadkommas
genom reflektion av VHF-/UHF-signaler mot månen.
\emph{EME-förbindelser} (eng. \emph{Earth--Moon--Earth}) kallas även
\emph{månstuds} (eng. \emph{Moon Bou\-n\-ce}).
EME-förbindelser kräver antenner med mycket hög riktverkan, mycket hög
sändareffekt och känsliga mottagare.

\newpage
\subsection{Markbaserade relästationer}
\index{repeater}

\mediumfig{images/cropped_pdfs/bild_2_7-12.pdf}{Markbaserad repeater}{fig:bildII7-12}

På VHF och högre frekvenser kan man, som tidigare beskrivits, endast
uppnå radiokontakter hitom den så kallade radiohorisonten.

För att överbrygga detta hinder används relästationer, se bild
\ssaref{fig:bildII7-12}.
Den slags relästation, som allmänt kallas \emph{repeater}, tar emot det den hör
på en viss fast frekvens och återutsänder detta på en viss annan fast frekvens.
Se bandplan i bilaga~\ssaref{bandplaner}.

\subsection{Rymdsatellit-baserade relästationer}
\index{satellit}
\index{OSCAR}

Radiovågor med tillräckligt hög frekvens kan passera genom jonosfärskikten.
Detta möjliggör radioförbindelser VHF/UHF/SHF mellan
stationer på jorden med hjälp av relästationer i rymdsatelliter.

För amatörradiotrafik över rymdsatelliter används vanligen den slags
relästation, som kallas \emph{transponder}.
En sådan tar emot allt det den hör inom ett helt frekvensband och återutsänder
detta i ett helt annat frekvensband.
På så sätt kan trafik över satellit ske på ett jämförbart sätt som vid
direktkontakt mellan jordbaserade stationer.

Satellitbaserade linjärtranspondrar med amatörradioutrustning finns i
OSCAR-satelliterna (OSCAR = Orbiting Satellite Carrying Amateur Radio).
Dessa har konstruerats och byggts av amatörradiogrupper.

OSCAR-satelliterna har många olika transpondrar i funktion, vilka var och en
arbetar med olika kombinationer av sändningsslag (moder) och frekvensband.
Detta kallas numera att de har olika konfiguration.

En vanlig transponderkonfiguration är CONFIG-V/U (f.d. MOD-J) där upplänken
är på VHF-bandet, till exempel \SIrange{145,900}{146,000}{\mega\hertz} och
nerlänken på UHF-bandet till exempel \SIrange{435,800}{435,900}{\mega\hertz}.
Varje upplänkfrekvens motsvarar en bestämd nerlänkfrekvens, till exempel 
145,950 upp och \qty{435,850}{\mega\hertz} ner.
Trafiken över transpondern kan därför ske i full duplex.

Man kan då prata och lyssa samtidigt i båda riktningarna, vilket
starkt förbättrar trafiken och gör samtalen roligare och intressantare.

En så kallad linjär transponder kan inte bara överföra FM, utan även SSB,
tontelegrafi och SSTV.
Dessutom även RTTY och andra digitala trafiksätt.

Nästan alla amatörradioband med tillräckligt hög frekvens används i
olika kombinationer som upp- och nerlänkar i de olika OSCAR-satelliterna.
AMSAT är den organisation, som fortlöpande informerar om amatörradiosatelliter.
Den svenska grenen på AMSAT är AMSAT-SM som är aktiva och har på sin webbplats
beskrivningar både för nybörjare och de som kommit lite längre om hur man
använder amatörsatelliter.

\mediumfig{images/cropped_pdfs/bild_2_7-13.pdf}{Transponder i rymdsatellit}{fig:bildII7-13}

Amatörradion utvecklas mycket snabbt genom den satellitbaserade
verksamheten och det kommer upp allt mer sofistikerade OSCAR-satelliter.
Tendensen är att man efter hand går över till allt
högre frekvensband och allt mera av digitala sändningsslag.

Med hjälp av satellit kan förbindelseavståndet bli mycket stort även
med enkel utrustning och små antenner.
En fördel med kommunikation över rymdsatellit är också att den till största
delen är oberoende av vågutbredningsvillkoren.
%
Se bild~\ssaref{fig:bildII7-13}.

% Avsnitt 8.7 Brus och länkbudget
\section{Brus och länkbudget}

\subsection{Allmänt}

Den mottagna signalens kvalitet kan ofta sammanfattas med dess signal-brus
förhållande.
För att kunna estimera det behöver man dels förstå själva länk-budgeten som
ger en uppfattning om hur stark signal man får, men även förstå de olika
bidragen av brus som sätter det effektiva brusgolvet.

\subsection{Brus}
\harecsection{\harec{a}{7.17}{7.17}, \harec{a}{7.18}{7.18}}
\index{brus}
\index{atmosfäriskt brus}
\index{brus!atmosfäriskt}
\index{galaktiskt brus}
\index{brus!galaktiskt}
\index{termiskt brus}
\index{brus!termiskt}

Det finns flera källor till brus, atmosfäriskt brus, galaktiskt brus samt
termiskt brus.

\emph{Atmosfäriskt brus} (eng. \emph{atmospheric noise}) uppstår på grund av
blixturladdningar.
Över hela jorden sker hela tiden blixtnedslag, och dess starka impulser sprider
sig precis som radiovågor och ger en grundläggande störning i kortvågsbandet.
Atmosfäriskt brus identifierades 1925 av Karl Jansky.

\emph{Galaktiskt brus} (eng. \emph{galactic noise}) kommer huvudsakligen från
centrum av Vintergatan, och är huvudsakligen termiskt brus från den stora
ansamlingen av stjärnor i mitten av Vintergatan.
Galaktiskt brus kommer från den delen av himlen som för stunden har mitten av
Vintergatan, så det är riktningskänsligt.

Termiskt brus är mottagarens interna brus, se \ssaref{termisktbrus}.

\subsection{Länkbudget}
\harecsection{\harec{a}{7.20}{7.20}}
\index{länkbudget}

För att kunna estimera den upplevda signalkvaliteten så gör man en så kallad
\emph{länkbudget} (eng. \emph{link budget}).
Länkbudgeten sammanställer hur signalstyrkan respektive brus varierar längs en
länk med dess förstärkningar och dämpningar.
I slutet av länkbudgeten kan sedan det upplevda signal-brus-förhållandet
enkelt estimeras.
En väl utförd länkbudget kan därför skapa god förståelse för länkens brister
så att förbättringar kan göras.

\subsubsection{Dominant bruskälla}
\harecsection{\harec{a}{7.20.1}{7.20.1}}
\index{bandbrus}
\index{band noise}
\index{brus!band}
\index{interntbrus}
\index{brus!internt}

I en noggrann modell så ska alla bruskällor, från källa till mottagare,
listas, justeras för gain och sammanställas.
I praktiken så har man en dominant störkälla, typiskt bruset på bandet eller
kort \emph{bandbruset} (eng. \emph{band noise}) eller
\emph{mottagarens interna brus} (eng. \emph{receiver noise}),
varvid de övriga bidragen har liten påverkan på den totala uträkningen.
Det är därför praktiskt att fort estimera om det är brus på bandet eller
mottagarens brus som dominerar, vartefter man enbart räknar med den
dominerande bruskällan.

Som tumregel kan man säga att för kortvåg är oftast bruset på bandet
den dominerande bruskällan, medan för högre band så kommer det interna
bruset att dominera, och dämpningar i kablar bli allt mer märkbart.

\subsubsection{Signal-brus-förhållande}
\harecsection{\harec{a}{7.1.2}{7.1.2}}
\index{signal-brus-förhållande}
\index{brus!signal-brus-förhållande}
\index{S/N}
\index{brus!S/N}

Upplevelsen av en signals kvalitet kan mätas på många sätt, dock är just
signalens brusmängd en viktig sådan relation och därför så använder man
begreppet \emph{signal-brus-förhållande} (eng.
\emph{signal to noise ratio, S/N}).

Signal-brus förhållandet uttrycks oftast i \unit{\decibel} och kan enkelt räknas
fram som skillnaden i nivå på signal och på brus, det vill säga signal minus
brus, räknat i \unit{\decibel}.
Med en signalnivå på \qty{45}{\decibel} och brusnivå på \qty{22}{\decibel} har
man således \qty{+23}{\decibel} S/N.

\subsubsection{Minimal signal-brus-förhållande}
\harecsection{\harec{a}{7.20.2}{7.20.2}}

Det är också till stor hjälp att fort etablera det \emph{minimala
signal-brus-förhållandet} (eng. \emph{minimum signal to noise ratio})
som man kan tolerera.
Genom att jämföra länkbudgeten mot detta kan man fort avgöra om det är
tillräckligt bra eller behöver ändras.

Har man för lågt signal-brus-förhållande mot minimum, så behöver man öka
förstärkningen eller oftast minska förlusterna i länkbudgeten.

\newpage
\subsubsection{Minimal mottagen signalstyrka}
\harecsection{\harec{a}{7.20.3}{7.20.3}}

Om den dominanta störkällan är det interna bruset och man har för lågt
signal-brus-förhållande, måste signalstyrkan in till mottagaren ökas
tills signalen är stark nog för att ge tillräckligt högt
signal-brus-förhållande.
Detta ger nivån för \emph{minimalt mottagen signalstyrka} (eng.
\emph{minimum receiver signal power}) som mottagaren kräver.

\subsubsection{Signaldämpning}
\harecsection{\harec{a}{7.1.1}{7.1.1}}
\index{dämpning}

Så väl kablar, filter, kopplingar och vågutbredning innebär
\emph{signaldämpning}.
Det innebär att man tappar energi i förhållande till den tillförda energin.
Förhållandet mellan uttagen och inmatad energi uttrycks oftast i form av
\unit{\decibel}.
Man ska vara noga att notera dämpningen är oftast relaterad till frekvensen,
så den ska uppskattas eller mätas för den frekvens som avses.

\subsubsection{Brusfaktor}
\label{brusfaktor}
\index{brus}
\index{brusfaktor}
\index{brus!brusfaktor}
\index{noise factor}
\index{NF}
\index{brus!noise factor}
\index{brus!NF}
\index{LNA}

För högre frekvenser tenderar brus domineras av mottagarens interna brus,
samtidigt som kabelförluster börjar bli märkbara.
För sådana fall kan det vara lämpligt att installera en
\emph{lågbrusig förstärkare} (eng. \emph{low noise amplifier, LNA}) före
mottagaren.
Även den förstärkaren har dock egenbrus som sedan förstärks.
\emph{Brusfaktor} (eng. \emph{noise factor, NF}) ger förhållandet mellan en
förstärkares egenbrus i förhållande till det termiska bruset för ett motstånd
på dess ingång.
Brusfaktor redovisas oftast i \unit{\decibel}, som varande dB över brusgolvet.

En förstärkares egenbrus kommer givetvis att förstärkas, och därför kommer
bruset på utgången vara brusfaktorn plus förstärkningen, räknat i \unit{\decibel}.
Exempelvis kommer en förstärkare med \qty{4,5}{\decibel} i brusfaktor och
\qty{20}{\decibel} förstärkning ha brus på \qty{24,5}{\decibel} över brusgolvet.
En efterföljande förstärkare med \qty{10}{\decibel} brusfaktor kommer inte
signifikant bidra med brus, eftersom föregående steg har \qty{14,5}{\decibel}
högre brus än egenbruset.
För detta fall är den första förstärkaren dominant.

Eftersom kabeldämpning kan vara signifikant, så kommer signalen dämpas genom
kabeln.
Givet att vi har \qty{15}{\decibel} dämpning i kabeln, en signal
\qty{40}{\decibel} över brusgolvet och en \qty{20}{\decibel} förstärkare med
brusfaktor \qty{4,5}{\decibel}, var ska vi sätta förstärkaren?

Om förstärkaren sitter efter kabeln så kommer signalen att dämpas först i
kabeln, bruset kommer att läggas på och sedan kommer det att förstärkas
\qty{20}{\decibel}.
Det ger \qty{40}{\decibel} minus \qty{15}{\decibel} plus \qty{20}{\decibel} för
signalen, det vill säga \qty{45}{\decibel}.
För bruset får vi \qty{4,5}{\decibel} plus \qty{20}{\decibel} det vill säga
\qty{24,5}{\decibel}.
För detta fallet får vi ett signal-brus-förhållande på \qty{45}{\decibel} minus
\qty{24,5}{\decibel}, det vill säga \qty{20,5}{\decibel}, givet att det är det
interna bruset som är dominerande.

Om förstärkaren sitter före kabeln så kommer signalen först förstärkas och
sedan dämpas i kabeln.
Det ger \qty{40}{\decibel} plus \qty{20}{\decibel} minus \qty{15}{\decibel} för
signalen, det vill säga \qty{45}{\decibel}.
För bruset får vi \qty{4,5}{\decibel} plus \qty{20}{\decibel} minus
\qty{15}{\decibel} det vill säga \qty{9,5}{\decibel}.
För detta fallet får vi ett signal-brus-förhållande på \qty{45}{\decibel} minus
\qty{9,5}{\decibel}, det vill säga \qty{35,5}{\decibel}, givet att det är det
interna bruset som är dominerande.

Med detta exempel ser vi hur en länkbudget hjälper oss att få
signal-brus-förhållandet att gå från \qty{20,5}{\decibel} till
\qty{35,5}{\decibel} enbart genom att ändra placeringen av förstärkaren i
systemet.

\subsubsection{Vägförlust}
\harecsection{\harec{a}{7.20.4}{7.20.4}}
\index{vägförlust}
\index{path loss}
\index{fresnelzon}
\index{närfält}

Den dämpning som signalen har i fri-rymds förlust, på grund av dess avtagande
fältstyrka kallas även för \emph{vägförlust} (eng. \emph{path loss}).
Fri-rymds förlusten beror förenklat på frekvens och avstånd, en enkel model
\cite[\S 19.1.2]{ARRLHDB2015}:
%%
\[L_{fs} = 32.45 + 20\log d + 20\log f\]
%%
där \(d\) är avståndet i \unit{\kilo\metre} och \(f\) är frekvensen i
\unit{\mega\hertz} och det ger \(L_{fs}\) är vägförlusten i \unit{\decibel}.
Fri-rymds förlusten avtar med kvadraten på avståndet, det vill säga
\qty{6}{\decibel} på dubblat avstånd och det är i allmänhet den dominerande
effekten när man lämnat antennens närfält.

Ytterligare förluster kan förekomma på grund av vegetation, delvis täckt
\emph{Fresnelzon}, studsar mot jonosfär med mera.
Fresnelzonen är den zon som befinner sig inom den ellips vars form definieras
av en våglängd längre väg än direkt väg mellan sändare och mottagare.
Merparten av energin mellan två antenner rör sig i denna fresnelzon och
således så om den regionen är påverkad av hinder så kommer signalen dämpas
märkbart.

\subsubsection{Antennförstärkning och kabelförluster}
\harecsection{\harec{a}{7.20.5}{7.20.5}}
\index{antennvinst}
\index{antenn!vinst}
\index{antennförstärkning}
\index{antenn!förstärkning}
\index{kabelförluster}

En antenns direktivitet ger antennen en \emph{antennförstärkning} (eng.
\emph{antenna gain}) då den i en viss riktning har en förmåga att ha högre
förstärkning än en enkel dipol.

Antennens förmåga att undertrycka andra signaler, till exempel som mätt med
fram-back-relationen, ger också en undertryckning av oönskade signaler och
atmosfäriskt brus.
Det kan därför vara värt att inte enbart mäta antennens förstärkning av
önskad signal, utan även räkna på dess förmåga att ta in oönskat brus och
störande signaler.

Anslutna kablar kan ha signifikant påverkan på både sänd- och mottagen
signalstyrka då \emph{kabelförlusterna} (eng. \emph{transmission line losses})
kommer dämpa signaler.
Kabelförluster beror på hur lång kabeln är, vilken kabel det är samt vid
vilken frekvens man använder.
Som regel har högre frekvenser högre dämpning.
Både storleken på kabeln och val av dielektrium påverkar förlusten i kabeln.

%% k7per
%% \newpage % layout
\subsubsection{Minsta sända signalstyrkan}
\harecsection{\harec{a}{7.20.6}{7.20.6}}

En sändare har en varierande uteffekt, och eftersom man försöker åstadkomma en
uppskattning på sämsta signal-brus-förhållandet så är det inte maxeffekten
eller ens medel effekten som blir den intressanta, utan den \emph{minsta sända
signalstrykan} (eng. \emph{minimum transmitter power}).
Genom att använda sig av den i beräkningen på sin länk-budget så försäkrar man
sig om att länk-budgeten hanterar sämsta tänkbara fall, och för de fall som
sändaren är starkare så får man alltså bättre signal än den lägsta man
tolererar.
På detta sätt bygger man sig marginaler i beräkningen.

\subsubsection{Sammanställning av länkbudget}

En komplett länkbudget fås genom att räkna på signalstyrka respektive brusnvå
för varje steg i kedjan, genom att gå igenom alla förstärkningar och förluster
längs vägen.
När man sedan har räknat fram mottagarens upplevda signalstyrka och brusnivå så
kan man räkna fram signal-brus förhållandet, se exemplet i \ssaref{brusfaktor}.

För att försäkra sig om att det fungerar brukar man räkna konservativt, det
vill säga man väljer de sämsta siffrorna, till exempel minsta mottagen effekt
och minsta sända effekt.

En väl utförd länkbudget ger god förståelse över var den svaga länken är,
och genom att experimentera med olika alternativa lösningar så kan man
förstå var man ska börja göra åtgärder och var det är lönlöst eller har
ringa påverkan.

%
%
% Kapitel 9 Mätteknik
% Avsnitt 9.1 Att mäta
\chapter{Mätteknik}

\noindent\emph{I forskning, utveckling och produktion är mätning en hörnpelare
  i verksamheten. Även inom mättekniken sker en snabb utveckling och
  digitaltekniken kommer alltmer till användning, men grunderna för
  mätning är desamma. I detta kapitel behandlas de viktigaste
  mättekniska begreppen som radioamatörer kan behöva känna till.}

\section{Att mäta}
\harecsection{\harec{a}{8.1}{8.1}, \harec{a}{8.1.1.1}{8.1.1.1}}

\subsection{Mäta likspänning}
\index{inre resistans}
\label{mäta_likspänning}

Vid spänningsmätning bestämmer man potentialskillnaden -- spänningen --
mellan två punkter.
Om det finns en spänning, så flyter motsvarande (mät)ström genom
instrumentet som presenterar mätströmmen som en spänning.

Mätströmmen påverkar emellertid spänningsfördelningen i kretsen och då
uppstår ett mätfel, vilket inte framgår av det visade mätvärdet.
Med kännedom om kretsens och instrumentets data kan man dock beräkna mätfelet.
En voltmeter ska ha hög \emph{inre resistans} för att mätfelet ska bli litet.

Endast vid mycket noggrann mätning kan man behöva räkna om det visade
mätvärdet med hänsyn till voltmeterns inre resistans och
förkopplingsresistansen -- om en sådan används.

\begin{center}
\begin{minipage}{0.19\columnwidth}
  %%\Huge{\fontencoding{U}\fontfamily{futs}\selectfont\char 66\relax}
  \fontfamily{futs}\Huge{\hspace{1ex}!}
\end{minipage}
\begin{minipage}{0.7\columnwidth}
På grund av den höga inre resistansen är en voltmeter endast lämpad
för spänningsmätning -- INTE för direkt strömmätning!
\end{minipage}
\end{center}

\subsubsection{Utöka mätområdet för en voltmeter}

Med hjälp av förkopplingsresistor i serie med voltmetern kan man mäta
högre spänning än den som voltmetern är gjord för.
Spänningen fördelas då proportionellt mellan förkopplingsresistorns resistans
och instrumentets inre resistans.
Ett exempel på detta finns i bild \ssaref{fig:BildII3-01}.

När förkopplingsresistor används måste mätvärdet räknas om med en
skalfaktor eller en skala med motsvarande gradering användas.
En voltmeter med valbar förkopplingsresistor kan därför ha flera skalor.
I digitala voltmetrar anpassas ''skalan'' oftast automatiskt.

\newpage % layout
\subsubsection{Inre resistansens inverkan}
\harecsection{\harec{a}{8.1.2.3}{8.1.2.3}}
\index{inre resistans}

Vid mätning av spänning kommer den inre resistansen hos voltmetern att lasta
kretsen, och därmed sänka spänningen och därmed kommer den uppmätta spänningen
vara lägre än den faktiska spänningen utan mätinstrumentet.

I gamla tider var den inre resistansen relativt låg, varvid påverkan blev större
än med moderna instrument.
Dock kan även moderna instrument påverka mätresultatet i kretsar som har väldigt
hög impedans.

\subsection{Mäta likström}
\index{strömmätning}
\index{inre resistans}

Vid \emph{strömmätning} bestämmer man strömstyrkan i en gren av en elektriskt
strömkrets.
Amperemetern ska kopplas i serie med den aktuella strömgrenen.
Det visade mätvärdet motsvarar strömstyrkan.
Amperemeterns inre resistans adderas emellertid till resistansen i strömgrenen
och då uppstår ett mätfel.
En amperemeter ska ha låg inre resistans för att mätfelet ska bli litet.

Endast vid mycket noggrann mätning kan man behöva räkna om det visade
mätvärdet med hänsyn till amperemeterns inre resistans och resistansen
i strömshunten -- om en sådan används.

\begin{center}
\begin{minipage}{0.19\columnwidth}
\Huge{\fontencoding{U}\fontfamily{futs}\selectfont\char 66\relax}
\end{minipage}
\begin{minipage}{0.7\columnwidth}
På grund av den låga inre resistansen ska en amperemeter
ALDRIG användas för spänningsmätning. Då förstörs den!
\end{minipage}
\end{center}

\subsubsection{Utöka mätområdet för en amperemeter}
\index{strömshunt}

Med en strömshunt (en resistor parallellt) över amperemetern kan man
mäta högre ström än den som amperemetern är gjord för.

Shunten dimensioneras så att större delen av strömmen leds förbi amperemetern.
Kvar är den mätström som behövs för att amperemetern ska göra fullt utslag.

Mätströmmen fördelar sig omvänt proportionellt mot instrumentets och
shuntens resistanser.

När en strömshunt används måste mätvärdet räknas om med en skalfaktor
eller en skala med motsvarande gradering användas.
En amperemeter med valbar shuntresistor kan därför har flera skalor.
I digitala amperemetrar anpassas ''skalan'' oftast automatiskt.

\newpage % layout
\subsection{Mäta växelspänning och växelström}

Grunderna för mätning av växelspänning och växelström är samma som för
likspänning och likström, men att bland annat en instrumentlikriktare oftast
behövs.

Beroende på frekvensen i strömkretsen och vilket slags värde man vill
mäta, används olika instrument.

Olika typer av instrument ger olika möjligheter, men också begränsningar.

\emph{Mjukjärnsinstrument} utan likriktare kan mäta växelströmmar ner
till cirka \qty{50}{\milli\ampere} och upp till cirka \qty{10}{\ampere}.
Frekvensen får dock inte vara högre än cirka \qty{100}{\hertz}.

\emph{Vridspoleinstrument} används dels direkt för likströmsmätning
och dels med likriktare även för växelströmsmätning.

Vridspoleinstrument med likriktare används ofta för frekvenser upp till cirka
\qty{10}{\kilo\hertz} och strömmar ner till \qty{0,1}{\milli\ampere}.
Noggrannheten är sällan bättre än 1,5~\% av fullt utslag.

Beroende på funktionsprincipen kan det skilja på hur instrument mäter,
vilket nödvändigtvis inte är detsamma som hur mätvärdet presenteras.

Mjukjärnsinstrument mäter effektivvärdet av en växelström medan ett
vridspoleinstrument med likriktare mäter likriktade medelvärdet.
Som exempel kan skalan i ett instrument med likriktare även graderas för
effektivvärdet för sinusformade förlopp.

För mätning av växelström används vanligen instrument med likriktare,
men för HF även instrument med termokors, vilka bygger på
termogalvanisk spänning mellan metaller.

\subsubsection{Frekvensens inverkan}
\harecsection{\harec{a}{8.1.2.1}{8.1.2.1}}

Frekvensen på den mätta signalen inverkar mer eller mindre på mätresultatet.
Till en del beror det på den instrumenttyp, som används.
En faktor är instrumentets gränsfrekvens, det vill säga hur högt i frekvens som
instrumentet fortfarande är rimligt rättvisande.
Detta kallas instrumentets bandbredd, vilken bör vara dokumenterad.

\subsubsection{Vågformens inverkan}
\harecsection{\harec{a}{8.1.2.2}{8.1.2.2}}

Även formen på den signal som mäts inverkar på mätresultatet och det
är viktigt att veta för vilken vågform som instrumentet presenterar mätvärdet.
Det vanligaste är att vågen förutsätts vara sinusformad, vilket ofta inte är
fallet i praktiken.
Det innebär att fel värde presenteras om vågformen är en annan än den
förutsatta.

\subsection{Mäta resistans}
\harecsection{\harec{a}{8.1.3}{8.1.3}}

Mätning av resistans är enklast att göra på en fristående komponent,
medan man vid mätning på en resistor i en strömkrets också måste ta
hänsyn till att andra komponenter i kretsen kan påverka mätresultatet.

Resistans kan mätas på flera sätt.
Det grundläggande är att mäta strömmen genom resistorn och spänningen över den
och sedan beräkna resistansen med Ohms lag.

Den vanligaste metoden är att använda en modern multimeter som kan mäta
resistans direkt.
En del kräver att man ställer in området för resistans, medan andra
kan göra det automatiskt.

Precisionsmätning av motstånd kan göras av mer avancerade instrument
där man använder 4-punkts\-mätning.
För 4-punktsmätning så är ström och spänninganslutningarna separerade
så att anslutningsledningarnas resistans inte ger spänning som inkluderas
i den mätta resistansen.
Istället så mäts spänningen så nära som möjligt på själva mätobjektet,
medan spänningsförlusten för strömledarna därmed kan elimineras.
Denna mätmetod är relevant framförallt för lågohmiga motstånd.

\subsection{Mäta effekt}
\harecsection{\harec{a}{8.1.4}{8.1.4}}
\label{mätaeffekt}
\index{PEP}
\index{Peak Envelope Power (PEP)}
\index{effekt!Peak Envelope Power (PEP)}

Effektformler vid lik- och växelström (medel-, effektiv- och toppvärden)
Vid likström:
%
\[
P = U \cdot I \quad \text{[W (watt)]} \quad \text{dvs. Joules lag}
\]
%%
Vid sinusformad växelström och resistiva belastningar
(För PEP-effekt se även avsnitt \ssaref{PEP-effekt}):
%%
\[
\begin{array}{ll}
\text{effektivvärde} \quad P & = \dfrac{U^2}{R} \\
&\\
\text{toppvärde}     \quad P_{PEP} & = \dfrac{U_{max}^2}{2R}
\end{array}
\]
%%
U = spänningens effektivvärde
R = resistansen

\subsection{Sändareffekt}

En sändares effekt kan mätas på olika sätt.
Den metod att mäta \emph{uteffekt} som är relevanta för radioamatören är
toppvärdeskännande sådana för mätning av \pep.
I föreskrifterna används sändareffekten \pep som uteffekt.
Därvid måste även \pep avses, fastän det inte uttryckligen uttalas.

Observera, att radioamatören måste beakta EMC-lagen.
Se vidare kapitel \ssaref{EMC-lagen}.

\subsection{Metoder för mätning av sändareffekt}
\index{PEP}
\index{Peak Envelope Power (PEP)}
\index{effekt!Peak Envelope Power (PEP)}

Tidigare har avhandlats effektberäkning i allmänhet.
Här nedan kommenteras mätning av sändareffekt i synnerhet.

Ett tillförlitligt sätt att mäta sändareffekt är att ansluta sändaren
till en konstlast med samma resistans som sändarens utgångsimpedans
och mäta spänningen över lasten med ett oscilloskop med tillräcklig bandbredd.
Då kan man se och mäta HF-spänningens topp-toppvärde och samtidigt se
signalens vågform.

Med spänningen och konstlastens impedans (resistans) bekanta så kan
uteffekten beräknas enligt formlerna i kapitel \ssaref{mätaeffekt}.

Den största HF-amplitud som uppstår momentant vid modulering motsvarar
PEP-effekten som kommer av \emph{Peak Envelope Power (PEP)}.

En mindre exakt metod att mäta HF-spänning är med voltmeter med likriktare.
Utifrån den uppmätta spänningen kan man beräkna effekten över en belastning.
På grund av instrumentets tröghet visas emellertid bara ett ''utjämnat''
toppvärde, vilket inte är det faktiska värde som instrumentet ''känner''.
Jämför med oscilloskopet som inte har denna visningströghet.

\tallfig{images/cropped_pdfs/bild_2_8-01.pdf}{Mätning av sändareffekt}{fig:bildII8-1}

Bild \ssaref{fig:bildII8-1} visar en voltmeter med likriktare, som kopplats till
en sändare över spänningsdelare.
Två alternativa delare visas; den ena består av resistorer och den andra av
kondensatorer.

Den resistiva delaren är bättre i den meningen att den är frekvensoberoende och
inte belastar sändaren kapacitivt.
Dessutom dämpas övertoner som bildas vid likriktningen.
I den kapacitiva delaren kan övertoner passera lättare.

Denna mätmetod är noggrann bara när impedansen är lika i sändaren,
kabeln till lasten och själva lasten.
Lasten kan vara en konstlast, en antenn etc. och ska ha ett känt värde för att
effekten ska kunna beräknas.

Ett sätt att skaffa underlag för beräkning av PEP-effekten är att mäta
HF-strömmen med ett termokorsinstrument och spänningen med en
toppvärdesvisande voltmeter.
Utifrån dessa värden beräknar man effekten.
Denna metod är dock inte så vanlig.

\newpage %layout
\subsection{Direktvisande effektmetrar}
\harecsection{\harec{a}{8.2.1.2}{8.2.1.2}}

Många föredrar direktvisande effektmätare.
En HF-voltmeter kan givetvis graderas för att visa effekt i stället för
spänning, men då med den viktiga förutsättningen att impedansen måste ha en
fastställt värde.

Om man avläser effekten genom en \qty{75}{\ohm}-kabel på ett instrument för
\qty{50}{\ohm}, så är det verkliga värdet ett annat än den avlästa.

De effektmetrar som förekommer i SVF-instrument är egentligen voltmetrar,
men med skalan graderad i effekt.

\subsection{Mäta ståendevågförhållande (SVF)}
\harecsection{\harec{a}{8.1.5}{8.1.5}}
\label{mäta_ståendevåg}


När till exempel en antennledning ansluts till en antenn och deras impedanser
inte är lika, så kommer en del av inmatade effekten i ledningen att
reflekteras tillbaka från antennen.

Det uppstår då en stående våg i ledningen. Förhållandet mellan inmatad
och reflekterad effekt uttrycks som ett \emph{ståendevågförhållande (SVF)}
(eng. \emph{Standing Wave Ratio (SWR)}).

Med en SVF-meter som sätts in mellan effektkälla och ledning kan man
mäta hur stor effekt som matas in i ledningen och hur stor effekt som
vänder tillbaka från slutet av ledningen.
SVF-värdet kan då bestämmas på något av följande sätt:

\begin{itemize}
\item Man mäter framåt- respektive bakåtgående effekt var för sig med
  en riktningskänslig effektmeter.
  Man beräknar därefter SVF eller tar fram det ur ett diagram.
\item Man använder ett instrument som beräknar eller visar SVF på något sätt.
\end{itemize}

\subsection{Studera vågformen}
\harecsection{\harec{a}{8.1.6}{8.1.6}}

Vågformen för snabba växelströmsförlopp studeras bäst med oscilloskop.

\subsection{Mäta frekvens}
\harecsection{\harec{a}{8.1.7}{8.1.7}}

Frekvensmätning gör man bäst med en så kallad frekvensräknare, som är ett
digitalt instrument.
Man kan också använda en så kallad absorbtionsvågmeter, som är mycket enkel och
inte alls lika exakt.
Vid frekvensmätning ansluter man instrumentet till mätobjektet med en
svag elektrisk eller magnetisk koppling.

\newpage %layout
\subsection{Mäta resonansfrekvens}
\harecsection{\harec{a}{8.1.8}{8.1.8}}

Mäta resonansfrekvensen för en passiv resonanskrets gör man klassiskt
med en så kallad dip-meter.
Idag använder man antingen en spektrumanalysator med tracking-generator, det
vill säga en SNA, eller en nätverksanalysator för att med bättre precision mäta
resonansfrekvenser.

\subsection{Mätfel}

Mätinstrument indelas i noggrannhetsklasser efter största tillåtna felvisning.
Klasserna är 0.1, 0.2, 0.5, 1.0, 1.5, 2.5 och 5.0 varvid klassen anges på
instrumentet.
Som exempel får ett instrument i klass 2.5 ha ett tillåtet mätfel av
\(\pm\)2,5~\% av fullt utslag.

Mätresultatet bestäms av flera faktorer; dels av instrumentets
så kallade mätonoggrannhet, dels av hur mätvärdet presenteras och slutligen
av hur noga användaren läser av.

Vid \emph{analog} visning presenteras mätvärdet med en visare mot en
graderad skala med en viss upplösning.
Visaren kan vara mekanisk eller optisk (ljusspalt).
Vid snabba mätvärdesändringar är instrumentets mekaniska tröghet en faktor
att ta hänsyn till.

Vid \emph{digital} visning presenteras mätvärdet med siffror eller som
längden på en pelare.
Det är förledande att se digital visning med siffror som mer exakt än analog,
men det är inte alls säkert.
Utöver instrumentets mätonoggrannhet, bestäms nämligen noggrannheten av hur
många siffror som mätresultatet presenteras i.

En oberäknelig källa till mätfel är elektromagnetiska fält från
apparater i närheten.

En ofta förbisedd felkälla är temperaturen i mätobjektet och/eller i
instrumentet, det kan vara av inkopplingstiden med mera.

\emph{Visningströgheten} är inget mätfel i sig men kan till nackdel
vid snabba förlopp.
Trögheten förekommer såväl vid analog som digital visning.
I det förra fallet är masströghet i instrumentets rörliga delar orsaken och i
det andra fallet är orsaken klockfrekvensen för instrumentets mikroprocessor.

% Avsnitt 9.2 Mätinstrument
\section{Mätinstrument}
\harecsection{\harec{a}{8.2}{8.2}}

\subsection{Att mäta är att veta}

Mätinstrument används för att, under kontrollerade former, testa och bekräfta
en funktion eller avsaknad av densamma, i en utrustning.
Det används också för att mäta olika komponenter för att verifiera dess
funktion och egenskaper.

Med mätinstrument vill man åstadkomma en så snarlik testmiljö som man kan
förvänta sig, att utrustningen som man testar, ska kunna hantera i
verkligheten.

De mätinstrument vi kommer att nämna nedan, används bland annat i sluttester,
som när, i vårt fall, radioutrustningen är hopmonterad och ska testas, innan den
levereras till kund.

Att tillverkarna använder mätinstrument enligt ovan, är sålunda en förutsättning
för att kunna veta att den utrustning man konstruerat uppfyller de krav som man
specificerat.

Dessa typer av instrument är också av intresse för sändaramatören, som kan
använda dessa för felsökning eller service.

\subsection{Presentation av mätvärden}

\tallfig{images/cropped_pdfs/bild_2_8-02.pdf}{Presentation av mätvärden}{fig:bildII8-2}

Mätvärden kan presenteras på olika sätt som illustreras i bild
\ssaref{fig:bildII8-2}.
De vanligaste sätten är optiska och då med digital eller analog visning.
Mätresultat kan även överföras till dator för vidare bearbetning och visning.

\subsection{Multimeter}
\harecsection{\harec{a}{8.2.1.1}{8.2.1.1}}
\index{multimeter}

Flera mätfunktioner kan utföras med samma basinstrument, som visas i
bild~\ssaref{fig:bildII8-2}, denna egenskap kallas för en \emph{multimeter}.
Genom omkoppling mellan olika tillsatser väljer man mätfunktion och mätområde.
Instrumentskalan utformas så att olika slags mätvärden kan avläsas.
Kombinationer med elektroniska förstärkare och digital visning etc. är nu
vanligt.

\subsection{Vridspoleinstrument}
\index{vridspoleinstrument}

\mediumfig{images/cropped_pdfs/bild_2_8-03.pdf}{Vridspoleinstrument}{fig:bildII8-3}

\emph{Vridspoleinstrument}, som illustreras i bild~\ssaref{fig:bildII8-3}, kan
bara användas för likströmsmätning, eftersom visarutslaget beror av
strömriktningen.
Instrumentet har låg effektförbrukning och stor noggrannhet.
Visningen är vanligen linjär, men kan göras annorlunda.

\paragraph{Funktion}
En spole är upplagrad i fältet av en hästskomagnet.
När den ström, som ska mätas, passerar genom den vridbara spolen så alstras ett
magnetfält även i denna.
De två magnetfälten påverkar varandra så att spolen vrider sig.
Spolen förses med en visare och en returfjäder.
Ju större ström det flyter genom spolen desto större blir visarutslaget.

\smallfig[0.3]{images/cropped_pdfs/bild_2_8-05.pdf}{Konstlast}{fig:bildII8-5}
\mediumbotfig{images/cropped_pdfs/bild_2_8-06.pdf}{Fältstyrkemätare}{fig:bildII8-6}

\subsection{Konstlast}
\label{konstlast}
\index{konstlast}
\index{dummy load}

En \emph{konstlast} (eng. \emph{dummy load}) är en alternativ last som kan
hantera en viss mängd effekt, konstlast illustreras i bild~\ssaref{fig:bildII8-5}.
En konstlast bör ingå i varje amatörradiostation.
Vid mätning och inställning av till exempel modulation och uteffekt, är det
lämpligt att belasta sändaren med dess nominella utgångsimpedans.
För att då undvika att energi strålas ut bör en väl skärmad konstlast användas.

I moderna amatörradiosändare med koaxialkabel utgång är utgångsimpedansen
\qty{50}{\ohm}.
Konstlasten ska då vara en \qty{50}{\ohm} resistor utan reaktiva egenskaper för
det intressanta frekvensområdet.
Den kan bestå av en eller flera sammankopplade resistorer, ofta parallellt för
att minska den induktiva komponenten.

Sändareffekten ska kunna tas upp utan att resistansen förändras nämnvärt.
Det är viktigt att resistorerna kyls effektivt med luft eller vätska i ett kärl
med tillräckligt utrymme, även när vätskan expanderar av värmen.
Vätskan får inte vara lättantändlig eller miljöfarlig.
Till exempel är oljor med PCB förbjudna!

\subsection{Fältstyrkemätare}
\index{fältstyrkemätare}
\label{fältstyrkemätare}

% \mediumbotfig{images/cropped_pdfs/bild_2_8-06.pdf}{Fältstyrkemätare}{fig:bildII8-6}

Styrkan av elektromagnetiska fält kan bestämmas med \emph{fältstyrkemätare}.

En fältstyrkemätare är en högfrekvensdetektor, vars utspänning visas med ett
instrument med skala.
Den selektiva kretsen kan bestå enbart av den avstämda antennen, men även av
ytterligare selektiva kretsar.
Instrumentet visar endast relativa värden och används till exempel för att
bestämma strålningsegenskaperna i sändarantenner och för antennjustering.
Mätresultatet påverkas även av utstrålning från andra sändare inom mätarens
bandbredd.
Bild~\ssaref{fig:bildII8-6} visar en sändare och en fältstyrkemätare.
Dessutom två enkla fältstyrkemätare.

\newpage
\subsection{Kalibreringsoscillator}
\index{kalibreringsoscillator}

\smallfig{images/cropped_pdfs/bild_2_8-07.pdf}{Kalibreringsoscillator i mottagare}{fig:bildII8-7}

En \emph{kalibreringsoscillator} (eng. \emph{calibration oscillator}) används
för att frekvenskalibrera andra apparaters inställningsskalor, som illustreras
i bild~\ssaref{fig:bildII8-7}.
Den är kristallstyrd och avger särskilt precisa och frekvensstabila signaler.

Oscillatorsignalen förvrängs avsiktligt, så att det utöver grundfrekvensen även
skapas harmoniska övertoner.
En oscillator med till exempel grundfrekvensen \qty{25}{\kilo\hertz} avger på så
sätt även frekvenserna \qty{50}{\kilo\hertz}, \qty{75}{\kilo\hertz},
\qty{100}{\kilo\hertz}, \qty{125}{\kilo\hertz} och så vidare.
Man får således en ''kalibreringsfrekvens'' för varje \qty{25}{\kilo\hertz}.

Detta övertonsspektrum kan sträcka flera \qty{100}{\mega\hertz} upp.
Man ''nollsvävar'' apparat mot närmaste kalibreringsfrekvens och kan kalibrera
till exempel VFO-skalan.

Användningsområdet är huvudsakligen kalibrering av äldre mottagare och gradering
av nya skalor och så vidare för densamma.
Dagens mottagare och sändare har syntesoscillator och då behövs normalt ingen
kalibreringsoscillator.

\paragraph{Not}
Äldre trafikmottagare har VFO med LC-krets och ofta en inbyggd
kalibreringsoscillator, vilken i sin tur kan behöva kalibreras.
Det enklaste sättet är då, att jämföra frekvensen på en känd rundradiosändare
på mellanvåg med kalibreringsoscillatorn.

\subsection{Brusmätbrygga}
\index{brusmätbrygga}
\index{Wheatstones brygga}

\smallfigpad{images/cropped_pdfs/bild_2_8-08.pdf}{Brusmätbrygga}{fig:bildII8-8}

\emph{Brusmätbryggan} används vid mätning i antennsystem, så som illustreras i
bild~\ssaref{fig:bildII8-8}.
Den består av en brusgenerator och en Wheatstonebrygga för mätning av
resistans och reaktans.

Till bryggan ansluts en antenn som mätobjekt och en mottagare som
nollindikeringsinstrument för brussignalen.
Mottagaren ställs in på den frekvens där mätvärden önskas.
Bruset hörs svagast när bryggan är injusterad.
Man kan då avläsa mätvärdena för \(R\) och \(X\).
Mäter man vid flera frekvenser, kan till exempel ett impedansdiagram upprättas.
Detta är med andra ord en äldre förlaga till en nätverksanalysator.

\subsection{Ståendevågmeter (SVF-meter)}
\harecsection{\harec{a}{8.2.1.3}{8.2.1.3}}
\index{ståendevågmeter}
\index{SWR-meter|see {ståendevågmeter}}
\index{ståendevåg-förhållande (SVF)}
\index{Standing Wave Ratio}
\index{framåtgående effekt}
\index{forward power|see {framåtgående effekt}}
\index{bakåtgående effekt}
\index{backward power, reflected power}
\index{SVF|see {ståendevåg-förhållande (SVF)}}
\index{SWR|see {Standing Wave Ratio}}
\label{SVF}

\mediumfig{images/cropped_pdfs/bild_2_8-09.pdf}{SVF-meter, princip och inkoppling}{fig:bildII8-9}

När en transmissionsledning eller apparat ansluts till en annan med
avvikande impedans, kommer HF-energi att reflekteras i övergången.

Denna reflekterade energi kan mätas med en \emph{ståendevågmeter}
(eng. \emph{SWR-meter}) så som illustreras i bild~\ssaref{fig:bildII8-9}.
Med \emph{ståendevåg-förhållande (SVF)} (eng. \emph{Standing Wave Ratio, SWR})
menas förhållandet mellan den effekt som flyter framåt respektive bakåt i en
transmissionsledning.
Användningområden för SVF-meter är:

\begin{itemize}
\item Mätning av \emph{framåtgående effekt} (eng. \emph{forward power}).
\item Mätning av \emph{bakåtgående effekt} (eng. \emph{backward power, reflected power}).
\item Bestämning av \emph{SVF} (eng. \emph{SWR}).
\item Bestämning av resulterande, relativ effekt.
\end{itemize}

\paragraph{Anmärkning} Vid bestämning av absolut effekt måste
anslutningsimpedansen vara lika i instrument och transmissionsledning.

SVF-metern är ett av de mest användbara instrumenten vid HF-mätningar.
En SVF-meter kan ha separata instrument för fram- respektive backeffekt eller
ett gemensamt.

SVF-metern kan vara ständigt inkopplad till exempel mellan sändare och antenn,
men ska då kunna tåla effektutvecklingen.
En SVF-meter kan alstra övertoner, vilka kan medföra störningar.
Orsaken är olinjäriteten halvledardioderna i instrumentet.

\subsection{Frekvensräknare}
\harecsection{\harec{a}{8.2.1.5}{8.2.1.5}}
\index{frekvensräknare}

\smallfig{images/cropped_pdfs/bild_2_8-10.pdf}{Frekvensräknare}{fig:bildII8-10}

\emph{Frekvensräknaren} (eng. \emph{frequency counter}), som är ett digitalt
instrument, används för att bestämma oscillatorfrekvensen i sändare,
mottagare med mera.

Bild~\ssaref{fig:bildII8-10} illustrerar den schematiska bilden av en
frekvensräknare.
I frekvensräknaren räknas antalet svängningar \(E\) (från engelskans events)
i den aktuella inkommande signalen under en bestämd tidsenhet \(t\).
Först förstärks signalen i en analog förstärkare och omvandlas till
kantvågspulser i ingångsstegets triggerenhet.
När varje mätning börjar så kommer en räknare räkna hur många triggerpulser
som passerat fram tills dess den inställda tiden löpt ut.
Moderna räknare mäter även hur den första pulsen (eng. \emph{start event})
respektive sista pulsen (eng. \emph{stop event}) skiljer i tid, så att den
egentliga tiden kan användas, för att få en hög upplösning.
Frekvensen kan nu \emph{estimeras} med formeln:
%%
\[f_{est} = \dfrac{E}{t}\]
%%
Gamla frekvensräknare hade en fix tid som den räknade över, och utan justering
av egentlig tid.
Dessa har en enkel princip och tiden valdes ofta för att få en enkel skalfaktor
mellan räknarens värde och frekvensen, genom att ha steg om
\qty{100}{\milli\second}, \qty{1}{\second}, \qty{10}{\second} och så vidare.
Dessa räknare har dock problemet att för låga frekvenser så krävs lång
observationstid för att försäkra sig om en tillräckligt bra numerisk precision.
En variant av detta som framkom var den så kallade reciproka räknaren, som är
den nu förhärskande principen när man behöver precision, i den så mäts både
hur många event och tiden.
Detta gör att man kan låta tidbasen enkelt varieras med en pot eller extern
signal.

Ytterligare en förbättring som kom är att interpolera tiden för den inledande
och den avslutande pulsen gentemot tidbasens klocka, för att därför kunna
justera mätningen med ett bättre estimat på tiden det egentligen tog för de
räknade eventen att hända.
Med tidsupplösning på 1~ps-nivå kan därför 12~siffrors noggrannhet presenteras
för en mätning över 1~sekund, medan för gamla frekvensräknare med sin
\qty{10}{\mega\hertz} oscillator gav \qty{100}{\nano\second} upplösning och
därmed enbart 7~siffrors noggrannhet för samma 1~sekund mätning.

De moderna frekvensräknarna har nu mer även filter som sammanställer flera
mätningar till en, och presenterar resultat överlappande.
Detta ger en uppfattad högre avläsningshastighet, men mätningarna är inte helt
oberoende.
Vissa frekvensräknare använder även linjärregression för att ytterligare
filtrera bort mätbrus.

Resultatet visas som siffror i ett fönster.
Noggrannheten i den så kallade tidbasen erhålls med en kristallstyrd oscillator
eller för dyrare instrument med en rubidiumnormal.
Man kan ofta ansluta en extern frekvensnormal med frekvens på \qty{10}{\mega\hertz},
vilket gör att man med moderna GPS-styrda oscillatorer kan få tillgång till
SI-definitionen av hertz till en nu mer modest kostnad även i ett hobbylabb.

\subsection{Dipmeter}
\index{dipmeter}

\smallfig{images/cropped_pdfs/bild_2_8-12.pdf}{Dip-meter}{fig:bildII8-12}

\smallfig{images/cropped_pdfs/bild_2_8-13.pdf}{Mätning med dip-meter}{fig:bildII8-13}

\emph{Dipmetern} är i princip en oscillator med variabel frekvens och utbytbara
induktorer för olika frekvensområden, så som visas i bild~\ssaref{fig:bildII8-12}.
Den används för att bestämma resonansfrekvensen på passiva och aktiva
resonanskretsar samt vid bestämning av induktanser och kapacitanser.
Noggrannheten är cirka 3~\%.

\paragraph{Funktion}
Instrumentet avger alternativt reagerar för en HF-signal med viss frekvens.
Frekvensen i dipmeterns resonanskrets är steglöst variabel och frekvensvärdet
kan avläsas på en skala.

Vid mätning av resonansfrekvensen i en passiv resonanskrets kopplas dipmeterns
induktor induktivt till kretsen så som visas i bild~\ssaref{fig:bildII8-13}.
När resonansfrekvensen i kretsen och dipmetern överensstämmer, ändras 
belastningen på dipmeterns resonanskrets varvid instrumentet uppvisar en
strömminskning -- en ''dip''.
Frekvensen avläses då på skalskivan.

Vid mätning på en aktiv resonanskrets, det vill säga som drivs av någon
HF-källa, uppstår i stället en strömökning vid resonans vilket också visas på
instrumentet.

Induktansen i en resonanskrets kan bestämmas med dip-metern, om kapacitansen
är bekant.
På motsvarande sätt kan en obekant kapacitans bestämmas om induktansen i
resonanskretsen är bekant.

Namnet griddipmeter kommer från elektronrörsepoken.
Ändringar i gallerströmmen (grid current) i ett oscillatorkopplat elektronrör
används som indikation på att en resonanskrets är i resonans.
Då minskar gallerströmmen -- det blir en ''ström-dip''.
Numera används en transistor i stället för röret och instrumentet benämns
dip-meter.

\subsection{Oscilloskop}
\harecsection{\harec{a}{8.2.1.6}{8.2.1.6}}
\index{oscilloskop}

\mediumfig{images/cropped_pdfs/bild_2_8-14.pdf}{Oscilloskop}{fig:bildII8-14}

\emph{Oscilloskopet} (eng. \emph{oscilloscope}) är ett mycket användbart
instrument.
Mycket snabba förlopp kan med fördel studeras på en oscilloskopskärm.

Spänningsförlopp kan visas som funktion av tiden.
Tillsammans med andra instrument kan frekvenskaraktäristiken i filter,
modulationskvalitet och så vidare åskådliggöras.

Oscilloskopet består av ett katodstrålerör, där styrningen av katodstrålen sker
med hjälp av X- och Y-förstärkare och en så kallad triggerförstärkare.
Den signal som ska mätas ansluts vanligen till Y-förstärkaren medan en
tidbasgenerator som alstrar en sågtandsformad signal ansluts till X-förstärkaren.
Bild~\ssaref{fig:bildII8-14} visar ett blockschema på oscilloskop.

Moderna oscilloskop digitaliserar signalen efter ingångsförstärkaren,
och läggs sedan i minne, där den sågtandsformade signalen är ersatt med en
räknare som placerar det i minnet.
Sen presenteras bilden på bildskärmen eller på en ansluten dator.
Gemensamt för analoga och digitala oscilloskop är i stort samma handhavande.

Man ansluter en eller flera signaler till ingångarna, justerar ingångssteget
så att hela vågformen fångas och att det är god amplitud, så den syns men inte
klipps.
Ibland väljer man att göra vågformerna mindre, för att man ska kunna
arrangera dem på ett bra sätt på skärmen.
Ibland väljer man att klippa vågen för att man enbart vill se tiden för en
kant tydligt.

En viktig sak för att få en tydlig bild på bildskärmen är att triggpunkten,
den punkt där mätningen av signalen börjar, är vald så att man inte får dubbla
eller otydliga bilder.
Valet av triggpunkt sker ofta automatiskt men om signalen som ska mätas är
väldigt flack eller har många nollgenomgångar kan triggpunkten bli felaktig.
För att lösa problemet med suddiga eller dubbla bilder så brukar man manuellt
justera triggpunkten så att man får en tydlig bild.

Man justerar också tidbasen för att ha rätt skala på tidsaxeln, så man ser en
eller ett fåtal cykler, eller ibland över längre tid för att se variationer.
Man kan också fördröja svepet för att kunna se en viss del efter triggpunkten.

Oscilloskop har nu mera ofta inbyggda funktioner för mätning.
Det är behändigt att snabbt få en uppfattning om periodtider, frekvenser,
amplituder med mera men dessvärre blir mätprecisionen ofta kraftigt lidande,
och det ska inte övertolkas.
Till exempel är frekvensmätningen inte bättre än hur bra placeringen av
markörerna är, så det är sällan tillförlitligheten är bättre än 2 siffrors
noggrannhet.

Rätt använt är oscilloskop dock ett fantastiskt mätinstrument.

\subsection{Spektrumanalysator}
\harecsection{\harec{a}{8.2.1.7}{8.2.1.7}}
\label{spektrumanalysator}

En \emph{spektrumanalysator} (eng. \emph{spectrum analyzer}) visar amplituden
för olika frekvenser över ett visst frekvensområde.
Detta är som kontrast till oscilloskopet som visar amplitud för en signal
över tiden.

Spektrumanalysatorn kan liknas vid en mottagare, men med en viktig skillnad:
där en mottagare har ett eller flera avstämda ingångssteg, som ska förhindra
mottagaren att påverkas av signaler som ligger mer eller mindre nära den
önskade signalen, har spektrumanalysatorn en vidöppen ingång.

Spektrumanalysatorn är ett mätinstrument, och ska kunna presentera de
signaler som matas in på dess ingång, utan att signalerna ska påverkas på
något sätt.
Detta ställer höga krav på spektrumanalysatorns konstruktion.
Den måste kunna tåla starka signaler, utan att en mätning på en svag signal i
närheten påverkas.

Spektrumanalysatorerna finns också i två olika typer.
Den ena arbetar med svepteknik, och sveper över ett viss del av
frekvensspektrumet.
Den andra typen kallas för \emph{realtidsanalysator}, och är kapabel att för
ett givet ögonblick spela in ett spektrum digitalt för senare analys av
innehållet.

Det vi fortsättningsvis beskriver i detta kapitel, är den svepande
spektrumanalysatorn, vilket också är den vanligaste typen för
service tillämpningar.

En spektrumanalysator består förenklat av en svepbar oscillator, variabelt
filter på mellanfrekvensen, en detektor och ett ställbart filter från detektorn.

Den variabla oscillatorn sveper så att det tänkta frekvensområdet täcks, ofta
med ett bestämt antal punkter, till exempel 801, där varje punkt är mätning på
en specifik frekvens.
Ibland kan man anpassa antalet punkter för att få mätningen gå snabbare,
mot att man får en lägre upplösning.

Filtret sätter bandbredden på mätningen, och det kan gå i 1--3 steg, till
exempel \qty{3}{\hertz}, \qty{10}{\hertz}, \qty{30}{\hertz}, osv. till
\qty{3}{\mega\hertz}.
För vissa specifika mätändamål, som till exempel för EMC mätningar, behöver man
ha filter av rätt bandbredd och dessa är extra optioner.

Filtret, vars inställning brukar kallas för \emph{Resolution Bandwidth},
fungerar som ett fönster, där fönstret släpper in de signaler som finns i det
spektrum som man önskar studera.

Om analysatorn inte skulle ha något filter skulle den ta in alla signaler som
finns inom dess specificerade frekvensområde.

Ett brett filter släpper igenom ett större frekvensområde och är användbart för
signaler med större bandbredd.
Ett smalare filter är att föredra för signaler med smalare bandbredd.

Ett brett filter innebär också att analysatorn kan svepa snabbare.
Det är då lättare att kunna detektera signaler som är kortvariga.
Ett smalare filter innebär att analysatorn måste svepa långsammare, men kan då
istället hitta signaler som inte skulle ha synts med det bredare filtret.

Detta filter har också en annan egenskap som är viktig -- ett brett filter
släpper också igenom mer brus, vilket påverkar den lägsta brusnivån som
analysatorn kan presentera.

För att nå högre känslighet kan man välja smalare filter.
Ju smalare filter, desto högre känslighet, men också ett långsammare svep över
det aktuella frekvensområdet.

\subsubsection{Fördjupning}

På en spektrumanalysator ställs stora krav på känslighet och förmåga att hantera
starka signaler i närheten av en svag men önskad signal utan att falska signaler
påverkar instrumentet.
Kraven har medfört att kostnaden för instrumentets uppbyggnad och ingående
komponenter under lång tid har varit mycket hög.

Under många decennier har sändaramatörer därför varit hänvisade till
marknaden för begagnade instrument som haft minst tio till tjugo år på nacken.

Det är bara för några år sedan som det har kommit produkter på marknaden som nu
kommit ner i kostnadsnivåer som inneburit att radioamatörer kan köpa dessa
instrument, i nyskick.

En modern spektrumanalysator av dyrare snitt erbjuder också en möjlighet att
analysera den modulerade signalen.
Sålunda förekommer instrument på marknaden som är specialdesignade att
analysera signaleringsinnehållet i olika system, till exempel Bluetooth, olika
Wi-Fi- och mobiltelefonsystem.

För en specifik mätning över ett visst frekvensspektrum, kan man ställa in en
så kallad startfrekvens, samt motsvarande stoppfrekvens för spektrumet ifråga.
Analysatorn kommer då att svepa, från startfrekvensen, till stoppfrekvensen.
Man kan också att välja en frekvens mitt i spektrumet, och därefter ett
valfritt så kallat span.
\emph{Span} betecknar det frekvensspann som är önskvärd för att man ska
kunna studera signalerna inom det aktuella spektrumet.

För att kunna göra bättre avläsningar kan man sätta markörer, så att man kan
avläsa frekvens och amplitud för den punkten.
Ibland sätter man dubbla frekvensen för att avläsa skillnaden i amplitud,
vilket kan vara relevant för filter eller avläsa den relativa styrkan på ett
sidband.

Detektorn kan vara topp-detektor eller RMS detektor, det kan finnas flera.
För specifika mätändamål som till exempel EMC mätningar behöver man specifik
detektor.
Det är viktigt att välja rätt detektor när signalen ska presenteras.
Valet av detektor kan påverka den presenterade nivån med flera \unit{\decibel}.

Det finns olika detektorer beroende på hur man vill ha signalen presenterad.
Det finns toppvärdesindikerande (Auto-peak, max peak, min peak) detektorer som
indikerar signalens toppvärden.
Det finns medelvärdesbildande (\emph{Average}, \emph{Sample}) detektorer som, om
instrumentet arbetar med digitalteknik, plockar ögonblickliga mätvärden
slumpmässigt.
Det finns också speciella detektorer (s.k. Quasi-Peak) som används för att
mäta enligt specifika EMC-mätningar.

En viktig detektor att komma ihåg är den så kallade RMS-detektorn.
Den utvecklades för att kunna mäta på digitalt modulerade signaler, ofta med
varierande fas- och amplitudinformation.

Denna detektor är att rekommendera för att mäta på digitalt modulerade signaler.
Den finns oftast i lite modernare analysatorer.

En vanlig Average- eller Sample-detektor enligt ovan, förväntar sig att
RF-signalens förlopp är i stort sett återkommande konstant, vilket är fallet
med analoga signaler som består av en bärvåg.

En digitalt modulerad signal har ett innehåll som förändras hela tiden, oftast
både till fas och amplitud.

RMS-detektorn läser in -- samplar -- den digitalt modulerade signalen och tar
konstant mätvärden ur den fasvarierande RF-signalen.
Den följer RF-signalens förändrade innehåll.

Denna detektor är därför utmärkt att använda för att mäta på digitalt modulerade
signaler, till exempel Bluetooth- eller Wifi-signaler, men även de digitala
system vi har inom amatörradion, då dessa signaler innehåller förändringar i fas
och amplitud.

Den kan självklart också användas för att mäta på analoga signaler.
Att den plockar ögonblicksmätvärden även på en analog signal gör alltså inget.

Efter detektorn finns ett filter, ofta benämnt med videobandbredd, som
medelvärdesbildar detektorns amplitudestimat över tiden.
Oftast regleras det automatiskt med bandbredden på filtret, eftersom smalare
filter behöver proportionerligt längre tid för att ge ett bra resultat.
Sveptiden beror därför på antalet punkter för frekvenser, bandbredd på filtret
och videobandbredden.
Ibland kan man styra videobandbredden manuellt för att få en längre tid, då
det kan vara gynnsamt för att få en tydligare bild, det vill säga ta bort brus
och störningar som enbart skapar variationer för att man inte observerar ett bra
medelvärde.
Videobandbredden påverkar alltså inte själva mätresultatet, utan är enbart till
för att användaren lättare ska kunna avläsa mätningen.

\subsection{Signalgeneratorn}
\harecsection{\harec{a}{8.2.1.4}{8.2.1.4}}

\emph{Signalgeneratorn} (eng. \emph{signal generator}) är ett instrument, som
namnet antyder, genererar en signal, i detta fall en radiofrekvent signal.

Detta instrument kan användas för att till exempel testa mottagare, eller för
att generera en eller flera kontrollerade signaler för att till exempel testa
förstärkarsteg.

Äldre signalgeneratorer var oftast uppbyggda runt en resonanskrets, och drev
ofta i frekvens när de värmdes upp.
De var sålunda inte stabila.
Senare generatorer arbetade med frekvenssyntes, och var att föredra i detta
sammanhang.

Den kan också användas för att generera en testsignal, som man matar in på en
mottagares ingång, för att sedan kunna följa signalen med en spektrumanalysator
(se \ssaref{spektrumanalysator}).

En bra signalgenerator ska ha förmågan att ge en så ren signal som möjligt,
där övertoner och sidband av olika slag är så låga som möjligt.
Ofta genererar generatorn ett egenbrus runt den inställda signalen.
Detta brus avtar, ju längre bort man kommer från den inställda signalen.
Detta brus ska förstås vara så lågt som möjligt.

En annan önskvärd parameter är möjligheten att kunna reglera den radiofrekventa
utnivån över ett stort område.
Signalgeneratorer innehåller oftast någon form av mätfunktion för att kunna
mäta nivån.

En fördel är om signalgeneratorn har en möjlighet att själv skapa modulation,
till exempel AM- eller FM-signaler.
Det kan emellanåt finna en inbyggd lågfrekvensgenerator, där man kan ställa in
önskad frekvens för att till exempel generera en ton.
I detta sammanhang brukar det också finnas möjlighet att justera
modulationsgraden för AM, eller deviationen för FM-signalen.

\subsection{Nätverksanalysator}
\index{nätverksanalysator}
\index{network analyzer}
\index{antennanalysator}
\index{skalär nätverksanalysator}
\index{Scalar Network Analyzer (SNA)}
\index{SNA}
\index{tracking generator}
\index{vektornätverksanalysator}
\index{Vector Network Analyzer (VNA)}
\index{VNA}

En \emph{nätverksanalysator} (eng. \emph{network analyser}) används för att
mäta hur mycket signal som går igenom en koppling, till exempel filter eller
förstärkare, eller hur mycket signal som reflekteras tillbaka från till exempel
en antenn.
Ibland kallas detta även för \emph{antennanalysator} i amatörradiosammanhang.

En nätverksanalysator som enbart kan mäta amplituder kallas ibland för
\emph{skalär nätverksanalysator} (eng. \emph{Scalar Network Analyzer, SNA}).
En spektrumanalysator med en så kallad \emph{tracking generator}, som genererar
en signal med samma frekvens som man analyserar, kan agera SNA.
En signalgenerator med svepfunktion kan också agera SNA.

En nätverksanalysator som mäter fasen både på utgående och inkommande signal
kan även mäta fasförskjutningen, och då kan man representera fasen både som
komplex storhet eller med polära koordinater, det vill säga amplitud och fas.
En sådan nätverksanalysator kallas för \emph{vektornätverksanalysator} (eng.
\emph{Vector Network Analyzer, VNA}).

Användningsmässigt liknar en nätverksanalysator en spektrum-analysator, men
med flera väsentliga skillnader.
För att få korrekt mätning av amplitud och fas läggs större vikt vid att göra
\emph{kalibrering} (eng. \emph{calibration}), något som görs för att kompensera
varierande amplitud och fas för olika frekvenser.
Vid kalibrering använder man ofta \emph{last} (eng. \emph{load}),
\emph{kortslutning} (eng. \emph{short}) samt \emph{öppen port} (eng.
\emph{open}) mätning av kalibreringsreferenser.
För två-ports mätning använder man även en \emph{överföring} (eng.
\emph{through}) för att få port-till-port egenskaperna korrekt.
Efter kalibrering av instrumentet så korrigeras mätningarna, och ibland kan
skillnaderna vara drastiska.

\newpage %layout
Nätverksanalysatorn har därtill ofta ett stort antal olika sätt att presentera
mätresultaten så att man kan mäta enligt \emph{scatter-modellen},
\emph{return loss (RL)}, \emph{VSWR}, \emph{Smith-diagram} och så vidare.
Det gör att en nätverksanalysator kan vara ett kraftfullt verktyg som korrekt
använt kan ge god insikt i hur en krets beter sig.

%
%
% Kapitel 10 Elektromagnetisk kompatibilitet
\chapter{Elektromagnetisk kompatibilitet}
\label{ch:EMC}
\index{EMC}

Det moderna samhället blir alltmer tekniskt avancerat och antalet elektroniska
apparater i hemmen och på arbetsplatserna ökar kraftigt.
Den ökande mängden och komplexiteten hos apparaterna kräver därför regler, som
styr både utförande och användning med rimligt bibehållen säkerhet och funktion.
Internationella och nationella väl preciserade regler för radio och
teleteknisk samexistens är numera helt nödvändiga.

% Avsnitt 10.1 Störningar och störkänslighet
\chapter{EMC}
\label{EMC}
\index{EMC}

\noindent Det moderna samhället blir alltmer tekniskt avancerat och antalet elektroniska
apparater i hemmen och på arbetsplatserna ökar kraftigt.
Den ökande mängden och komplexiteten hos apparaterna kräver därför regler, som
styr både utförande och användning med rimligt bibehållen säkerhet och funktion.
Internationella och nationella väl preciserade regler för radio- och
teletekniskt samexistens är numera helt nödvändiga.

\section{Störningar och störkänslighet}

\subsection{Om EMC-lagen}
\label{EMC-lagen}
\index{EMC!lag}
\index{EMC!förordning}
\index{EMC!apparater}
\index{EMC!störningar}

Samlingsbegreppet är \emph{Electromagnetic Compatibility} (EMC), det vill säga
en apparats förmåga att fungera tillfredsställande i sin elektromagnetiska
omgivning så att den:

\begin{itemize}
\item inte alstrar en elektromagnetisk störning som överskrider en nivå som
  tillåter radio- eller teleutrustning eller annan utrustning att fungera som
  avsett

\item har en sådan tålighet att den elektromagnetiska störning som kan
  förväntas vid avsedd användning inte medför att utrustningens funktion
  försämras i en oacceptabel utsträckning.
\end{itemize}

Lagen om elektromagnetisk kompatibilitet, \emph{SFS 1992:1512} \cite{SFS1992:1512}
ger regeringen eller den myndighet regeringen bestämmer rätt att i fråga om
kommunikationer eller näringsverksamhet eller skydd för liv, personlig säkerhet
eller hälsa meddela föreskrifter om EMC.
Förordning om elektromagnetisk kompatibilitet \emph{SFS 2016:363}
\cite{SFS2016:363} definierar nyckelbegreppen; apparater, EMC, elektromagnetisk
störning och tålighet.

Lagen och förordningen tillsammans med Elsäkerhetsverkets föreskrifter
\emph{ELSÄK-FS} samt direktivet om elektromagnetisk kompatibilitet implementerar
EU-direktiv 2014/30/EU i Sverige.
Elsäkerhetsverket är ansvarig myndighet, med rätt att utfärda föreskrifter om
bland annat skyddskraven, kontroll och märkning samt om vissa undantag.

Ovanstående handlar om störningar orsakade av apparater eller störningar på
apparaters funktion.
Sådana störningar kan anmälas till Elsäkerhetsverket.
Störningar orsakade av radiosändare eller radiomottagare behandlas i avsnitt
\ssaref{LEK}.


\subsection{Utdrag ur LEK}
\label{LEK}
\index{LEK}
\index{störning}
\index{störning!skadlig}
\index{störning!tillåten}

Post- och telestyrelsens föreskrifter om undantag från tillståndsplikt för
användning av vissa radiosändare \emph{PTSFS 2022:19} \cite{PTSFS2022:19}
hänvisar till lag om elektronisk kommunikation \emph{LEK} \emph{SFS 2022:482}
\cite{SFS2022:482}.
Där kan följande läsas om åtgärder vid störningar:
%%
\begin{quote}
	3~kap. 23~\S{} Om det till följd av tillståndshavarens användning av en
	radiosändare uppkommer en otillåten skadlig störning, ska tillståndshavaren
	omedelbart se till att störningen upphör eller i möjligaste mån minskar.
	Den som använder en radiomottagare som stör användningen av en annan
	radiomottagare har motsvarande skyldighet.
\end{quote}
%%
Skadlig störning definieras i 1~kap. 7~\S{} som:
%%
\begin{quote}
	en störning som äventyrar funktionen hos en radionavigationstjänst eller
	någon annan säkerhetstjänst, eller som på annat sätt allvarligt försämrar,
	hindrar eller upprepat avbryter en radiokommunikationstjänst som fungerar i
	enlighet med gällande bestämmelser, inbegripet störning av befintliga eller
	planerade tjänster på nationellt tilldelade frekvenser
\end{quote}
%%
Tillåten störning definieras i förarbetena till \emph{LEK} som en störning
orsakad av användares delning av frekvens och anses då vara tillåten.
Observera dock att användare med sekundär status inte får störa användare med
primär status vid delning av frekvens eller frekvensband.
%%
I \emph{LEK} definieras även radioanläggning:
\index{radioanläggning}
\begin{quote}
	anordning som möjliggör radiokommunikation eller bestämning av position,
	hastighet eller andra kännetecken hos ett föremål genom sändning av radiovågor
	(radiosändare) eller mottagning av radiovågor (radiomottagare)
\end{quote}

\subsection{Utstrålning från amatörradiosändare}
\index{uteffekt!begränsning}

Vad som sägs i 3~kap. 4~\S{} Lag om elektronisk kommunikation och skrivningen i
Post- och telestyrelsens föreskrifter om undantag från tillståndsplikt för
användning av vissa radiosändare 3~kap. 14~\S{} \emph{De tekniska egenskaperna hos
	amatörradiosändaren ska anpassas så att de inte stör användningen av andra
	radioanläggningar}.
Tillsammans med skrivningarna i Strålsäkerhetsmyndighetens \emph{SSMFS 2008:18}
och Förordning om elektromagnetisk kompatibilitet \emph{SFS 2016:363}.
Medför detta att sändareffekten alltid ska anpassas så att styrkan av utstrålade
fält inte förorsakar störningar eller för höga nivåer av elektromagnetiska fält.

Den enligt undantagsföreskrifterna högsta tillåtna uteffekten kan alltså inte
användas hinderslöst.

Om störningarna inte kan avhjälpas kan PTS komma att anvisa om restriktioner
(begränsningar i sändningstillståndet), det kan vara sändningsförbud under
vissa tider, på vissa frekvenser, över viss sändareffekt etc.

%% k7per: PM?
\subsection{PM vid störningsproblem}
\index{störning!PM}

\begin{itemize}
	\item Störningar är alltid förenade med obehag och ställer grannsämjan på prov.
	Håll dig väl med dem som bor i omgivningen! Observera även att gällande EMC
	har myndigheter inte rättighet att få tillträde till bostäder.
	\item Om det väcks klagomål på dig om störningar, ska du först
	kontrollera din egen sändare och antennanläggning.
	\item Be därefter att få undersöka antennanläggning och apparater hos
	den som besväras av störningar.
	\item Om du ser en lösning, berätta om vad som kan göras.
	Kom överens om vad som får göras.
	Ändra då inte något inne i apparater, men prova gärna ut yttre,
	kompletterande filter etc.
	\item Om det inte går att komma till rätta med störningarna bör de som
	levererat och installerat anläggningen anlitas.
	\item Störningsanmälan gällande radiokommunikation kan göras på PTS webbplats.
	\item Störningsanmälan gällande produkter kan göras till Elsäkerhetsverket.
\end{itemize}

\subsection{Arbeta aktivt med avstörning}
\index{störning!avstörningslåda}

\begin{itemize}
	\item Låna hem en av SSA:s avstörningslådor och försök att finna en lösning.
	I lådan finns ett sortiment av frekvensfilter för avstörning.
	\item Undvik att störa i onödan.
	Sänk sändareffekten och begränsa sändningstiden under utprovningen av en
	lösning.
\end{itemize}
%%
Lyckas du inte själv med att störa av:
%%
\begin{itemize}
	\item Ta gärna hjälp av en radioamatör med erfarenhet av avstörning.
	\item Anlita annan sakkunnig hjälp.
\end{itemize}

% Avsnitt 10.2 Störningar i elektronik
%\newpage
\section{Störningar i elektronik}
\index{Electromagnetic Interference (EMI)}
\index{EMI}
\index{Electromagnetic Susceptibility (EMS)}
\index{EMS}

Liksom att radiomottagning kan ''störas'' av sändningar som inte är av
intresse, så kan störningar i form av radiovågor från olika slags
elektrisk utrustning försvåra mottagning eller andra funktioner.

Utstrålning från till exempel datorer, kabel-TV, hushållsmaskiner,
tändgnistor från oljebrännare, bilar och mopeder etc. är radiovågor.
Elektriska apparater kan alltså både störa och störas genom
radiovågor, även om de inte är definierade som \emph{radioanläggning},
det vill säga radiosändare och/eller radiomottagare.

Störningar som uppstår av elektromagnetiska fält kallas
\emph{Electromagnetic Interference (EMI)}.
Känsligheten för sådana störningar kallas för
\emph{Electromagnetic Susceptibility (EMS)}.

\subsection{Blockering}
\harecsection{\harec{a}{9.1.1}{9.1.1}}
\index{blockering}
\index{störning!blockering}
\label{blockering}

I de flesta mottagare finns en automatisk förstärkningsreglering.
Om insignalerna blir för starka, så räcker regleringen inte till.
Då överstyrs förstärkarstegen så att de arbetar olinjärt.
Detta kallas blockering och kan medföra att mottagaren tystnar eller en TV-bild
försvinner.

Ett sätt att undvika blockering är att koppla en \emph{dämpsats}
(eng. \emph{attenuator}) till mottagaringången.
En sådan sänker dock signalstyrkan över hela frekvensområdet, inte bara för en
viss signalfrekvens.

\subsection{Interferens}
\harecsection{\harec{a}{9.1.2}{9.1.2}}
\index{interferens}
\index{störning!interferens}

När den önskade signalen störs av en annan signal nära i frekvens, kallas det
\emph{interferens} (eng. \emph{interference}).
I mottagaringången finns frekvensfilter, som undertrycker ej önskade signaler,
om de inte ligger alltför nära.
Om ingången inte är tillräckligt selektiv, kan det behövas en tillsats som
förbättrar selektiviteten.

\subsection{Intermodulation}
\harecsection{\harec{a}{9.1.3}{9.1.3}}
\index{intermodulation}
\index{störning!intermodulation}

Blandningsprodukter av signaler i en mottagare eller sändare kallas för
\emph{intermodulation} och kan höras som falska signaler i en mottagare.
(se även kapitel~\ssaref{intermodulation})

% \newpage
\subsection{LF-detektering}
\harecsection{\harec{a}{9.1.4}{9.1.4}}
\index{störning!LF-detektering}
\index{LF-detektering}

HF-signaler kan komma in genom in- och utgångarna för LF samt genom nätkabeln.
Dessutom förekommer direktinstrålning av radiovågor genom apparathöljet, om
detta inte har tillräckligt avskärmande verkan.
\emph{LF-detektering} uppstår när HF-signaler demoduleras i diodsträckor i den
störda apparatens komponenter.
Detta sker oavsett vilken frekvens som sändaren eller mottagaren är inställd på.
Den uppstår särskilt vid AM- eller SSB-modulerade sändningar
samt av transienter vid bärvågsnyckling av sändare.

Problem med LF-detektering kan minskas genom att minska uteffekten från sändaren
eller genom att flytta antennen så att fältstyrkan minskar.
Ofta är det inte möjligt att förhindra LF-detektering utan ingrepp i den störda
apparaten.
Sådana ingrepp bör endast utföras av fackman.

% Avsnitt 10.3 Störningsorsaker
\section{Störningsorsaker}
\label{Störningsorsaker}
\subsection{Störningar från sändare}
\harecsection{\harec{a}{9.2.1}{9.2.1}, \harec{a}{9.2.2}{9.2.2}}
\index{störning!av sändare}

HF-förstärkare, till exempel i sändarslutsteg, kan komma i oönskad självsvängning,
vilket kan uppstå av flera orsaker; det kan vara bristande avkoppling av
matningsspänningar, induktiv och/eller kapacitiv återkoppling etc.

Effektförstärkare kan även överstyras.
Då uppstår intermodulation och övertoner som strålas ut på oönskade frekvenser.
I många fall kan störningar undvikas med en eller flera av följande åtgärder:

\begin{itemize}
\item Använd inte mer effekt än vad som behövs.
\item Undvik att överstyra sändarslutsteget (kontrolleras t.ex. med
  ALC-mätaren).
\item Förse sändarutgången med lågpassfilter.
  På så sätt undertrycks övertoner.
\item Anpassa sändarens och antennanläggningens impedanser till varandra.
  Stäm av sändarens \(\pi\)-filter och/eller en separat antennanpassningsenhet
  rätt.
  En felinställd sändare kan medföra oavsiktligt utstrålning.
\item Koppla in balanseringsnät (balun) mellan osymmetriska antennledningar
  (koaxialkablar) och symmetriska antenner.
\item Placera antennen högt och fritt och så långt från personer och
  störningskänslig utrustning som möjligt.
  Fältstyrkan är nämligen högst närmast antennen.
  Se kapitel~\ssaref{EMF} om elektriska fält.
\item Undvik direkt HF-instrålning på elnätet genom att använda nätfilter.
\item Använd ''mjuk'' nyckling av bärvågen (avrundade telegrafitecken).
  Vid hård nyckling alstras övertoner i form av knäppar som hörs långt vid
  sidan av sändningsfrekvensen. Se även kapitel~\ssaref{Nycklingsfilter}.
\end{itemize}

\subsection{Störningar på radiomottagning}
\harecsection{\harec{a}{9.2.3.1}{9.2.3.1}, \harec{a}{9.2.3.2}{9.2.3.2}, \harec{a}{9.2.3.3}{9.2.3.3}}
\index{störning!radiomottagning}

I regel uppstår störningar på radiomottagning först när instrålade signaler
uppnått en viss styrka -- immunitetsnivån för HF.
Man kan tala om tre slags HF-immunitet hos mottagare:

\begin{itemize}
\item mot signaler genom antenningången
\item mot signaler genom övriga anslutna ledningar, till exempel högtalar-
  och nätledningar
\item mot elektriska och/eller magnetiska fält som strålar direkt in i
  apparaten.
\end{itemize}

I de båda första fallen kan det hjälpa med komplettering med hög- och/eller
lågpassfilter och skärmströmsfilter.

Störningar orsakade av instrålning är svårast att avhjälpa och fordrar ingrepp i
mottagaren, vilket bör överlåtas till en fackman med tillgång till
tillverkarens serviceinstruktioner.

\subsection{Störningar på TV-mottagning}
\index{störning!TV-mottagning}
\index{störning!digital-TV}

Störningar från radiosändare kan yttra sig till exempel på följande sätt för
digital TV:

Sändningar, främst på 2-meter men även på 70-cm, kan orsaka blockering och
bildstörningar vid mottagning av digital-TV.
TV-bilden tappar då information, det blir pixlingar (fyrkantiga rutor), grönt
brus eller hela bilden fryses eller försvinner kortvarigt.
För analog TV i till exempel kabel-TV-nät kan störningarna yttra sig på
följande sätt:

\begin{itemize}
\item Vid sändning av amplitudmodulerade signaler, till exempel AM och SSB,
  uppstår ljudförvrängning i ljudkanalen samt ränder med mera i bilden.
\item Vid sändning av FM och CW uppstår ljudstörningar samt
  kontrastvariationer, interferensmönster (moire-effekter) med mera i bilden.
\end{itemize}

Problem med störningar av den här typen har minskat betydligt sedan digital-TV
infördes och de flesta TV-sändningar numera sker på VHF- och UHF-banden.

Störningar i TV som orsakas av sändare på lägre frekvenser kan i många fall
avhjälpas med frekvensfilter.
Ett lågpassfilter efter en kortvågsändare kan till exempel dimensioneras att
endast släppa igenom signaler under cirka \qty{35}{\mega\hertz}.
Läs mer om lågpassfilter i kapitel~\ssaref{Lågpassfilter}.

Ett högpassfilter före en TV-mottagare kan till exempel dimensioneras att
endast släppa igenom signaler med frekvenser över cirka \qty{35}{\mega\hertz}.
Läs mer om högpassfilter i kapitel~\ssaref{Högpassfilter}.

Om inte mottagning i TV-band I och II är av intresse, så kan ett högpassfilter
med en gränsfrekvens av cirka \qty{160}{\mega\hertz} sättas in.
Det dämpar den oönskade utstrålning från sändare i HF- och lägre VHF-området,
det vill säga upp till och med \SIrange{144}{146}{\mega\hertz} amatörband.
Däremot släpps TV-band III (\SIrange{174}{230}{\mega\hertz}) och TV-banden IV
och V igenom (\SIrange{470}{890}{\mega\hertz}).

Ytterligare avstörningsmedel kan sättas in om det uppstår störningar av
amatörradiosändningar.
Det kan vara skärmströmsfilter på antennkablar, bandspärrar samt sug- och
spärrkretsar avstämda till störfrekvensen, bandpassfilter avstämt till
nyttofrekvensen. Läs mer om filter i kapitel~\ssaref{spärrfilter}.

Ett vanligt störningsfall är att en bredbandig antennförstärkare blir
överstyrd eller blockerad av en sändare. Se även kapitel~\ssaref{blockering}.

%% k7per
%% \newpage % layout
\begin{itemize}
\item Försök att undvika antennförstärkare.
\item Försök att undvika dåligt skärmade skarvar och antennkontakter.
\item Skaffa en bättre TV-antenn som även kan ta emot TV-sändningar på VHF.
  Många hushåll har idag endast en UHF-antenn och har därför dålig
  antennsignal på VHF-bandet där sändningar för HD-TV sker i många områden.
\end{itemize}

\subsection{Störningar på LF-apparater}
\index{störning!LF-apparater}

Störningar av HF-instrålning i ljudbandspelare, LF-förstärkare, telefonapparater
etc. kan ofta stoppas med avkopplingskondensatorer och HF-drosslar.
Moderna avstörningsdrosslar innehåller oftast något ferritmaterial i form av
rör, stavar eller ringar.

% Avsnitt 10.4 Avstörningsmetoder
\section{Avstörningsmetoder}
\index{störning!avstörningsmetoder}
\label{avstörning}
\index{avstörning}

\mediumfig[0.5]{images/cropped_pdfs/bild_2_9-01.pdf}{Nätfilter}{fig:bildII9-1}
\mediumbotfig{images/cropped_pdfs/bild_2_9-02.pdf}{Lågpassfilter för sändare}{fig:bildII9-2}

\subsection{Allmänt}
För att prova ut ett filter, som bäst löser ett visst radiostörningsproblem,
kan man behöva tillgång till ett filtersortiment.
Som exempel nämns bland annat filter i SSA:s avstörningslådor.

\subsection{Nätfilter}
\harecsection{\harec{a}{9.3.1.1}{9.3.1.1}}
\index{nätfilter}
\index{avstörning!nätfilter}

Nätledningar kan fungera som antenn.
I sändarfallet kan HF-signaler komma ut i elnätet genom nätledningen och störa
andra apparater både genom direktanslutning och genom strålning.
I mottagarfallet kan HF-signaler uppfångas av nätledningen, ledas in i
apparaterna och LF-detekteras där.
För att förhindra sådana störningar behövs ett nätfilter.

Nätfiltret ska vara dimensionerat för den nätström, som apparaten är avsäkrad
för och bör anslutas så nära apparaten som möjligt.
Om filtret inte kan placeras där, kan det vara nödvändigt att även skärma
nätledningen mellan filtret och apparaten och jorda skärmen.

% \mediumfig[0.5]{images/cropped_pdfs/bild_2_9-01.pdf}{Nätfilter}{fig:bildII9-1}

Om ledningen förses med till exempel en serieinduktans -- en drossel -- så
dämpas HF-signalerna.
En drossel kan man göra till exempel genom att linda upp några varv av nätsladden
närmast apparaten på toroider eller en eller flera sammanlagda ferritstavar.
I svåra fall kan det behövas ett bredbandigt nätfilter, liknande
det på bild~\ssaref{fig:bildII9-1}.

Det kan förekomma kraftiga spänningstransienter (spänningsstötar) på elnätet.
Dessa transienter kan leda till felfunktioner i anslutna apparater.
För att förebygga sådana fel kan man koppla in ett överspänningsfilter, som kan
vara separat eller sammanbyggt med nätfiltret.

%% \newpage % layout
\subsection{Lågpassfilter}
\label{Lågpassfilter}
\index{lågpassfilter}
\index{avstörning!lågpassfilter}

% \mediumbotfig{images/cropped_pdfs/bild_2_9-02.pdf}{Lågpassfilter för sändare}{fig:bildII9-2}

Lågpassfilter släpper igenom signaler med frekvenser under filtrets
gränsfrekvens.

Ett lågpassfilter med lämpligt vald gränsfrekvens dämpar till exempel
övertonsutstrålningen från en sändare, vars sändarfrekvens ligger under filtrets
gränsfrekvens medan övertonerna ligger över dess gränsfrekvens.

Övertoner kan dämpas med lågpassfilter.
En överton är i detta sammanhang en multipel av sändningsfrekvensen
(grundtonen) exempelvis för
\qty{3,5}{\mega\hertz} grundtonen = (1:a harmoniska) \qty{3,5}{\mega\hertz},
1:a överton = (2:a harmoniska) \qty{7,0}{\mega\hertz},
2:a överton = (3:e harmoniska) \qty{10,5}{\mega\hertz} osv.

Viktigt för avsedd filterverkan är, att filtret ansluts med korrekt
impedansanpassning och med kortast möjliga ledningar.
Detta gäller för övrigt alla filter.

Utstrålning utanför sändningsslagets tillåtna bandbredd anses som
''icke önskad utstrålning''.
Vidare gäller att sådan utstrålning från amatörradiosändare ska hållas så låg
som dagens amatörradioteknik medger.
Bild~\ssaref{fig:bildII9-2} visar principen för lågpassfiltret TP~30 för kortvåg,
med gränsfrekvensen \qty{36}{\mega\hertz}, att kopplas mellan sändaren och
antennledningen.
Med denna gränsfrekvens dämpas övertoner från sändare så att risken för
TV-störningar minskar.

\mediumfig{images/cropped_pdfs/bild_2_9-03.pdf}{Högpassfilter för VHF/UHF-mottagare}{fig:bildII9-3}

%\newpage
\subsection{Högpassfilter}
\label{Högpassfilter}
\index{högpassfilter}
\index{avstörning!högpassfilter}
\index{avstörning!antennförstärkare}

Högpassfilter släpper igenom signaler med frekvenser över filtrets
gränsfrekvens.

Bild~\ssaref{fig:bildII9-3} visar principen för högpassfiltret HP~40-S med
gränsfrekvensen \qty{47}{\mega\hertz}, att kopplas in mellan antennledningen och
en mottagare för VHF eller högre frekvenser.

Störningar kommer inte alltid ''utifrån''.
De kan till exempel alstras i bredbandiga antennförstärkare, vilka lätt
överstyrs av alla slags signaler från ett stort frekvensområde.
Man kan då koppla in ett högpassfilter före bredbandsförstärkaren, men en
bättre lösning är att byta till en väl skärmad passbands- eller ännu hellre
kanalförstärkare.

Koaxialkablar med täta skärmar och rätt monterade anslutningskontakter är också
viktigt för en lyckad avstörning.

%\clearpage
\mediumplustopfig{images/cropped_pdfs/bild_2_9-05.pdf}{Spärrfilter för mottagare}{fig:bildII9-5}
\mediumplustopfig{images/cropped_pdfs/bild_2_9-04.pdf}{Ingångsimpedansen i resonanskretsar}{fig:bildII9-4}

\subsection{Spärrfilter och sugkretsar}
\label{spärrfilter}
\index{spärrfilter}
\index{avstörning!spärrfilter}
\label{Sugkretsar}
\index{sugkretsar}
\index{avstörning!sugkretsar}
\index{stub}
\index{avstörning!stub}

\mediumplustopfig{images/cropped_pdfs/bild_2_9-06.pdf}{Sugkretsar för mottagare}{fig:bildII9-6}

Om en störande signal råkar finnas inom passbandet för mottagaren kan
man undertrycka -- ''spärra'' -- den signalen med ett spärr- eller sugfilter.
Vilket man väljer är inte kritiskt.

Den störande signalen kan ''spärras'' med en parallellresonanskrets i
serie med mottagaringången, se bild~\ssaref{fig:bildII9-5}.
Kretsen består av en induktans och en kapacitans.

Om man använder en stub som resonanskrets -- till exempel en koaxialkabel -- så
ska den ha längden \(\lambda/4\) och vara ''kortsluten'' eller ha
längden \(\lambda/2\) och vara ''öppen''.
Bild~\ssaref{fig:bildII9-4} visar exempel på hur ingångsimpedansen kan användas.

Man kan även kortsluta -- ''suga bort'' den störande signalen med en
serieresonanskrets parallellt över mottagaringången, se bild~\ssaref{fig:bildII9-6}.
Om man då använder en stub, så ska den ha längden \(\lambda/4\) och
vara ''öppen'' eller ha längden \(\lambda/2\) och vara ''kortsluten''.

Den störande signalen kan undertryckas ytterligare med fler stubar,
som ordnas som i bild~\ssaref{fig:bildII9-6}.
Filtret består då av öppna \(\lambda/4\)-stubar, som utgör avgreningar från
antennkabeln med ett avstånd av \(\lambda/4\).

(Om stubarna i detta filter kortsluts, så bildas ett bandpassfilter i stället).

Exempel på kommersiella spärrfilter är SF~145-S för \qty{144}{\mega\hertz} och
SF~435-S, för \qty{435}{\mega\hertz} amatörband.
De är avsedda att kopplas in före mottagare som störs av amatörradiosändningar.

\begin{description}
\item[SF~145-S] spärrar amatörbandet \SIrange{144}{148}{\mega\hertz} och släpper
igenom banden 0--120 och \SIrange{174}{870}{\mega\hertz}.

\item[SF~435-S] spärrar amatörbandet \SIrange{430}{440}{\mega\hertz} och släpper
igenom 0--350 och \SIrange{470}{870}{\mega\hertz}.
\end{description}

\subsection{Nät- och skärmströmfilter för mottagning}
\index{gemensam överföring}
\index{gemensam strömöverföring}
\index{CM}
\index{Common Mode (CM)}
\index{common mode current}

\smallfig[.35]{images/cropped_pdfs/bild_2_9-07.pdf}{Nät- och skärmströmfilter}{fig:bildII9-7}

Bild~\ssaref{fig:bildII9-7} visar nät- och skärmströmfilter.
Det är vanligt att \emph{gemensam strömöverföring} (eng.
\emph{common mode current}) uppstår som läckage från utrustning och antenn.
Detta kallas ofta ledningsbunden störning.
Det gör att att antennkabeln kan också fungera som antenn.
Särskilt i skärmskarvar kan HF-strömmar läcka ut och in.
De kan då passera förbi eventuella antennförstärkare, filter etc. och orsaka
störningar.

I enkla fall kan gemensam ström stoppas med att linda upp kabeln några varv på
ferritstavar eller genom en stor ferritring som på bilden.
En nätkabel, så kallad sladdställ, får inte kapas och skarvas.

\newpage
\subsection{Phono-ingångsfilter (TBA~302)}
\index{avstörning!phonofilter}

\smallfigpad{images/cropped_pdfs/bild_2_9-08.pdf}{Phonoingångsfilter}{fig:bildII9-8}

\smallfig{images/cropped_pdfs/bild_2_9-09.pdf}{Högtalarledningsfilter}{fig:bildII9-9}

Bild~\ssaref{fig:bildII9-8} visar phonoingångsfilter.
Störande påverkan från radiosändningar kan uppstå om anslutningsledningarna
till phono-ingången i LF-förstärkare är dåligt skärmade och avkopplade.
Sådana störningar kan avhjälpas med ett filter.

\newpage
\subsection{Högtalarledningsfilter (EM 502-B)}
\index{avstörning!högtalarledning}

Bild~\ssaref{fig:bildII9-9} visar högtalarledningsfilter.
HF-instrålning på högtalarledningar kan ha en störande påverkan.
Detta kan undvikas genom koppla in HF-drosslar i ledningarna.
Dessa drosslar bör vara skärmade så att de inte verkar som antenner istället.

I enklare fall kan det räcka med att byta till skärmade högtalarkablar
eller att linda upp en sträcka av ledningarna på en ferritkärna.

\newpage
\subsection{Avkoppling av HF-signaler}
\harecsection{\harec{a}{9.3.1.2}{9.3.1.2}}
\index{avstörning!avkoppling}

\smallfig[.4]{images/cropped_pdfs/bild_2_9-10b.pdf}{HF-avkopplad bas på tre sätt}{fig:bildII9-10b}

\smallfig[.2]{images/cropped_pdfs/bild_2_9-10a.pdf}{HF-avkopplat styrgaller}{fig:bildII9-10a}

\smallfig[.3]{images/cropped_pdfs/bild_2_9-11.pdf}{Parasitfilter i HF-förstärkare}{fig:bildII9-11}

Med avkoppling av en signal menas att den avleds från en signalväg till en
annan.
Vid avstörning avkopplas vanligen den störande signalen till systemjord.

Störimmuniteten i mottagare kan alltså förbättras genom att LF-ingångarna
HF-avkopplas med kondensatorer och/eller HF-spärras med drosslar.

I svåra störningsfall kan det också bli nödvändigt med HF-avskärmning av
LF-ingångsstegen, liksom med ytterligare avstörningsfilter inne i förstärkaren.
Sådana åtgärder innebär emellertid att konstruktionsändringar har gjorts.
Apparatens elsäkerhetsmärkningar är då ogiltiga.

Bild~\ssaref{fig:bildII9-10b} och \ssaref{fig:bildII9-10a} visar några sätt att
avkoppla en oönskad signal från styrgallret i ett elektronrör respektive från
basen i en transistor.

\subsection{Parasitfilter}
\index{avstörning!parasitfilter}

Bild~\ssaref{fig:bildII9-11} visar parasitfilter i HF-förstärkare.
Förstärkarsteg kan råka i självsvängning, ofta på frekvenser i VHF/UHF-området.
Ett sätt att stoppa det är med så kallat parasitfilter.

\subsection{Nycklingsfilter}
\label{Nycklingsfilter}
\index{nycklingsfilter}
\index{avstörning!nycklingsfilter}

Bild~\ssaref{fig:bildII9-12} visar nycklingsfilter.
När en bärvåg nycklas, så bildas övertoner.
Blandningsprodukter av övertonerna och bärvågen hörs som knäppar på
omkringliggande frekvenser.
Märk att övertoner uppstår vid all bärvågsnyckling -- inte bara vid
morsetelegrafering!

När övergångstiden är kort (hård nyckling), så bildas fler övertoner
än när den är längre (mjuk nyckling).
Knäpparna kan till en del dämpas med ett nycklingsfilter där dels
insvängningsförloppet bromsas med en drossel i serie med nycklingskontakten och
dels ursvängningsförloppet med en seriekrets av en resistor och en kondensator,
kopplade parallellt över nycklingskontakten.

\smallfig{images/cropped_pdfs/bild_2_9-12.pdf}{Nycklingsfilter}{fig:bildII9-12}

%% k7per
%% \newpage % layout
\subsection{Förbättrad skärmning}
\harecsection{\harec{a}{9.3.1.3}{9.3.1.3}}
\index{avstörning!skärmning}

HF-energi kan i olyckliga fall även stråla ut genom sändarens hölje
och in genom andra apparaters hölje.
Det medför att apparaternas skärmningar och jordning måste förbättras.
Följ då elsäkerhetsbestämmelserna!
Se även avsnitt~\ssaref{elektriskafält}, \ssaref{elektromagnetiskafält} samt
\ssaref{jordning}.
%
%
% Kapitel 11 Elektromagnetiska fält
\chapter[EMF gränsvärden]{Elektromagnetiska gränsvärden}
\label{EMF}
\index{elektromagnetiska fält (EMF)}
\index{EMF}
En amatörradiostation skickar ut radiovågor, signaler, för att kommunicera
trådlöst med hela världen.
Radiovågorna kallas även elektromagnetiska fält (EMF)
(eng. \emph{Electromagnetic Field, EMF}).
Runt alla antenner som sänder ut radiovågor bildas elektromagnetiska fält av den
energi som skickas in i antennerna från radiosändaren.

\textbf{Radiovågorna från en amatörradiostation,} de elektro\-magnetiska fälten, klassas
som \emph{icke-joniserande strålning} och är som sådan strålning inte
tillräckligt energirik för att orsaka annat än uppvärmning av kroppens vävnad.

I allmänhet har studier visat att de nivåer av elektromagnetiska fält som
allmänheten kan utsättas för i närheten av en amatörradiostation ligger långt
under de värden där kroppstemperaturen skulle öka.

\index{icke-joniserande strålning}
\index{strålning!icke-joniserande}
\textbf{Icke-joniserande strålning,} som optisk strålning (infraröd strålning,
synligt~ljus och ultraviolett~strålning) och elektromagnetiska fält (radiovågor
och mikrovågor) är normalt inte lika energirik som joniserande strålning.
När elektromagnetisk strålning absorberas i biologisk vävnad eller material är
den dominerande effekten därför endast en temperaturhöjning i vävnaden eller
materialet.

\index{joniserande strålning}
\index{strålning!joniserande}
\textbf{Joniserande strålning,} partikelstrålning eller elektromagnetisk strålning, som
har tillräcklig energi för att rycka loss elektroner från de atomer som den
träffar och förvandla dem till positivt laddade joner, jonisering.
Exempel på joniserande strålning är röntgenstrålning och strålning från
radioaktiva ämnen.
Energin hos joniserande strålning kan vara så hög att den kan tränga in i
kroppen och påverka cellstruktur samt arvsmassa (DNA) i biologiskt material.

\index{WHO|see {World Health Organization (WHO)}}
\index{World Health Organization (WHO)}
\index{International Commission on Non-Ionizing Radiation Protection (ICNIRP)}
\index{ICNIRP}
Inom World Health Organization (WHO) finns ett program som kallas
''The International EMF Project'' och där samlas all vetenskaplig
information som finns om biologiska effekter orsakade av elektromagnetiska fält.
''International Commission on Non-Ionizing Radiation Protection'', (ICNIRP)
är en fristående organisation (erkänd av WHO) som bland annat använder denna
information för att utveckla riktlinjer för begränsning av exponeringsnivån för
elektromagnetiska fält.
Dessa riktlinjer används av många länder.

\index{Strålsäkerhetsmyndigheten (SSM)}
\index{SSM|see {Strålsäkerhetsmyndigheten (SSM)}}
\index{International Commission on Non-Ionizing Radiation Protection (ICNIRP)}
\index{ICNIRP}
Strålsäkerhetsmyndigheten (SSM) är den myndighet som har det formella ansvaret
för strålskydd i Sverige.
Myndigheten ska bland annat förebygga akuta skador och minska risken för sena
hälsoeffekter hos allmänheten till följd av exponering för elektromagnetiska
fält.

SSM har tagit fram allmänna råd SSMFS 2008:18~\cite{SSMFS2008:18} för
begränsning av allmänhetens exponering för elektromagnetiska fält.
De allmänna råden anger vilka referensvärden som gäller i Sverige.
Råden utgår från rekommendationer i EU-direktiv 1999/519/EG~\cite{1999/519/EG}.
EU-direktivet följer i sin tur de riktlinjer för begränsning av
elektromagnetiska fält som sammanställts av ICNIRP.

Eftersom grunden i amatörradioutövandet är att generera elektromagnetiska fält
för att kommunicera via radio så är kunskapen om EMF viktig.
Med de möjligheter radioamatörer har, måste de allmänna råden gällande EMF
följas.
Förståelsen för hur fält uppträder och hur de kan begränsas anses vara
fundamental kunskap för radioamatörer.

% Avsnitt 11.2 Fält
% Avsnitt 11.3 Allmännaråd
% Avsnitt 11.4 Utvärdering av EMF
% Avsnitt 11.5 Egenkontroll
% Avsnitt 11.6 Sammanfattning
\chapter[EMF gränsvärden]{Elektromagnetiska gränsvärden}
\label{EMF}

\section{Inledning}
\index{elektromagnetiska fält (EMF)}
\index{EMF}
En amatörradiostation skickar ut radiovågor, signaler, för att kommunicera
trådlöst med hela världen.
Radiovågorna kallas även elektromagnetiska fält (EMF)
(eng. \emph{Electromagnetic Field (EMF)}).
Runt alla antenner som sänder ut radiovågor bildas elektromagnetiska fält av den
energi som skickas in i antennerna från radiosändaren.

\textbf{Radiovågorna från en amatörradiostation,} de elektro\-magnetiska fälten, klassas
som \emph{icke-joniserande strålning} och är som sådan strålning inte
tillräckligt energirik för att orsaka annat än uppvärmning av kroppens vävnad.

I allmänhet har studier visat att de nivåer av elektromagnetiska fält som
allmänheten kan utsättas för i närheten av en amatörradiostation ligger långt
under de värden där kroppstemperaturen skulle öka.

\index{icke-joniserande strålning}
\index{strålning!icke-joniserande}
\textbf{Icke-joniserande strålning,} som optisk strålning (infraröd strålning,
synligt~ljus och ultraviolett~strålning) och elektromagnetiska fält (radiovågor
och mikrovågor) är normalt inte lika energirik som joniserande strålning.
När elektromagnetisk strålning absorberas i biologisk vävnad eller material är
den dominerande effekten därför endast en temperaturhöjning i vävnaden eller
materialet.

\index{joniserande strålning}
\index{strålning!joniserande}
\textbf{Joniserande strålning,} partikelstrålning eller elektromagnetisk strålning, som
har tillräcklig energi för att rycka loss elektroner från de atomer som den
träffar och förvandla dem till positivt laddade joner, jonisering.
Exempel på joniserande strålning är röntgenstrålning och strålning från
radioaktiva ämnen.
Energin hos joniserande strålning kan vara så hög att den kan tränga in i
kroppen och påverka cellstruktur samt arvsmassa (DNA) i biologiskt material.

\index{WHO|see {World Health Organization (WHO)}}
\index{World Health Organization (WHO)}
\index{International Commission on Non-Ionizing Radiation Protection (ICNIRP)}
\index{ICNIRP}
Inom World Health Organization (WHO) finns ett program som kallas
''The International EMF Project'' och där samlas all vetenskaplig
information som finns om biologiska effekter orsakade av elektromagnetiska fält.
''International Commission on Non-Ionizing Radiation Protection'', (ICNIRP)
är en fristående organisation (erkänd av WHO) som bland annat använder denna
information för att utveckla riktlinjer för begränsning av exponeringsnivån för
elektromagnetiska fält.
Dessa riktlinjer används av många länder.

\index{Strålsäkerhetsmyndigheten (SSM)}
\index{SSM|see {Strålsäkerhetsmyndigheten (SSM)}}
\index{International Commission on Non-Ionizing Radiation Protection (ICNIRP)}
\index{ICNIRP}
Strålsäkerhetsmyndigheten (SSM) är den myndighet som har det formella ansvaret
för strålskydd i Sverige.
Myndigheten ska bland annat förebygga akuta skador och minska risken för sena
hälsoeffekter hos allmänheten till följd av exponering för elektromagnetiska
fält.

SSM har tagit fram allmänna råd SSMFS 2008:18 \cite{SSMFS2008:18} för
begränsning av allmänhetens exponering för elektromagnetiska fält.
De allmänna råden anger vilka referensvärden som gäller i Sverige.
Råden utgår från rekommendationer i EU-direktiv 1999/519/EG \cite{1999/519/EG}.
EU-direktivet följer i sin tur de riktlinjer för begränsning av
elektromagnetiska fält som sammanställts av ICNIRP.

Eftersom grunden i amatörradioutövandet är att generera elektromagnetiska fält
för att kommunicera via radio så är kunskapen om EMF viktig.
Med de möjligheter radioamatörer har, måste de allmänna råden gällande EMF
följas.
Förståelsen för hur fält uppträder och hur de kan begränsas anses vara
fundamental kunskap för radioamatörer.

\section{Fält}
\index{elektriskt fält (E)}
\index{magnetiskt fält (H)}
För att ange nivån på det elektriska fältet (E) används enheten
''volt per meter'' (V/m).
Det magnetiska fältet (H) nivå anges i enheten ''ampere per meter'' (A/m).

Antennens uppgift är att så effektivt som möjligt omvandla den högfrekventa
strömmen i matarkabeln till en elektromagnetisk våg som utbreder sig i luften.

Den sammansatta elektromagnetiska vågen uppträder inte direkt vid antennen utan
uppstår i det som man kallar fjärrfältet.
Detta sker genom växelverkan mellan de elektriska och magnetiska fält som
utgår från antennen.
Teorierna som beskriver hur denna växelverkan fungerar är komplicerade
men det viktiga att förstå är att det finns en gräns mellan vad man
kallar fjärrfältet, längre bort från antennen och närfältet nära antennen.

\index{fjärrfält}
I fjärrfältet kan man tack vare växelverkan mellan det elektriska- och det
magnetiska fältet mäta vilket som helst av dem.
I och med att det elektromagnetiska fältet sprider ut sig över en större yta så
avtar styrkan i fältet med avståndet från antennen.
Det sammansatta elektromagnetiska fältet som passerat gränsen till fjärrfältet
avtar linjärt med avståndet, dubbleras avståndet halveras fältstyrkan.
Det spelar ingen roll om antennen är helt rundstrålande eller koncentrerar
effekten i en riktning, det elektromagnetiska fältet avtar på samma sätt.

\index{närfält}
I närfältet behöver man på grund av fältens komplicerade inbördes förhållande
mäta både det elektriska och det magnetiska fältet för att få en uppfattning
om storleken på det radiofrekventa fältet.
I antennens närhet varierar nivåerna på de olika fälten kraftigt och på vissa
punkter kan höga fältstyrkenivåer mätas upp.

Om antennen har stor utsträckning i förhållande till använd våglängd kan ibland
fjärrfältsformler användas för att överslagsmässigt beräkna fältstyrkenivå i
antennens närfält.
För kompakta antenner (t.ex. små loopar) krävs komplicerade beräkningar
med hjälp av antennsimuleringsprogram.

Beroende på den antenntyp som genererar fältet är det antingen ett elektriskt
eller magnetiskt fält som dominerar i närfältet.
Elektrisk fältdominans genereras av antenntyper som bygger på
spänningsskillnader (t.ex. dipol) och magnetisk fältdominans av antenner
med strömflöde (t.ex. små loopar).

Eftersom alla elektriska ledare kan betraktas som antenner kommer dessa att
kunna generera fält, oavsett om det är tänkt att det ska vara en antenn eller
inte.
Man bör ha detta i åtanke vid installation av matarledning till antennen för
att undvika att högfrekvent ström flyter tillbaka till stationen på utsidan av
ledningen.
Även de apparater man använder för att generera radiosignaler kan ha dålig
skärmning och därigenom leds högfrekvent ström till apparaternas utsida.

Det finns alltså en risk att fältstyrkorna kan vara betydande i närheten av
sändare och framför allt vid slutsteg med tillhörande kablage.

\section{Allmänna råd}
\index{EMF!allmänna~råd}

SSM har gett ut allmänna råd för begränsning av allmänhetens exponering
för elektromagnetiska fält SSMFS~2008:18 \cite{SSMFS2008:18}.
Syftet med råden är att skydda allmänheten från akuta
skadliga biologiska effekter vid exponering för elektromagnetiska fält.
I råden anges grundläggande begränsningar och härledda referensvärden.

\begin{quote}
	De grundläggande begränsningarna säkerställer att elektriska eller
	magnetiska fenomen som kan uppträda i kroppen inte stör funktioner i
	nervsystemet eller ger upphov till skadlig värmeutveckling.
\end{quote}

De grundläggande begränsningarna är, enligt internationella rekommendationer,
satta vid ungefär två procent av de nivåer vid vilka akuta biologiska effekter
är vetenskapligt säkerställda.

Från de grundläggande begränsningarna har härletts referensvärden som utgörs
av storheter som är mätbara utanför människokroppen.
Referensvärdena ska säkerställa att de grundläggande begränsningarna inte
överskrids.

\begin{quote}
	Om uppmätta värden överstiger referensvärdena, innebär detta inte nödvändigtvis
	att de grundläggande begränsningarna överskrids. I sådana fall gäller enligt
	dessa allmänna råd de grundläggande begränsningarna.
\end{quote}

\newpage

I EU-direktivet 1999/519/EG \cite{1999/519/EG} skrivs att i sådana fall skall det
göras en bedömning huruvida exponeringsnivån ligger under den grundläggande
begränsningen.

Referensvärdena i de allmänna råden bör inte överskridas i något område där
allmänheten kan vistas under sådana tider att begränsningarna är av betydelse.

\index{EMF!akuta biologiska effekter}
Det finns två huvudsakliga akuta biologiska effekter som kan förekomma vid
kraftig exponering för elektromagnetiska fält.
Fält med frekvens upp till cirka \qty{10}{\mega\hertz} kan om strömtätheten blir
hög i kroppen påverka det centrala nervsystemet.
Fält med frekvenser från \qty{100}{\kilo\hertz} till \qty{10}{\giga\hertz} kan
vid höga nivåer leda till en uppvärmning av kroppen.

\index{Specific Absorption Rate (SAR)}
\index{SAR}
När elektromagnetisk strålning absorberas i biologisk vävnad kan vävnaden värmas
upp.
Detta benämns ''Specific Absorption Rate'' (SAR) som mäts i enheten watt per
kilogram (\unit{\watt\per\kilo\gram}) eller milliwatt per gram
(\unit{\milli\watt\per\gram}).
SAR definieras som den energi, medelvärdesbildad över hela kroppen eller delar
av kroppen som absorberas per tidsenhet och per massenhet biologisk vävnad.

Då uppvärmningen av kroppsvävnad inte går snabbt räknar man med den medeleffekt
som under en viss tid orsakar uppvärmning.
För frekvenser mellan \qty{100}{\kilo\hertz} och \qty{10}{\giga\hertz} beräknas
SAR-värdet som medelvärdet under en sexminutersperiod.
För beräkning av SAR-värde på frekvenser överstigande \qty{10}{\giga\hertz}
hänvisas till formler för beräkning enligt SSMFS 2008:18.

Beroende på kroppens storlek i förhållande till det elektromagnetiska fältets
riktning och våglängd skapas resonansfenomen på grund av att kroppen fungerar
som en antenn.
Detta påverkar uppvärmningen på så sätt att vid frekvenser som är nära kroppens
eller kroppsdelens elektriska resonansfrekvens absorberas effekten lättare och
maximal uppvärmning uppstår.
Hos vuxna ligger denna resonansfrekvens mellan 70 och \qty{90}{\mega\hertz} om
personen står upp är och isolerad från något som kan jämföras med ett jordplan.
Även de olika kroppsdelarna kan vara resonanta.
En vuxen persons huvud är till exempel resonant vid cirka \qty{400}{\mega\hertz}.

Kroppens storlek avgör alltså vid vilken frekvens den absorberar mest effekt och
vid frekvenser över och under resonansfrekvensen så minskar uppvärmningen
orsakad av det elektromagnetiska fältet.

\index{EMF!referensvärden}
Referensvärdena tar hänsyn till detta faktum och det mest restriktiva
frekvensområdet ligger inom området 10 till \qty{400}{\mega\hertz} där effekt
lättast absorberas av kroppen.

I frekvensområdet 10 till \qty{110}{\mega\hertz} finns även en begränsning till
\qty{45}{\milli\ampere} för inducerad ström i varje extremitet i syfte att
begränsa det lokala SAR-värdet.

\newpage
\mediumfig[0.87]{images/emfbild-000}{Referensvärden för begränsning av elektriska fält (100~kHz--10~GHz)}{fig:emf1}
\mediumfig[0.87]{images/emfbild-001}{Referensvärden för begränsning av magnetiska fält (100~kHz--10~GHz)}{fig:emf2}

Bild~\ssaref{fig:emf1} illustrerar referensvärden för begränsning av elektriska
fält på platser där allmänheten kan vistas (100~kHz--10~GHz), med amatörband
och fältstyrkenivå angivna, till exempel \qty{10,15}{\mega\hertz} har en högsta
tillåtna elektriskt fältstyrka på \qty{28}{\volt\per\metre}.

\newpage
Bild~\ssaref{fig:emf2} illustrerar referensvärden för begränsning av magnetiska
fält på platser där allmänheten kan vistas (100~kHz--10~GHz), med amatörband
och fältstyrkenivå angivna, till exempel \qty{10,15}{\mega\hertz} har en högsta
tillåtna magnetisk fältstyrka på \qty{73}{\milli\ampere\per\metre}.

\clearpage
\section{Utvärdering av EMF}
\index{EMF!utvärdering}

För att kunna utvärdera att den egna radiostationen vid sändning ger
elektromagnetiska fält som understiger referensvärdena behöver man känna till
de parametrar som är avgörande för styrkan på det elektromagnetiska fältet:

\begin{itemize}
  \item Antennens förstärkning (G).
  \item Sändningens medeleffekt (P).
  \item Transmissionsledningens förluster (k).
  \item Distansen (d).
\end{itemize}

\subsection{Antennen}
Antennen tar emot signalen från sändaren via en matningskabel och
omvandlar denna signal till en elektromagnetisk våg.
Hur effektivt antennen omvandlar signalen från sändaren kan enklast förklaras
med begreppen förstärkning eller antennvinst.

Man måste alltså känna till vilken förstärkning antennen har uttryckt i linjära
faktorer i förhållande till en isotrop antenn.

Antennförstärkning i förhållande till en isotrop antenn uttrycks vanligen i dBi.
Detta medför att en vanlig dipolantenn som används som referens för 0\,dBd har
en förstärkning på 2,15\,dBi jämfört med en isotrop antenn.

Alla värden på antennförstärkning uttryckt i dBd ska därför ökas med 2,15 för
att kunna användas i tabell~\ssaref{tab:forst} som visar förhållandet mellan
förstärkning i \unit{\decibel} och linjära faktorer.

\begin{table*}[ht]
  \begin{center}
    \begin{tabular}{|l|ccccccccccc|}
	\hline
	dB     &  0  &  1  &  2 & 2,15 &  3  &  4  &  5  &  6  &  7  &  8  &  9  \\ \hline
	G & 1,0 & 1,3 & 1,6 & 1,64 & 2,0 & 2,5 & 3,2 & 4,0 & 5,0 & 6,3 & 7,9 \\ \hline
	dB     &  10  &  11  &  12  &  13  &  14  &  15  &  16  &  17  &  18  &  19  &  20 \\ \hline
	G & 10,0 & 12,6 & 15,8 & 20,0 & 25,1 & 31,6 & 39,8 & 50,1 & 63,1 & 79,4 & 100,0 \\ \hline
    \end{tabular}
    \caption{G = Antennens förstärkning i linjära faktorer}
    \label{tab:forst}
  \end{center}
\end{table*}

För en antenn med förstärkningen 7\,dBi ska alltså värdet 5,0 användas.

%%\textbf{G = Antennens förstärkning i linjära faktorer}

\subsection{Sändareffekten}
Alla SAR-värden ska beräknas som ett medelvärde under en period av sex minuter.
För att kunna utföra en beräkning av effektens medelvärde behövs utöver
PEP-effekt kännedom om de två faktorer som påverkar medeleffekten.
Faktorerna har därför betydelse för nivån på det elektromagnetiska fältet och
påverkar därigenom den genomsnittliga exponeringen för EMF.

\subsubsection{Modulationsfaktor}
\index{EMF!modulationsfaktor}
\index{modulationsfaktor}

Beroende på trafiksätt så blir medeleffekten olika.
Används FM så medför det modulationssättet att man använder max uteffekt
kontinuerligt jämfört med SSB där medeleffekten beror på hur man talar.

Tabell~\ssaref{tab:modfakt} ger de faktorer som enligt OET bulletin 65 supplement b,
\cite{OETbul65b} används i USA för att räkna ut medeleffekten på grund
av modulationsfaktorn.

\begin{table}[H]
  \begin{center}
    \begin{tabular}{lc}
	\textbf{Trafiksätt} & \textbf{Modulationsfaktor} \\ 
	\hline
	\emph{SSB} & 0,2 \\ 
	\emph{CW} & 0,4 \\ 
	\emph{SSB med processing} & 0,5 \\ 
	\emph{FM} & 1,0 \\ 
	\emph{MGM (t.ex. RTTY,PSK)} & 1,0 \\ 
	\emph{Bärvåg} & 1,0 \\ 
    \end{tabular}
    \caption{Modulationsfaktor per trafiksätt}
    \label{tab:modfakt}
  \end{center}
\end{table}

\subsubsection{Intermittensfaktor}
\index{EMF!intermittensfaktor}
\index{intermittensfaktor}

Vid vanlig amatörradioanvändning sänder man inte kontinuerligt då växling
mellan sändning och lyssning sker regelbundet.
Sänder man och tar emot lika mycket under en sexminutersperiod så blir faktorn
0,5 men om man lyssnar mycket mer och sänder sällan blir faktorn mindre.
Se tabell~\ssaref{tab:intfakt} för fler exempel.

\begin{table}[H]
  \begin{center}
    \begin{tabular}{|c|c|c|}
	\hline
	Sändning  & Mottagning & Intermittensfaktor \\
	(minuter) & (minuter)  & \\ \hline
	1 & 5 & 0,17 \\ \hline
	2 & 4 & 0,33 \\ \hline
	3 & 3 & 0,50 \\ \hline
	4 & 2 & 0,67 \\ \hline
	5 & 1 & 0,83 \\ \hline
	6 & 0 & 1,00 \\ \hline
    \end{tabular}
    \caption{Intermittensfaktor}
    \label{tab:intfakt}
  \end{center}
\end{table}

\subsubsection{Medeleffekt}
\index{EMF!medeleffekt}

För att räkna ut vilken medeleffekt som används ska man ta hänsyn
till både modulationsfaktor och intermittensfaktor enligt följande

\(\textit{Medeleffekt} = \textit{Maxeffekten} \cdot \textit{Modulationsfaktor} \cdot \textit{Intermittensfaktor}\)

\noindent\textbf{P = Medeleffekten under en sexminutersperiod}

\subsection{Kabeldämpning}
\index{EMF!kabeldämpning}

När uteffekten mäts vid sändaren och fältet genereras av effekten som
når antennen måste även den dämpning som matarledaren har vara känd.
Annars överskattas den genererade fältstyrkan.

Även här måste linjära faktorer användas.
Förlusterna i en kabel har negativa värden uttryckt i \unit{\decibel} vilket
medför att faktorerna i tabell~\ssaref{tab:feedannut} blir mindre än ett.

\begin{table*}[ht]
  \begin{center}
    \begin{tabular}{|l|c|c|c|c|c|c|c|c|c|c|c|}
	\hline
	dB & 0,0  & 0,5  & 1,0  & 1,5  & 2,0  & 2,5  & 3,0  & 3,5  & 4,0  & 4,5  & 5,0 \\ \hline
	k  & 1,00 & 0,89 & 0,79 & 0,71 & 0,63 & 0,56 & 0,50 & 0,45 & 0,40 & 0,35 & 0,32 \\ \hline
    \end{tabular}
    \caption{k = Matarkabels dämpning i linjära termer}
    \label{tab:feedannut}
  \end{center}
\end{table*}

För en kabel med dämpningen \qty{2,5}{\decibel} ska alltså värdet 0,56 användas.

\subsection{Distans}
\index{EMF!distans}

För att kunna beräkna nivån på det elektromagnetiska fältet på en utvald plats
behöver man veta distansen till den sändande antennen.

Enligt Strålsäkerhetsmyndighetens allmänna råd så bör inte referensvärdena
överskridas på platser där allmänheten vistas.
En bedömning bör därför göras över distanserna från den sändande antennen till
platser allmänheten riskerar att exponeras för elektromagnetiska fält.
\\[1ex] % layout
\noindent\textbf{d = Distansen från antennen till platsen där fältstyrkan ska bestämmas}

%% k7per
%% \newpage % layout
\subsection{Beräkning}
\index{EMF!beräkning}

Beräkning av det elektromagnetiska fältet kan med enkelhet bara
genomföras i fjärrfältet från en antenn.
I fjärrfältet vet vi sedan tidigare att man antingen kan utvärdera det
elektriska eller det magnetiska fältet.
Av denna anledning beskrivs här enbart beräkning av det elektriska fältets del av
det elektromagnetiska fältet.
Ett vedertaget avstånd från antennen där fjärrfältsberäkningar kan genomföras
är
%% \(d=\dfrac{\lambda}{6}\).
\(d=\lambda / 6\). Se tabell~\ssaref{tab:fjfltgr}.

Följande formler gäller enbart för beräkning av korrekt fältstyrka i
fjärrfältet men kan för enklare antenner användas för att uppskatta den
fältstyrka som uppträder i närfältet.

%% k7per: Where is this table referenced?
\begin{table*}[ht]
  \begin{center}
    \begin{tabular}{|l|c|c|c|c|c|c|c|c|c|c|}
	\hline
	Band [m] & 160 & 80 & 40 & 30 & 20 & 17 & 15 & 12 & 10 & 6 \\ \hline
	Fjärrfältsgräns [m] & 27 & 13 & 6,7 & 5 & 3,3 & 2,8 & 2,5 & 2 & 1,7 & 1 \\ \hline
    \end{tabular}
    \caption{Fjärrfältsgräns per band}
    \label{tab:fjfltgr}
  \end{center}
\end{table*}

\noindent\textbf{E = Det elektromagnetiska fältets storlek i fjärrfältet}

Det elektromagnetiska fältets storlek (i fjärrfältet) räknas ut från
effekten (medelvärde), antennförstärkningen, matarledningens dämpning
och avståndet enligt följande förenklade formel.
%%
\[E=\dfrac{\sqrt{30 \cdot P \cdot G \cdot k}}{d}\]
%%
Genom enkel matematik kan man då använda samma formel för att räkna
ut på vilket avstånd man genererar en viss fältstyrka.
%%
\[d=\dfrac{\sqrt{30 \cdot P \cdot G \cdot k}}{E}\]
%%
Denna beräkning är enbart relevant för huvudloben.
Fältet under antennen beräknas inte, och därför kan resultatet inte användas
för att bedöma höjd på eller säkerhetsavstånd till antenntorn.
Använd dataprogram för att få bra bedömning på hur en antenn beter sig,
särskilt med avseende på antenner med riktverkan.

\subsubsection{Exempel 1:}

Vilken medelfältstyrka genererar man på ett visst avstånd från antennen?

En riktantenn för \qty{144}{\mega\hertz} med förstärkning enligt databladet på
14,92\,dBi (31 gånger).
Max uteffekt är \qty{1000}{\watt} och trafiksättet är MGM (t.ex. RTTY, PSK) med
30 sekunders intervaller.
Den valda matarledningen har en dämpning på \qty{2,5}{\decibel} (0,56~gånger).
Avståndet från antennen till beräkningspunkten är \qty{15}{\metre}.

\[P_{medel} = P_{pep} \cdot k_{mod} \cdot k_{if}
= 1000 \cdot 1 \cdot 0,5 = \qty{500}{\watt}\]
\[k_{mod} = modulationsfaktor\]
\[k_{if} = intermittensfaktor\]
\[G = 31 \quad k = 0,56 \quad d = 15\]
% \[E = \dfrac{\sqrt{30 \cdot P \cdot G \cdot k}}{d}
% = \dfrac{\sqrt{30 \cdot 500 \cdot 31 \cdot 0,56}}{15}
% = \qty{34,02}{\volt\per\metre}\]
\begin{align*}
  E &= \dfrac{\sqrt{30 \cdot P \cdot G \cdot k}}{d} =\\
&= \dfrac{\sqrt{30 \cdot 500 \cdot 31 \cdot 0,56}}{15}
= \qty{34,02}{\volt\per\metre}
\end{align*}

Då referensvärdet på denna frekvens är \qty{28}{\volt\per\metre}, överskrider
amatörradiosändningen referensvärdet på detta avstånd.

\subsubsection{Exempel 2:}

På vilket avstånd från antennen når man referensvärdet?

En riktantenn för \qty{144}{\mega\hertz} med förstärkning enligt databladet på
14,92\,dBi (31 gånger).
Max uteffekt är \qty{1000}{\watt} och trafiksättet är MGM (t.ex. RTTY, PSK) med
30~sekunders intervaller.
Den valda matarledningen har en dämpning på \qty{2,5}{\decibel} (0,56~gånger).
Referensvärdet för \qty{144}{\mega\hertz} är \qty{28}{\volt\per\metre}.

\[P_{medel} = P_{pep} \cdot k_{mod} \cdot k_{if}
= 1000 \cdot 1 \cdot 0,5 = \qty{500}{\watt}\]
\[k_{mod} = \textit{modulationsfaktor}\]
\[k_{if} = \textit{intermittensfaktor}\]
\[G = 31 \quad k = 0,56 \quad E = 28\]
%% \[d = \dfrac{\sqrt{30 \cdot P \cdot G \cdot k}}{d}
%% = \dfrac{\sqrt{30 \cdot 500 \cdot 31 \cdot 0,56}}{28}
%% = \qty{18,22}{\metre}\]
\begin{align*}
  d &= \dfrac{\sqrt{30 \cdot P \cdot G \cdot k}}{d} =\\
&= \dfrac{\sqrt{30 \cdot 500 \cdot 31 \cdot 0,56}}{28}
  = \qty{18,22}{\metre}
  \end{align*}

För att följa de allmänna råden bör allmänheten inte kunna vistas i huvudloben
framför antennen på ett avstånd mindre än \qty{19}{\metre} då sändning utförs
enligt exemplet.

\subsubsection{Exempel 3:}

På vilket avstånd från antennen når man referensvärdet?

En dipolantenn för \qty{3,75}{\mega\hertz} har jämfört med en isotrop antenn
förstärkningen 2,15\,dBi (cirka 1,6 gånger).
Max uteffekt är \qty{100}{\watt} och trafiksättet är SSB med normala TX/RX
intervaller.
Den valda matarledningen har en dämpning på \qty{0,5}{\decibel} (0,89 gånger).
Referensvärdet för \qty{3,75}{\mega\hertz} är \qty{45}{\volt\per\metre}.

\[P_{medel} = P_{pep} \cdot k_{mod} \cdot k_{if}
= 100 \cdot 0,5 \cdot 0,5 = \qty{25}{\watt}\]
\[k_{mod} = modulationsfaktor\]
\[k_{if} = intermittensfaktor\]
\[G = 1,6 \quad k = 0,89 \quad E = 45\]
\[d = \dfrac{\sqrt{30 \cdot P \cdot G \cdot k}}{E} = \dfrac{\sqrt{30 \cdot 25 \cdot 1,6 \cdot 0,89}}{45}
= \qty{0,74}{\metre}\]

Här konstaterar vi att det uträknade avståndet ligger i närfältet från antennen
(inom 13~meter).
En dipol är en enklare antenntyp så vi kan anta att värdet är användbart för att
kunna utvärdera exponeringen.

För att följa de allmänna råden bör människor inte ha tillträde till nån del av
antennen närmare än \qty{0,74}{\metre} då sändning utförs.

\section{Egenkontroll}
\index{EMF!egenkontroll}
\index{EMF!utvärdering}

För att utvärdera sin egen station så finns det några olika vägar att gå:

\begin{itemize}
\item Räkna ut fältstyrkan eller säkerhetsavståndet med sina egna
  parametrar enligt exemplen ovan.
\item Jämföra med andras utvärderingar.
\item Använda programvara som är speciellt gjort för att räkna ut på
  vilket avstånd referensvärdet nås under givna förutsättningar enligt
  exempel 2 ovan.
\item Använda värden från tabeller där olika typiska antenner är beskrivna.
\item Använda antennsimuleringsprogram som har möjlighet att även
  beräkna fältstyrka.
\item Mäta fältstyrkan (speciellt då man utvärderar i närfältet från
  antennen).
\end{itemize}

Man bör då tänka på vilket avstånd man har till platser där allmänheten har
tillträde, sin effektanvändning, vilka antenntyper och vilka trafiksätt man
använder.

\subsection{Räkna manuellt}

Enligt exemplen ovan är det ganska enkelt att göra en uppskattning av
de fältstyrkor som genereras av sin egen amatörradioanvändning.

\subsection{Räkna med specialprogram}

Istället för att själv använda miniräknaren kan man använda program
som är speciellt framtagna för detta ändamål.

Ett exempel på ett sådant program är ICNIRPcalc som är framtaget av en
representant från den tyska amatörradioföreningen (DARC).
I programmet finns redan olika antenntyper och det finns även möjlighet att
lägga in egna antenner för att göra korrekta beräkningar.
Detta program finns att ladda ner från SSA:s webbplats för EMC/EMF-frågor.

\subsection{Tabellvärden}
Utifrån den typ av antenn man själv använder kan man jämföra med
typiska värden från andras beräkningar och göra en hyfsad uppskattning
av sig egen situation.

\subsection{Antennsimulering}
Vissa program för antennsimulering har även funktioner för att beräkna
fältstyrkenivåer runt antennen och kan i vissa fall beräkna fältstyrkan
även i närfältet.

\subsection{Mäta fältstyrka}
Att mäta fältstyrka kräver tillgång till kalibrerad mätutrustning som
ger mätvärden som är tillförlitliga nog för att med säkerhet kunna användas
vid utvärdering av fältstyrkenivån.

\section{Sammanfattning}
Strålsäkerhetsmyndigheten (SSM) har i sina allmänna råd angett referensvärden
som ska begränsa allmänhetens exponering för elektromagnetiska fält (EMF).

Dessa begränsningar och sändaramatörens möjligheter att generera kraftiga
elektromagnetiska fält innebär att vi som sändaramatörer måste förstå
och kunna hantera området elektromagnetiska fält (EMF).

Alla sändande antenner kommer att ha ett elektromagnetiskt fält (EMF)
runt sig.
Detta elektromagnetiska fält (EMF) är beroende på vilken typ av antenn som
används och den signal som skickas in i antennen.
Hur man bedömer storleken på dessa fält är avgörande för att kunna
begränsa exponeringen av EMF från en amatörradiostation.

En egenkontroll bör genomföras för att kunna bedöma den fältbild som
amatörradioutövandet orsakar runt sin station.
Eftersom amatörradio är en experimentell verksamhet så måste alla förstå hur
olika förändringar i sin installation och användning påverkar denna fältbild.

Vilken metod man än väljer för sin egenkontroll är det lämpligt att
göra den tydligt och lättförståelig.
Detta är viktigt eftersom man bör spara sina resultat och då ha möjlighet att
göra om sin utvärdering när man har förändrat något eller några av de värden
som skulle kunna påverka resultatet.

% \newpage % layout
\subsection{Praktisk hantering}
Vid all användning av amatörradioutrustning måste man göra en bedömning
av vilka fältstyrkor man genererar och vilka som kan bli exponerade.
Det kan vara frågan om människor i omedelbar närhet eller människor på
längre avstånd.
I alla fall bör man fundera på om man valt rätt sätt att generera den
elektromagnetiska fältstyrka som man behöver, eller om det finns ett bättre
och effektivare sätt som möjliggör att man når motstationen utan att onödigtvis
exponera någon annan för elektromagnetiska fält.

Det finns vissa installationer som man bör undvika och andra som kan
rekommenderas för att hålla nivåerna på exponering så låga som möjligt:

\begin{itemize}
\item Antenner som sitter nära människor, exempelvis balkongantenner, kan ge
  mycket högre exponering än antenner som sitter högt monterade i en mast.

\item Riktantenner för höga frekvenser har ofta hög förstärkning, och
  kan ge höga fältstyrkor i huvudriktningen.
  Då måste man se till att det inte är möjligt att rikta denna typ av antenn
  mot platser där människor kan exponeras.

\item Inomhusantenner hamnar alltid nära människor och bör enbart användas med
  låg effekt då de kan ge mycket hög exponering.
  De kommer också ta emot störningar från hemelektronik (nätadaptrar, datorer
  etc.) vilket också gör antennplaceringen mycket olämplig.

\item Antenner ovanför huskroppar bör endast användas med låg effekt.
  Trådantenner för lägre frekvenser rakt ovanför bostadshus kommer att
  vara nära människor i byggnaden.

\item Om man har behov av att använda hög effekt så måste man också se
  till att effekten används så bra som möjligt.
  Det är direkt olämpligt att kompensera en dålig antenn med högre effekt då
  det oftast resulterar i höga fältstyrkor på fel ställe.

\item Högre fältstyrka kan för det mesta enklast åstadkommas med en
  antenn som riktar signalen i den riktning man vill kommunicera.
  Det är oftast mycket dyrare och mer komplicerat att öka uteffekten för att nå
  samma resultat.

%%k7per
%% \newpage % layout
\item Osymmetriska antenner kan ge mantelströmmar i matningsledningen.
  Det innebär att en HF-ström flyter från antennen tillbaka på matarledningen
  och kan ge höga fältstyrkor längs hela kabellängden.
  Bättre är det då att använda symmetriska antenner, exempelvis en mittmatad
  halvvågsdipol.
  En strömbalun (även common mode choke, RF-choke) där antennen ansluts till
  matarledningen undertrycker denna HF-ström och därmed kommer
  matningsledningen sluta att agera radierande element, varvid fältstyrkorna
  längs matningsledningen sjunker.

\item Vissa antenner, så som T-antenn, använder dock obalansen då
  matningsledningen agerar radierande element.
  I dessa fall ska den delen av matningsledningen som agerar radierande element
  betraktas som sådant även i EMF-sammanhang och säkerhetsavstånd ska iakttas.
  Det är rekommenderat att använda en strömbalun för att isolera antennen från
  radiostationen med avseende på mantelburen HF-ström.

\item Även symmetriska antenner kan ha strömmar på utsidan av matarledningen.
  Dra därför matarledningen så långt bort från människor som möjligt.

\item Använd inte effektförstärkare eller antennavstämningsenhet utan
  hölje då fältstyrkorna runt utrustningen kan nå höga nivåer.

\item Vid antennplaceringar nära människor så kan det bli omöjligt att
  använda hög effekt.
\end{itemize}

Det finns som synes många sätt att göra rätt men också många sätt att göra fel
när det gäller att hantera den fältstyrka vi vill generera för att upprätthålla
radiokommunikation.
Innan man börjar sin amatörradiosändning är det viktigt att ha förståelse för
de fält som genereras och att kunna begränsa dem där så behövs.
Det går att finna mer information om elektromagnetiska fält på:

\begin{itemize}
\item Strålskyddsmyndighetens webbplats där återfinns även SSMFS 2008:18 \cite{SSMFS2008:18}.

\item Arbetsmiljöverkets webbplats där finns även AFS 2016:3 Arbetsmiljöverkets
föreskrifter om elektromagnetiska fält och allmänna råd om tillämpningen av
föreskrifterna.

\item Folkhälsomyndighetens webbplats.

\item Federal Communications Commission (FCC) OET bulletin 65 supplement B \cite{OETbul65b}.

\item EU-direktiv 1999/519/EG \cite{1999/519/EG}
\end{itemize}

%
%
% Kapitel 12 Elsäkerhet
\chapter{Elsäkerhet}
\label{ch:elsakerhet}
\index{elsäkerhet}

Människokroppen är ett komplicerat elektrokemiskt system, som främst
kontrolleras av hjärnan.
Musklerna styrs av svaga elektriska strömimpulser genom nervsystemet.
Främmande strömmar genom kroppen kan störa kroppsfunktioner och kan i olyckliga
fall göra stor skada.
Styrkan och frekvensen på strömmarna avgör skadans art och omfattning.

% Avsnitt 12.1 Människokroppen
\chapter{Elsäkerhet}
\label{ch:elsakerhet}
\index{elsäkerhet}

\section{Människokroppen}
\harecsection{\harec{a}{10.1}{10.1}}

\subsection{Elektrisk chock}

Människokroppen är ett komplicerat elektrokemiskt system, som främst
kontrolleras av hjärnan.
Musklerna styrs av svaga elektriska strömimpulser genom nervsystemet.
Främmande strömmar genom kroppen kan störa kroppsfunktioner och kan i olyckliga
fall göra stor skada.
Styrkan och frekvensen på strömmarna avgör skadans art och omfattning.

\paragraph{Elektrisk chock kan döda av flera orsaker.}
En orsak är att hjärtrytmen störs.
Hjärtkammarflimmer och hjärtstillestånd kan lätt uppstå.
Flimmer innebär att hjärtat arbetar okontrollerat och med kraftigt nedsatt
eller helt upphävd pumpfunktion.
Hjärtstillestånd inträffar lätt av hög spänning.
Av otillräcklig blodtillförsel blir det syrebrist i hjärncellerna, som då
skadas snabbt.
Medvetslöshet inträder redan efter ett fåtal sekunder.

En annan orsak är andningsstillestånd genom att andningscentrum blockeras.
Det kan hända när strömmen från en högspänningskondensator går genom kroppen.

\subsection{Hjärt- och lungräddning, HLR}
\index{hjärt- och lungräddning}
\index{HLR|see {hjärt- och lungräddning}}

Vid hjärtstillestånd, hjärtkammarflimmer eller andningsstillestånd ska
hjärt- och lungräddning påbörjas omedelbart då obotliga hjärnskador av
syrebrist kan uppstå inom några få minuter.
Finns en hjärtstartare, AED, i närheten bör den användas så skyndsamt som
möjligt.

\textbf{Glöm inte att ringa efter hjälp! Ring 112!}

Broschyren \emph{Vägledning vid elskada} kan laddas ner eller beställas från
Elsäkerhetsverkets webbplats
<\href{https://www.elsakerhetsverket.se}{\texttt{www.elsakerhetsverket.se}}>.

Vårdguiden 1177 <\href{https://www.1177.se}{\texttt{www.1177.se}}> har
instruktioner för hjärt- och lungräddning (HLR).

Svenska rådet för Hjärt- och Lungräddning \\
<\href{https://www.hlr.nu}{\texttt{www.hlr.nu}}> har beskrivningar och
instruktionsfilmer för hjärt- och lungräddning.

\subsection{Resistansen genom människo\-kroppen}

Vid kontakt med ett strömförande föremål kommer kroppen att bli en del av
strömkretsen.
Det flyter då en främmande ström genom kroppen.

Strömstyrkan följer Ohms lag och beror av strömkällans spänning och inre
resistans samt av övergångsresistansen i huden och kroppens inre resistans.

Övergångsresistansen minskar med fuktigare hud samt med större kontaktyta och
större kontakttryck.
Beröringsspänningen inverkar också.
Vid spänningar över cirka \qty{75}{\volt} minskar övergångsresistansen med
ökande spänning.
Vid allvarliga brännskador minskar övergångsresistansen särskilt mycket.
Den totala resistansen genom kroppen blir då nära lika med dess inre resistans
-- ungefär \qty{500}{\ohm}.

\begin{center}
\begin{minipage}{0.19\columnwidth}
\Huge{\fontencoding{U}\fontfamily{futs}\selectfont\char 66\relax}
\end{minipage}
\begin{minipage}{0.7\columnwidth}
  Experimentera inte med detta! Det kan vara livsfarligt.
\end{minipage}
\end{center}


\subsection{Strömmens inverkan på människan}

Sjukvården skiljer på verkan av strömstöt, strömgenomgång och ljusbåge.

En strömstöt kan tyckas ofarlig men kan leda till okontrollerade rörelser,
fallskada eller beröring av andra spänningsförande föremål.

Vid en strömgenomgång utjämnas en elektrisk potentialskillnad genom kroppen
vilket utöver hjärtstillestånd, hjärtkammarflimmer, och andningsstillestånd
kan leda till blodpropp, muskelskador, njurskador eller inre brännskador.

Vid en ljusbågsolycka ökar risken för kraftiga brännskador på grund av den
höga temperaturen i ljusbågen.
En ljusbåge kan även orsaka skador på ögonen på grund av bländning eller den
stora mängden UV-ljus.

%% \begin{center}
%% \begin{minipage}{0.19\columnwidth}
%% \Huge{\fontencoding{U}\fontfamily{futs}\selectfont\char 66\relax}
%% \end{minipage}
%% \begin{minipage}{0.7\columnwidth}
%% Personer som drabbats av olycka med
%% \end{minipage}
%% \end{center}
%% \begin{itemize}
%% \item högspänning
%% \item lågspänning med strömgenomgång genom bålen
%% \item som är omtöcknade eller medvetslösa efter strömolycka
%% \item som har drabbats av brännskada
%% \item som visar tecken på nervskada till exempel förlamning
%% \end{itemize}
%% \textbf{ska omedelbart till sjukhus för akut behandling.}

\bigskip
\noindent
\begin{minipage}{0.19\columnwidth}
\Huge{\fontencoding{U}\fontfamily{futs}\selectfont\char 66\relax}
\end{minipage}
\begin{minipage}{0.7\columnwidth}
Personer som drabbats av olycka med
\begin{itemize}
\item högspänning
\item lågspänning med strömgenomgång genom bålen
\item som är omtöcknade eller medvetslösa efter strömolycka
\item som har drabbats av brännskada
\item som visar tecken på nervskada till exempel förlamning
\end{itemize}
\textbf{ska omedelbart till sjukhus för akut behandling.}
\end{minipage}

\vspace{1ex}
%Starka strömmar ger häftiga muskelkramper och/eller brännskador.
Häftiga muskelkramper och/eller brännskador kan uppkomma av starka strömmar.
Muskelkramp kan förekomma redan vid strömmar under \qty{10}{\milli\ampere}.
För vuxna, friska människor är det direkt farligt när strömmen överstiger
detta värde.
För unga eller sjuka kan strömmar under \qty{10}{\milli\ampere} vara direkt
farliga.

Strömstyrkan påverkar kroppen olika från fall till fall och det är osäkert
vilken strömstyrka som är farlig.
Det finns både de som överlevt höga strömmar och de som inte har klarat någon
milliampere.
Strömmar som går genom hjärta eller hjärna är särskilt farliga.
När man arbetar med elektriska apparater under spänning, bör man för säkerhets
skull hålla den ena handen i fickan!

\subsection{Påverkan av elektromagnetiska fält}

Undersökningar har visat att vistelse i starka elektromagnetiska fält
kan kan påverka människan.
Personer som har varit utsatta för kraftig exponering av fält har bland annat
klagat över svettningar och huvudvärk.
Det forskas omkring dessa fenomen.

Elektromagnetiska fält kan förorsaka fel i elektronikutrustningar.
Halvledare är särskilt känsliga för kraftfält.
Det är möjligt att känsliga instrument, hjärtstimulatorer (pacemaker) etc. kan
påverkas av högfrekventa elektromagnetiska fält från radiosändare.
När du använder en sändare, mobiltelefon etc. och någon får svårigheter med
hjärta eller andning så ska du omedelbart stänga av din apparat helt!
Med tiden utvecklas störningsokänsligare elektronik, men säker mot störningar
kan man aldrig vara. Se vidare i kapitel~\ssaref{EMF}.

\subsection{Normer för fältstyrkor}

Det finns flera olika normer och rekommendationer för elektromagnetiska
fältstyrkor. Några av dessa normer har till exempel syftet att olika slags
apparater ska kunna samexistera och därför fungera utan att påverkas av
elektromagnetiska fält eller stråla ut elektromagnetiska fält överstigande
givna gränsvärden (EMC).

Andra normer och råd har till syftet att skydda arbetstagare eller individer
ur allmänheten från akuta biologiska effekter när de exponeras för
elektromagnetiska fält.

Strålsäkerhetsmyndigheten har genom utgivandet av SSMFS 2008:18 publicerat
allmänna råd om begränsning av allmänhetens exponering för elektromagnetiska
fält.
Dessa råd bygger på rekommendationer från Europeiska unionens råd.
Se vidare i kapitel~\ssaref{EMF}.

% Avsnitt 12.2 Allmänna elnätet
\section{Allmänna elnätet}
\harecsection{\harec{a}{10.2}{10.2}}
\label{jordning}
\index{elnätet}

Elektrisk energi levereras till förbrukarna över transformatorstationer där
högspänning först transformeras till lågspänning.
Från transformatorstationerna förgrenas lågspänningsnätet till serviceskåp ute
i kvarter och byar.

I Sverige är fördelningstransformatorns sekundärlindningar oftast sammankopplade
till ett Y (s.k. Y- eller stjärnkoppling) där mittpunkten är jordad.

De i Sverige vanligast förekommande 3-fas lågspänningsnäten har huvudspänningen
\qty{400}{\volt} och fasspänningen \qty{230}{\volt}.
Spänningen mellan fasledarna kallas för huvudspänning och spänningen mellan
respektive fasledare och nolledaren kallas för fasspänning.

Bruksföremålen i huset ansluts oftast 1-fasigt, det vill säga mellan någon av
fasledarna och nolledaren.
Någorlunda lika belastning mellan faserna är önskvärd.
Mer effektkrävande apparater som el-pannor och spisar ansluts därför till alla
tre faserna (3-fasigt).
Amatörradioutrustningar ansluts oftast 1-fasigt.

Nybyggnad, förändring eller reparation av starkströmsanläggning,
fast anslutning av elektrisk utrustning till en starkströmsanläggning
eller att koppla loss fast ansluten elektrisk utrustning från en
starkströmsanläggning, klassas som elinstallationsarbete och får endast
utföras av person som har auktorisation som elinstallatör eller av
yrkesverksam som omfattas av ett elinstallationsföretags egenkontroll.

Om du har \emph{erforderlig kunskap} om elsäkerhet får du

\begin{itemize}
\item byta ut en elkopplare (strömbrytare) för högst \qty{16}{\ampere} \qty{400}{\volt}
\item byta ut ett anslutningsdon (vägguttag, lamputtag, stickpropp,
skarvuttag eller liknande) för högst \qty{16}{\ampere} \qty{400}{\volt}
\item byta ut en ljusarmatur i torrt icke brandfarligt utrymme i bostäder
\item utföra, ändra eller reparera en starkströmsanläggning som ingår i en
skyddsklenspänningskrets med nominell spänning om högst \qty{50}{\volt} med
effekt om högst \qty{200}{VA} och ström begränsad av säkring på högst \qty{10}{\ampere}
\item byta säkring
\item byta ljuskälla (lampa, lysrör eller liknande)
\item reparera apparater
\item reparera och tillverka apparatkablar och skarvsladdar.
\end{itemize}

\begin{center}
\begin{minipage}{0.19\columnwidth}
\Huge{\fontencoding{U}\fontfamily{futs}\selectfont\char 66\relax}
\end{minipage}
\begin{minipage}{0.7\columnwidth}
\textbf{Kom ihåg, att auktoriserad installatör ska anlitas för arbete
i fasta installationer.}
\end{minipage}
\end{center}

\smallfig[0.1]{images/cropped_pdfs/CE-mark.pdf}{CE-märke}{fig:CE-mark}

% \newpage % layout
\subsection{Radioamatören och hembyggd elektronik}
\index{hembyggd elektronik}
\index{praktiska råd för självbyggaren}
\index{CE-märkning}

Enligt \emph{radioutrustningslagen} SFS 2016:392 \cite{SFS2016:392} ska
radioutrustning som släpps ut eller tillhandahålls på marknaden inom EU ska vara
konstruerad och tillverkad så att den uppfyller föreskrivna krav, ha en
EU-försäkran om överensstämmelse och vara CE-märkt.

När CE-märket bild~\ssaref{fig:CE-mark} sätts på en produkt eller en
radioutrustning så innebär det att tillverkaren eller importören intygar att
alla föreskrivna krav är uppfyllda.

I många fall har det dock vid kontroll av CE-märkt utrustning funnits brister
i elsäkerhet och elektromagnetisk kompatibilitet (EMC) villkoren för CE-märkning
har inte varit uppfyllda. Lär mer om EMC i avsnitt~\ssaref{EMC-lagen}.

Som \emph{radioutrustning} räknas en elektrisk eller elektronisk produkt som
avsiktligt avger eller tar emot radiovågor för radiokommunikation eller
radiobestämning, eller en elektrisk eller elektronisk produkt som måste
kompletteras med ett tillbehör, såsom en antenn, för att avsiktligt avge
eller ta emot radiovågor för radiokommunikation eller radiobestämning.

Lagens tillämpningsområde och definitioner anger att lagen inte omfattar
radioutrustning som används av radioamatörer för amatörradiotrafik, under
förutsättning att utrustningen inte tillhandahålls på marknaden.
Radioutrustning som används av radioamatörer för amatörradiotrafik ska inte
anses tillhandahållen om det är:

\begin{itemize}
  \item radiobyggsatser som är avsedda att byggas samman och användas av
  radioamatörer
  \item radioutrustning som har modifierats av radioamatörer för att
  avvändas av radioamatörer
  \item utrustning som har konstruerats av enskilda radioamatörer för
  experimentella och vetenskapliga ändamål i samband med amatörradio.
\end{itemize}

Detta innebär att du som radioamatör, utöver vanlig elektronik, får bygga
och använda en radioutrustning.
Du är då ansvarig för att den utrustning du byggt är säker att använda och inte
orsakar störningar.
Detta innebär dock inte att du får göra följande:

\begin{itemize}
  \item Bygga en sändare för användning utanför amatörradiobanden.
  \item Modifiera en amatörradiosändare för användning utanför amatörradiobanden
  \item Modifiera en CE-märkt sändare utanför amatörradiobanden.
  \item Återställa en CE-märkt sändare till ursprunget efter modifiering till
    amatörradiosändare på amatörradiobanden.
  \item Montera avstörningsfilter inuti en CE-märkt apparat.
\end{itemize}

Radioutrustningen får vara avsedd att anslutas till en starkströmsanläggning
om utrustningen vid användning inte orsakar någon typ av skada på person
egendom eller husdjur.
Kom även ihåg att \qty{12}{\volt} från ett bilbatteri räknas som en
starkströmsanläggning.

När en elektrisk eller elektronisk apparat konstrueras eller byggs finns det
ett antal punkter som ska uppmärksammas för att apparaten ska vara säker att
använda oavsett hur den är avsedd att strömförsörjas.
Som stöd för hur en apparat kunde byggas för att uppfylla kraven gav
dåvarande SEMKO ut \emph{Praktiska råd för självbyggaren}.
Nedanstående punkter bygger på dessa praktiska råd:

\begin{itemize}
\item Höljet ska vara anpassat till apparaten och inte vara öppningsbart
  utan verktyg.

\item Höljet ska vara försett med nödvändiga ventilationshål för att
  undvika överhettning.
  Observera att spänningsförande delar inte får vara nåbara genom
  ventilationshålen.

\item Höljet får inte bli så varmt att skada kan uppstå på människa
  eller egendom.

\item Är höljet eller chassiet till en elnätsansluten apparat av ledande
  material och apparaten inte har förstärkt isolering så ska \emph{utsatta}
  delar som riskerar att spänningssättas vid fel anslutas till skyddsjord.

\item Kabeln för nätanslutning ska vara försedd med en för ändamålet lämplig
  dragavlastning som även skyddar kabeln mot nötning när den passerar höljet.

\item Komponenter i apparaten ska vara dimensionerade och godkända
  för den effekt de utvecklar och för den spänning och strömstyrka de
  ansluts till.
  \emph{Not: Ett tips är att ha god marginal vad gäller värmetålighet då det
    ger ökad livslängd och större säkerhetsmarginaler.}

\item Apparaten ska vara försedd med korrekt dimensionerad säkring
  som skydd mot kortslutning och överbelastning.

\item Elnätsansluten apparat ska vara försedd med 2-polig nätströmbrytare.

\item Spänningsförande delar i apparaten ska vara försedda med
  beröringsskydd som skyddar mot oavsiktlig beröring.

\item Komponenter i apparaten ska monteras fast och placeras på lämpliga
  inbördes avstånd så att risken för störningar, överslag, kortslutning eller
  överhettning minimeras.

\item Kablar och ledningar för starkström ska skyddas mot varma komponenter,
  nötning och skarpa kanter samt förläggas separerade från ledningar för
  klenspänning och signaler.
\end{itemize}

\begin{center}
\begin{minipage}{0.19\columnwidth}
\Huge{\fontencoding{U}\fontfamily{futs}\selectfont\char 66\relax}
\end{minipage}
\begin{minipage}{0.7\columnwidth}
\textbf{Sträva efter att alltid ansluta din apparat via vägguttag
	skyddade av jordfelsbrytare.}
\end{minipage}
\end{center}


\subsection{Strömbrytare}

Kraftförsörjningen av radiostationens apparater bör ske över en
gemensam huvudströmbrytare, som lätt kan nås.
En indikatorlampa får gärna markera att den brytaren är tillslagen och att
stationen är under spänning.
Informera familjen och övriga i din omgivning om hur den brytaren fungerar.
Det är en säkerhetsåtgärd om något skulle hända.

Apparaternas nätströmbrytare ska vara utförda för den aktuella arbetsspänningen
och ha ett godkänt utförande.

\begin{description}
\item[Vid 1-fassystem] ska nätströmbrytaren i apparaterna vara 2-polig och bryta fas-
och N-ledare, men aldrig PE-ledaren.

\item[Vid 3-fassystem] ska nätströmbrytaren vara 3-polig och bryta fasledarna, men
aldrig N-ledare och PE-ledare.
\end{description}

\begin{center}
\begin{minipage}{0.19\columnwidth}
\Huge{\fontencoding{U}\fontfamily{futs}\selectfont\char 66\relax}
\end{minipage}
\begin{minipage}{0.7\columnwidth}
\textbf{Kom ihåg, att auktoriserad installatör ska anlitas för arbete
i fasta installationer.}
\end{minipage}
\end{center}
%%\noindent\textbf{Kom ihåg, att en auktoriserad elinstallatör ska
%%  anlitas vid ingrepp i fasta elinstallationer.}

\subsection{Liten terminologi vid elinstallationer}
\begin{description}[style=nextline]
\item[Gruppcentral] Den säkringscentral som följer efter elmätaren,
  till exempel i villor och lägenheter.

\item[Gruppledningar] Ledningar efter en gruppcentral, dvs.
  ledningar till belysning, el-spisar, uttag med mera.

\item[Fasledare] En ledare som för fasspänning.

\item[Nolledare (N-ledare)] En ledare som är ansluten till elnätets så kallade
  nollpunkt (nollskena) och som normalt inte ska föra spänning till jord.

\item[Skyddsledare (PE-ledare)] De ledare i kablar och sladdar, som är
  speciellt avsedda för skyddsjordning.

\item[Bruksföremål] Ett i princip flyttbart elanslutet föremål,
  till exempel handverktyg och radioapparater.

\item[Förstärkt isolering] Vissa bruksföremål tillverkas med en så god
  isolering att de inte behöver skyddsjordas.
  Så isolerade får anslutningsledningen förses med en speciell stickpropp,
  som passar i vägguttag, såväl med som utan jorddon.
  Sådana bruksföremål är märkta med Fi-märket bild~\ssaref{fig:Fi-mark} och får
  inte ändras så att de kan skyddsjordas.
\end{description}

\smallfig[0.1]{images/cropped_pdfs/Fi-mark.pdf}{Dubbel isolering, Fi-märke}{fig:Fi-mark}

Bild~\ssaref{fig:Fi-mark} visar Fi-märket, symbolen som finns på all elektrisk
utrustning som har dubbel isolering.

\subsection{Färgkoder för fas, noll- och skyddsledare}

Isoleringsmaterialet omkring gruppledarna i fasta elinstallationer har
färger som fyller en viktig funktion.
Tyvärr har användningen av dessa färger ändrats flera gånger under årens lopp,
vilket skapar risker för förväxling.
Ledarnas färger och funktion får aldrig förväxlas då det kan medföra fara för
allvarlig skada genom brand, elchock eller ljusbåge.

Fasledaren har numera brun färg vid nyinstallation, men har tidigare varit
både svart, grå, vit eller röd.
N-ledaren (nollan) har numera blå färg vid nyinstallation, men har tidigare
varit både svart och vit.
Skyddsledaren (PE-ledare) med gul/grön längsgående randig färgmärkning är
alltid en skyddsjordledare och får endast användas för det ändamålet.
I äldre installationer kan emellertid skyddsledarens isolering vara till
exempel röd.

Det är till fas och N-ledarna i vägguttagen, som man kopplar apparaterna för
att få ström.
Helst ska uttagen vara i skyddsjordat utförande det vill säga med ett
jordningsbleck.
Detta bleck är anslutet till den gul/gröna ledaren för skyddsjord.

\subsection{Uttag och stickproppar med jorddon}

Jorddonet ger förbindelse med elsystemets skyddsjord (PE).
Det är tidigare rummets utförande som avgjorde om vägg- och lamputtagen skulle
ha uttag med jorddon.
Kök och tvättstugor med ledande plåtbänkar, vattenkranar och så vidare anses
som riskfyllda rum och måste ha uttag med jorddon.
Samma gäller källare och liknande andra rum med ledande golv, väggar och
inredningar.
Bostadsrum var klassade som inte särskilt riskfyllda och har därför tidigare
inte försetts med lamp- och vägguttag med jorddon.

Vid nybyggnation är emellertid numera alla uttag är av skyddsjordat utförande!
Det rekommenderas att installera skyddsjordade vägguttag för radiostationen.
Observera då, att alla uttag i det rummet ska vara skyddsjordade!

\subsection{Skyddsjordning}

Att jorda är det vanliga uttrycket för att ansluta ett föremål till skyddsjord.
Men uttrycket används även lite slarvigt i andra fall utan att syfta på
skyddsjordning av elsäkerhetsskäl.

Metallhöljen på elektrisk utrustning kan av olika anledningar bli
spänningsförande och är då en elsäkerhetsrisk.
För att minska risken för farlig spänningssättning av metallhöljet ansluts
höljet till skyddsjord.

\begin{center}
\begin{minipage}{0.19\columnwidth}
\Huge{\fontencoding{U}\fontfamily{futs}\selectfont\char 66\relax}
\end{minipage}
\begin{minipage}{0.7\columnwidth}
Om det blir isolationsfel mellan en strömförande del och höljet kommer
säkringen att bryta strömtillförseln och risken för skada minskar.
\textbf{PE-ledaren får därför aldrig brytas!}
\end{minipage}
\end{center}


\noindent\emph{För skyddsjordning finns särskilda föreskrifter.
  Kontakta därför en auktoriserad elinstallatör.}

\subsection{Jordfelsbrytare}
\index{jordfelsbrytare}

Jordfelsbrytare är en automatisk strömbrytare som snabbt bryter strömmen
då strömmen till och från en apparat är olika.
Detta kan inträffa vid ett jordfel eller vid överledning i en skyddsjordad
apparat eller i andra fall när inkommande ström och utgående ström genom
jordfelsbrytaren inte är lika stora.
Jordfelsbrytaren kan skydda dig:

\begin{itemize}
\item vid isolations- och jordfel
\item om chassiet på en apparat blir strömförande
\item om du kommer åt spänningsförande delar och jord samtidigt
\item om vägguttagen saknar skyddsjord
\item om du använder en apparat på ett felaktigt sätt i våtutrymmen
\item om du installerat en apparat på att felaktigt sätt
\item om apparatens kabel skadats
\item mot och minimera risken för brand.
\end{itemize}

Jordfelsbrytaren \textbf{skyddar inte} för strömmar som går genom fasledare
och neutralledare eller genom fas till fasledare (3-fas).

Jordfelsbrytare får inte ersätta skyddsjordning, men kan under särskilda
förutsättningar komplettera skyddsjordningen som en extra säkerhetsåtgärd.
Vid nyinstallation av bostäder är det numera krav på att minst en
jordfelsbrytare ska installeras.
Beställ gärna installation av jordfelsbrytare i äldre anläggningar!

\subsection{Särjordning}
\index{särjordning}

Särjordning är ett uttryck för att jorda apparater till en separat jordpunkt,
det görs via separat jordlina till ett jordtag, det vill säga jordplåt eller
jordspett.
Särjordning ska ske på rätt sätt eftersom det avsedda skyddet annars kan bli en
fara.

\emph{Om du har planer på särjordning, fråga en auktoriserad installatör.}

\subsection{Jordning av antennsystem}

I brist på annan jordpunkt är det frestande att ansluta antennjordledaren till
PE-ledarens anslutningsbleck i vägguttaget eller till ett värmeelement med
förhoppning att på så sätt få ett bättre HF-jordplan för antennen.
Detta är emellertid ett dåligt exempel på särjordning, som både kan innebära
säkerhetsrisker och medföra störningsproblem.

\subsection{Snabba och tröga säkringar}
\index{säkringar}

Det finns snabba och tröga säkringar.
Snabba säkringar är det som normalt används.
Tröga säkringar för samma strömstyrka kan behövas för apparater som har
speciellt hög startström, till exempel stora nättransformatorer med toroidkärna.

Säkringarna ska kunna bryta tillräcklig hög spänning, annars blir det
en kvarstående ljusbåge i dem vid säkringsbrott.
Använd säkringar med rätta strömvärden och välj en säkring med lite marginal
till belastningsströmmen så att säkringen inte löser ut under normal drift.

Det är förbjudet att laga säkringar då det kan orsaka brand.

% Avsnitt 12.3 Faror
\input{koncept/chapter12-3}
% Avsnitt 12.4 Åska
\input{koncept/chapter12-4}
%
%
% Kapitel 13 Trafikreglemente
\chapter{Trafikreglemente}

% Avsnitt 13.1 Fonetiska alfabet
\section{Fonetiska alfabet}
%% k7per: Should these not be referenced?  I can add a "silent" harec macro that only adds to the "index"?
\harecsection{\harec{b}{1.1}{1.1} --
%\harec{b}{1.2}{1.2}
%\harec{b}{1.3}{1.3}
%\harec{b}{1.4}{1.4}
%\harec{b}{1.5}{1.5}
%\harec{b}{1.6}{1.6}
%\harec{b}{1.7}{1.7}
%\harec{b}{1.8}{1.8}
%\harec{b}{1.9}{1.9}
%\harec{b}{1.10}{1.10}
%\harec{b}{1.11}{1.11}
%\harec{b}{1.12}{1.12}
%\harec{b}{1.13}{1.13}
%\harec{b}{1.14}{1.14}
%\harec{b}{1.15}{1.15}
%\harec{b}{1.16}{1.16}
%\harec{b}{1.17}{1.17}
%\harec{b}{1.18}{1.18}
%\harec{b}{1.19}{1.19}
%\harec{b}{1.20}{1.20}
%\harec{b}{1.21}{1.21}
%\harec{b}{1.22}{1.22}
%\harec{b}{1.23}{1.23}
%\harec{b}{1.24}{1.24}
%\harec{b}{1.25}{1.25}
\harec{b}{1.26}{1.26}}
\label{fonetiska_alfabet}

Ibland behöver man göra förtydliganden genom att bokstavera.
Det internationella finns i ITU radioreglemente (RR) \cite[Appendix 14]{ITU-RR}
och kravställs i CEPT för HAREC \cite[Annex 6]{TR6102}.

Svenska radioamatörer ska kunna två fonetiska alfabet, dels det
internationella och dels det svenska.

Det kan vara värt att veta att det förekommer slang med andra ord vid
bokstavering.
Det finns därtill bokstavering på flera språk.
Medan dessa inte ska användas vid internationell trafik, så kan det vara bra
att känna till.

\begin{table}[htbp]
  \small
  \begin{tabular}{llll}
      & Kodord     & Uttal                                 & Svenskt kodord  \\ \hline
    A & Alfa       & \underline{all} fa                    & Adam            \\
    B & Bravo      & \underline{bra} vo                    & Bertil           \\
    C & Charlie    & \underline{tjar} li                   & Cesar          \\
    D & Delta      & \underline{dell} ta                   & David           \\
    E & Echo       & \underline{eck} å                     & Erik            \\
    F & Foxtrot    & \underline{fåcks} trått               & Filip           \\
    G & Golf       & \underline{gålf}                      & Gustav          \\
    H & Hotel      & hå \underline{tell}                   & Helge           \\
    I & India      & \underline{in} dia                    & Ivar            \\
    J & Juliett    & \underline{djo} li \underline{ett}    & Johan           \\
    K & Kilo       & \underline{ki} lå                     & Kalle           \\
    L & Lima       & \underline{li} ma                     & Ludvig          \\
    M & Mike       & majk                                  & Martin          \\
    N & November   & no \underline{vem} bö(rr)             & Niklas          \\
    O & Oscar      & \underline{åssk} a(rr)                & Olof            \\
    P & Papa       & pa \underline{pa}                     & Petter          \\
    Q & Quebec     & ke \underline{beck}                   & Qvintus         \\
    R & Romeo      & \underline{rå} mio                    & Rudolf          \\
    S & Sierra     & si \underline{err} ra                 & Sigurd          \\
    T & Tango      & \underline{täng} gå                   & Tore            \\
    U & Uniform    & \underline{jo} ni form                & Urban           \\
    V & Victor     & \underline{vick} tö(rr)               & Viktor          \\
    W & Whiskey    & \underline{oiss} ki                   & Wilhelm         \\
    X & X-ray      & \underline{ecks} rej                  & Xerxes          \\
    Y & Yankee     & \underline{jäng} ki                   & Yngve           \\
    Z & Zulu       & \underline{zo} lo                     & Zäta            \\
    Å & Alfa Alfa  & \underline{all} fa \underline{all} fa & Åke             \\
    Ä & Alfa Echo  & \underline{all} fa \underline{eck} å  & Ärlig           \\
    Ö & Oscar Echo & \underline{åssk} a \underline{eck} å  & Östen           \\
      &            &                                                         \\
    0 & Zero       & \underline{ze} ro                     & Nolla           \\
    1 & One        & o \underline{ann}                     & Ett (ej etta) \\
    2 & Two        & to                                    & Tvåa            \\
    3 & Three      & tri                                   & Trea            \\
    4 & Four       & får                                   & Fyra            \\
    5 & Five       & fajv                                  & Femma           \\
    6 & Six        & sicks                                 & Sexa            \\
    7 & Seven      & \underline{se} ven                    & Sju (ej sjua) \\
    8 & Eight      & ejt                                   & Åtta            \\
    9 & Nine       & \underline{naj} nö(rr)                & Nia             \\
      &            &                                                         \\
    , & Decimal    & \underline{de} si mal                 & Komma           \\
    . & Stop       & stopp                                 & Punkt           \\
  \end{tabular}
	\caption{Det internationella och svenska fonetiska alfabetet}
  	\label{tab:bokstavering-internationell}
	\label{tab:bokstavering-svenska}
\end{table}

% Avsnitt 13.2 Q-koden
\input{koncept/chapter13-2}
% Avsnitt 13.3 Trafikförkortningar
\section{Trafikförkortningar}
\label{trafikförkortningar}
\harecsection{\harec{b}{3.1}{3.1} --
%\harec{b}{3.2}{3.2}
%\harec{b}{3.3}{3.3}
%\harec{b}{3.4}{3.4}
%\harec{b}{3.5}{3.5}
%\harec{b}{3.6}{3.6}
%\harec{b}{3.7}{3.7}
%\harec{b}{3.8}{3.8}
%\harec{b}{3.9}{3.9}
%\harec{b}{3.10}{3.10}
%\harec{b}{3.11}{3.11}
\harec{b}{3.12}{3.12}}

Utöver Q-koden och klartext används vid morsetelegrafering även andra
trafikförkortningar.
Eftersom det internationella radiospråket är engelska, är förkortningar av
engelska ord vanligast.
Förkortningar bör emellertid inte användas i onödan.
En ovan operatör vid motstationen kan då få svårt att förstå meddelandet.

\subsection{Urval för radioamatörer}

I CEPT-rekommendation T/R 61-02 nämns utöver Q-koden följande övriga
trafikförkortningar, som berör amatörradio.
Radioamatörerna använder i praktiken många fler trafikförkortningar än dessa.

I reglementsprovet för amatörradiocertifikat ingår frågor om
trafikförkortningar, se tabell~\ssaref{tab:trafikforkortningar}.

% k7per, fixa bredd
\begin{table}
  \begin{tabular}{lll}
    Förkort- & & \\
    ning & Engelskt uttryck & Svensk betydelse \\
    \hline
    \textbf{BK} & break & avbryt(-a) (sändningen) \\
    \textbf{CQ} & ''seek you'' & allmänt anrop, till alla \\
    \textbf{CW} & continuous waves & telegrafi (A1A) \\
    \textbf{DE} & franska ''de'' & från ..... (anropssignal) \\
    \textbf{K}  & come & ''kom'' \\
    \textbf{MSG} & message & meddelande, telegram \\
    \textbf{PSE} & please & var god (att \dots) \\
    \textbf{R} & received & allt uppfattat, mottaget \\
    \textbf{RST} & readability, & läsbarhet \\
   & signal-strength, & signalstryka \\
   & tone-report & ton \\
    \textbf{RX} & receiver & mottagare \\
    \textbf{TX} & transmitter & sändare \\
    \textbf{UR} & your & din, ditt, dina, er \\
  \end{tabular}
\caption{Trafikförkortningar -- urval för radioamatörer}
\label{tab:trafikforkortningar}
\end{table}

Utöver ovanstående trafikförkortningar upptas i CEPT-rekommendationen
även följande bokstavskombinationer, vilka används i teleprintertrafik
i stället för motsvarande morsetecken, slagna utan teckenmellanrum.
(Strecket ovanför bokstäverna betecknar att det inte finns något mellanrum).

\begin{tabular}{lll}
  \textoverline{AR} & sluttecken & \(+\) \\
  \textoverline{VA} eller \textoverline{SK} & avslutningstecken & @ \\
\end{tabular}

Ett exempel på en avsnitt ur en amatörradiosändning, där
trafikförkortningar används särskilt flitigt:

''gm es tnx vy much om fer ur rprt. u are cmg in hr ufb. my tx is
.... and rx .... ant 3 el beam . condx hr gud mni dx stns hrd . wl nw
nil so tks es 73''

I klartext ser exemplet ut så här: ''good morning and thank you very
much Old Man for your report.
You are coming in here ultra fine business.
My transmitter is .....  and receiver .. ... antenna is a 3 element beam.
Conditions here are good many stations heard.
Well now nothing for you so thanks and kindest regards''

% Avsnitt 13.4 Internationell nödtrafik
\section{Internationell nödtrafik}
%% k7per: Refs?
\harecsection{\harec{b}{4.1}{4.1} --
%\harec{b}{4.2}{4.2}
%\harec{b}{4.3}{4.3}
\harec{b}{4.4}{4.4}}

\subsection{Nödsignaler}
\label{subsec:noedsignaler}
\index{nödsignaler}
\index{nödanrop}
\index{SOS}
\index{MAYDAY}

I CEPT-rekommendation T/R~61-02~\cite{TR6102} ställs krav på att radioamatörer
ska känna till de internationella nödsignalerna SOS och MAYDAY.

Nödsignalen på morsetelegrafi består av teckendelarna \Mcharsep\MSOS %. . . --- --- --- . . .
sända i en följd, där längden på de långa teckendelarna betonas så att de klart
skiljer sig från de korta.

Signalen skrivs som bokstäverna \textoverline{SOS} med ett streck ovanför.

Nödsignalen på radiotelefoni består av ordet MAYDAY uttalat som det franska
uttrycket ''m'aider''. I Sverige kan man även ropa ''NÖDANROP''.

\subsection{Internationella nödfrekvenser}
\index{nödfrekvens}
\label{nödfrekvens}

Nödsignaler på telefoni sänds i första hand på frekvenserna:

\begin{itemize}
  \item \qty{121,5}{\mega\hertz} AM (Flygradio).
  \item \qty{156,8}{\mega\hertz} FM (Marin VHF kanal 16).
\end{itemize}

En äldre nödfrekvens är \qty{2182}{\kilo\hertz}, men den är ej längre en primär
frekvens för nöd- och säkerhetstrafik.
Det finns inte längre krav på att fartyg ska ha radiopassning på frekvensen
vilket framgår av \emph{Transportstyrelsens föreskrifter och allmänna råd
	om radioutrustning på fartyg} TSFS 2009:95 \cite[\S22]{TSFS2009:95}.
(Läs mer om nödfrekvens i avsnitt~\ssaref{subsec:noedtrafik})

\begin{historiabox}
Kustradion i Sverige upphörde med sin radiopassning, vakthållning, av frekvensen
i början av 2005 och US Coast Guard slutade radiopassningen i augusti 2013.
\end{historiabox}

\begin{center}
\begin{minipage}{0.19\columnwidth}
\Huge{\fontencoding{U}\fontfamily{futs}\selectfont\char 66\relax}
\end{minipage}
\begin{minipage}{0.7\columnwidth}
I händelse av nöd, med omedelbart behov av assistans, är det därför
olämpligt att i första hand söka hjälp på frekvensen \qty{2182}{\kilo\hertz}.
\end{minipage}
\end{center}

Frekvensen \qty{2182}{\kilo\hertz} är fortfarande reserverad i ITU
Radioreglemente (RR) \cite{ITU-RR} för ''Distress and safety communications''
och radiopliktiga fartyg som trafikerar vatten i och utanför kustnära områden
ska ha radioutrustning för frekvensen.

\subsection{Nödtrafik}
\index{nödtrafik}
\index{nödfrekvens}
\index{GMDSS}
\label{subsec:noedtrafik}

I CEPT-rekommendation T/R~61-02~\cite{TR6102} ställs krav på att radioamatörer
ska känna till bestämmelser om nödtrafik och användningen av
amatörradiostationer vid naturkatastrofer.

ITU Radioreglemente (RR) \cite{ITU-RR} har sedan WRC-07 inte längre information
om ''Distress and safety communications'' för annat än
GMDSS (\emph{Global Maritime Distress and Safety System})

Med ''nödfrekvens'' avses en frekvens som radiopassas av exempelvis flyg- eller
sjöräddningscentral 24/7 (dygnet runt, året runt).
Även om termen ''nödfrekvens'' ibland förekommer i svenska bandplaner för
amatörradio, så finns inga egentliga sådana frekvenser inom amatörradiobanden.

År 1998 hölls en internationell konferens i Tammerfors i Finland
(\emph{ICET-98}).
Konferensen ledde fram till Tampere-konventionen ''The Tampere Convention on
the Provision of Telecommunication Resources for Disaster Mitigation and Relief
Operations'' \cite{TampereConvention}.
Konventionen trädde i kraft 8~januari 2005.

I enlighet med konventionen har IARU infört rekommendationer om regionala och
globala frekvenser för \emph{Emergency Centre of Activity}.
Det vill säga centerfrekvenser för radiokommunikation som kan användas i
händelse av naturkatastrofer.

\begin{center}
\begin{minipage}{0.19\columnwidth}
\Huge{\fontfamily{futs}\Huge{\hspace{1ex}!}}
\end{minipage}
\begin{minipage}{0.7\columnwidth}
IARU:s rekommendationer och förändringen av ITU RR innebär alltså att
det inte finns någon speciell nödsignal för amatörradiobanden och inga
nödfrekvenser inom amatörradiobanden.
\end{minipage}
\end{center}

För vidare läsning rekommenderas
\emph{IARU Emergency Telecommunications Guide} \cite{IARU-ETG}.

\newpage % layout
\subsection{Om du hör en nödsignal på radio}
\label{hör_nödtrafik}

Avbryt omedelbart din egen sändning när du hör en nödsignal. Lyssna på
nödmeddelandet och \textbf{skriv ner} vad som sägs.
Notera position, frekvens, tidpunkt etc. Anmäl vad du hört på följande sätt.

\subsubsection{Nödsignal från radioamatör i utlandet}

Om du uppfattar ett nödanrop från Sveriges närområde, så som Nordsjön och
Östersjön eller närliggande länder, så använd sunt förnuft och ring 112 och
berätta att du uppfattat en nödsignal från utlandet via radio.

För de fall att det är längre bort i Europa, kan det kanske vara läge att vara
lite mer restriktiv.

Oavsett vilket bör man först avvakta en stund för att övervaka om anropet verkar
besvaras av någon annan samt anteckna informationen i meddelandet innan man
själv besvarar det.

Det är aldrig fel vid uppfattandet av nödtrafik att kontakta 112 och påtala vad
som har uppfattats.
De kan sedan avgöra hur ärendet ska hanteras.

\subsubsection{Nödsignal från svenskt landområde}
\label{sv. nödsignal}

Ring 112 för att kalla på räddningstjänst, ambulans, polis, sjöräddning,
flygräddning etc.               %
Ditt telefonnummer visas automatiskt i larmoperatörens display.
För att undvika missförstånd och feldirigering av räddningsinsatserna
\textbf{måste} du meddela operatören att nödanropet kommit via radio.
Själva olycksplatsen kan ligga i ett helt annat riktnummerområde än det som ditt
telefonsamtal kommer ifrån.

\subsubsection{Nödsignal från fartyg eller luftfarkost}
\label{nödsignal fartyg}

Om nödsignalen inte besvaras av någon kust- eller markstation, ring 112
och begär sjöräddning respektive flygräddning och meddela dina iakttagelser.

\begin{center}
\begin{minipage}{0.19\columnwidth}
\Huge{\fontfamily{futs}\Huge{\hspace{1ex}!}}
\end{minipage}
\begin{minipage}{0.7\columnwidth}
Vidarebefordra nödmeddelandet utan att ändra på det!
\end{minipage}
\end{center}

\subsection{Du själv sänder nödsignal över radio}
\label{nödtrafik egen}

I första hand bör du välja andra signalvägar än amatörradio, så som fast och
mobil telefoni, båt- eller flygradio eller därför avsedda nödsändare om möjligt.

Välj gärna en frekvens med mycket trafik utifall du inte använder en
nödfrekvens, så det finns en chans att den är bevakad så att någon kan höra
ditt nödanrop.
Att använda en repeater för att höras bättre är en bra strategi.

Uppträd lugnt och sansat, när du kallar på hjälp över radion.
Tänk först och sänd sedan. Färsök få med så mycket som möjligt av det som listas
under ''åtgärder'' nedan.

\newpage % layout
\subsection{Åtgärder}

%% k7per: Om en förkortning uttalas som ordet larma, ska det inte skrivas some Larma?  Blir konstigt här... :-)
Nyckelordet för dina åtgärder är LARMA:
\index{LARMA}

\begin{description}
\item[Läge] Ange olycksplatsens läge. Du kan ange gatu- eller vägnamn eller riktmärken som
  till exempel vägkorset, gränsen, bron, järnvägen etc.
\item[Analysera] Gör en överblick över olycksplatsen och tala om vad som hänt.
  Några skadade? Några innestängda? Brinner det? Släpps farliga ämnen ut?
\item[Ropa] Ropa på hjälp. Använd gärna en repeater på 2-metersbandet så att du når många,
  men även andra frekvenser kan användas. Anropa med NÖDANROP FRÅN SMXxxx.
  Fråga efter någon med telefon. Ge inte upp om du inte får svar genast.
\item[Meddela] Meddela när du fått kontakt med någon med telefon, sänd NÖDTRAFIK PÅGÅR
  för att freda frekvensen och NÖDMEDDELANDET med de viktigaste uppgifterna.
  Begär att uppgifterna repeteras och ta löfte på att de sänds vidare.
  Begär att få veta när så har skett. Påminn annars!
\item[Avvakta] Vänta på platsen tills hjälp har anlänt.
  Passa radion så att du kan svara på frågor. Behövs inte längre din hjälp, avsluta då med
  NÖDTRAFIK UPPHÖR FRAN SMXxxx \dots \\
  KLART SLUT.
\end{description}

% Avsnitt 13.5 Anropssignaler
% Avsnitt 13.6 Användning av anropssignal
% Avsnitt 13.7 Exempel på kontakt
% Avsnitt 13.8 Innehåll i förbindelse
% Avsnitt 13.9 Hederskod
% Avsnitt 13.10 Ordningsregler
\section{Anropssignaler}
\label{anropssignaler}

\subsection{Anropssignalernas syfte}

Alla radiosändare ska vara identifierbara, så att man kan veta vem
som sänder \cite[\S19.1]{ITU-RR}.
Identifiering görs med hjälp av en anropssignal, som är en kombination av
bokstäver, (A--Z) och siffror (0--9). \cite[\S19.45]{ITU-RR}.
Ett tecken är antingen en bokstav eller siffra.
Nationella bokstäver som Å, Ä och Ö samt andra specialtecken används inte.
Anropssignaler är internationellt koordinerade och unika, vilket är nödvändigt
när signalerna kan komma att höras över hela världen.
Systemet är gemensamt för kommersiell trafik och amatörradio, men vi kommer
enbart beröra de anropssignaler som är aktuella för amatörradio.

\begin{itemize}
\item Alla sändningar med falsk eller missledande identifiering är förbjuden
\cite[\S19.2]{ITU-RR}!

\item Alla amatörradiosändningar ska vara identifierade \cite[\S19.4, \S19.5]{ITU-RR}.
\end{itemize}

Identifiering sker normalt i tal eller på morsetelegrafi, men även andra former
kan förekomma som är anpassade till modulationsmetoden som används.

Det finns flera sätt på vilka personen bakom en anropssignal kan identifieras.
För svenska anropssignaler tillhandahåller SSA en Callbook
<\href{https://www.ssa.se/}{\texttt{www.ssa.se}}>.
En annan populär variant är QRZ
<\href{https://www.qrz.com/}{\texttt{www.qrz.com}}> där man kan registrera sig.
Anropssignalen används även för online-loggning av kontakter, så som Logbook of
the World (LoTW) <\href{https://lotw.arrl.org/}{\texttt{lotw.arrl.org}}>.

\subsection{Anropssignalernas sammansättning}

Varje land disponerar en eller flera serier med unika anropssignaler för all
sin radiotrafik.
Dessa utformas enligt ITU Radioreglemente (RR) \cite[\S19]{ITU-RR} på sätt,
som beror på syftet med varje särskild radiostation.
I RR finns definitioner för olika slags stationer, till exempel stationer för
fast radio, landmobila stationer, stationer i fartyg, i sjöräddningsfarkoster,
i flygplan, amatörradiostationer och så vidare.

\subsection{Identifiering av amatörradiostationer}
\harecsection{\harec{b}{5.1}{5.1}, \harec{b}{5.3}{5.3}}

En radiostation ska identifieras med den anropssignal, som tilldelats av det
egna landets teleadministration (myndighet).
I Sverige är det Post- och telestyrelsen (PTS) som har ansvaret och som genom
beslut har delegerat handläggningen av amatörradiosignaler till Föreningen
Sveriges Sändareamatörer (SSA).
Anropssignalen meddelas i det amatörradiocertifikat som erhålls efter godkänt
kompetensprov.

Anropssignaler för amatörradio är uppbyggda av ett prefix, en siffra och ett
suffix på följande sätt \cite[\S19.68, \S19.69]{ITU-RR}:

\begin{itemize}
\item Prefixet består vanligtvis av två tecken, exempelvis SM~(Sverige), 9A~(Kroatien)
eller S5~(Slovenien).
\item Prefixet kan ibland bestå av en ensam bokstav, som i så fall måste vara någon
av B, F, G, I, K, M, N, R eller W.
\end{itemize}

Sverige är tilldelat prefix i serierna SA--SM, 7S och 8S
\cite[Appendix 42]{ITU-RR}, se tabell~\ssaref{tab:seprefix}.

Prefixet följs av en siffra och ett suffix. Suffixet består av minst ett och
högst fyra tecken, där det sista tecknet inte får vara en siffra.

Anropssignaler för speciella ändamål, exempelvis för att fira något jubileum,
kan ha suffix som består av fler än fyra tecken \cite[\S19.68A]{ITU-RR}.
Sådana anropssignaler, eller andra som inte följer formatmallen, behöver i så
fall godkännas av PTS innan de kan tilldelas av SSA.

\begin{exempelbox}
\signal{DL65DARC} är en eventsignal för tyska (DL) amatörradioföreningen
DARC:s 65-års jubileum.
\end{exempelbox}

PTS regler för tilldelning av svenska anropssignaler kan skilja sig från
grundreglerna i RR som anges ovan, men följer i allmänhet dessa.

Anropssignaler för svenska amatörradiostationer är uppbyggda på följande
sätt, varvid med distrikt avses amatörradiodistrikt.

\begin{table*}[ht]
  \begin{center}
    \begin{tabular}{lll}
      \emph{enskilda radioamatörer} & \textbf{SA} &
      + distriktssiffra + treställigt suffix (grundsignal) \\
      \emph{enskilda radioamatörer} & \textbf{SM} &
      + distriktssiffra + två- eller treställigt suffix (grundsignal) \\
      \emph{amatörradioklubbar} & \textbf{SA} &
      + distriktssiffra + tvåställigt suffix \\
      \emph{amatörradioklubbar} & \textbf{SK} &
      + distriktssiffra + tvåställigt suffix \\
      \emph{militära förband och FRO} & \textbf{SL} &
      + distriktssiffra + två- eller treställigt suffix \\
    \end{tabular}
    \caption{Svenska anropssignalprefix}
    \label{tab:seprefix}
  \end{center}
\end{table*}

Signalserien SM är tilldelad av Televerket och sedermera PTS fram till 2009.
Signalserien SA är tilldelad av SSA från 2004.
Äldre anropssignaler i SM-serien är tilldelade med tvåställiga suffix, medan
nyare SM- och SA-signaler har treställiga suffix.

Utöver grundsignalen finns även extra anropssignaler tilldelade i de övriga
tillgängliga serierna.

\begin{exempelbox}
	\begin{itemize}
		\item \signal{SM0XXX} är en radioamatör som fått sin tilldelning av PTS.
		\item \signal{SA0XXX} är en radioamatör som fått sin tilldelning av SSA.
		\item \signal{SK2XX} är en amatörklubb.
		\item \signal{SM7X} är en radioamatör med kort anropssignal.
	\end{itemize}
\end{exempelbox}

\medskip

Sverige är indelat i amatörradiodistrikt med följande numrering och
utsträckning:

\begin{center}
\begin{tabular}{rp{6cm}}
\emph{Distrikt} & \emph{Utsträckning} \\
\textbf{0} & Stockholms (AB) län \\
\textbf{1} & Gotlands (I) län \\
\textbf{2} & Västerbottens (AC) och Norrbottens (BD) län \\
\textbf{3} & Gävleborgs (X), Jämtlands (Z) och Västernorrlands (Y) län \\
\textbf{4} & Örebro (T), Värmlands (S) och Dalarnas (W) län \\
\textbf{5} & Östergötlands (E), Södermanlands (D), Västmanlands (U) och Uppsala (C) län\\
\textbf{6} & Hallands (N) och Västra Götalands (O) län \\
\textbf{7} & Skåne (M), Blekinge (K), Kronobergs (G), Jönköpings (F) och Kalmar (H) län.\\
\end{tabular}
\end{center}

Distriktssiffran i anropssignalen bestäms av det län som hemadressen är belägen inom.
Vid sändning utanför hemadressen bör det framgå av tillägg till anropssignalen.

\begin{exempelbox}
	\begin{itemize}
		\item \signal{SA0XXX} är en radioamatör hemmahörande i Stockholms län.
		\item \signal{SM7YYY} är en radioamatör hemmahörande i Jönköpings län.
		\item \signal{SK7AX} är en amatörklubb hemmahörande i Jönköping län.
	\end{itemize}
\end{exempelbox}
\medskip

I Post- och telestyrelsens föreskrifter sägs dock inte vilken distriktssiffra
som ska användas, när sändning sker från annan plats än hemortsadressen.

Med stöd av praxis rekommenderar dock SSA att följande regler tillämpas:

\begin{itemize}
\item Vid trafik från en regelbundet använd fritidsbostad kan i
  anropssignalen användas den distriktssiffra som utvisar var
  fritidsbostaden är belägen.

\item Vid trafik från annan tillfällig plats bör anropssignalen
  åtföljas av snedstreck och siffran för det distrikt varifrån
  sändningen görs. Till exempel \signal{SM0XYZ} i distrikt 6 blir \signal{SM0XYZ/6}
  vilket låter som ``S M nolla X Y Z streck sexa.''

\item Vid trafik från mobil station bör den ordinarie anropssignalen
  även åtföljas av \signal{/M}. Till exempel \signal{SM0XYZ} mobil i distrikt 6 blir \signal{SM0XYZ/6/M} vilket låter som ``S M nolla X Y Z streck sexa mobil.''

\item Vid trafik från mobil station inom hemorten kan dock den extra
  distriktssiffran utelämnas.  Till exempel \signal{SM9XYZ} mobil hemma vid blir \signal{SM0XYZ/M} vilket låter som ``S M nolla X Y Z mobil.''

%% k7per???   
\item Vid trafik från sjöfarkost bör den ordinarie anropssignalen
  åtföjas av \signal{/MM} vilket låter som ``maritime mobil.''

%% k7per???   
\item Vid trafik från luftfarkost bör den ordinarie anropssignalen
  åtföljas av \signal{/AM} vilket låter som ``aeromobil.''

\item Vid trafik från svensk farkost på internationellt territorium
 kan distriktssiffran 8 användas.

\item Vid sändning från ett annat lands territorium gäller det landets
  bestämmelser.
  Vid osäkerhet -- vänd dig till SSA!

\item Utländsk radioamatör på besök i Sverige ska använda sin
  anropssignal från det egna landet, föregånget av \signal{SM/}. Till exempel \signal{SM/LA9XX} vilket låter som ``S M streck L A nia X X'' \cite{TR6101}.
\end{itemize}

\subsection{Nationella prefix}
\harecsection{\harec{b}{5.4}{5.4}}

Tabell~\ssaref{tab:landsprefix} visar några viktiga nationella prefix att kunna.

\begin{table*}[ht]
  \begin{center}
    \begin{minipage}{.3\linewidth}
      \begin{tabular}{ll}
        \emph{Prefix} & \emph{Land} \\
        \hline
        DL            & Tyskland    \\
        EA            & Spanien        \\
        EA8           & Kanarieöarna   \\
        ES            & Estland     \\
        F             & Frankrike   \\
        G             & Storbritannien \\
        HB            & Schweiz     \\
        HS            & Thailand    \\
        I             & Italien     \\
        JA            & Japan          \\
      \end{tabular}
    \end{minipage}
    \begin{minipage}{.3\linewidth}
      \begin{tabular}{ll}
        \emph{Prefix} & \emph{Land} \\
        \hline
        K             & USA         \\
        LA            & Norge       \\
        LU            & Argentina      \\
        LY            & Litauen        \\
        OH            & Finland     \\
        OH0           & Åland          \\
        OK            & Tjeckien    \\
        ON            & Belgien     \\
        OZ            & Danmark     \\
        PA            & Holland        \\
      \end{tabular}
    \end{minipage}
    \begin{minipage}{.3\linewidth}
      \begin{tabular}{ll}
        \emph{Prefix} & \emph{Land} \\
        \hline
        PY            & Brasilien   \\
        S5            & Slovenien   \\
        SP            & Polen       \\
        SV            & Grekland       \\
        UA            & Ryssland    \\
        VE            & Kanada      \\
        VK            & Australien  \\
        YL            & Lettland    \\
        ZL            & Nya Zeeland    \\
        ZS            & Sydafrika   \\
      \end{tabular}
    \end{minipage}
    \caption{Landsprefix}
    \label{tab:landsprefix}
  \end{center}
\end{table*}

\section{Användning av anropssignal}
\harecsection{\harec{b}{5.2}{5.2}, \harec{b}{7.2.2}{7.2.2}}

Både motstationens och den egna anropssignalen ska användas i början
och slutet av varje sändning.
Under sändningen ska anropssignalen upprepas ''med korta mellanrum'', utan
närmare precisering av mellanrummet.
Även om man inte har kontakt med en motstation, ska den egna anropssignalen
anges vid varje sändning.
Se vidare i PTS föreskrifter.

\section{Exempel på kontakt}
\harecsection{\harec{b}{7.2.1}{7.2.1}}

Det finns många sätt att genomföra en radiokontakt, men det finns några
grundregler för hur man uppträder och utväxlar samtal.
Ett trevligt och kamratligt uppträdande är en hederssak inom amatörradion.
Det behöver inte bli stelt för den skull!

Allmänt anrop är ett sätt att kalla på någon
-- vem som helst -- att kommunicera med.
På telegrafi låter det så här:

-- CQ CQ CQ de SM0XYZ SM0XYZ K

Det vill säga anropet först och därefter den egna anropssignalen.
På telefoni låter det så här:

-- Allmänt anrop, allmänt anrop, allmänt anrop från SM0XYZ Kom

Glöm inte Kom i slutet.
Riktat anrop gör man, när man vill tala med någon särskild station.
Då sänder man först anropssignalen på den station, som man vill tala med och
därefter sin egen anropssignal.
På telegrafi låter det så här:

-- SM0ZYX SM0ZYX de SM0XYZ SM0XYZ K

På telefoni låter det så här:

-- SM0ZYX SM0ZYX från SM0XYZ SM0XYZ Kom

Motstationen svarar förhoppningsvis på anropet, alltså

-- SM0XYZ från SM0ZYX Kom

\subsection{Upprättad förbindelse}

När en station svarat på anrop, lämnar man först sin signalrapport enligt
RST-koden och presenterar sig med sitt förnamn och berättar var man finns.
Motstationen kvitterar troligen med sina motsvarande uppgifter.

När man överlämnar ordet till motstationen avslutar man meningen med Kom och
lyssnar.
Om man har en telegrafiförbindelse och bara vill att den station man har
förbindelse med ska svara kan man sända KN (eng. \emph{come named station}).

Om förbindelsen varar länge, är det lämpligt att upprepa anropssignalerna
ungefär var tionde minut vid överlämning.

-- SM0ZYX från SM0XYZ Kom

\subsection{Avsluta förbindelse}

När man så småningom avslutar kontakten tackar man för sig på och utbyter
avskedshälsningar. Då kan det låta så här:

-- Tack för en trevlig förbindelse och på återhörande. SM0ZYX från
SM0XYZ. Klart Slut.

Träna med din instruktör på att klara olika slags trafiksituationer!

\subsection{Second operator}
\index{Second operator}
\label{secondoperator}

Den som självständigt använder en amatörradiosändare ska ha ett
amatörradiocertifikat.
Det finns ett undantag från kravet på amatörradiocertifikat då en person
tillfälligt använder en amatörradiosändare under uppsikt av någon som har ett
amatörradiocertifikat.
Detta kallas \emph{second operator} och innebär att en person som saknar
amatörradiocertifikat kan agera operatör jämte en person som har ett.

I Sverige är det reglerat i undantagsföreskriften PTSFS 2022:19 som tas upp i
avsnitt~\ssaref{PTSFS2022:19}.
Detta medger att man kan förevisas hobbyn och även träna under kontrollerade
förhållanden.
För att detta ska fungera krävs att den med amatörradiocertifikat instruerar
om hur man ska bete sig i etern, hur handhavandet går till och kan övervaka
att detta följs.

Självklart används anropssignalen för innehavaren av amatörradiocertifikatet.
Det är bra att det tydligt framgår att det är en second operator som är aktiv.
Antingen ropar amatören upp och sedan berättar att han lämnar över till second
operator Simon.
Alternativt kan en second operator göra anropen själv och då ropa till exempel
''SM5XYZ second operator Anna''.

Möjligheten att använda second operator ska användas med klokhet, och kan rätt
använd skapa en god förståelse för hobbyn och utgöra en morot för att få både
ungdomar och vuxna intresserade av amatörradio.

\subsection{CQ DX och split}
\label{cq dx och split}
\index{CQ DX}
\index{split}
\index{DX expedition}
\index{rar DX}
\index{pile-up}

Det förekommer att man hör någon ropa \emph{''CQ DX''}, vilket betyder att
stationen söker långväga kontakter, i allmänhet utanför sin egen världsdel.

-- ''CQ DX, CQ DX, CQ DX, SM0XYZ calling CQ DX and standing by''

I detta fallet är det SM0XYZ som söker att nå någon utanför Europa.
Är du själv inte ett DX, det vill säga om du befinner dig i samma världsdel så
ska du undvika att svara.

Ibland genomförs så kallade \emph{DX expeditioner} då man beger sig till en
plats som sällan aktiveras.
Man brukar tala om \emph{rara DX} (eng. \emph{rare DX}), då ett ovanligt
landområde aktiveras, som många vill ha i sin logg.

En station som ropar CQ kan få svar från många stationer samtidigt.
Då uppstår ett sammelsurium av signaler som kallas för \emph{pile-up}.

När stationen betar av en pile-up kan stationen även fråga \emph{''QRZ?''},
alltså ''vem där''?

Ett rart DX kan drabbas av enorma pile-ups, och det kan bli svårt för
motstationerna att höra DX-stationen bland alla andra som ropar samtidigt.
Det kan kan också vara svårt för DX-stationen att urskilja vilka motstationer
som svarar, om alla svarar på samma frekvens.
En strategi för att få detta att fungera effektivare är att köra split
\cite{LowBandDX}, det vill säga att DX-stationen sänder och lyssnar på olika
frekvenser (men fortfarande inom samma frekvensband).

Oftast väljer DX-stationen att lyssna på en frekvens som ligger några kilohertz
högre än den egna sändningsfrekvensen, och anger detta genom att sända
exempelvis \emph{''listening up''} eller \emph{''listening five up''}.

Genom att använda sig av split undviker DX-stationen att störas ut av sin egen
pile-up.
DX-stationen kan också välja sprida ut sin pile-up, genom att inte lyssna på
endast en frekvens, utan genom att svepa över ett lite större område.

\emph{''Listening five to ten up''} betyder då att DX-stationen lyssnar i ett
område mellan 5 och \qty{10}{\kilo\hertz} över den egna sändningsfrekvensen, och
motstationerna får försöka gissa var i detta frekvensområde som DX-stationen
lyssnar just för tillfället.

För trafik på morsetelegrafi använder man vanligtvis ett mindre avstånd mellan
sändar- och mottagarfrekvens än för trafik på SSB-telefoni, eftersom
telegrafisignalerna upptar mindre bandbredd.

Moderna transceivrar har nästan alltid möjlighet att ställa in split genom att
man använder ''VFO A/B'', ''RIT/XIT'' eller ''clarifier''.
Mer avancerade transceivrar kan ha möjlighet att separera de två frekvenserna i
hörlurarnas vänster- respektive högerkanal.

När en station ropar CQ och gör paus för anropande stationer, ange då din egen
signal kort och tydligt en gång i varje pass.
Istället för att ropa flera gånger varje pass, skrika eller på annat sätt
ta utrymme, ha tålamod och vänta ut bra tillfälle.
Ropa inte under den tid som DX-stationen sänder sitt CQ. Då hör han dig ju ändå inte.

Det kan vara nyttigt att lyssna in sig på operatörens stil.
Var medveten om att DX-stationen kan höra helt andra stationer starkare än vad
du hör, eftersom konditionerna kan vara helt annorlunda för DX-stationen.

Vid stora pile-ups kan operatören välja att bara lyssna efter vissa stationer,
och därför fråga efter ''only number five stations please'' eller efter ''only
European stations please''.
Detta syftar till att dela upp en stor pile-up för en chans att lättare
uppfatta vilka som anropar.

Vid uppdelning efter nummer kommer operatören avverka några stationer med ett
visst nummer i anropssignalen, för att sedan gå vidare till nästa och så
vidare, tills 0 till 9 är genomgångna.

Alternativet att gå efter regioner eller landsprefix kan vara att föredra om
operatören upplever att konditionerna dit snart försvinner och därför vill ge
dem en extra förtur innan de helt tappar chansen.

\paragraph{Lär dig ''DX:arens ordningsregler'':}

\begin{itemize}
\item Jag ska lyssna, och lyssna, och sedan lyssna lite till.
\item Jag ska ropa endast om jag kan läsa DX-sta\-tion\-en ordentligt.
\item Jag ska icke lita på cluster-information, utan vara helt klar över DX-stationens anropssignal innan jag ropar.
\item Jag ska icke störa DX-stationen, eller någon som anropar denne, och jag ska aldrig stämma av på DX:ets egen frekvens, eller i det segment där denne lyssnar.
\item Jag ska vänta tills DX:et avslutat föregående kontakt innan jag ropar själv.
\item Jag ska alltid ange min fullständiga anropssignal.
\item Jag ska ropa och lyssna med lämpliga intervaller.
\item Jag ska icke ropa kontinuerligt.
\item Jag ska icke ropa då DX:et svarar någon annan än mig.
\item Jag ska icke ropa då DX:et frågar efter en anropssignal som icke liknar min egen.
\item Jag ska icke ropa då DX:et söker efter ett annat geografiskt område än mitt eget.
\item När DX:et svarar mig så ska jag icke upprepa min anropssignal, annat än om jag tror att denne ej uppfattat den korrekt.
\item Jag ska vara tacksam om och när jag får kontakt.
\item Jag ska respektera mina amatörkamrater och uppträda så jag förtjänar deras respekt.
\end{itemize}

Försök inte agera polis och rätta andra stationer som du anser bryter mot reglerna!

\section{Innehåll i förbindelse}
\label{innehåll i förbindelse}
\harecsection{\harec{b}{7.2.3}{7.2.3}}

Tidigare har det i Sverige varit reglerat vad innehållet får vara i
förbindelser, eller snarare vad de inte får innehålla.
Den regleringen är numera borttagen.
Man ska vara medveten om att samma regler och förutsättningar inte gäller i
alla länder och för deras radioamatörer.
Därför uppmanas du att använda sunt förnuft, hålla god ton och respektera alla
amatörer.
Se även IARU etik och trafikmetoder.

\subsection{Tystnadsplikt}
\index{tystnadsplikt}
\index{LEK}

Innehållet i en radioförbindelse skyddas av
\emph{Lag om elektronisk kommunikation (LEK)} \cite{SFS2022:482}.
I LEK regleras tystnadsplikt för radiobefordrade meddelanden i kapitel~6.

\begin{quote}
	Den som i annat fall än som avses i 31~\S{} första stycket och 32~\S{} i
	radiomottagare har avlyssnat eller på annat sätt med användande av sådan
	mottagare fått tillgång till ett radiobefordrat meddelande i ett
	elektroniskt kommunikationsnät som inte är avsett för honom eller henne
	själv eller för allmänheten får inte obehörigen föra det vidare.
	Lag (2022:482).\cite[kap 9, \S33]{SFS2022:482}
\end{quote}

Tystnadsplikten gäller alla radiomeddelanden som avlyssnats, oavsett ursprung.

Detta innebär att om du själv varit part i radiomeddelandet eller om
radiomeddelandet var en nyhetsbulletin avsett för många så får du föra det vidare.

En stor del av radioamatörhobbyn bygger dock på radiokommunikation med andra och
att andra kan höra dig när du sänder.
En radioamatör kan därför inte anses vara omedveten om att någon annan lyssnar
på det som sänds ut.
Därför är mycket accepterat inom amatörradio som annars skulle vara förbjudet.

Tips om rara DX, tips om någon som ropar CQ, QSL från lyssnaramatörer, att
berätta att du hörde någon ha förbindelse med någon annan anses därför normalt
inte vara ett brott mot tystnadsplikten.

Att koppla en radiomottagare till webben så att någon kan lyssna på radiotrafik
i realtid är tillåtet.

\emph{Observera även texten i andra punkten i 6~kap.~20~\S{} gällande den som i
	samband med tillhandahållande av ett elektronisk kommunikationstjänst har fått
	del av eller tillgång till innehållet i ett elektroniskt meddelande inte
	obehörigen får föra vidare eller utnyttja det han fått del av eller tillgång
	till.}

Detta kan vara aktuellt då någon som tillhandahåller en elektronisk
kommunikationstjänst från punkt A till punkt B får tillgång till innehållet i
ett elektroniskt meddelande när det har lämnat punkt A och innan det når fram
till punkt B.

\subsection{Inspelning av radiomeddelande}
\index{inspelning}
\index{GDPR}
\index{Dataskyddsförordningen}

Radiosamtal som du själv deltar i får spelas in utan att andra deltagare i
samtalet informeras om att du spelar in samtalet.

Grundregeln är att inspelning av radiomeddelanden är tillåten såvida inte
inspelningen är förbjuden för att skydda personers personliga integritet.

Uppspelning av de inspelade meddelandena får inte bryta mot bestämmelserna om
tystnadsplikt.
Det vill säga att meddelandet inte obehörigen får föras vidare.

Alla radiomeddelanden får inte spelas in.
Lagstiftningen skiljer även på analoga- och digitala inspelningar.
Dataskyddsförordningen~\cite{GDPR} samt 4~kap.~9a~\S{} i
Brottsbalken~\cite{SFS1962:700} är exempel på lagar som begränsar inspelning av
avlyssnade radiomeddelanden.

Av ovanstående följer att det inte är tillåtet att lagra inspelad radiotrafik
för senare lyssning via webbaserade medier då det kan anses kränka den
personliga integriteten.

\subsection{Kryptering av radiomeddelande}
\label{kryptering av radiomeddelande}
\index{kryptering}

Inom Sveriges gränser är kryptering av radiomeddelanden på amatörradiofrekvenser
tillåten under villkor att en anropssignal regelbundet sänds ut, anropssignalen
ska då kunna avläsas med kända tekniker.
Trots detta rekommenderas inte användning av kryptering för amatörradiotrafik.

Tekniken för kryptering av radiomeddelanden har blivit mera lättillgänglig i
samband med införandet av digitala radiosystem typ DMR (Digital Mobile Radio) på
amatörbanden.
Ett flertal av dessa radiosystem är dock ihopkopplade via internationella
nätverk och därigenom hörbara i flera länder där kryptering inte är tillåten.

Användningen av krypteringsteknik på amatörradiofrekvenser riskerar därför att
medföra begränsningar i de rättigheter vi har enligt PTSFS 2022:19.

\newpage % layout
\section[Hederskod]{Radioamatörens hederskod}
\harecsection{\harec{b}{7.1.1}{7.1.1}}
\index{Radioamatörens hederskod}

Radioamatören är

\begin{description}
  \item[HÄNSYNSFULL] Han agerar aldrig medvetet på ett sätt som minskar nöjet för andra.

  \item[LOJAL] Han erbjuder lojalitet, uppmuntran och stöd åt andra amatörer, lokala klubbar,
    IARU organisationen i hans land genom vilken amatörradio i hans land
    representeras nationellt och internationellt.

  \item[PROGRESSIV] Han håller sin station på en hög teknisk nivå. Den är välbyggd och effektiv.
    Hans operationsteknik är oantastlig.

  \item[VÄNLIG] Han kommunicerar sakta och tålmodigt när så begärs; erbjuder kamratligt stöd och ger nybörjaren goda råd; vänlig assistans, samarbete och omtanke i andras intresse. Detta är kännetecknen för amatörandan.

  \item[BALANSERAD] Radio är en hobby och får aldrig orsaka konflikt i förpliktelser gentemot
    familj, arbete, skola eller samhälle.

  \item[PATRIOTISK] Hans station och hans kunnande står alltid till förfogande för att
    assistera land och samhälle.
\end{description}

%% \begin{tabular}{lp{6cm}}
%%   \textbf{HÄNSYNSFULL} &
%%   Han agerar aldrig medvetet på ett sätt som minskar nöjet för andra. \\
%%   & \\
  
%%   \textbf{LOJAL} &
%%   Han erbjuder lojalitet, uppmuntran och stöd åt andra amatörer, lokala klubbar,
%%   IARU organisationen i hans land genom vilken amatörradio i hans land
%%   representeras nationellt och internationellt.\\
%%   & \\
  
%%   \textbf{PROGRESSIV} &
%%   Han håller sin station på en hög teknisk nivå.
%%   Den är välbyggd och effektiv.
%%   Hans operationsteknik är oantastlig.\\
%%   & \\
  
%%   \textbf{VÄNLIG} &
%%   Han kommunicerar sakta och tålmodigt när så begärs;
%%   erbjuder kamratligt stöd och ger nybörjaren goda råd;
%%   vänlig assistans, samarbete och omtanke i andras intresse.
%%   Detta är kännetecknen för amatörandan.\\
%%   & \\
  
%%   \textbf{BALANSERAD} &
%%   Radio är en hobby och får aldrig orsaka konflikt i förpliktelser gentemot
%%   familj, arbete, skola eller samhälle.\\
%%   & \\
  
%%   \textbf{PATRIOTISK} &
%%   Hans station och hans kunnande står alltid till förfogande för att
%%   assistera land och samhälle.\\
%% \end{tabular}

\emph{-- anpassad från den ursprungliga Amateur's Code, skriven av Paul M. Segal, W9EEA, 1928.}

\section[Ordningsregler]{Radioamatörens ordnings\-regler}
\harecsection{\harec{b}{7.1.2}{7.1.2}}

\subsection{Grundläggande principer}
\textbf{Grundläggande principer} som ska styra vårt \textbf{uppträdande} på
amatörbanden är:

\begin{description}
\item[Samhörighet, broderskap och kompiskänsla:] många, många av oss
  är aktiva i etern (vår spelplan).
  Vi är aldrig ensamma.
  Alla andra amatörer är våra kollegor, våra bröder och systrar, våra vänner.
  Agera därefter.
  Var alltid hänsynsfull.

\item[Tolerans:] inte alla amatörer delar nödvändigtvis samma
  uppfattning som du, och din uppfattning är kanske inte den bästa.
  Förstå att det finns andra med en annan uppfattning om ett visst tema.
  Var tolerant.
  Du har inte denna värld för dig själv.

\item[Anständighet:] aldrig får svordomar och oanständigheter yttras
  på banden.
  Ett sådant beteende säger ingenting om den person som de är avsedda för men
  mycket om den person som uttalar dem.
  Behåll ditt lugn i alla situationer.

\item[Förståelse:] Var snäll och förstå att alla inte är så smarta,
  så professionella eller så mycket expert som du.
  Om du vill göra något åt detta agera positivt (hur kan jag hjälpa till,
  hur kan jag förbättra, hur kan jag lära ut) i stället för negativt
  (med svordomar, förolämpningar etc.).
\end{description}

\subsection{Risken för konflikter}
\textbf{Endast en spelplan, etern}: alla radioamatörer vill spela sitt spel
eller utöva sin sport men det måste göras på en enda spelplan: våra amatörband.
Hundratusentals spelare på en enda spelplan leder ibland till konflikter.

Ett exempel: Plötsligt hör du någon ropa CQ på din frekvens (den frekvens du
har kört på en stund).
Hur är detta möjligt?
Du har varit igång här mer än en halvtimme på en helt ren frekvens!
Jo, visst är det möjligt; den där andra stationen tror kanske också att du stör
honom på HANS frekvens.
Kanske har skippet eller konditionerna ändrats?

\subsection{Hur undvika konflikter?}
\begin{itemize}
\item Genom att förklara för alla spelare vilka regler som gäller och genom
  att motivera dem att tillämpa dessa regler.
  De flesta konflikter orsakas av \textbf{okunskap}:
  många spelare känner inte till reglerna tillräckligt väl.

\item Dessutom hanteras många konflikter dåligt återigen på grund av
  \textbf{okunskap}.

\item Den IARU-etikhandbok som finns översatt på SSA:s webbplats avser att
  åtgärda denna brist på kunnande i huvudsak genom att lära ut hur man kan
  undvika konflikter av alla slag.
\end{itemize}

\subsection{Moraliska aspekter}
\begin{itemize}
\item I de flesta länder bryr sig myndigheterna inte om i detalj hur
  amatörerna uppför sig på banden, förutsatt att de håller sig till reglerna
  som myndigheten fastslagit.
\item Radioamatörerna anses vara \textbf{självstyrande}, detta betyder att
  självdisciplin måste utgöra basen i vårt agerande. Det betyder emellertid
  inte att radioamatörerna har en egen polisiär funktion!
\end{itemize}

\subsection{Förhållningsregler}

Vad menar vi med \textbf{förhållningsregler} (code of conduct)?
De är en uppsättning regler baserade på såväl \textbf{etiska} principer som
\textbf{trafikmässiga hänsyn}.

\begin{description}
\item[Etik] Etik bestämmer vår attityd och vårt allmänna uppförande
  som radioamatörer.
  Etik har med moral att göra.
  Etik utgör principerna för moral.

  Exempel: etiken säger oss att aldrig medvetet störa andra stationers
  radiotrafik.
  Detta är en moralisk regel.
  Det är omoraliskt att inte följa denna regel, likvärdigt med att fuska i en
  tävling.
\item[Praktiska regler] för att hantera alla olika aspekter av
  vårt uppförande behövs utöver etik också en uppsättning regler baserade på
  \textbf{trafikmässiga hänsyn} och på \textbf{praxis och sedvänja}.
  För att undvika konflikter behöver vi också praktiska regler som styr
  vårt beteende på amatörbanden eftersom vårt huvudintresse är att köra
  radio på de olika banden.
  Vi avser här mycket \textbf{praktiska regler} och \textbf{riktlinjer} för
  situationer som ej är etikrelaterade.
  De flesta trafikmetoder (hur genomföra ett QSO, var får man köra,
  vad betyder QRZ, hur använda Q-koderna) hör hit.
  Respekt för dessa trafikmetoder säkerställer optimalt resultat och
  effektivitet i våra QSO och kommer att vara nyckeln till att undvika
  konflikter.
  Dessa trafikmetoder har tillkommit som ett resultat av daglig radiotrafik
  under många år och som ett resultat av den pågående tekniska utvecklingen.
\end{description}

% Avsnitt 13.11 Bandplaner
% Avsnitt 13.12 Svenska bandplaner
\section{Bandplaner}
\label{bandplaner}

\subsection{Introduktion}

Det allra vanligaste är att en radiostation eller ett nät av stationer tilldelas
en eller ett fåtal frekvenser samt väl preciserade villkor i övrigt.
Amatörradio är däremot en radiotjänst, som tilldelas inte bara enstaka
frekvenser utan hela frekvensband samt inom dessa band förhållandevis stor
frihet till personligt val av frekvens, sändningsslag etc.
Därvid kan den enskilde radioamatören inte ställa anspråk på ostörda frekvenser.
I stället är det upp till radioamatörerna, att själva samråda och rekommendera
varandra om hur de tilldelade frekvensbanden bör fördelas på olika slags
användning.
Denna fördelning av trafiken kallas bandplan.

Frekvensbanden och effekt-gränser sätts av varje lands ansvarig myndighet,
kallad administration, och i Sverige är det Post- och telestyrelsen (PTS) som
meddelar i \emph{Post- och telestyrelsens föreskrifter om undantag från
tillståndsplikt för användning av vissa radiosändare} PTSFS 2022:19
\cite{PTSFS2022:19}.
För att koordinera frekvensanvändningen samarbetar olika länders ansvariga
myndigheter i Internationella Teleunionen (ITU) för att ha en gemensam
utgångspunkt för sina beslut, som publiceras som
ITU Radioreglemente \cite{ITU-RR}.

ITU:s Radioreglemente är dock inte ett tvingande dokument för
administrationerna, och därmed gäller det inte per automatik över
nationell lag, utan man behöver alltid hålla sig bekant med vad gällande lag
och föreskrifter säger.

För att radioamatörer sedan ska använda banden på ett liknande sätt, så har
sedan IARU skrivit ihop en bandplan, som enbart kan tolkas som en rekommendation
inom den ram som nationella lagar och föreskrifter sätter på radiosändning.

Det är viktigt att förstå dessa samband rätt, för det förekommer tyvärr att
betydelsen av dokument övertolkas, eftersom det kan resultera i att man sänder
på frekvenser som inte är tilldelat amatörtjänsten, eller sända med högre
effekt än tillåtet.

\subsection{IARU:s bandplaner, syfte och ändamål}
\harecsection{\harec{b}{6.1}{6.1}, \harec{b}{6.2}{6.2}}
\index{Internationella Amatörradiounionen (IARU)}
\index{IARU}

Internationella Amatörradiounionen (IARU) är det organ på internationell
nivå, där samråd om amatörradions intressen sker regelbundet, dels i
arbetsgrupper med olika inriktning och dels i generella konferenser.
IARU har som syfte att

\begin{itemize}
  \item verka för att av ITU tilldelade frekvensband för amatörradio bevaras
  \item förbättra amatör- och amatörsatellittjänsternas status inom tilldelade
  frekvensband
  \item verka för tilldelning av ytterligare frekvensband för amatörradio
  \item frekvensplanera amatörradiotrafiken inom tilldelade amatörradioband
  genom samråd och rekommendationer.
\end{itemize}

Syftet med en bandplan är att ge utrymme för alla aspekter inom amatörradio:
självträning, kommunikation och tekniska undersökningar.

Radioamatörernas bandplaner siktar på att ge möjlighet till så många olika
amatöraktiviteter som möjligt, såväl sändningsslag som tekniker, både nu och i
framtiden.
För att utnyttja banden på bästa sätt är det normalt att minsta möjliga
bandbredd samt optimal sändarutrustning och teknik används.

För att alla ska kunna utöva amatörradio med ett minimum av störningar,
förutsätts att man använder utrustningar som är ''state of the art''.
God insikt i frekvensplanering, tillräckliga resurser, gott anseende samt
internationellt samarbete behövs för att främja amatörradion.
De flesta nationella amatörradioorganisationer har sedan många år ett
världsomfattande samarbete genom sitt organ International Amateur Radio Union
(IARU) som är organiserat som tre regioner.
Dessa regioner sammanfaller geografiskt med ITU:s regioner.
Region~1 omfattar Afrika, Europa och västra Asien.

% \newpage % layout
\section{Svenska bandplaner}

Tilldelningen av frekvensband för amatörradioanvändning sker enligt
överenskommelser mellan telemyndigheterna i de länder som är anslutna till ITU.
Tilldelningen är därvid i stort sett lika i de flesta länder.
Av olika skäl förekommer dock skillnader såväl mellan ITU-regioner som länder.

I Sverige regleras amatörradioanvändningen främst genom Post- och
telestyrelsens (PTS) föreskrifter PTSFS 2022:19 \cite{PTSFS2022:19} samt genom lag
om elektronisk kommunikation \textbf{LEK} SFS 2022:482 \cite{SFS2022:482}.
I anslutning till frekvenstilldelningen anges tillåten effekt och
amatörradiostatus i respektive band.
Inom denna ram är det upp till radioamatörerna själva att utnyttja sina
möjligheter bästa sätt.
Bandplaner fungerar som radioamatörernas rekommendationer till varandra.
Endast i minsta utsträckning medverkar PTS till reglering inom dessa planer.
IARU Region~1 bandplanerna finns i bilaga \ssaref{IARU bandplan} och svensk
frekvensplan finns i bilaga \ssaref{svensk frekvensplan}.

% \newpage % layout
Föreningen Sveriges Sändareamatörer SSA -- företräder de svenska
radioamatörerna i IARU Region~1.

Mer information om de rådande bandplanerna för HF, VHF och UHF kan du finna i
appendix~\ssaref{bandplaner2} men kontrollera alltid mot en officiell källa innan
du använder dem för sändning.

%
%
% Kapitel 14 Bestämmelser
\chapter{Bestämmelser}

Tekniskt sett kan radioamatörerna världen över, med hjälp av sina
radiostationer, tämligen lätt skapa kontakt med varandra.
Därvid krävs att reglerna i de länder som berörs vid kontakten respekteras.

En hel serie både internationella och nationella regler styr
radiokommunikationerna i en nation.
Varje radioamatör ska känna till och följa dessa regler så långt de har
anslutning till amatörradio.
Vissa länder -- till exempel CEPT-länderna -- har i någon utsträckning
harmoniserat sina bestämmelser inbördes.
Nationella avvikelser förekommer likväl och reglerna i det land, som man gör
radiosändningar ifrån, ska alltid följas.

% Avsnitt 14.1 ITU RR
\section{ITU Radioreglemente (RR)}
\label{ITU radioreglemente}
\index{Internationella Teleunionen (ITU)}
\index{ITU}
\index{ITU RR}
\index{ITU Radioreglemente (ITU RR)}
\index{RR}
\index{radioreglemente}
\index{PTS}

Internationella Teleunionen (ITU) är det internationella samarbetsorgan där
olika länders myndigheter (administrationer) för telekommunikation samarbetar
och koordinerar sig, bland annat genom gemensamt regelverk och standarder.
Det är viktigt för att koordinera användning av spektrum och signalerna i det.

ITU Radioreglemente (RR) \cite{ITU-RR} är det övergripande regelverket för att
koordinera spektrumanvändning, det vill säga för alla former av
radiorelaterad verksamhet.
Det är det gemensamma ramverk som används, och varje land utgår från det för att
sedan skriva de nationella föreskrifterna och tilldelningarna.
Dock, den ansvariga myndigheten behöver inte strikt följa ITU RR och det
förekommer flera fall där saker i ITU RR ser tillåtna ut men de nationella
föreskrifterna inte tillåter samma sak.
Man ska därför inte tolka ITU RR som gällande istället för de nationella
föreskrifterna utan snarare en utgångspunkt.
Det kan ha skett förändringar i ITU RR men där existerande frekvensallokeringar
nationellt förhindrar möjligheten att följa ITU RR.
Omvänt så är det ofta svårt för de nationella föreskrifterna att gå utanför
ITU RR eftersom det kan kräva svåra förhandlingar, och därför försöker man ofta
få till i förändringen av ITU RR istället.

I Sverige är det Post- och telestyrelsen (PTS) som är ansvarig för
administration gällande telekommunikation och spektrum Det är deras
föreskrifter som reglerar all radio inklusive amatörtjänsten.
Där ITU RR nämner begreppet ''administration'' så avses för Sveriges del PTS.

Som del av ITU RR definieras ''Amateur services'' \cite[Article 25]{ITU-RR}.
Amatör- och Amatörsatellittjänsterna är radiokommunikationstjänster
med syfte att tillhandahålla nödvändig kommunikation i händelse av
naturkatastrofer, träna operatörer och tekniker i radio- och
telekommunikationsteknik till ingen kostnad för stat och samhälle,
bidra till att tidsenlig radiokommunikation främjas och att förbättra
internationell förståelse och välvilja.

\subsection{Artikel 1 (RR) Termer och definitioner}
\harecsection{\harec{c}{1.1}{1.1}, \harec{c}{1.2}{1.2}}
\index{amatörtjänst}
\index{amatörsatellittjänst}
\index{amatörradiostation}
\index{radiostation}
\label{amatörradio definitioner}

1.56 (RR) \emph{Amatörtjänst} \cite[1.56]{ITU-RR}\\
En radiokommunikationstjänst avsedd för självutbildning, inbördes
kommunikation och tekniska undersökningar bedriven av amatörer, det
vill säga av behöriga personer intresserade av radioteknik,
endast av personligt intresse och utan ekonomiskt syfte.

1.57 (RR) \emph{Amatörsatellittjänst} \cite[1.57]{ITU-RR}\\
En radiokommunikationstjänst som använder rymdstationer på
jordsatelliter för samma ändamål som för \emph{Amatörradiotjänsten}.

1.96 (RR) \emph{Amatörradiostation} \cite[1.96]{ITU-RR}\\
Radiostation inom \emph{amatörradiotjänst}.

\subsection{Artikel 25 (RR) Amateur services}
\harecsection{\harec{c}{1.3}{1.3}, \harec{c}{1.4}{1.4}}

\subsubsection{Sektion I. Amatörtjänst}
25.1 \S1 Radiokommunikation mellan amatörstationer i olika länder
skall vara tillåten, om inte administrationen i en av de berörda
nationerna har meddelat att den är emot sådan radiokommunikation.
\cite[25.1]{ITU-RR}

25.2 \S2 1) Sändning mellan amatörstationer i olika länder skall vara
begränsad till spontan kommunikation med syftet att nyttja amatörtjänsten,
som definierad i 1.56, och av personlig karaktär.
\cite[25.2]{ITU-RR}

25.2A \S2 1A) Sändning mellan amatörstationer i olika länder skall
inte vara kodad med syfte att dölja dess mening, annat än för kontrollsignaler
utbytta mellan jordstation och satellitstation i amatörradiotjänst.
\cite[25.2A]{ITU-RR}

25.3 \S2 2) Amatörradiostationer får användas för internationell
radiokommunikation för tredje parts räkning enbart vid nöd eller
krishantering.
\cite[25.3]{ITU-RR}

25.5 \S3 1) Administrationerna avgör huruvida en person som söker licens
att använda en amatörstation skall bevisa sin förmåga att sända och ta
emot text i morsesignaler.
\cite[25.5]{ITU-RR}

25.6 \S3 2) Administrationerna skall kontrollera de handhavandemässiga och
tekniska kvalifikationerna hos varje person som önskar använda en
amatörradiostation. En guide för den kompetens som krävs kan man finna i
senaste upplagan av ITU-R rekommendation M.1544.
\cite[25.6]{ITU-RR}

25.7 \S4 Den högsta effekten från en amatörstation skall fastställas
av berörda administrationer.
\cite[25.7]{ITU-RR}

25.8 \S5 1) Alla allmänna regler i överenskommelsen och de i denna
artikel skall tillämpas på amatörradiostationer.
\cite[25.8]{ITU-RR}

25.9 \S5 2) Under loppet av sändningarna skall amatörstationer sända
sina anropssignaler med korta mellanrum.
\cite[25.9]{ITU-RR}

25.9A \S5A Administrationer uppmuntras att vidta nödvändiga steg för att
tillåta amatörstationer att förbereda sig för och möta kommunikationsbehov
vid katastroftillstånd.
\cite[25.9A]{ITU-RR}

25.9B \S5B En administration kan avgöra huruvida en person som har tillstånd
att använda en amatörstation hos en annan administration kan tillåtas använda
en amatörstation medan denna person befinner sig på tillfälligt besök landet,
samt vilka villkor och begränsningar de väljer att ange.
\cite[25.9B]{ITU-RR}

\subsection{Sektion II. Amatörsatellittjänst}

25.10 \S6 Bestämmelserna i Sektion 1 i denna artikel skall gälla i all
tillämplig omfattning även för amatörsatellittjänst.
\cite[25.10]{ITU-RR}

25.10 \S7 Administrationer som godkänner rymdstationer i amatörsatellittjänst
ska tillse att tillfredsställande jordkontrollstationer upprättas före
uppskjutningen för att säkerställa att varje rapporterad skadlig störning
skall kunna avbrytas omedelbart av den bemyndigande administrationen.
Se 22.1. \cite[25.11]{ITU-RR}\footnote{22 behandlar ''Space Services''}

\subsection{Artikel 5 Frekvenstilldelning}

\subsubsection{Inledning}

5.1 I Unionens alla dokument där termerna \emph{allocation},
\emph{allotment} och \emph{assignment} används skall de ha den
betydelse som ges i 1.16 till 1.18, varvid termerna på de tre
arbetsspråken skall vara som följer (franska, engelska och spanska):
\cite[5.1]{ITU-RR}
Frekvensfördelning till:

\begin{center}
\begin{tabular}{lll}
  Tjänster & Allocation & (tilldelning) \\
  Områden & Allotment & (fördelning) \\
  Stationer & Assignment & (anvisning) .... etc. \\
\end{tabular}
\end{center}

(För enkelhetens skull återges här endast betydelserna på engelska språket).

\subsubsection{Sektion I. Regioner och områden}
\harecsection{\harec{c}{1.5}{1.5}}
\label{regioner}

5.2 För tilldelning av frekvenser har världen delats in i tre
Regioner så som visas på följande karta och som beskrivs i 5.3 till
5.9 ... etc.
\cite[5.2]{ITU-RR}

\emph{ Det innebär att tilldelning, fördelning och anvisning av frekvenser
  mycket väl kan skilja mellan ITU-regionerna.
  Skillnaderna förklaras till exempel av regionalt olika behovsstruktur, befolkning
  etc.}

\emph{Det förekommer också likheter.
  På nedanstående karta har markerats en tropisk zon, vilket förklaras av den
  annorlunda vågutbredningen där.
  Till exempel behöver särskild hänsyn tas vid frekvenstilldelning (allokering) till
  rundradiotjänsten i zonen.}

\mediumfig[0.9]{images/ITU_Regions-Map.png}{ITU Regionkarta (ur RRB-2)}{fig:bildIII2-1}

% Avsnitt 14.2 CEPT
\section{CEPT}
\label{sec:CEPT}

\subsection{Begreppet CEPT}

Vid sidan av folkrättsligt bindande avtal såsom den internationella
telekonventionen (ITC) -- har det internationella samarbetet lett till
överenskommelser som inte är tvingande.
Sådana avtal görs bland annat inom \emph{CEPT}.

\emph{CEPT} betyder \emph{Conf\'erence Europ\'eenne des administrations des
postes et t\'el\'ecommunications}, det vill säga Europeiska konferensen för
post- och teleadministrationerna.
''Konferens'' är att förstå som ett ständigt arbetande samarbetsorgan.

Arbetet inom CEPT har huvudsakligen karaktär av ömsesidiga programförklaringar
mellan länder.
Trots att dessa viljeförklaringar eller rekommendationer inte är bindande har de
visat sig värdefulla för utvecklingen av det internationella samarbetet.

\subsection{CEPT-rekommendationerna}

Länder anslutna till CEPT förenklar numera handläggningen av
tillståndsärenden om amatörradio genom att ömsesidigt bekräfta och
inom sitt land tillämpa rekommendationer som länderna utformat i
samråd.
Det innebär att svenska amatörradiobestämmelser kan harmoniseras till andra
länders.
För kompetenskrav vid examinering av radioamatörer finns CEPT-rekommendationen
T/R~61-02 \cite{TR6102}.

\subsubsection{CEPT-rekommendation T/R 61-01}
\harecsection{\harec{c}{2.1}{2.1}, \harec{c}{2.2}{2.2}, \harec{c}{2.3}{2.3}}

Rekommendationen T/R 61-01 \cite{TR6101} möjliggör för radioamatörer från
CEPT-länderna att utöva amatörradio under korta besök i andra CEPT-länder, utan
att behöva ett tillfälligt tillstånd från det besökta CEPT-landet.
Erfarenheterna av detta system är goda.

\subsubsection{CEPT-rekommendation T/R 61-02}

Rekommendationen T/R 61-02 \cite{TR6102} innebär att administrationerna i
CEPT-länder utger ömsesidigt erkända amatörradiocertifikat (Harmonised Amateur
Radio Examination Certificate -- HAREC) till de personer som vid nationella
prov uppfyller rekommendationens kunskapskrav.

Radioamatörer med ett CEPT-certifikat (HAREC) får utöva amatörradio i annat
land som accepterat T/R 61-01 och får tilldelas ett tillstånd av det landet
utan att behöva genomgå ytterligare kunskapsprov.

Det svenska amatörradiocertifikatet motsvarar kraven för HAREC och Sverige
tillämpar T/R 61-01 och T/R 61-02.

% Avsnitt 14.3 Svensk lag och föreskrift
\section{Svensk lag och föreskrift}
\label{svensk lag och föreskrift}

Lagar, föreskrifter och anvisningar tillämpas för
  amatörradioanvändning.
Märk, att ändringar kan förekomma.
\textbf{Använd därför aktuella versioner!}


\subsection{Lag om elektronisk kommunikation}
\harecsection{\harec{c}{3.1}{3.1}}
\index{LEK}

\emph{Lag (2022:482) om elektronisk kommunikation}~\cite{SFS2022:482} reglerar
all radiokommunikation Sverige.
Tillstånd behövs för all radiosändning som inte är undantagen tillståndsplikt.
Lagen förkortas ofta LEK.

\emph{Post- och telestyrelsen} (PTS) är enligt förordning (2022:511) om
elektronisk kommunikation den svenska myndighet som handlägger ärenden gällande
telekommunikation.
PTS ska bland annat svara för att möjligheterna till radiokommunikationer
utnyttjas effektivt och har därvid att beakta den internationella regleringen
inom området.
Regleringen av amatörradioanvändningen begränsas nu till den minsta omfattning
som följer av internationella avtal och europeiska rekommendationer,
CEPT-rekommendationer.

\newpage %layout
\subsection{Post- och telestyrelsens föreskrifter om undantag från tillståndsplikt för användning av vissa radiosändare}
\harecsection{\harec{c}{3.2}{3.2}}
\index{amatörradiocertifikat}
\index{amatörradiosändare}
\index{amatörradiotrafik}
\index{antennvinst}
\index{antennförstärkning}
\index{ERP}
\index{effektivt utstrålad effekt (ERP)}
\index{antenn!effektivt utstrålad effekt (ERP)}
\index{Effective Radiated Power (ERP)}
\index{antenn!Effective Radiated Power (ERP)}
\index{ERP}
\index{antenn!ERP}
\index{ekvivalent isotropiskt utstrålad effekt (EIRP)}
\index{antenn!ekvivalent isotropiskt utstrålad effekt (EIRP)}
\index{Equivalent Isotropically Radiated Power (EIRP)}
\index{antenn!Equivalent Isotropically Radiated Power (EIRP)}
\index{EIRP}
\index{antenn!EIRP}
\index{PEP}
\label{PTSFS2022:19}
\index{T/R 61-02}
\index{PTSFS 2022:19}

Post- och telestyrelsen föreskriver i PTSFS~2022:19~\cite{PTSFS2022:19} med stöd
av 5~\S{} Förordningen (2022:511)~\cite{SFS2022:511} om elektronisk kommunikation
att användningen av amatörradiosändare är undantagen tillståndsplikt.
Notera att PTS med viss regelbundenhet uppdaterar undantagsföreskrifterna,
och därför bör man kontrollera på PTS webbplats vad som är den senaste versionen
och använda den när den trätt i kraft.

I undantagsföreskriften~\cite{PTSFS2022:19} finns följande definitioner som är
relevanta för amatörradiotjänsten:

% \newpage % layout
\begin{description}
\item[amatörradiocertifikat] kunskapsbevis utfärdat eller godkänt av
Post- och telestyrelsen, som utvisar att godkänt kunskapsprov avlagts.

\item[amatörradiosändare] radiosändare som är avsedd att användas av personer
som har amatörradiocertifikat, för sändning på frekvenser som är avsedda för
amatörradiotrafik.

\item[amatörradiotrafik] icke yrkesmässig radiotrafik för övning,
kommunikation och tekniska undersökningar, bedriven i personligt radiotekniskt
intresse och utan vinstsyfte.

\item[antennvinst] förstärkning i förhållande till en referensantenn som
antingen är isotropisk eller en dipol och som mäts i dBi eller dBd.
Antennvinsten anger hur bra riktverkan en antenn har.

\item[\eirp] equivalent isotropically radiated power (ekvivalent
isotropiskt utstrålad effekt).

\item[\erp] effective radiated power (effektivt utstrålad effekt relativt en
halvvågsdipol).

\item[\pep] peak envelope power.
\end{description}
%%
Vidare anges ytterligare villkor i 3~kap. 26~\S{} av undantagsföreskriften
\cite{PTSFS2022:19}:
%%

De tekniska egenskaperna hos amatörradiosändaren ska anpassas så att de inte
stör användningen av andra radioanläggningar.
Den som använder en amatörradiosändare ska ha ett amatörradiocertifikat.
För att få ett amatörradiocertifikat krävs kunskaper i enlighet med Annex 6 i
CEPT Rekommendation T/R~61-02~\cite{TR6102}.

%%
Undantag från kravet på amatörradiocertifikat gäller för den som under en
tidsbegränsad period utbildar sig för att få ett sådant certifikat och för den
som under en förevisning tillfälligt använder amatörradiosändare, under
förutsättning att användningen av radiosändaren sker under uppsikt av en
innehavare av amatörradiocertifikat.
(Läs mer om användningen i avsnitt~\ssaref{secondoperator})

%\newpage % layout
Den som innehar amatörradiocertifikat ska ha en egen anropssignal.
Denna framgår av certifikatet, eller tidigare av amatörradiotillståndet.
Mottagare- och sändarestationens anropssignaler ska sändas i början och i
slutet av varje radioförbindelse.
Anropssignalerna ska också upprepas med korta mellanrum under pågående
radioförbindelse. Under de utbildnings- och förevisningstillfällen som anges i
stycket ovan ska anropssignal användas som tillhör den innehavare av
amatörradiocertifikat som har uppsikt över användningen av radiosändaren.
Vid dessa tillfällen får även anropssignal som tillhör den amatörradioförening
eller institution som anordnar utbildnings- eller förevisningstillfället
användas om företrädare för föreningen eller institutionen har uppsikt över
användningen av radiosändaren.

\newpage
Automatiska amatörradiosändare, till exempel en radiofyr, repeater eller
sändare för positionering ska alltid kunna identifieras genom att en
anropssignal regelbundet sänds med morsetelegrafi, röstmeddelande eller
på annat sätt.
Anropssignalen ska ange vem som är ansvarig för den automatiska sändaren.
Den som startar eller använder automatiska amatörradiosändare ska ha eget
amatörradiocertifikat och ska använda egen anropssignal.
Sådan start och användning får även utföras av den som inte har
amatörradiocertifikat, om det sker under uppsikt av en innehavare av
amatörradiocertifikat och dennes anropssignal används.

\subsection{Litteraturhänvisning om lagar och föreskrifter}

\begin{itemize}
\item CEPT rekommendation T/R~61-01~\cite{TR6101}
\item CEPT rekommendation T/R~61-02~\cite{TR6102}
\item Lag (2022:482) om elektronisk kommunikation~\cite{SFS2022:482}
\item Förordning (2022:511) om elektronisk kommunikation~\cite{SFS2022:511}
\item Post- och telestyrelsens föreskrifter om undantag från tillståndsplikt för
användning av vissa radiosändare PTSFS 2022:19~\cite{PTSFS2022:19}
\end{itemize}

%
%
% Kapitel 15 Att skriva loggbok
% Avsnitt 15.1 Ändamål
% Avsnitt 15.2 Kunna visa hur man för en loggbok
% Avsnitt 15.3 Föra in data
% Avsnitt 15.4 Rapportkoder
\chapter{Att skriva loggbok}

%\section{Loggbok}

\section{Ändamål}
\harecsection{\harec{c}{3.3.2}{3.3.2}}
\index{stationsdagbok}
\index{loggbok}

Dina radioförbindelser och övriga händelser med radiostationen bör antecknas i
en \emph{stationsdagbok} även känd som \emph{loggbok}.
Tidigare fanns myndighetskrav på att föra loggbok, men det finns inte numer.

Amatörradioverksamheten bygger på förtroende och då är det viktigt att själv
kunna dokumentera sin verksamhet till exempel i störningssituationer med mera.
Loggen används också för att kunna visa när man har varit aktiv.

Helt i eget intresse är det ju också trevligt med en loggbok.
Tänk bara på hur bra det är att ha alla underlag för tävlingar och diplom med
mera dokumenterade.

\section{Kunna visa hur man för en loggbok}
\harecsection{\harec{c}{3.3.1}{3.3.1}}

I tabell~\ssaref{tab:loggblad} visas ett exempel på hur en förenklad loggsida kan
se ut med ett par radioförbindelser (QSO) inskrivna.

Fundera på följande:
\begin{enumerate}
\item Halv tre på eftermiddagen den tionde oktober gör Arne (SM6XYZ)
  ett allmänt anrop på den lokala repeatern på 2-metersbandet.
  Eva (SM6ZYX) som är på väg hem från skolan svarar.
  Arne berättar att han precis har byggt sitt nya slutsteg på \qty{25}{\watt}
  färdigt och frågar Eva om det hörs någon skillnad när han kopplar ur det.
  Efter lite småprat om allt möjligt säger de 73 till varandra och då har det
  gått sju minuter sen de började.
  Fyll i loggboken åt Arne!
\item Gör ett låtsas-QSO med en kurskamrat.
  Bokstavera era ''anropssignaler''.
  För in i loggen.
\end{enumerate}

%% k7per, make it floating...
\begin{table*}[b]
  \qquad
  \begin{tabular}{|c|c|c|c|c|c|c|c|c|c|c|}
    \hline
    & band / & \multicolumn{2}{|c|}{tid-UTC} & anrops- & \multicolumn{2}{|c|}{RST} & &
                                                                                       \multicolumn{2}{|c|}{QSL} & \\
    datum & frekvens & start & slut & signal & sänt & fick & namn och QTH & s & m & anmärkning \\
    \hline
    \hline
    20171021 & 80 & 06:55 & 07:13 & SK0HQ & 59 & 59 & Anders & & & HQ-nätet \\
    \hline
    20171021 & 80 & 07:15 & 07:38 & SM0ZXY & 579 & 559 & Eva Sollentuna & & & \\
    \hline
    & & & & & & & & & & \\
    \hline
    & & & & & & & & & & \\
    \hline
  \end{tabular}
  \caption{Exempel på loggblad}
  \label{tab:loggblad}
\end{table*}

\newpage %layout
\section{Föra in data}
\harecsection{\harec{c}{3.3.3}{3.3.3}}

Det man skriver upp i loggen är
\begin{itemize}
  \item tiden i början och i slutet av förbindelsen (glöm inte datum)
  \item motstationens anropssignal
  \item din effekt (ineffekt, PEP eller utstrålad effekt)
  \item frekvensband, ev frekvens
  \item sändningsslag (FM, SSB, CW, paketradio etc)
  \item uppgift om varifrån man sände (eget QTH)
  \item signalrapporter (rapportkoder).
\end{itemize}

Allmänna uppgifter om motstationen, till exempel signalrapport, namn, QTH,
motpartens utrustning, QSL-adress och så vidare brukar också vara bra att ha
med.

Man bör också skriva upp när man har gjort allmänt anrop, sänt ut bärvåg för
prov, experiment och annat som kan vara av intresse.

Om någon annan radioamatör använder din station ska du också skriva upp hens
namn och anropssignal.

\section{Rapportkoder}
\index{rapportkoder}

Man blir ofta ombedd av motstationen att lämna en så kallad signalrapport på
dennes sändning.
Omvänt är det bra att få en signalrapport på den egna sändningen.

För rapportering mellan radioamatörer används RST-koden.

För lyssnarrapporter till exempel till rundradiostationer, förekommer ett
kodsystem, som kallas för SINPO eller SINPFEMO. Se bilaga \ssaref{Rapportkoder}.

%
%
