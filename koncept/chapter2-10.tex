\section{Värmeutveckling}

\harecsection{\harec{a}{2.7}{2.7}}

\index{värmeutveckling}
\index{heat dissipation}

\subsection{Värmeledning}

\harecsection{\harec{a}{2.7.1}{2.7.1}}

\index{värmeledning}
\index{heat transfer}
\index{termisk resistans}
\index{symbol!\(R_\Theta\) termisk resistans}
\index{omgivande temperatur}
\index{ambient temperature}
\index{symbol!\(T_A\) ambient temperature}
\index{elsäkerhet}
\index{kallödning}

Vi har tidigare betraktat Joules lag för effektutveckling i motstånd.
Det är dags att börja utveckla en lite mer komplett syn på värmeutveckling.
Ett motstånd som utvecklar 1~watt kommer stiga i temperatur till dess att
jämvikt uppstår mellan motståndets förmåga att avleda värme och
omgivningstemperaturen.

\emph{Termisk resistans} (eng. \emph{thermal resistance}) är ett mått på
hur bra ett material är på att leda värme. Den betecknas med symbolen \(R_\Theta\),
och anges i enheten kelvin per watt.
Temperaturen \(T_k\) för en komponent beror på medeleffekten \(P\) som den
producerar i värme, den termiska resistansen samt den \emph{omgivande temperaturen}
(eng. \emph{ambient temperature}) \(T_A\) enligt:
%%
\[T_k = T_A + R_\Theta \cdot P\]
%%
De termiska resistanserna för komponent, kylpasta, isolerskiva och kylfläns
kan summeras precis som resistanser för vanliga motstånd och det
sammanlagda värdet används sedan för att beräkna temperaturen på en
komponent eller för att dimensionera en kylfläns.

\subsection{Konvektion}
\harecsection{\harec{a}{2.7.2}{2.7.2}}
\index{konvektion}
\index{kylfläns}
\index{heatpipe}

\emph{Konvektion} (eng. \emph{convection}) är när värme skapar ett
naturligt flöde i vätska eller gas, oftast luft. När luft värms upp 
vill den expandera, varvid densiteten sjunker och luften vill stiga uppåt.
Kallare luft strömmar då till och kan därmed kyla värmekällan.
En stor temperaturskillnad medför att konvektionen ökar och innebär därmed en
bättre kylning.

För exempelvis transistorer kan värmealstringen ske på en sådan liten yta att
konvektion från komponenten inte räcker för att kyla bort den producerade
värmen.
Därför monterar man dem på en \emph{kylfläns} (eng. \emph{heat sink}) som
fördelar värmen över en större yta så att verkan av konvektion ökar.

En effektiv metod för att transportera värme är via en så kallad \emph{heat pipe}.
Det är ett rör innehållande en vätska som förångas vid en temperatur strax över
rumstemperatur och som då effektivt leder överskottsvärme till ett plats där den
kan kylas bort.
Heat pipe används numera ofta i datorer och solfångare.

Om värme produceras på en liten yta kan man behöva hjälpa konvektionen, vilket
ibland kallas för \emph{forcerad konvektion} (eng. \emph{forced convection}).
Med hjälp av en fläkt blåses luft mot eller sugs förbi kylflänsen vilket ökar
värmeutbytet.
Eftersom fläktar skapar oljud brukar man försöka anpassa fläktens varvtal i
förhållande till temperaturen, men även en variation av varvtal kan uppfattas
som störande.
Andra åtgärder för att minska ljudnivån är att skapa släta ytor för luften så
det inte bildas luftvirvlar eller att styra in- och utgående luftflöde med
bafflar.

Ett problem som kan uppstå är att utrustning som är gjord för självkonvektion
blir placerad eller monterad så att luft inte kan flöda fritt runt
utrustningen. Detta kan leda till överhettning på motsvarande sätt som när
en fläkt för forcerad kylning går sönder.
Dålig termisk kontakt mellan transistor och kylfläns är ett annat exempel på hur
dålig värmeledning skapar problem med överhettning.

\subsection{Värmealstring}
\harecsection{\harec{a}{2.7.4}{2.7.4}}
\index{värmealstring}

\emph{Värmealstring} kan ske på fler ställen än i motstånd. Lite förenklat
kan man säga att alla komponenter har förluster som producerar värme. Genom
lämpligt val av komponenter och korrekt dimensionering kan vi undvika att
producera onödiga värmeförluster. Kraftaggregat och effektsteg är exempel
på apparater där det går större strömmar vilka ofrånkomligen också alstrar
mera värme.
Lägre förluster skapar man genom att helt enkelt ha bättre ledningsförmåga,
lägre resistans.

Även halvledare skapar värme, och även här gäller Joules lag med spänning
gånger ström. I exempelvis ett effektsteg kommer transistorn utveckla en
effekt motsvarande spänningen över transistorn gånger strömmen genom den.
Onödigt hög spänning och ström skapar högre värmeutveckling, vilket är en
anledning till att man gärna undviker slutsteg som arbetar i klass A till fördel
för slutsteg som arbetar i klass AB, B eller C.

Bristande värmeavledning leder ofta till katastrofala fel, som till exempel
sönderbrända motstånd och transistorer. Även ledare kan brinna av när man
har för liten ledararea, och därmed för hög resistans för den ström som
ska gå genom den. Av det skälet finns dimensioneringsregler, till exempel
krav på minsta arean av koppar i ledare, helt enkelt för att det inte ska
uppstå brand.

En annan effekt av värmeledning är att det kan ibland vara svårt att löda
på kretskort, framför allt vid ledare som går mot stora kopparytor som
har en relativt god värmeledningsförmåga. Ibland konstruerar man små mönster
''thermals'' runt sådana lödpunkter för att minska värmeavledningen.
Ett effektivt sätt att kunna löda och framförallt löda av från sådana
kort är att man förvärmer hela kretskortet eller området runt om
lödpunkten. Då kommer temperaturskillnaden mellan lödpennans spets och
omgivningen att minska och det krävs inte lika stor effekt för att få
upp lödpunkten i rätt temperatur för att kunna genomföra lödningen med god
\emph{vätning} och därmed undvika att det bildas en kallödning.

\subsection{Värme i transistor}
\harecsection{\harec{a}{2.7.3}{2.7.3}}

\mediumtopfig{macros/bild_tx_heat.pdf}{Värmealstring i en transistor. Transistorspänning $U_{ce}$ och transistoreffekt $P_t$ varierar med vinkeln hos sinussignalen för resistiv last.}{fig:power1}

För att förstå värmealstring i en transistor börjar vi med att titta
på effektförbrukningen, \(P_t\), i en NPN-transistor som kan beskrivas
som

\[P_t \approx U_{be}\cdot I_b + U_{ce}\cdot I_c,\]
%
där \(U_{be}\) är spänningen från bas till emittor, \(I_b\)
strömmen genom bas,  \(U_{ce}\) spänningen från kollektor till
emittor, och \(I_c\) strömmen genom kollektor.
Oftast är strömmen genom basen försumbar.

För att gå lite djupare tänker vi oss ett exempel med en transistor i
enkel Klass A förstärkarkrets, se kapitel~\ssaref{klassabc}.
Transistorn har en \qty{12}{\volt} matningsspänning och har en
vilospänning på \qty{6}{\volt} för att få marginal mot \qty{0}{\volt}
och \qty{+12}{\volt}.
Den genererar en sinussignal med topp-till-topp-värdet \(10\ V_{pp}\)
in i en resistiv last.

Spänningen kollektor till emittor, \(U_{ce}\), och strömmen, \(I_c\),
är 180 grader ur fas så den resulterande effekten, \(P_t\), har sina
maxvärden mellan spänningens max- och minvärden, se bild
\ssaref{fig:power1}.
