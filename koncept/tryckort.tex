\onecolumn\newgeometry{left=4cm,right=4cm}
\pagenumbering{gobble}
\vspace{10em}
\title{KonCEPT för amatörradiocertifikat}
\begin{center}
\Large{KonCEPT FÖR AMATÖRRADIOCERTIFIKAT}

Föreningen Sveriges Sändareamatörer\\[2\baselineskip]
\end{center}

\noindent \textbf{Andra upplagan, sjunde tryckningen}\\
\noindent ISBN: 978-91-86368-23-4\\
\noindent Version \revision
\bigskip

\noindent Det här verket är licenserat under Creative Commons:\newline
\noindent Erkännande, Icke kommersiell, Dela lika\\
\noindent (CC BY-NC-SA) 4.0 Internationell.
\begin{figure}[h]
    \includegraphics[width=4cm]{images/cc-by-nc-sa}
\end{figure}

\vfill

\noindent Denna faktabok omfattar det av Post- och tele\-styrel\-sen anvisade
kompetensområdet för amatörradiocertifikat.

\bigskip

\noindent Så till vida innehåller boken ämnen såsom grundläggande ellära, elektronik, komponenter,
kretsar, radioteknik, elsäkerhet, regler, bestämmelser, bandplaner och tra\-fik\-metoder.
Det finns även inlärningsanvisningar för morsesignalering för den
som vill lära sig telegrafi.

\bigskip

\noindent I bilagorna finns bland annat grundläggande matematik
och frekvensplaner för ama\-törradiotrafik. 

\vfill

\noindent Tryckt i Sollentuna, 2023.

\bigskip

\noindent \textbf{Förlag}

\smallskip

\noindent\textit{Föreningen Sveriges Sändareamatörer (SSA)}

\smallskip\noindent Box 45, SE-191 21 Sollentuna

\smallskip\noindent Telefon \href{tel:+46709585702}{+46 70 958 57 02}

\smallskip\noindent E-post \href{mailto:hq@ssa.se}{hq@ssa.se}

\restoregeometry\twocolumn
\pagenumbering{arabic}
