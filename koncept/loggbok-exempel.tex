\section{Kunna visa hur man för en loggbok}
\harecsection{\harec{c}{3.3.1}{3.3.1}}

I tabell~\ssaref{tab:loggblad} visas ett exempel på hur en förenklad loggsida kan
se ut med ett par radioförbindelser (QSO) inskrivna.

Fundera på följande:
\begin{enumerate}
\item Halv tre på eftermiddagen den tionde oktober gör Arne (SM6XYZ)
  ett allmänt anrop på den lokala repeatern på 2-metersbandet.
  Eva (SM6ZYX) som är på väg hem från skolan svarar.
  Arne berättar att han precis har byggt sitt nya slutsteg på \qty{25}{\watt}
  färdigt och frågar Eva om det hörs någon skillnad när han kopplar ur det.
  Efter lite småprat om allt möjligt säger de 73 till varandra och då har det
  gått sju minuter sen de började.
  Fyll i loggboken åt Arne!
\item Gör ett låtsas-QSO med en kurskamrat.
  Bokstavera era ''anropssignaler''.
  För in i loggen.
\end{enumerate}

%% k7per, make it floating...
\begin{table*}[b]
  \qquad
  \begin{tabular}{|c|c|c|c|c|c|c|c|c|c|c|}
    \hline
    & band / & \multicolumn{2}{|c|}{tid-UTC} & anrops- & \multicolumn{2}{|c|}{RST} & &
                                                                                       \multicolumn{2}{|c|}{QSL} & \\
    datum & frekvens & start & slut & signal & sänt & fick & namn och QTH & s & m & anmärkning \\
    \hline
    \hline
    20171021 & 80 & 06:55 & 07:13 & SK0HQ & 59 & 59 & Anders & & & HQ-nätet \\
    \hline
    20171021 & 80 & 07:15 & 07:38 & SM0ZXY & 579 & 559 & Eva Sollentuna & & & \\
    \hline
    & & & & & & & & & & \\
    \hline
    & & & & & & & & & & \\
    \hline
  \end{tabular}
  \caption{Exempel på loggblad}
  \label{tab:loggblad}
\end{table*}
