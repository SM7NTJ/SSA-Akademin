\newpage %layout
\section{Föra in data}
\harecsection{\harec{c}{3.3.3}{3.3.3}}

Det man skriver upp i loggen är
\begin{itemize}
  \item tiden i början och i slutet av förbindelsen (glöm inte datum)
  \item motstationens anropssignal
  \item din effekt (ineffekt, PEP eller utstrålad effekt)
  \item frekvensband, ev frekvens
  \item sändningsslag (FM, SSB, CW, paketradio etc)
  \item uppgift om varifrån man sände (eget QTH)
  \item signalrapporter (rapportkoder).
\end{itemize}

Allmänna uppgifter om motstationen, till exempel signalrapport, namn, QTH,
motpartens utrustning, QSL-adress och så vidare brukar också vara bra att ha
med.

Man bör också skriva upp när man har gjort allmänt anrop, sänt ut bärvåg för
prov, experiment och annat som kan vara av intresse.

Om någon annan radioamatör använder din station ska du också skriva upp hens
namn och anropssignal.
