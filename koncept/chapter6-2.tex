\newpage
\smallfig{images/cropped_pdfs/bild_2_5-09.pdf}{Separat sändare och mottagare}{fig:bildII5-9}

\smallfig{images/cropped_pdfs/bild_2_5-10.pdf}{Transceiver med samma VFO}{fig:bildII5-10}

\section{Transceiver}
\index{transceiver}
\index{sändtagare|see {transceiver}}

En \emph{transceiver} -- transmitter receiver -- är både en sändare och
mottagare med delvis gemensamma funktioner.
Dessa kan till exempel vara oscillatorer, signalbehandlingskretsar, filter,
strömförsörjning och så vidare, vilket innebär besparing av ingående
komponenter, men också vissa funktionella begränsningar.

Transceiverkoncept är numera vad som används allra mest av radioamatörer.
Eftersom man på olika vis önskar sig så många sändar- och mottagarfunktioner
som möjligt inom samma skal, så kan det vara svårt att undvika kompromisser.
Så kan till exempel en specialiserad, separat mottagare ha bättre eller fler
egenskaper än i en transceiver.

\subsection{Jämförelse mellan stationskoncept}

Bild~\ssaref{fig:bildII5-9} visar i stort en station med skilda sändar- och
mottagarfunktioner, men att antennen är gemensam.
Bild~\ssaref{fig:bildII5-10} visar i stort en transceiver där VFO och antenn är
gemensamma, men i övrigt med skilda funktioner.
Bild~\ssaref{fig:bildII5-11} visar samma transceiver, men med ett mer detaljerat
blockschema.

\mediumfig{images/cropped_pdfs/bild_2_5-11.pdf}{Direktblandad transceiver med gemensam VFO}{fig:bildII5-11}

\subsection{Simplex}
\index{simplex}

En station som växelvis sänder och tar emot på en frekvens använder
trafikmetoden \emph{simplex}.
Detta är den vanligaste trafikmetoden på kortvåg.

\subsection{Halvduplex}
\index{halvduplex}
\index{split}

En station som växelvis sänder och tar emot på två skilda frekvenser använder
trafikmetoden \emph{halvduplex}.
Trafikmetoden används oftast vid trafik via repeater men även vid \emph{pile up}
på kortvåg, metoden kallas då \emph{split}.
Se vidare avsnitt \ssaref{cq dx och split}.

\subsection{Duplex}
\index{full duplex}
\index{duplex}
\index{duplexfilter}
\index{utsläckning}
\index{notch}
\label{duplex}

En stations sägs sända \emph{duplex} eller \emph{full duplex} när den kan
samtidigt sända och ta emot på två olika frekvenser.

Duplex-operation kräver i allmänhet stor isolation mellan sändare och mottagare,
något som ofta åstadkoms med kavitetsfilter kopplade mellan sändare och antenn
och mottagare och antenn.
Om gemensam antenn används, så kopplas dessa kavitetsfilter ihop till vad som
kallas \emph{duplexfilter}.

För en lyckad duplex-operation krävs i allmänhet mer än \qty{100}{\decibel}
isolation mellan sändare och mottagare.
Mottagarens kavitetsfilter trimmas så att det får en djup \emph{utsläckning}
(eng. \emph{notch}) vid sändarens frekvens, men med så lite förlust som möjligt
på mottagarens frekvens.
Sändarens kavitetsfilter trimmas så att det får en djup utsläckning/notch vid
mottagarens frekvens, för att på så sätt minimera att sändarens fasbrus höjer
brusgolvet för mottagaren, men med så liten förlust som möjligt på sändarens
frekvens.

\subsection{CW-transceiver med direktblandare}
\index{CW}
\index{transceiver!CW}
\index{direktblandare}
\index{Receiver Incremental Tuning (RIT)}
\index{RIT}
\index{keyed operated xmitter (KOX)}
\index{KOX}

Bild~\ssaref{fig:bildII5-11} visar en enkel transceiver för telegrafi.
Sändaren är en rak sändare och mottagaren arbetar med direktblandning.
För 1-kanaltrafik räcker det med en gemensam VFO för sändning och mottagning.
Om motstationen svarar exakt på sändningsfrekvensen, vilken ju är
VFO-frekvensen, så erhålls svävningsnoll i mottagaren.
För att få hörbara morsetecken är mottagaren utrustad med
\emph{Receiver Incremental Tuning (RIT)}, som ändrar VFO-frekvensen med
cirka \qty{800}{\hertz} vid mottagning.

I konstruktionen finns en anordning kallad \emph{Key Operated Xmitter (KOX)}.
Denna kopplar om transceivern till sändning när telegrafnyckeln trycks ner och
till mottagning igen efter en viss tid sedan nyckeln har släppts upp.
Telegrafnyckeln styr också en tongenerator som ljuder i takt med de sända
morsetecknen, så kallad medhörning.

Denna transceiver är utförd för endast ett frekvensband och i övrigt
mycket enkel.

\subsection{Kristallstyrd FM-transceiver för VHF}
\index{frekvensmodulation}
\index{transceiver!FM}
\index{dubbelsuperheterodyn}

Bild~\ssaref{fig:bildII5-12} visar en kristallstyrd FM-sändare med
frekvensomkopplare för kanalval inom \SIrange{144}{146}{\mega\hertz}-bandet.

En kristallfrekvens av cirka \qty{12}{\mega\hertz} multipliceras 12 gånger i en
kedja av förstärkarsteg för att ge sändningsfrekvensen.
Bilden visar räkneexempel för två frekvenskanaler.
Det frekvenssving i oscillatorn, som alstras av modulatorn,
multipliceras också med 12.
För ett sving av \qty{3}{\kilo\hertz} på bärvågen är svinget på oscillatorn bara
\qty{250}{\hertz}.

Efter mikrofonförstärkaren följer en amplitudbegränsare, som ska
hålla deviationen inom ett givet maxvärde, oavsett signalstyrkan från
mikrofonen.
Därefter följer ett lågpassfilter, som dels dämpar de övertoner som
uppstår vid amplitudbegränsningen och dels begränsar de höga frekvenserna
i den modulerade signalen.
Båda åtgärderna begränsar bandbredden.

Mottagaren är en \emph{dubbelsuperheterodyn}, ofta kallad för dubbelsuper.
Den mottagna signalen passerar genom ett förselektionsfilter och en
HF-förstärkare för att i 1:a blandaren blandas med en lokal signal.

\mediumfig{images/cropped_pdfs/bild_2_5-12.pdf}{Kristallstyrd 6-kanals FM-transceiver för VHF}{fig:bildII5-12}

En kristallstyrd lokaloscillator med efterföljande
frekvensmultipliceringssteg alstrar denna signal.

Lokaloscillatorkedjans utfrekvens läggs \qty{10,7}{\mega\hertz} över eller under
mottagningsfrekvensen och mellanfrekvensen efter den 1:a blandningen blir då
\qty{10,7}{\mega\hertz}.
Skilda oscillatorer används vid sändning respektive mottagning varför
styrkristallerna för sändning respektive mottagning på en given kanal får
olika frekvens.
Vid omkoppling till en annan kanal väljs ett annat kristallpar, vilket
lämpligen sker med samma omkopplare.

Den relativt höga 1:a mellanfrekvensen \qty{10,7}{\mega\hertz} ger ett så stort
avstånd till spegelfrekvensen, att bandbredden i förselektionsfiltren
är tillräckligt smal för att undertrycka spegelfrekvensen.
Av samma skäl bör 1:a mellanfrekvensen i en UHF-mottagare väljas ytterligare
tre gånger högre.
Den relativt låga 2:a mellanfrekvensen medger en god närselektering
redan med enkla bandfilter.
En eventuell MF-förstärkare ger tillräcklig signalstyrka till FM-demodulatorn.

För denna lösning behövs det två styrkristaller för varje
frekvenskanal, vilket av kostnadsskäl kan vara en nackdel.

\subsection{PLL-styrd FM-transceiver för VHF}
\index{PLL}
\index{frekvensmodulation}
\index{transceiver!PLL}
\index{transceiver!FM}
\index{dubbelsuperheterodyn}
\index{simplex}

\mediumfig{images/cropped_pdfs/bild_2_5-13.pdf}{PLL-styrd FM-transceiver för VHF}{fig:bildII5-13}

Den PLL-styrda sändare som redan beskrivits i bild~\ssaref{fig:bildII5-7} har här
kompletterats med en svingbegränsare och ett lågpassfilter i modulatorn.
Liksom i den station med kanalkristaller, som beskrivits i
bild~\ssaref{fig:bildII5-13}, är mottagaren även i detta fall en dubbelsuper.

VCO används även som lokaloscillator i mottagaren.
Eftersom sändaren och mottagaren ska användas på samma frekvens
(simplextrafik), måste i detta koncept VCO-frekvensen vara olika vid
sändning och mottagning.
Eftersom mottagarens mellanfrekvens MF är \qty{10,7}{\mega\hertz} måste nämligen
VCO ligga \qty{10,7}{\mega\hertz} högre eller lägre vid mottagning än vid
sändning.
Vid sändning däremot, är VCO-frekvensen densamma som sändningsfrekvensen.

Den programmerbara frekvensdelaren i PLL-kret\-sen arbetar därför med
olika delningstal vid sändning respektive mottagning, se tabell~\ssaref{tab:delningstal}.
Inställningen av divisorn kan ske med kanalomkopplare, tumhjulssats,
knappsats eller ''VFO-ratt'' + digitalräknare och så vidare.
PLL-styrningen ger dessutom möjligheter, till exempel att ordna en automatisk
avsökning över ett önskat frekvensområde -- så kallad scanning.

\begin{table}[ht]
\begin{center}
  \begin{tabular}{ll|lll}
    \multicolumn{2}{l|}{Sändning} &
    \multicolumn{3}{l}{Mottagning} \\
    QRG & Deln.- & QRG & VCO & Deln.- \\
    MHZ & tal    & MHz & MHz & tal \\
    \hline
    \multicolumn{2}{p{7em}|}{Simplexkanaler, exempel} & & & \\
    144,000 & 5760 & 144,000 & 154,700 & 6188 \\
    144,025 & 5761 & 144,025 & 154,725 & 6189 \\
    & & & & \\
    \multicolumn{2}{p{7em}|}{Repeaterkanaler, exempel} & & & \\
    145,000 & 5800 & 145,600 & 156,300 & 6252 \\
    145,025 & 5801 & 145,625 & 156,325 & 6253 \\
  \end{tabular}
\end{center}
\caption{Exmpel på användning av olika delningstal vid simplex- och repeaterkanaler.}
\label{tab:delningstal}
\end{table}

VCO-frekvensen är lika vid sändning och mottagning medan delningstalet
bestämmer arbetsfrekvensen.

\newpage
\subsection{Kortvågstransceiver för SSB och CW}
\index{SSB}
\index{transceiver!SSB}
\index{CW}
\index{transceiver!CW}
\index{talstyrd sändning}
\index{Voice Operated Xmitter (VOX)}
\index{VOX}

\mediumfig{images/cropped_pdfs/bild_2_5-14.pdf}{SSB-transceiver för kortvåg}{fig:bildII5-14}

Vi har redan beskrivit en KV-sändare och KV-mot\-tag\-are för SSB.
I det koncept på en kortvågstransceiver, som visas här i
bild~\ssaref{fig:bildII5-14}, ingår en super-VFO i signalberedningen.
VFO-signalen (\SIrange{5}{5,5}{\mega\hertz}) blandas med signalen från en
kristallstyrd CO, vars frekvens är valbar med en bandomkopplare.
Samtidigt kopplas ett bandpassfilter in efter blandaren i super-VFO, som svarar
till det aktuella frekvensbandet.

Till exempel i \qty{21}{\mega\hertz}-bandet är VFO-filtrets passband
\SIrange{12}{12,5}{\mega\hertz}.
När en VFO-signal \SIrange{12}{12,5}{\mega\hertz} blandas med en
\qty{9}{\mega\hertz} SSB modulerad signal erhålls en frekvens i området
\SIrange{3}{3,5}{\mega\hertz} och en frekvens i området
\SIrange{21}{21,5}{\mega\hertz}.
Den önskade av dessa frekvenser filtreras fram med omkopplingsbara
bandpassfilter, vilket sker med den bandkopplare som nämnts tidigare.

I den enkla kortvågssändare som beskrivits tidigare är det
tillräckligt med en enda sats av omkopplingsbara bandpassfilter.
Det större antalet filter i den här beskrivna utrustningen behövs för att
även kunna använda super-VFO som en del i mottagaren, vilken arbetar
som enkelsuper.
Eftersom en MF på \qty{9}{\mega\hertz} används även i mottagaren kommer
mottagning och sändning att kunna ske på samma frekvens.

Mottagaren beskrivs inte närmare.
Med lämpliga omkopplingsanordningar kan vissa funktionsblock i
transceivern användas både vid mottagning och sändning.
Bild~\ssaref{fig:bildII5-14} visar en SSB-transceiver där passbandfilter i
förkretsar, MF-filter och kristalloscillatorer har dubbel användning.
Funktionsblocken visas inplacerade i sina alternativa
funktioner, däremot inte omkopplingsanordningarna.

Vid sändning och mottagning av CW förbikopplas den balanserade modulatorn och
kristallfiltret i signalbehandlingskretsarna för \qty{9}{\mega\hertz}.
För mottagning av CW ändras BFO-frekvensen i mottagaren så att
det hörs en svävningston när en bärvåg tas emot.
Utan denna frekvensändring skulle endast bärvågsbruset höras.

Även en RIT och en \emph{talstyrd sändning} (eng.
\emph{Voice Operated Xmitter (VOX)}) är inritade.


\newpage
\mediumplustopfig{images/cropped_pdfs/bild_2_5-15.pdf}{PLL-styrd SSB-transceiver för kortvåg}{fig:bildII5-15}

\subsection{PLL-styrd kortvågstransceiver}
\index{PLL}
\index{transceiver!PLL}

En modern transceiver i den högre prisklassen finns i bild~\ssaref{fig:bildII5-15},
i så kallad ''all-mode''-utförande, erbjuder många funktionella möjligheter.
Flera av dem kommer emellertid endast till användning i speciella situationer.
Konceptet för en sådan transceiver beskrivs här i stort.
Huvudprincipen för signalbehandlingen kan beskrivas som en PLL-styrd
dubbelsuper.
SSB-signalen bereds på \qty{9}{\mega\hertz}-nivån och flyttas därefter upp till
\qty{70}{\mega\hertz}-nivån genom frekvensblandning och filtrering.
De möjliga sändningsfrekvenserna mellan 0,5 och \qty{30}{\mega\hertz} skapas
genom att blanda den fasta SSB-signalen med en variabel frekvens från VCO.
Den steglösa frekvenstäckningen som innefattar mellanvågs- och kortvågsområdet
är emellertid endast avsedd för mottagningsfunktionen i transceivern.
För sändningsfunktionen kan tillkomma blockeringskretsar, som förhindrar
sändning utanför tillåtna frekvensband.

Denna förenklade beskrivning omfattar inte kristalloscillatorerna för 9 och
\qty{61}{\mega\hertz} i fasregleringskretsen och inte heller SSB-modulatorn,
FM-modulatorn och anordningarna för CW-sändning.

Mottagaren är en dubbelsuper med hög 1:a MF-frekvens.
Mottagare för höga frekvenser kan till och med utföras som en trippelsuper.
Samma bandpassfilter, blandare och kristallfilter används både vid sändning
och mottagning.

Genom lämplig programmering av frekvensdelaren kan sändning och
mottagning ske på samma frekvens eller på skilda frekvenser
(split-trafik).

En extra VFO-funktion kan åstadkommas genom att frekvensdelaren
programmeras med delningstal som hämtas från ett digitalt minne.
Den extra VFO-funktionen kan sedan efterjusteras genom att ändra
delningstalet med frekvensratten.
Minnet blir ännu mer användbart, om det förutom frekvenser också kan lagra
uppgifter till exempel om sändningsslag och andra inställningar.

\newpage % layout
\subsection{Sammanfattning}

Till skillnad från den raka sändaren är den här beskrivna PLL-styrda
transceivern mycket komplicerad.
Den tekniska utvecklingen går fort.
Nya, bättre och mer invecklade apparater utvecklas ständigt.
Men det är inte alls nödvändigt att använda det senaste och mest
avancerade inom apparattekniken för att utöva amatörradio.
Det går mycket bra att börja med enkla medel och med liten ekonomisk insats.

% \newpage % layout 
Det finns ett stort utbud av begagnade apparater som i olika avseenden
är konkurrenskraftiga med senare konstruktioner.
Det ligger i amatörradions traditioner att ta tillvara tillgänglig
utrustning och förbättra denna efter bästa förmåga.

Ytterst beror resultatet och framgången mest på radiooperatörens
skicklighet, val av frekvens, antenn och tillfälle.
