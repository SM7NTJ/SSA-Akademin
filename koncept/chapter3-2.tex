\section{Filter}
\harecsection{\harec{a}{3.2}{3.2}, \harec{a}{3.2.9}{3.2.9}}
\index{filter}
\index{filter!frekvensfilter}
\index{frekvensgång}
\index{filter!frekvensgång}
\index{frequency response}
\label{filter}

Frekvensfilter, eller mer allmänt \emph{filter}, används inom radiotekniken för
många olika ändamål, till exempel för att
\begin{itemize}
  \item eliminera störande signaler
  \item öka avstämningsskärpan (selektiviteten) i mottagare och sändare
  \item framhäva eller dämpa ett sidband i en AM-signal med mera.
\end{itemize}

\emph{Frekvensgången} (eng. \emph{frequency response}) är ett mått på ett
filters förmåga att släppa igenom olika mycket av olika frekvenser.
Frekvensgången presenteras i allmänhet som en kurva med amplitud av genomsläppt
sinussignal som funktion av frekvensen.

Beroende på frekvensgången indelas filtren i olika typer, varav de vanligaste
presenteras här.

Beroende på det tekniska utförandet finns dels så kallade passiva filter vilka
använder extern energi för sin funktion, och dels aktiva filter vilka i princip
är förstärkare som likaledes använder passiva kretsar.
Här presenteras för enkelhets skull passiva filter.

Man skiljer även mellan analoga filter och digitala filter.
Vi beskriver här först några olika typer av klassiska analoga filter.

\subsection{Högpassfilter (HP)}
\harecsection{\harec{a}{3.2.8b}{3.2.8b}, \harec{a}{3.2.9}{3.2.9a}}
\index{högpassfilter}
\index{filter!högpass (HP)}
\index{highpass filter}
\index{HP}

\mediumtopfig{images/cropped_pdfs/bild_2_3-22.pdf}{Högpassfilter}{fig:BildII3-22}


Ett \emph{högpassfilter} (eng. \emph{highpass filter (HP)},
bild~\ssaref{fig:BildII3-22}) släpper igenom signaler med höga frekvenser och
dämpar de med låga frekvenser.

\paragraph{Exempel} En frekvensberoende spänningsdelare som LC-högpassfilter.

Vid låga frekvenser är \(X_C\) stor och \(X_L\) liten.
Över \(X_L\) uppstår då ett litet spänningsfall -- en låg utgångsspänning \(U_a\).
Resultatet blir att låga frekvenser dämpas.

Vid höga frekvenser är \(X_C\) liten och \(X_L\) stor.
Över \(X_L\) uppstår då ett stort spänningsfall -- en hög utgångsspänning
\(U_a\).
Resultatet blir att höga frekvenser släpps igenom.

\(X_L\) kan bytas ut mot en resistor \(R\), men då blir passbandskurvan inte lika brant.

\paragraph{Gränsfrekvens}

Gränsfrekvensen \(f_g\) beror av kapacitansen \(C\), induktansen \(L\) samt
resistansen \(R\).

\paragraph{LC-högpass}
\begin{gather*}
  f_g = \frac{1}{2\pi \sqrt{LC}} \\
  f_g\ \text{[Hz]} \quad L\ \text{[H]} \quad C\ \text{[F]}
\end{gather*}

\paragraph{RC-högpass}
\begin{gather*}
  f_g = \frac{1}{2\pi RC}\\
  f_g\ \text{[Hz]} \quad R\ [\unit{\ohm}] \quad C\ \text{[F]}
\end{gather*}

\paragraph{Räkneexempel}
\begin{enumerate}
\item \(L = 4\ \text{H} \quad C = 1\ \unit{\micro\farad} \quad f_g =\ ?\)
  \[
  f_g = \frac{1}{2\pi \sqrt{4 \cdot 10^{-6}}} = \frac{500}{2\pi }
  = 79,6\ \text{Hz}
  \]
\item \(R = \qty{1}{\kilo\ohm} \quad C = 10\ \text{nF} \quad f_g =\ ?\)
  \[
    f_g = \frac{1}{2\pi  \cdot 1 \cdot 10^3 \cdot 10 \cdot 10^{-9}}
    = \frac{10^5}{2\pi } = \qty{15,9}{\kilo\hertz}
  \]
\end{enumerate}


\subsection{Lågpassfilter (LP)}
\harecsection{\harec{a}{3.2.8a}{3.2.8a}, \harec{a}{3.2.9}{3.2.9b}}
\index{lågpassfilter}
\index{filter!lågpass (LP)}
\index{lowpass filter}
\index{LP|see {lågpassfilter}}
\label{lågpassfilter}

\mediumtopfig{images/cropped_pdfs/bild_2_3-23.pdf}{Lågpassfilter}{fig:BildII3-23}

Om induktor och kondensator respektive resistor och kondensator i ett
högpassfilter byter plats, som i bild~\ssaref{fig:BildII3-23}, så får man i
stället ett LC-lågpassfilter respektive ett RC-lågpassfilter.

Ett \emph{lågpassfilter} (eng. \emph{lowpass filter (LP)}) släpper igenom
signaler med låga frekvenser och dämpar de med höga frekvenser.

\paragraph{Exempel} En frekvensberoende spänningsdelare som LC-lågpassfilter.

Vid låga frekvenser är \(X_C\) stor och \(X_L\) liten.
Över \(X_L\) uppstår då ett litet spänningsfall -- en hög utgångsspänning
\(U_a\).
Resultatet blir att låga frekvenser släpps igenom.

Vid höga frekvenser är \(X_C\) liten och \(X_L\) stor.
Över \(X_L\) uppstår då ett stort spänningsfall -- en låg utgångsspänning
\(U_a\).
Resultatet blir att höga frekvenser dämpas.

\paragraph{Gränsfrekvens}

Samma formler används vid beräkning av gränsfrekvensen både i lågpass- och
högpassfilter, således

\paragraph{LC-lågpass}
\begin{gather*}
  f_g = \frac{1}{2\pi \sqrt{LC}} \\
  f_g\ \text{[Hz]} \quad L\ \text{[H]} \quad C\ \text{[F]}
\end{gather*}

\paragraph{RC-lågpass}
\begin{gather*}
  f_g = \frac{1}{2\pi {RC}} \\
  f_g\ \text{[Hz]} \quad R\ [\unit{\ohm}] \quad C\ \text{[F]}
\end{gather*}

\subsection{Bandpassfilter (BP)}
\harecsection{\harec{a}{3.2.7}{3.2.7}, \harec{a}{3.2.8c}{3.2.8c}, \harec{a}{3.2.9}{3.2.9c}}
\index{bandpassfilter}
\index{filter!bandpass (BP)}
\index{bandpass filter}
\index{BP}

\mediumtopfig{images/cropped_pdfs/bild_2_3-24.pdf}{Bandpassfilter}{fig:BildII3-24}

Ett \emph{bandpassfilter} (eng. \emph{bandpass filter}) släpper igenom signaler
bara inom ett visst frekvensområde medan signaler utanför detta frekvensområde dämpas.

Bandpassfiltret består i enklaste fall av två resonanskretsar av LC-typ, vilka
är avstämda till angränsande frekvenser.
Kretsarna är kopplade induktivt, kapacitivt eller galvaniskt så som illustreras
i bild~\ssaref{fig:BildII3-24}.

Beroende på kopplingsgrad eller dämpning skiljer man mellan underkritisk
koppling (lös koppling), kritisk koppling och överkritisk koppling
(fast koppling).

I bild~\ssaref{fig:BildII3-24} visas hur passbandet påverkas bland annat av
kopplingsgraden.
Lös koppling liten bandbredd.
Kritisk koppling -- större bandbredd.
Fast koppling -- stor bandbredd.

\mediumtopfig[0.6]{images/cropped_pdfs/bild_2_3-25.pdf}{Passfilter}{fig:BildII3-25}
\mediumtopfig{images/cropped_pdfs/bild_2_3-26.pdf}{Bandspärrfilter}{fig:BildII3-26}
\mediumherefig{images/cropped_pdfs/bild_2_3-27.pdf}{Spärrfilter (2 sorter)}{fig:BildII3-27}
\newpage

\subsection{Passfilter}
\harecsection{\harec{a}{3.2.9}{3.2.9d}}
\index{passfilter}
\index{filter!bandpass (BP)}
\index{pass filter|see {passfilter}}
\index{BP}

Passkretsen eller passfilter stäms av till en viss frekvens och erbjuder där
en mycket låg impedans så som illustreras i bild~\ssaref{fig:BildII3-25}.
Passkretsen kopplas i serie med signalvägen och låter signaler med
frekvenser inom filtrets passband att passera.


\subsection{Bandspärrfilter}
\harecsection{\harec{a}{3.2.8d}{3.2.8d}, \harec{a}{3.2.9}{3.2.9e}}
\index{bandspärrfilter}
\index{filter!bandspärr (BR)}
\index{band reject filter (BR)}
\index{BR}

Om serie- och parallellkretsarna i ett bandpassfilter byter plats får man
i stället ett bandspärrfilter så som illustreras i bild~\ssaref{fig:BildII3-26}.
Ett sådant spärrar signaler inom ett visst frekvensområde, men släpper igenom
signaler utom detta område.

\newpage
\subsection{Spärrfilter}
\index{bandspärrfilter}
\index{filter!bandspärr (BR)}
\index{band reject filter (BR)}
\index{BR}
\index{spärrkrets}
\index{sugkrets}


\subsubsection{Spärrkrets}
Spärrkretsen stäms av till en viss frekvens och erbjuder där en mycket hög
impedans.
Spärrkretsen kopplas i serie med signalvägen och spärrar en signal med samma
frekvens som resonansfrekvensen, så som illustreras i bild~\ssaref{fig:BildII3-27}.

\subsubsection{Sugkrets}
Sugkretsen stäms av till en viss frekvens och erbjuder där en mycket låg
impedans.
Sugkretsen kopplas parallellt med signalvägen och kortsluter (suger bort) en
signal med samma frekvens som resonansfrekvensen, så som illustreras i bild
\ssaref{fig:BildII3-27}.

\mediumfig[0.8]{images/cropped_pdfs/bild_2_3-30.pdf}{Mekaniskt filter}{fig:BildII3-30}

\subsection{Kvartskristall}

\harecsection{\harec{a}{3.2.11}{3.2.11}}
\index{kvartskristall}
\index{quartz crystal}
\index{crystal}
\index{Q-värde}
\index{resonator}

\smallfig{images/cropped_pdfs/bild_2_3-28.pdf}{Kvartskristall}{fig:BildII3-28}

En \emph{kvartskristall} (eng. \emph{quartz crystal} eller \emph{crystal}),
egentligen en slipad skiva av kvarts, kan fungera som en
elektromekanisk svängningskropp (resonator), vars egenskaper liknar dem i en
LC-krets.
Detta illustreras i bild~\ssaref{fig:BildII3-28}.

Den låga inre resistansen gör att Q-värdet i en kvartskristall är bättre än
10000.
Som jämförelse är Q-värdet i en LC-krets oftast sämre än 1000.

Många moderna kvartskristaller kan uppvisa olastat Q-värde på 100000.

\vspace{12pt} % Undgår brytning av nästa titelrad

\subsection{Bandfilter med kvartskristaller}
\index{kristallfilter}
\index{crystal filter}
\index{keramiska resonatorer}
\index{ceramic resonators}
\label{bandfilter_kristall}

\smallfigpad{images/cropped_pdfs/bild_2_3-29.pdf}{Bandfilter med kvartskristaller}{fig:BildII3-29}

Bild~\ssaref{fig:BildII3-29} visar hur kvartskristaller kan kombineras till
filter, ofta refererade till som \emph{kristallfilter} (eng.
\emph{crystal filter}), med önskad bandbredd.
Även utföranden med \emph{keramiska resonatorer} (eng.
\emph{ceramic resonators}) finns.
Resonatorerna är avstämda till var sin bestämda frekvens och hela komplexet
bidrar på så sätt till att bilda passband eller andra egenskaper på samma sätt
som med sammankopplade LC-kretsar.

\subsection{Mekaniska filter}
\index{mekaniskt filter}
\index{mechanical filter}
\index{mekanisk resonator}
\index{resonator!mekanisk}

Med en elektromekanisk givare kan man få en kropp (resonator) att svänga på sin
resonansfrekvens.
Med ännu en elektromekanisk givare kan man känna av svängningarna och
åter omvandla dem till elektriska signaler.
Bild~\ssaref{fig:BildII3-30} illustrerar ett sådant arrangemang.
Hela anordningen fungerar som en \emph{elektromekanisk resonator} (eng.
\emph{mechanical resonator}), vars egenskaper liknar dem i en LC-krets.

% 

Resonatorerna kan kombineras till filterkomplex med önskad bandbredd där
resonatorerna är avstämda till var sin bestämd frekvens.
Hela komplexet bidrar på så sätt till att bilda ett passband på samma sätt som
med sammankopplade LC-kretsar.
Beroende på tillämpningen finns olika frekvenslägen i intervallet
\SIrange{60}{600}{\kilo\hertz}.

\emph{Mekaniska filter} (eng. \emph{mechanical filter}) användes mest förr som
mellanfrekvensfilter i högvärdiga radioutrustningar, men har numera till stor
del ersatts av bandfilter med kvartskristaller där arbetsområdet kan ligga
avsevärt högre i frekvens.

\newpage % layout
\subsection{Kavitetsfilter}
\index{kavitetsfilter}
\index{cavity filter}
\index{filter!kavitet}

\smallfig[0.3]{images/cropped_pdfs/bild_2_3-31.pdf}{Kavitetsfilter}{fig:BildII3-31}

Resonanskretsars dimensioner minskar med ökande frekvens.
Vid mycket hög frekvens kan induktorns varvtal i en LC-krets ha minskat till
ett enda varv samtidigt som kapacitansen inom detta enda varv kan räcka för
önskad resonansfrekvens.

En sådan resonanskrets kan bland annat ha formen av en ledare mitt inne i En
elektriskt ledande kavitet, så som illustreras i bild~\ssaref{fig:BildII3-31}.
Ledarens längd tillsammans med kavitetens insida bildar induktorn.
Mellan ledaren och kavitetens insida råder en kapacitans, som kan
kompletteras/justeras med en extra kondensator.

%% k7per: Remove explicit formatting.
%% \newpage % layout

Inkommande och utgående signaler ansluts till filtrets mittledare över
induktionsslingor, kondensatorer eller direkt galvaniskt.
\emph{Kavitetsfilter} (eng. \emph{cavity filter}) kan kopplas ihop för att till
exempel bilda bandfilter eller frekvensdelare.

Kavitetsfilter används ofta på sändare eftersom de med sina låga förluster kan
hantera stora effekter samt åstadkomma djupa utsläckningar.
Dessa egenskaper gör dem oerhört lämpliga som duplexfilter till repeatrar.

\subsection{Helixfilter}
\index{helixfilter}
\index{filter!helix}

När ett kompakt kavitetsfilter behövs kan man öka reaktansen i mittledaren
både induktivt och kapacitivt genom att utforma den som en spiral (helix).
Detta sker dock på bekostnad av Q-värdet.
Flera kavitetsfilter kan kopplas ihop för att bilda till exempel bandfilter
eller spärrfilter.

\subsection{Pi-filter}
\harecsection{\harec{a}{3.2.10a}{3.2.10a}}
\index{Pi-filter}
\index{filter!Pi-filter}

\smallfigpad[0.35]{images/cropped_pdfs/bild_2_3-32.pdf}{Pi-filter}{fig:BildII3-32}

För att överföra HF-signaler med bästa verkningsgrad är det viktigt med god
impedansanpassning mellan de olika kretsarna.
Om anslutningsimpedansen är lika i båda kretsarna behövs inga extra åtgärder.
Om impedanserna däremot är olika behövs korrigeringsnät (filter).

Ofta är nätet Pi-format så som bild~\ssaref{fig:BildII3-32} visar och består av
induktanser och kapacitanser.
Ett Pi-format nät kan sägas bestå av två L-formade nät ställda mot varandra, där
den seriella delen är gemensam (på bilden en induktor).

%% k7per
% \newpage % layout
\subsection{T-filter}
\harecsection{\harec{a}{3.2.10b}{3.2.10b}}
\index{T-filter}
\index{filter!T-filter}

\smallfig[0.35]{images/cropped_pdfs/bild_2_3-33.pdf}{T-filter (två varianter)}{fig:BildII3-33}
\smallfigpad[0.3]{images/cropped_pdfs/bild_2_3-34.pdf}{Halvledardioder}{fig:BildII3-34}

Ett nät kan också vara T-format, som bild~\ssaref{fig:BildII3-33} visar, och bestå
av induktanser och kapacitanser.
Ett sådant nät kan sägas bestå av två L-formade nät ställda ''rygg mot rygg''.
Då är den parallella delen gemensam.
På bilden visas två alternativ.

När den parallella delen är kapacitiv, blir huvudkaraktären ett lågpassfilter.
När den parallella delen i stället är induktiv blir huvudkaraktären ett
högpassfilter.

Ett Pi- eller T-filter kan fungera som
\begin{itemize}
  \item resonanskrets
  \item impedanstransformator (anpassning)
  \item neutralisering av reaktans
\end{itemize}

\subsection{Icke-ideala komponenter}
\harecsection{\harec{a}{3.2.12}{3.2.12}}

I verkligheten är alla analoga komponenter även behäftade med oönskade
egenskaper, även kallade parasitiska egenskaper.

Ett motstånd uppvisar inte enbart en strikt resistiv egenskap, utan för högre
frekvenser kommer även en parasitisk seriekopplad induktans att göra sig påmind.

En kondensator har inte perfekt isolation.
Genom en parasitisk resistans parallellkopplad med kondensatorplattorna flyter
en läckström, som kommer att ladda ur kondensatorn.

En induktor är inte perfekt förlustfri, utan den har en parasitisk serieresistans.

För kondensatorer och induktorer kommer resistansen att påverka deras Q-värde.
Ett högt Q-värde innebär att man har låg förlust.
Förlusterna kommer att göra sig påminda när man bygger kretsar med dessa
komponenter.
Till exempel kommer en LC krets i praktiken alltid att vara en LCR-krets, där
förlusterna i spole och kondensator ger en förlust i kretsen och begränsar hur
högt Q-värde som kan uppnås.
När en resonator belastas ökar förlusten ytterligare och därmed sjunker Q-värdet.

% \mediumminusbotfig{images/cropped_pdfs/bild_2_3-35.pdf}{Halv- och helvågslikriktning}{fig:BildII3-35}

\subsection{Digitala filter}
\harecsection{\harec{a}{3.2.13}{3.2.13}}
\index{digitala filter}

Utvecklingen går mot att allt mer signalbehandling sker digitalt. 
\emph{Digitala filter} kan utnyttjas genom att signalen först konverteras till
digital form och sedan efter filtreringen konverteras tillbaka till analog form.
Digitala filter har många fördelar.
Man kan konstruera komplicerade och skarpa filter som behåller sina egenskaper
över tiden, där klassiska analoga kan behöva trimmas både individuellt vid
tillverkning och över tiden för att upprätthålla sina egenskaper.

För mer om digitala filter se \ssaref{DSP} samt \ssaref{digitala filter}.
